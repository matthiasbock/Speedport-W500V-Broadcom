\documentclass[b5paper]{book}
\usepackage{makeidx}
\usepackage{amssymb}
\usepackage{color}
\usepackage{alltt}
\usepackage{graphicx}
\usepackage{layout}
\def\union{\cup}
\def\intersect{\cap}
\def\getsrandom{\stackrel{\rm R}{\gets}}
\def\cross{\times}
\def\cat{\hspace{0.5em} \| \hspace{0.5em}}
\def\catn{$\|$}
\def\divides{\hspace{0.3em} | \hspace{0.3em}}
\def\nequiv{\not\equiv}
\def\approx{\raisebox{0.2ex}{\mbox{\small $\sim$}}}
\def\lcm{{\rm lcm}}
\def\gcd{{\rm gcd}}
\def\log{{\rm log}}
\def\ord{{\rm ord}}
\def\abs{{\mathit abs}}
\def\rep{{\mathit rep}}
\def\mod{{\mathit\ mod\ }}
\renewcommand{\pmod}[1]{\ ({\rm mod\ }{#1})}
\newcommand{\floor}[1]{\left\lfloor{#1}\right\rfloor}
\newcommand{\ceil}[1]{\left\lceil{#1}\right\rceil}
\def\Or{{\rm\ or\ }}
\def\And{{\rm\ and\ }}
\def\iff{\hspace{1em}\Longleftrightarrow\hspace{1em}}
\def\implies{\Rightarrow}
\def\undefined{{\rm ``undefined"}}
\def\Proof{\vspace{1ex}\noindent {\bf Proof:}\hspace{1em}}
\let\oldphi\phi
\def\phi{\varphi}
\def\Pr{{\rm Pr}}
\newcommand{\str}[1]{{\mathbf{#1}}}
\def\F{{\mathbb F}}
\def\N{{\mathbb N}}
\def\Z{{\mathbb Z}}
\def\R{{\mathbb R}}
\def\C{{\mathbb C}}
\def\Q{{\mathbb Q}}
\definecolor{DGray}{gray}{0.5}
\newcommand{\url}[1]{\mbox{$<${#1}$>$}}
\newcommand{\emailaddr}[1]{\mbox{$<${#1}$>$}}
\def\twiddle{\raisebox{0.3ex}{\mbox{\tiny $\sim$}}}
\def\gap{\vspace{0.5ex}}
\makeindex
\begin{document}
\frontmatter
\pagestyle{empty}
\title{Multiple-Precision Integer Arithmetic, \\ A Case Study Involving the LibTomMath Project \\ - DRAFT - }
\author{\mbox{
%\begin{small}
\begin{tabular}{c}
Tom St Denis \\
Algonquin College \\
\\
Mads Rasmussen \\
Open Communications Security \\
\\
Gregory Rose \\
Qualcomm \\
\end{tabular}
%\end{small}
}
}
\maketitle
This text in its entirety is copyrighted \copyright{}2003 by Tom St Denis.  It may not be redistributed 
electronically or otherwise without the sole permission of the author.  The text is freely re distributable as long as
it is packaged along with the LibTomMath project in a non-commercial project.  Contact the
author for other redistribution rights.

This text corresponds to the v0.17 release of the LibTomMath project.

\begin{alltt}
Tom St Denis
111 Banning Rd
Ottawa, Ontario
K2L 1C3
Canada

Phone: 1-613-836-3160
Email: tomstdenis@iahu.ca
\end{alltt}

This text is formatted to the international B5 paper size of 176mm wide by 250mm tall using the \LaTeX{} 
{\em book} macro package and the Perl {\em booker} package.

\tableofcontents
\listoffigures
\chapter*{Preface}
Blah.

\mainmatter
\pagestyle{headings}
\chapter{Introduction}
\section{Multiple Precision Arithmetic}
\subsection{The Need for Multiple Precision Arithmetic}
The most prevalent use for multiple precision arithmetic (\textit{often referred to as bignum math}) is within public
key cryptography.   Algorithms such as RSA, Diffie-Hellman and Elliptic Curve Cryptography require large integers in order to 
resist known cryptanalytic attacks.  Typical modern programming languages such as C and Java only provide small 
single-precision data types which are incapable of precisely representing integers which are often hundreds of bits long.

For example, consider multiplying $1,234,567$ by $9,876,543$ in C with an ``unsigned long'' data type.  With an 
x86 machine the result is $4,136,875,833$ while the true result is $12,193,254,061,881$.  The original inputs 
were approximately $21$ and $24$ bits respectively.  If the C language cannot multiply two relatively small values 
together precisely how does anyone expect it to multiply two values which are considerably larger?

Most advancements in fast multiple precision arithmetic stems from the desire for faster cryptographic primitives.  However, cryptography
is not the only field of study that can benefit fast large integer routines.  Another auxiliary use for multiple precision integers is 
high precision floating point data types.  The basic IEEE standard floating point type is made up of an integer mantissa $q$ and an exponent $e$.  
Numbers are given in the form $n = q \cdot b^e$ where $b = 2$ is convention.  Since IEEE is meant to be implemented in 
hardware the precision of the mantissa is often fairly small (\textit{roughly 23 bits}).  Since the mantissa is merely an 
integer a large multiple precision integer could be used.  In effect very high precision floating point arithmetic 
could be performed.  This would be useful where scientific applications must minimize the total output error over long simulations.  

\subsection{Multiple Precision Arithmetic}
\index{multiple precision}
Multiple precision arithmetic attempts to the solve the shortcomings of single precision data types such as those from
the C and Java programming languages.  In essence multiple precision arithmetic is a set of operations that can be 
performed on members of an algebraic group whose precision is not fixed.  The algorithms when implemented to be multiple
precision can allow a developer to work with any practical precision required.

Typically the arithmetic is performed over the ring of integers denoted by a $\Z$ and referred to casually as ``bignum'' 
routines.  However, it is possible to have rings of polynomials as well typically denoted by $\Z/p\Z \left [ X \right ]$ 
which could have variable precision (\textit{or degree}).  This text will discuss implementation of the former, however,
implementing polynomial basis routines should be relatively easy after reading this text.

\subsection{Benefits of Multiple Precision Arithmetic}
\index{precision} \index{accuracy}
Precision is defined loosely as the proximity to the real value a given representation is.  Accuracy is defined as the 
reproducibility of the result.  For example, the calculation $1/3 = 0.25$ is imprecise but can be accurate provided 
it is reproducible.

The benefit of multiple precision representations over single precision representations is that 
often no precision is lost while representing the result of an operation which requires excess precision.  For example, 
the multiplication of two $n$-bit integers requires at least $2n$ bits to represent the result.  A multiple precision 
system would augment the precision of the destination to accomodate the result while a single precision system would
truncate excess bits to maintain a fixed level of precision.

Multiple precision representations allow for the precision to be very high (\textit{if not exacting}) but at a cost of
modest computer resources.  The only reasonable case where a multiple precision system will lose precision is when
emulating a floating point data type.  However, with multiple precision integer arithmetic no precision is lost.

\subsection{Basis of Operations}
At the heart of all multiple precision integer operations are the ``long-hand'' algorithms we all learnt as children 
in grade school.  For example, to multiply $1,234$ by $981$ the student is not taught to memorize the times table for 
$1,234$ instead they are taught how to long-multiply.  That is to multiply each column using simple single digit 
multiplications and add the resulting products by column.  The representation that most are familiar with is known as 
decimal or formally as radix-10. A radix-$n$ representation simply means there are $n$ possible values per digit.  
For example, binary would be a radix-2 representation.

In essence computer based multiple precision arithmetic is very much the same.  The most notable difference is the usage
of a binary friendly radix.  That is to use a radix of the form $2^k$ where $k$ is typically the size of a machine 
register.  Also occasionally more optimal algorithms are used to perform certain operations such as multiplication and 
squaring instead of traditional long-hand algorithms.

\section{Purpose of This Text}
The purpose of this text is to instruct the reader regarding how to implement multiple precision algorithms.  That is 
to not only explain the core theoretical algorithms but also the various ``house keeping'' tasks that are neglected by
authors of other texts on the subject.  Texts such as Knuths' ``The Art of Computer Programming, vol 2.'' and the 
Handbook of Applied Cryptography (\textit{HAC}) give considerably detailed explanations of the theoretical aspects of 
the algorithms and very little regarding the practical aspects.  

That is how an algorithm is explained and how it is actually implemented are two very different 
realities.  For example, algorithm 14.7 on page 594 of HAC lists a relatively simple algorithm for performing multiple 
precision integer addition.  However, what the description lacks is any discussion concerning the fact that the two 
integer inputs may be of differing magnitudes.  Similarly the division routine (\textit{Algorithm 14.20, pp. 598}) 
does not discuss how to handle sign or handle the dividends decreasing magnitude in the main loop (\textit{Step \#3}).

As well as the numerous practical oversights both of the texts do not discuss several key optimal algorithms required 
such as ``Comba'' and Karatsuba multipliers and fast modular inversion.  These optimal algorithms are considerably
vital to achieve any form of useful performance in non-trivial applications.  

To solve this problem the focus of this text is on the practical aspects of implementing the algorithms that 
constitute a multiple precision integer package with light cursory discussions on the theoretical aspects.  As a case 
study the ``LibTomMath''\footnote{Available freely at http://math.libtomcrypt.org} package is used to demonstrate 
algorithms with implementations that have been field tested and work very well.

\section{Discussion and Notation}
\subsection{Notation}
A multiple precision integer of $n$-digits shall be denoted as $x = (x_n ... x_1 x_0)_{ \beta }$ to be the 
multiple precision notation for the integer $x \equiv \sum_{i=0}^{n} x_i\beta^i$.  The elements of the array $x$ are
said to be the radix $\beta$ digits of the integer.  For example, $x = (15,0,7)_{\beta}$ would represent the 
integer $15\cdot\beta^2 + 0\cdot\beta^1 + 7\cdot\beta^0$.  

A ``mp\_int'' shall refer to a composite structure which contains the digits of the integer as well as auxilary data
required to manipulate the data.  These additional members are discussed in chapter three.  For the purposes of this text
a ``multiple precision integer'' and a ``mp\_int'' are synonymous.

\index{single-precision} \index{double-precision} \index{mp\_digit} \index{mp\_word}
For the purposes of this text a single-precision variable must be able to represent integers in the range $0 \le x < 2 \beta$ while
a double-precision variable must be able to represent integers in the range $0 \le x < 2 \beta^2$.  Within the source code that will be
presented the data type \textbf{mp\_digit} will represent a single-precision type while \textbf{mp\_word} will represent a 
double-precision type.  In several algorithms (\textit{notably the Comba routines}) temporary results 
will be stored in a double-precision arrays.  For the purposes of this text $x_j$ will refer to the 
$j$'th digit of a single-precision array and $\hat x_j$ will refer to the $j$'th digit of a double-precision
array.

\subsection{Work Effort}
\index{big-O}
To measure the efficiency of various algorithms a modified big-O notation is used.  In this system all 
single precision operations are considered to have the same cost\footnote{Except where explicitly noted.}.  
That is a single precision addition, multiplication and division are assumed to take the same time to 
complete.  While this is generally not true in practice it will simplify the discussions considerably.

Some algorithms have slight advantages over others which is why some constants will not be removed in 
the notation.  For example, a normal multiplication requires $O(n^2)$ work while a squaring requires 
$O({{n^2 + n}\over 2})$ work.  In standard big-O notation these would be said to be equivalent.  However, in the 
context of the this text the magnitude of the inputs will not approach an infinite size.  This means the conventional limit 
notation wisdom does not apply to the cancellation of constants.

Throughout the discussions various ``work levels'' will be discussed.  These levels are the $O(1)$,
$O(n)$, $O(n^2)$, ..., $O(n^k)$ work efforts.  For example, operations at the $O(n^k)$ ``level'' are said to be
executed more frequently than operations at the $O(n^m)$ ``level'' when $k > m$.  Obviously most optimizations will pay
off the most at the higher levels since they represent the bulk of the effort required.  

\section{Exercises}
Within the more advanced chapters a section will be set aside to give the reader some challenging exercises.  These exercises are not 
designed to be prize winning problems yet instead to be thought provoking.  Wherever possible the problems are foreward minded stating 
problems that will be answered in subsequent chapters.  The reader is encouraged to finish the exercises as they appear to get a 
better understanding of the subject material.  

Similar to the exercises of \cite{TAOCPV2} as explained on pp.\textit{ix} these exercises are given a scoring system.  However, unlike 
\cite{TAOCPV2} the problems do not get nearly as hard as often.  The scoring of these exercises ranges from one (\textit{the easiest}) to
five (\textit{the hardest}).  The following table sumarizes the scoring.

\vspace{5mm}
\begin{tabular}{cl}
$\left [ 1 \right ]$ & An easy problem that should only take the reader a manner of \\
                     & minutes to solve.  Usually does not involve much computer time. \\
                     & \\
$\left [ 2 \right ]$ & An easy problem that involves a marginal amount of computer \\
                     & time usage.  Usually requires a program to be written to \\
                     & solve the problem. \\
                     & \\
$\left [ 3 \right ]$ & A moderately hard problem that requires a non-trivial amount \\
                     & of work.  Usually involves trivial research and development of \\
                     & new theory from the perspective of a student. \\
                     & \\
$\left [ 4 \right ]$ & A moderately hard problem that involves a non-trivial amount \\
                     & of work and research.  The solution to which will demonstrate \\
                     & a higher mastery of the subject matter. \\
                     & \\
$\left [ 5 \right ]$ & A hard problem that involves concepts that are non-trivial.  \\
                     & Solutions to these problems will demonstrate a complete mastery \\
                     & of the given subject. \\
                     & \\
\end{tabular}

Essentially problems at the first level are meant to be simple questions that the reader can answer quickly without programming a solution or
devising new theory.  These problems are quick tests to see if the material is understood.  Problems at the second level are also
designed to be easy but will require a program or algorithm to be implemented to arrive at the answer.  

Problems at the third level are meant to be a bit more difficult.  Often the answer is fairly obvious but arriving at an exacting solution
requires some thought and skill.  These problems will almost always involve devising a new algorithm or implementing a variation of
another algorithm.

Problems at the fourth level are meant to be even more difficult as well as involve some research.  The reader will most likely not know
the answer right away nor will this text provide the exact details of the answer (\textit{or at least not until a subsequent chapter}).  Problems
at the fifth level are meant to be the hardest problems relative to all the other problems in the chapter.  People who can correctly 
answer fifth level problems have a mastery of the subject matter at hand.

Often problems will be tied together.  The purpose of this is to start a chain of thought that will be discussed in future chapters.  The reader
is encouraged to answer the follow-up problems and try to draw the relevence of problems.

\chapter{Introduction to LibTomMath}

\section{What is the LibTomMath?}
LibTomMath is a free and open source multiple precision number theoretic library written in portable ISO C
source code.  By portable it is meant that the library does not contain any code that is platform dependent or otherwise
problematic to use on any given platform.  The library has been successfully tested under numerous operating systems 
including Solaris, MacOS, Windows, Linux, PalmOS and on standalone hardware such as the Gameboy Advance.  The 
library is designed to contain enough functionality to be able to develop number theoretic applications such as public 
key cryptosystems.

\section{Goals of the LibTomMath}

Even though the library is written entirely in portable ISO C considerable care has been taken to 
optimize the algorithm implementations within the library.  Specifically the code has been written to work well with
the GNU C Compiler (\textit{GCC}) on both x86 and ARMv4 processors.  Wherever possible optimal 
algorithms (\textit{such as Karatsuba multiplication, sliding window exponentiation and Montgomery reduction.}) have 
been provided to make the library as efficient as possible.  Even with the optimal and sometimes specialized 
algorithms that have been included the API has been kept as simple as possible.  Often generic place holder routines 
will make use of specialized algorithms automatically without the developers attention.  One such example
is the generic multiplication algorithm \textbf{mp\_mul()} which will automatically use Karatsuba multiplication if the 
inputs are of a specific size.

Making LibTomMath as efficient as possible is not the only goal of the LibTomMath project.  Ideally the library should 
be source compatible with another popular library which makes it more attractive for developers to use.  In this case the
MPI library was used as a API template for all the basic functions.

The project is also meant to act as a learning tool for students.  The logic being that no easy to follow ``bignum'' 
library exists which can be used to teach computer science students how to perform fast and reliable multiple precision 
arithmetic.  To this end the source code has been given quite a few comments and algorithm discussion points.  Often 
where applicable routines have more comments than lines of code.

\section{Choice of LibTomMath}
LibTomMath was chosen as the case study of this text not only because the author of both projects is one and the same but
for more worthy reasons.  Other libraries such as GMP, MPI, LIP and OpenSSL have multiple precision 
integer arithmetic routines but would not be ideal for this text for numerous reasons as will be explained in the 
following sub-sections.

\subsection{Code Base}
The LibTomMath code base is all portable ISO C source code.  This means that there are no platform dependent conditional
segments of code littered throughout the source.  This clean and uncluttered approach to the library means that a
developer can more readily ascertain the true intent of a given section of source code without trying to keep track of
what conditional code will be used.

The code base of LibTomMath is also exceptionally well organized.  Each function is in its own separate source code file 
which allows the reader to find a given function very fast.  When compiled with GCC for the x86 processor the entire 
library is a mere 87,760 bytes (\textit{$116,182$ bytes for ARMv4 processors}).  This includes every single function 
LibTomMath provides from basic arithmetic to various number theoretic functions such as modular exponentiation, various 
reduction algorithms and Jacobi symbol computation.  

By comparison MPI which has fewer number theoretic functions than LibTomMath compiled with the same conditions is 
45,429 bytes (\textit{$54,536$ for ARMv4}).  GMP which has rather large collection of functions with the default 
configuration on an x86 Athlon is 2,950,688 bytes.  Note that while LibTomMath has fewer functions than GMP it has been
been used as the sole basis for several public key cryptosystems without having to seek additional outside functions
to supplement the library.

\subsection{API Simplicity}
LibTomMath is designed after the MPI library and shares the API design.  Quite often programs that use MPI will build 
with LibTomMath without change. The function names are relatively straight forward as to what they perform.  Almost all of the 
functions except for a few minor exceptions which as will be discussed are for good reasons share the same parameter passing 
convention.  The learning curve is fairly shallow with the API provided which is an extremely valuable benefit for the 
student and developer alike.  

The LIP library is an example of a library with an API that is awkward to work with.  LIP uses function names that are often ``compressed'' to 
illegible short hand.  LibTomMath does not share this fault.

\subsection{Optimizations}
While LibTomMath is certainly not the fastest library (\textit{GMP often beats LibTomMath by a factor of two}) it does
feature a set of optimal algorithms for tasks ranging from modular reduction to squaring.  GMP and LIP also feature
such optimizations while MPI only uses baseline algorithms with no optimizations.

LibTomMath is almost always a magnitude faster than the MPI library at computationally expensive tasks such as modular
exponentiation.  In the grand scheme of ``bignum'' libraries LibTomMath is faster than the average library and usually  
slower than the best libraries such as GMP and OpenSSL by a small factor.

\subsection{Portability and Stability}
LibTomMath will build ``out of the box'' on any platform equipped with a modern version of the GNU C Compiler 
(\textit{GCC}).  This means that without changes the library will build without configuration or setting up any 
variables.  LIP and MPI will build ``out of the box'' as well but have numerous known bugs.  Most notably the author of 
MPI is not working on his library anymore.  

GMP requires a configuration script to run and will not build out of the box.   GMP and LibTomMath are still in active
development and are very stable across a variety of platforms.

\subsection{Choice}
LibTomMath is a relatively compact, well documented, highly optimized and portable library which seems only natural for
the case study of this text.  Various source files from the LibTomMath project will be included within the text.  However, the 
reader is encouraged to download their own copy of the library to actually be able to work with the library.  

\chapter{Getting Started}
\section{Library Basics}
To get the ``ball rolling'' so to speak a primitive data type and a series of primitive algorithms must be established.  First a data
type that will hold the information required to maintain a multiple precision integer must be designed.  With this basic data type of a series
of low level algorithms for initializing, clearing, growing and clamping integers can be developed to form the basis of the entire
package of algorithms.

\section{The mp\_int structure}
First the data type for storing multiple precision integers must be designed.  This data type must be able to hold information to 
maintain an array of digits, how many are actually used in the representation and the sign.  The ISO C standard does not provide for 
any such data type but it does provide for making composite data types known as structures.  The following is the structure definition 
used within LibTomMath.

\index{mp\_int}
\begin{verbatim}
typedef struct  {
    int used, alloc, sign;
    mp_digit *dp;
} mp_int;
\end{verbatim}

The \textbf{used} parameter denotes how many digits of the array \textbf{dp} are actually being used.  The array 
\textbf{dp} holds the digits that represent the integer desired.  The \textbf{alloc} parameter denotes how 
many digits are available in the array to use by functions before it has to increase in size.  When the \textbf{used} count 
of a result would exceed the \textbf{alloc} count all LibTomMath routines will automatically increase the size of the 
array to accommodate the precision of the result.  The \textbf{sign} parameter denotes the sign as either zero/positive 
(\textbf{MP\_ZPOS}) or negative (\textbf{MP\_NEG}).  

\section{Argument Passing}
A convention of arugment passing must be adopted early on in the development of any library.  Making the function prototypes
consistent will help eliminate many headaches in the future as the library grows to significant complexity.  In LibTomMath the multiple precision 
integer functions accept parameters from left to right as pointers to mp\_int structures.  That means that the source operands are 
placed on the left and the destination on the right.   Consider the following examples.

\begin{verbatim}
   mp_mul(&a, &b, &c);   /* c = a * b */
   mp_add(&a, &b, &a);   /* a = a + b */
   mp_sqr(&a, &b);       /* b = a * a */
\end{verbatim}

The left to right order is a fairly natural way to implement the functions since it lets the developer read aloud the
functions and make sense of them.  For example, the first function would read ``multiply a and b and store in c''.

Certain libraries (\textit{LIP by Lenstra for instance}) accept parameters the other way around.  That is the destination
on the left and arguments on the right.  In truth it is entirely a matter of preference.  

Another very useful design consideration is whether to allow argument sources to also be a destination.  For example, the
second example (\textit{mp\_add}) adds $a$ to $b$ and stores in $a$.  This is an important feature to implement since it
allows the higher up functions to cut down on the number of variables.  However, to implement this feature specific
care has to be given to ensure the destination is not written before the source is fully read.

\section{Return Values}
A well implemented library, no matter what its purpose, should trap as many runtime errors as possible and return them to the 
caller.  By catching runtime errors a library can be guaranteed to prevent undefined behaviour within reason.  In a multiple precision 
library the only errors that are bound to occur are related to inappropriate inputs (\textit{division by zero for instance}) or 
memory allocation errors.

In LibTomMath any function that can cause a runtime error will return an error as an \textbf{int} data type with one of the 
following values.

\index{MP\_OKAY} \index{MP\_VAL} \index{MP\_MEM}
\begin{center}
\begin{tabular}{|l|l|}
\hline \textbf{Value} & \textbf{Meaning} \\
\hline \textbf{MP\_OKAY} & The function was successful \\
\hline \textbf{MP\_VAL}  & One of the input value(s) was invalid \\
\hline \textbf{MP\_MEM}  & The function ran out of heap memory \\
\hline
\end{tabular}
\end{center}

When an error is detected within a function it should free any memory they allocated and return as soon as possible.  The goal
is to leave the system in the same state the system was when the function was called.  Error checking with this style of API is fairly simple.

\begin{verbatim}
   int err;
   if ((err = mp_add(&a, &b, &c)) != MP_OKAY) {
      printf("Error: %d\n", err);
      exit(EXIT_FAILURE);
   }
\end{verbatim}

The GMP library uses C style \textit{signals} to flag errors which is of questionable use.  Not all errors are fatal 
and it is not ideal to force developers to have signal handlers for such cases.

\section{Initialization and Clearing}
The logical starting point when actually writing multiple precision integer functions is the initialization and 
clearing of the integers.  These two functions will be used by far the most throughout the algorithms whenever 
temporary integers are required.

Given the basic mp\_int structure an initialization routine must first allocate memory to hold the digits of
the integer.  Often it is optimal to allocate a sufficiently large pre-set number of digits even considering
the initial integer will represent zero.  If only a single digit were allocated quite a few re-allocations
would occur for the majority of inputs.  There exists a tradeoff between how many default digits to allocate
and how many re-allocations are tolerable.  

If the memory for the digits has been successfully allocated then the rest of the members of the structure must
be initialized.  Since the initial state is to represent a zero integer the digits allocated must all be zeroed.  The
\textbf{used} count set to zero and \textbf{sign} set to \textbf{MP\_ZPOS}.

\subsection{Initializing an mp\_int}
To initialize an mp\_int the mp\_init algorithm shall be used.  The purpose of this algorithm is to allocate 
the memory required and initialize the integer to a default representation of zero.

\begin{figure}[here]
\begin{center}
\begin{tabular}{l}
\hline Algorithm \textbf{mp\_init}. \\
\textbf{Input}.   An mp\_int $a$ \\
\textbf{Output}.  Allocate memory for the digits and set to a zero state. \\
\hline \\
1.  Allocate memory for \textbf{MP\_PREC} digits. \\
2.  If the allocation failed then return(\textit{MP\_MEM}) \\
3.  for $n$ from $0$ to $MP\_PREC - 1$ do  \\
\hspace{3mm}3.1  $a_n \leftarrow 0$\\
4.  $a.sign \leftarrow MP\_ZPOS$\\
5.  $a.used \leftarrow 0$\\
6.  $a.alloc \leftarrow MP\_PREC$\\
7.  Return(\textit{MP\_OKAY})\\
\hline
\end{tabular}
\end{center}
\caption{Algorithm mp\_init}
\end{figure}

\textbf{Algorithm mp\_init.}
The \textbf{MP\_PREC} variable is a simple constant used to dictate minimal precision of allocated integers.  It is ideally at least equal to $32$ but 
can be any reasonable power of two.  Step one and two allocate the memory and account for it.  If the allocation fails the algorithm returns
immediately to signal the failure.  Step three will ensure that all the digits are in the default state of zero.  Finally steps 
four through six set the default settings of the \textbf{sign}, \textbf{used} and \textbf{alloc} members of the mp\_int structure.

\index{bn\_mp\_init.c}
\vspace{+3mm}\begin{small}
\hspace{-5.1mm}{\bf File}: bn\_mp\_init.c
\vspace{-3mm}
\begin{alltt}
016   
017   /* init a new bigint */
018   int
019   mp_init (mp_int * a)
020   \{
021     /* allocate ram required and clear it */
022     a->dp = OPT_CAST calloc (sizeof (mp_digit), MP_PREC);
023     if (a->dp == NULL) \{
024       return MP_MEM;
025     \}
026   
027     /* set the used to zero, allocated digit to the default precision
028      * and sign to positive */
029     a->used  = 0;
030     a->alloc = MP_PREC;
031     a->sign  = MP_ZPOS;
032   
033     return MP_OKAY;
034   \}
\end{alltt}
\end{small}

The \textbf{OPT\_CAST} type cast on line 22 is designed to allow C++ compilers to build the code out of
the box.  Microsoft C V5.00 is known to cause problems without the cast.  Also note that if the memory
allocation fails the other members of the mp\_int will be in an undefined state.  The code from 
line 29 to line 31 sets the default state for a mp\_int which is zero, positive and no used digits.

\subsection{Clearing an mp\_int}
When an mp\_int is no longer required the memory allocated for it can be cleared from the heap with 
the mp\_clear algorithm.

\begin{figure}[here]
\begin{center}
\begin{tabular}{l}
\hline Algorithm \textbf{mp\_clear}. \\
\textbf{Input}.   An mp\_int $a$ \\
\textbf{Output}.  The memory for $a$ is cleared. \\
\hline \\
1.  If $a$ has been previously freed then return(\textit{MP\_OKAY}). \\
2.  Free the digits of $a$ and mark $a$ as freed. \\
3.  $a.used \leftarrow 0$ \\
4.  $a.alloc \leftarrow 0$ \\
5.  Return(\textit{MP\_OKAY}). \\
\hline
\end{tabular}
\end{center}
\caption{Algorithm mp\_clear}
\end{figure}

\textbf{Algorithm mp\_clear.}
In steps one and two the memory for the digits are only free'd if they had not been previously released before.  
This is more of concern for the implementation since it is used to prevent ``double-free'' errors.  It also helps catch
code errors where mp\_ints are used after being cleared.  Simiarly steps three and four set the 
\textbf{used} and \textbf{alloc} to known values which would be easy to spot during debugging.  For example, if an mp\_int is expected
to be non-zero and its \textbf{used} member observed to be zero (\textit{due to being cleared}) then an obvious bug in the code has been
spotted.

\index{bn\_mp\_clear.c}
\vspace{+3mm}\begin{small}
\hspace{-5.1mm}{\bf File}: bn\_mp\_clear.c
\vspace{-3mm}
\begin{alltt}
016   
017   /* clear one (frees)  */
018   void
019   mp_clear (mp_int * a)
020   \{
021     if (a->dp != NULL) \{
022   
023       /* first zero the digits */
024       memset (a->dp, 0, sizeof (mp_digit) * a->used);
025   
026       /* free ram */
027       free (a->dp);
028   
029       /* reset members to make debugging easier */
030       a->dp = NULL;
031       a->alloc = a->used = 0;
032     \}
033   \}
\end{alltt}
\end{small}

The \textbf{if} statement on line 21 prevents the heap from being corrupted if a user double-frees an 
mp\_int.  For example, a trivial case of this bug would be as follows.

\begin{verbatim}
mp_int a;
mp_init(&a);
mp_clear(&a);
mp_clear(&a);
\end{verbatim}

Without that check the code would try to free the memory allocated for the digits twice which will cause most standard C
libraries to cause a fault.  Also by setting the pointer to \textbf{NULL} it helps debug code that may inadvertently 
free the mp\_int before it is truly not needed.  The allocated digits are set to zero before being freed on line 24.  
This is ideal for cryptographic situations where the mp\_int is a secret parameter.

The following snippet is an example of using both the init and clear functions.  

\begin{small}
\begin{verbatim}
#include <tommath.h>
#include <stdio.h>
#include <stdlib.h>
int main(void)
{
   mp_int num;
   int err;
   
   /* init the bignum */
   if ((err = mp_init(&num)) != MP_OKAY) {
      printf("Error: %d\n", err);
      return EXIT_FAILURE;
   }
   
   /* do work with it ... */
   
   /* clear up */
   mp_clear(&num);
   
   return EXIT_SUCCESS;
}
\end{verbatim}
\end{small}

\section{Other Initialization Routines}

It is often helpful to have specialized initialization algorithms to simplify the design of other algorithms.  For example, an 
initialization followed by a copy is a common operation when temporary copies of integers are required.  It is quite
beneficial to have a series of simple helper functions available.

\subsection{Initializing Variable Sized mp\_int Structures}
Occasionally the number of digits required will be known in advance of an initialization.  In these
cases the mp\_init\_size algorithm can be of use.  The purpose of this algorithm is similar to mp\_init except that 
it will allocate \textit{at least} a specified number of digits.  This is ideal to prevent re-allocations when the 
input size is known.

\newpage\begin{figure}[here]
\begin{center}
\begin{tabular}{l}
\hline Algorithm \textbf{mp\_init\_size}. \\
\textbf{Input}.   An mp\_int $a$ and the requested number of digits $b$\\
\textbf{Output}.  $a$ is initialized to hold at least $b$ digits. \\
\hline \\
1.  $u \leftarrow b\mbox{ (mod }MP\_PREC\mbox{)}$ \\
2.  $v \leftarrow b + 2 \cdot MP\_PREC - u$ \\
3.  Allocate $v$ digits. \\
4.  If the allocation failed then return(\textit{MP\_MEM}). \\
5.  for $n$ from $0$ to $v - 1$ do \\
\hspace{3mm}5.1  $a_n \leftarrow 0$ \\
6.  $a.sign \leftarrow MP\_ZPOS$\\
7.  $a.used \leftarrow 0$\\
8.  $a.alloc \leftarrow v$\\
9.  Return(\textit{MP\_OKAY})\\
\hline
\end{tabular}
\end{center}
\caption{Algorithm mp\_init\_size}
\end{figure}

\textbf{Algorithm mp\_init\_size.}
The value of $v$ is calculated to be at least the requested amount of digits $b$ plus additional padding.  The padding is calculated
to be at least \textbf{MP\_PREC} digits plus enough digits to make the digit count a multiple of \textbf{MP\_PREC}.  This padding is used to 
prevent trivial allocations from becomming a bottleneck in the rest of the algorithms that depend on this.

\index{bn\_mp\_init\_size.c}
\vspace{+3mm}\begin{small}
\hspace{-5.1mm}{\bf File}: bn\_mp\_init\_size.c
\vspace{-3mm}
\begin{alltt}
016   
017   /* init a mp_init and grow it to a given size */
018   int
019   mp_init_size (mp_int * a, int size)
020   \{
021   
022     /* pad size so there are always extra digits */
023     size += (MP_PREC * 2) - (size & (MP_PREC - 1));    
024     
025     /* alloc mem */
026     a->dp = OPT_CAST calloc (sizeof (mp_digit), size);
027     if (a->dp == NULL) \{
028       return MP_MEM;
029     \}
030     a->used = 0;
031     a->alloc = size;
032     a->sign = MP_ZPOS;
033   
034     return MP_OKAY;
035   \}
\end{alltt}
\end{small}

Line 23 will ensure that the number of digits actually allocated is padded up to the next multiple of 
\textbf{MP\_PREC} plus an additional \textbf{MP\_PREC}.  This ensures that the number of allocated digit is 
always greater than the amount requested.  As a result it prevents many trivial memory allocations.  The value of 
\textbf{MP\_PREC} is defined in ``tommath.h'' and must be a power of two.

\subsection{Creating a Clone}
Another common sequence of operations is to make a local temporary copy of an argument.  To initialize then copy a mp\_int will be known as 
creating a clone.  This is useful within functions that need to modify an integer argument but do not wish to actually modify the original copy.  
The mp\_init\_copy algorithm will perform this very task.

\begin{figure}[here]
\begin{center}
\begin{tabular}{l}
\hline Algorithm \textbf{mp\_init\_copy}. \\
\textbf{Input}.   An mp\_int $a$ and $b$\\
\textbf{Output}.  $a$ is initialized to be a copy of $b$. \\
\hline \\
1.  Init $a$.  (\textit{hint: use mp\_init}) \\
2.  If the init of $a$ was unsuccessful return(\textit{MP\_MEM}) \\
3.  Copy $b$ to $a$.  (\textit{hint: use mp\_copy}) \\
4.  Return the status of the copy operation. \\
\hline
\end{tabular}
\end{center}
\caption{Algorithm mp\_init\_copy}
\end{figure}

\textbf{Algorithm mp\_init\_copy.}
This algorithm will initialize a mp\_int variable and copy another previously initialized mp\_int variable into it.  The algorithm will
detect when the initialization fails and returns the error to the calling algorithm.  As such this algorithm will perform two operations
in one step.  

\index{bn\_mp\_init\_copy.c}
\vspace{+3mm}\begin{small}
\hspace{-5.1mm}{\bf File}: bn\_mp\_init\_copy.c
\vspace{-3mm}
\begin{alltt}
016   
017   /* creates "a" then copies b into it */
018   int
019   mp_init_copy (mp_int * a, mp_int * b)
020   \{
021     int     res;
022   
023     if ((res = mp_init (a)) != MP_OKAY) \{
024       return res;
025     \}
026     return mp_copy (b, a);
027   \}
\end{alltt}
\end{small}

This will initialize \textbf{a} and make it a verbatim copy of the contents of \textbf{b}.  Note that 
\textbf{a} will have its own memory allocated which means that \textbf{b} may be cleared after the call
and \textbf{a} will be left intact.  

\subsection{Multiple Integer Initializations}
Occasionally a function will require a series of mp\_int data types to be made available.  The mp\_init\_multi algorithm
is provided to simplify such cases.  The purpose of this algorithm is to initialize a variable length array of mp\_int 
structures at once.  As a result algorithms that require multiple integers only has to use 
one algorithm to initialize all the mp\_int variables.

\begin{figure}[here]
\begin{center}
\begin{tabular}{l}
\hline Algorithm \textbf{mp\_init\_multi}. \\
\textbf{Input}.   Variable length array of mp\_int variables of length $k$. \\
\textbf{Output}.  The array is initialized such that each each mp\_int is ready to use. \\
\hline \\
1.  for $n$ from 0 to $k - 1$ do \\
\hspace{+3mm}1.1.  Initialize the $n$'th mp\_int (\textit{hint: use mp\_init}) \\
\hspace{+3mm}1.2.  If initialization failed then do \\
\hspace{+6mm}1.2.1.  for $j$ from $0$ to $n$ do \\
\hspace{+9mm}1.2.1.1.  Free the $j$'th mp\_int (\textit{hint: use mp\_clear}) \\
\hspace{+6mm}1.2.2.   Return(\textit{MP\_MEM}) \\
2.  Return(\textit{MP\_OKAY}) \\
\hline
\end{tabular}
\end{center}
\caption{Algorithm mp\_init\_multi}
\end{figure}

\textbf{Algorithm mp\_init\_multi.}
The algorithm will initialize the array of mp\_int variables one at a time.  As soon as an runtime error is detected (\textit{step 1.2}) all of
the previously initialized variables are cleared.  The goal is an ``all or nothing'' initialization which allows for quick recovery from runtime 
errors.

\subsection{Multiple Integer Clearing}
Similarly to clear a variable length list of mp\_int structures the mp\_clear\_multi algorithm will be used.

\index{bn\_mp\_multi.c}
\vspace{+3mm}\begin{small}
\hspace{-5.1mm}{\bf File}: bn\_mp\_multi.c
\vspace{-3mm}
\begin{alltt}
016   #include <stdarg.h>
017   
018   int mp_init_multi(mp_int *mp, ...) 
019   \{
020       mp_err res = MP_OKAY;      /* Assume ok until proven otherwise */
021       int n = 0;                 /* Number of ok inits */
022       mp_int* cur_arg = mp;
023       va_list args;
024   
025       va_start(args, mp);        /* init args to next argument from caller */
026       while (cur_arg != NULL) \{
027           if (mp_init(cur_arg) != MP_OKAY) \{
028               /* Oops - error! Back-track and mp_clear what we already
029                  succeeded in init-ing, then return error.
030               */
031               va_list clean_args;
032               
033               /* end the current list */
034               va_end(args);
035               
036               /* now start cleaning up */            
037               cur_arg = mp;
038               va_start(clean_args, mp);
039               while (n--) \{
040                   mp_clear(cur_arg);
041                   cur_arg = va_arg(clean_args, mp_int*);
042               \}
043               va_end(clean_args);
044               res = MP_MEM;
045               break;
046           \}
047           n++;
048           cur_arg = va_arg(args, mp_int*);
049       \}
050       va_end(args);
051       return res;                /* Assumed ok, if error flagged above. */
052   \}
053   
054   void mp_clear_multi(mp_int *mp, ...) 
055   \{
056       mp_int* next_mp = mp;
057       va_list args;
058       va_start(args, mp);
059       while (next_mp != NULL) \{
060           mp_clear(next_mp);
061           next_mp = va_arg(args, mp_int*);
062       \}
063       va_end(args);
064   \}
\end{alltt}
\end{small}

Consider the following snippet which demonstrates how to use both routines.
\begin{small}
\begin{verbatim}
#include <tommath.h>
#include <stdio.h>
#include <stdlib.h>
int main(void)
{
   mp_int num1, num2, num3;
   int err;
   
   if ((err = mp_init_multi(&num1, &num2, &num3, NULL)) !- MP_OKAY) {
      printf("Error: %d\n", err);
      return EXIT_FAILURE;
   }
   
   /* at this point num1/num2/num3 are ready */
   
   /* free them */
   mp_clear_multi(&num1, &num2, &num3, NULL);
   
   return EXIT_SUCCESS;
}
\end{verbatim}
\end{small}

\section{Maintenance}
A small useful collection of mp\_int maintenance functions will also prove useful.  

\subsection{Augmenting Integer Precision}
When storing a value in an mp\_int sufficient digits must be available to accomodate the entire value without
loss of precision.  Quite often the size of the array given by the \textbf{alloc} member is large enough to simply
increase the \textbf{used} digit count.  However, when the size of the array is too small it must be re-sized 
appropriately to accomodate the result.  The mp\_grow algorithm will provide this functionality.

\begin{figure}[here]
\begin{center}
\begin{tabular}{l}
\hline Algorithm \textbf{mp\_grow}. \\
\textbf{Input}.   An mp\_int $a$ and an integer $b$. \\
\textbf{Output}.  $a$ is expanded to accomodate $b$ digits. \\
\hline \\
1.  if $a.alloc \ge b$ then return(\textit{MP\_OKAY}) \\
2.  $u \leftarrow b\mbox{ (mod }MP\_PREC\mbox{)}$ \\
3.  $v \leftarrow b + 2 \cdot MP\_PREC - u$ \\
4.  Re-Allocate the array of digits $a$ to size $v$ \\
5.  If the allocation failed then return(\textit{MP\_MEM}). \\
6.  for n from a.alloc to $v - 1$ do  \\
\hspace{+3mm}6.1  $a_n \leftarrow 0$ \\
7.  $a.alloc \leftarrow v$ \\
8.  Return(\textit{MP\_OKAY}) \\
\hline
\end{tabular}
\end{center}
\caption{Algorithm mp\_grow}
\end{figure}

\textbf{Algorithm mp\_grow.}
Step one will prevent a re-allocation from being performed if it was not required.  This is useful to prevent mp\_ints
from growing excessively in code that erroneously calls mp\_grow.  Similar to mp\_init\_size the requested digit count
is padded to provide more digits than requested.  

In step four it is assumed that the reallocation leaves the lower $a.alloc$ digits intact.  Much akin to how the 
\textit{realloc} function from the standard C library works.  Since the newly allocated digits are assumed to contain
undefined values they are also initially zeroed.

\index{bn\_mp\_grow.c}
\vspace{+3mm}\begin{small}
\hspace{-5.1mm}{\bf File}: bn\_mp\_grow.c
\vspace{-3mm}
\begin{alltt}
016   
017   /* grow as required */
018   int
019   mp_grow (mp_int * a, int size)
020   \{
021     int     i;
022   
023     /* if the alloc size is smaller alloc more ram */
024     if (a->alloc < size) \{
025       /* ensure there are always at least MP_PREC digits extra on top */
026       size += (MP_PREC * 2) - (size & (MP_PREC - 1));     
027   
028       a->dp = OPT_CAST realloc (a->dp, sizeof (mp_digit) * size);
029       if (a->dp == NULL) \{
030         return MP_MEM;
031       \}
032   
033       /* zero excess digits */
034       i        = a->alloc;
035       a->alloc = size;
036       for (; i < a->alloc; i++) \{
037         a->dp[i] = 0;
038       \}
039     \}
040     return MP_OKAY;
041   \}
\end{alltt}
\end{small}

The first step is to see if we actually need to perform a re-allocation at all.  This is tested for on line 
24.  Similar to mp\_init\_size the same code on line 26 was used to resize the 
digits requested.  A simple for loop from line 34 to line 38 will zero all digits that were above the 
old \textbf{alloc} limit to make sure the integer is in a known state.

\subsection{Clamping Excess Digits}
When a function anticipates a result will be $n$ digits it is simpler to assume this is true within the body of 
the function.  For example, a multiplication of a $i$ digit number by a $j$ digit produces a result of at most 
$i + j + 1$ digits.  It is entirely possible that the result is $i + j$ though, with no final carry into the last 
position.  However, suppose the destination had to be first expanded (\textit{via mp\_grow}) to accomodate $i + j$
digits than further expanded to accomodate the final carry.  That would be a considerable waste of time since heap
operations are relatively slow.

The ideal solution is to always assume the result is $i + j + 1$ and fix up the \textbf{used} count after the function
terminates.  This way a single heap operation (\textit{at most}) is required.  However, if the result was not checked
there would be an excess high order zero digit.  

For example, suppose the product of two integers was $x_n = (0x_{n-1}x_{n-2}...x_0)_{\beta}$.  The leading zero digit 
will not contribute to the precision of the result.  In fact, through subsequent operations more leading zero digits would
accumulate to the point the size of the integer would be prohibitive.  As a result even though the precision is very 
low the representation is excessively large.  

The mp\_clamp algorithm is designed to solve this very problem.  It will trim leading zeros by decrementing the 
\textbf{used} count until a non-zero leading digit is found.  Also in this system, zero is considered to be a positive 
number which means that if the \textbf{used} count is decremented to zero the sign must be set to \textbf{MP\_ZPOS}.

\begin{figure}[here]
\begin{center}
\begin{tabular}{l}
\hline Algorithm \textbf{mp\_clamp}. \\
\textbf{Input}.   An mp\_int $a$ \\
\textbf{Output}.  Any excess leading zero digits of $a$ are removed \\
\hline \\
1.  while $a.used > 0$ and $a_{a.used - 1} = 0$ do \\
\hspace{+3mm}1.1  $a.used \leftarrow a.used - 1$ \\
2.  if $a.used = 0$ then do \\
\hspace{+3mm}2.1  $a.sign \leftarrow MP\_ZPOS$ \\
\hline \\
\end{tabular}
\end{center}
\caption{Algorithm mp\_clamp}
\end{figure}

\textbf{Algorithm mp\_clamp.}
As can be expected this algorithm is very simple.  The loop on step one is indended to be iterate only once or twice at
the most.  For example, for cases where there is not a carry to fill the last position.  Step two fixes the sign for 
when all of the digits are zero to ensure that the mp\_int is valid at all times.

\index{bn\_mp\_clamp.c}
\vspace{+3mm}\begin{small}
\hspace{-5.1mm}{\bf File}: bn\_mp\_clamp.c
\vspace{-3mm}
\begin{alltt}
016   
017   /* trim unused digits 
018    *
019    * This is used to ensure that leading zero digits are
020    * trimed and the leading "used" digit will be non-zero
021    * Typically very fast.  Also fixes the sign if there
022    * are no more leading digits
023    */
024   void
025   mp_clamp (mp_int * a)
026   \{
027     while (a->used > 0 && a->dp[a->used - 1] == 0) \{
028       --(a->used);
029     \}
030     if (a->used == 0) \{
031       a->sign = MP_ZPOS;
032     \}
033   \}
\end{alltt}
\end{small}

Note on line 27 how to test for the \textbf{used} count is made on the left of the \&\& operator.  In the C programming
language the terms to \&\& are evaluated left to right with a boolean short-circuit if any condition fails.  This is 
important since if the \textbf{used} is zero the test on the right would fetch below the array.  That is obviously 
undesirable.  The parenthesis on line 28 is used to make sure the \textbf{used} count is decremented and not
the pointer ``a''.  

\section*{Exercises}
\begin{tabular}{cl}
$\left [ 1 \right ]$ & Discuss the relevance of the \textbf{used} member of the mp\_int structure. \\
                     & \\
$\left [ 1 \right ]$ & Discuss the consequences of not using padding when performing allocations.  \\
                     & \\
$\left [ 2 \right ]$ & Estimate an ideal value for \textbf{MP\_PREC} when performing 1024-bit RSA \\
                     & encryption when $\beta = 2^{28}$.  \\
                     & \\
$\left [ 1 \right ]$ & Discuss the relevance of the algorithm mp\_clamp.  What does it prevent? \\
                     & \\
$\left [ 1 \right ]$ & Give an example of when the algorithm  mp\_init\_copy might be useful. \\
                     & \\
\end{tabular}


\chapter{Basic Operations}
\section{Copying an Integer}
After the various house-keeping routines are in place, simpl algorithms can be designed to take advantage of them.  Being able
to make a verbatim copy of an integer is a very useful function to have.  To copy an integer the mp\_copy algorithm will be used.

\newpage\begin{figure}[here]
\begin{center}
\begin{tabular}{l}
\hline Algorithm \textbf{mp\_copy}. \\
\textbf{Input}.  An mp\_int $a$ and $b$. \\
\textbf{Output}.  Store a copy of $a$ in $b$. \\
\hline \\
1.  Check if $a$ and $b$ point to the same location in memory. \\
2.  If true then return(\textit{MP\_OKAY}). \\
3.  If $b.alloc < a.used$ then grow $b$ to $a.used$ digits.  (\textit{hint: use mp\_grow}) \\
4.  If failed to grow then return(\textit{MP\_MEM}). \\
5.  for $n$ from 0 to $a.used - 1$ do \\
\hspace{3mm}5.1  $b_{n} \leftarrow a_{n}$ \\
6.  if $a.used < b.used - 1$ then \\ 
\hspace{3mm}6.1.  for $n$ from $a.used$ to $b.used - 1$ do \\
\hspace{6mm}6.1.1  $b_{n} \leftarrow 0$ \\
7.  $b.used \leftarrow a.used$ \\
8.  $b.sign \leftarrow a.sign$ \\
9.  return(\textit{MP\_OKAY}) \\
\hline
\end{tabular}
\end{center}
\caption{Algorithm mp\_copy}
\end{figure}

\textbf{Algorithm mp\_copy.}
Step 1 and 2 make sure that the two mp\_ints are unique.  This allows the user to call the copy function with
potentially the same input and not waste time.  Step 3 and 4 ensure that the destination is large enough to
hold a copy of the input $a$.  Note that the \textbf{used} member of $b$ may be smaller than the \textbf{used}
member of $a$ but a memory re-allocation is only required if the \textbf{alloc} member of $b$ is smaller.  This
prevents trivial memory reallocations.

Step 5 copies the digits from $a$ to $b$ while step 6 ensures that if initially $\vert b \vert > \vert a \vert$,
the leading digits of $b$ will be zeroed.  Finally steps 7 and 8 copies the \textbf{used} and \textbf{sign} members over 
which completes the copy operation.

\index{bn\_mp\_copy.c}
\vspace{+3mm}\begin{small}
\hspace{-5.1mm}{\bf File}: bn\_mp\_copy.c
\vspace{-3mm}
\begin{alltt}
016   
017   /* copy, b = a */
018   int
019   mp_copy (mp_int * a, mp_int * b)
020   \{
021     int     res, n;
022   
023     /* if dst == src do nothing */
024     if (a == b || a->dp == b->dp) \{
025       return MP_OKAY;
026     \}
027   
028     /* grow dest */
029     if ((res = mp_grow (b, a->used)) != MP_OKAY) \{
030       return res;
031     \}
032   
033     /* zero b and copy the parameters over */
034     \{
035       register mp_digit *tmpa, *tmpb;
036   
037       /* pointer aliases */
038       tmpa = a->dp;
039       tmpb = b->dp;
040   
041       /* copy all the digits */
042       for (n = 0; n < a->used; n++) \{
043         *tmpb++ = *tmpa++;
044       \}
045   
046       /* clear high digits */
047       for (; n < b->used; n++) \{
048         *tmpb++ = 0;
049       \}
050     \}
051     b->used = a->used;
052     b->sign = a->sign;
053     return MP_OKAY;
054   \}
\end{alltt}
\end{small}

Source lines 23-31 do the initial house keeping.  That is to see if the input is unique and if so to 
make sure there is enough room.  If not enough space is available it returns the error and leaves the destination variable
intact.

The inner loop of the copy operation is contained between lines 34 and 50.  Many LibTomMath routines are designed with this source code style
in mind, making aliases to shorten lengthy pointers (\textit{see line 38 and 39}) for rapid to use.  Also the
use of nested braces creates a simple way to denote various portions of code that reside on various work levels.  Here, the copy loop is at the 
$O(n)$ level.  

\section{Zeroing an Integer}
Reseting an mp\_int to the default state is a common step in many algorithms.  The mp\_zero algorithm will be the algorithm used to
perform this task.

\begin{figure}[here]
\begin{center}
\begin{tabular}{l}
\hline Algorithm \textbf{mp\_zero}. \\
\textbf{Input}.   An mp\_int $a$ \\
\textbf{Output}.  Zero the contents of $a$ \\
\hline \\
1.  $a.used \leftarrow 0$ \\
2.  $a.sign \leftarrow$ MP\_ZPOS \\
3.  for $n$ from 0 to $a.alloc - 1$ do \\
\hspace{3mm}3.1  $a_n \leftarrow 0$ \\
\hline
\end{tabular}
\end{center}
\caption{Algorithm mp\_zero}
\end{figure}

\textbf{Algorithm mp\_zero.}
This algorithm simply resets a mp\_int to the default state.  

\index{bn\_mp\_zero.c}
\vspace{+3mm}\begin{small}
\hspace{-5.1mm}{\bf File}: bn\_mp\_zero.c
\vspace{-3mm}
\begin{alltt}
016   
017   /* set to zero */
018   void
019   mp_zero (mp_int * a)
020   \{
021     a->sign = MP_ZPOS;
022     a->used = 0;
023     memset (a->dp, 0, sizeof (mp_digit) * a->alloc);
024   \}
\end{alltt}
\end{small}

After the function is completed, all of the digits are zeroed, the \textbf{used} count is zeroed and the 
\textbf{sign} variable is set to \textbf{MP\_ZPOS}.

\section{Sign Manipulation}
\subsection{Absolute Value}
With the mp\_int representation of an integer, calculating the absolute value is trivial.  The mp\_abs algorithm will compute
the absolute value of an mp\_int.

\begin{figure}[here]
\begin{center}
\begin{tabular}{l}
\hline Algorithm \textbf{mp\_abs}. \\
\textbf{Input}.   An mp\_int $a$ \\
\textbf{Output}.  Computes $b = \vert a \vert$ \\
\hline \\
1.  Copy $a$ to $b$.  (\textit{hint: use mp\_copy}) \\
2.  If the copy failed return(\textit{MP\_MEM}). \\
3.  $b.sign \leftarrow MP\_ZPOS$ \\
4.  Return(\textit{MP\_OKAY}) \\
\hline
\end{tabular}
\end{center}
\caption{Algorithm mp\_abs}
\end{figure}

\textbf{Algorithm mp\_abs.}
This algorithm computes the absolute of an mp\_int input.  As can be expected the algorithm is very trivial.

\index{bn\_mp\_abs.c}
\vspace{+3mm}\begin{small}
\hspace{-5.1mm}{\bf File}: bn\_mp\_abs.c
\vspace{-3mm}
\begin{alltt}
016   
017   /* b = |a| 
018    *
019    * Simple function copies the input and fixes the sign to positive
020    */
021   int
022   mp_abs (mp_int * a, mp_int * b)
023   \{
024     int     res;
025     if ((res = mp_copy (a, b)) != MP_OKAY) \{
026       return res;
027     \}
028     b->sign = MP_ZPOS;
029     return MP_OKAY;
030   \}
\end{alltt}
\end{small}

\subsection{Integer Negation}
With the mp\_int representation of an integer, calculating the negation is also trivial.  The mp\_neg algorithm will compute
the negative of an mp\_int input.

\newpage\begin{figure}[here]
\begin{center}
\begin{tabular}{l}
\hline Algorithm \textbf{mp\_neg}. \\
\textbf{Input}.   An mp\_int $a$ \\
\textbf{Output}.  Computes $b = -a$ \\
\hline \\
1.  Copy $a$ to $b$.  (\textit{hint: use mp\_copy}) \\
2.  If the copy failed return(\textit{MP\_MEM}). \\
3.  If $a.sign = MP\_ZPOS$ then do \\
\hspace{3mm}3.1  $b.sign = MP\_NEG$. \\
4.  else do \\
\hspace{3mm}4.1  $b.sign = MP\_ZPOS$. \\
5.  Return(\textit{MP\_OKAY}) \\
\hline
\end{tabular}
\end{center}
\caption{Algorithm mp\_neg}
\end{figure}

\textbf{Algorithm mp\_neg.}
This algorithm computes the negation of an input.  

\index{bn\_mp\_neg.c}
\vspace{+3mm}\begin{small}
\hspace{-5.1mm}{\bf File}: bn\_mp\_neg.c
\vspace{-3mm}
\begin{alltt}
016   
017   /* b = -a */
018   int
019   mp_neg (mp_int * a, mp_int * b)
020   \{
021     int     res;
022     if ((res = mp_copy (a, b)) != MP_OKAY) \{
023       return res;
024     \}
025     b->sign = (a->sign == MP_ZPOS) ? MP_NEG : MP_ZPOS;
026     return MP_OKAY;
027   \}
\end{alltt}
\end{small}

\section{Small Constants}
\subsection{Setting Small Constants}
Often a mp\_int must be set to a relatively small value such as $1$ or $2$.  For these cases the mp\_set algorithm is useful.

\newpage\begin{figure}
\begin{center}
\begin{tabular}{l}
\hline Algorithm \textbf{mp\_set}. \\
\textbf{Input}.   An mp\_int $a$ and a digit $b$ \\
\textbf{Output}.  Make $a$ equivalent to $b$ \\
\hline \\
1.  Zero $a$ (\textit{hint: use mp\_zero}). \\
2.  $a_0 \leftarrow b \mbox{ (mod }\beta\mbox{)}$ \\
3.  $a.used \leftarrow  \left \lbrace \begin{array}{ll}
                              1 &  \mbox{if }a_0 > 0 \\
                              0 &  \mbox{if }a_0 = 0 
                              \end{array} \right .$ \\
\hline                              
\end{tabular}
\end{center}
\caption{Algorithm mp\_set}
\end{figure}

\textbf{Algorithm mp\_set.}
This algorithm sets a mp\_int to a small single digit value.  Step number 1 ensures that the integer is reset to the default state.  The
single digit is set (\textit{modulo $\beta$}) and the \textbf{used} count is adjusted accordingly.

\index{bn\_mp\_set.c}
\vspace{+3mm}\begin{small}
\hspace{-5.1mm}{\bf File}: bn\_mp\_set.c
\vspace{-3mm}
\begin{alltt}
016   
017   /* set to a digit */
018   void
019   mp_set (mp_int * a, mp_digit b)
020   \{
021     mp_zero (a);
022     a->dp[0] = b & MP_MASK;
023     a->used = (a->dp[0] != 0) ? 1 : 0;
024   \}
\end{alltt}
\end{small}

Line 21 calls mp\_zero() to clear the mp\_int and reset the sign.  Line 22 actually copies digit 
into the least significant location.  Note the usage of a new constant \textbf{MP\_MASK}.  This constant is used to quickly
reduce an integer modulo $\beta$.  Since $\beta = 2^k$ it suffices to perform a binary AND with $MP\_MASK = 2^k - 1$ to perform
the reduction.  Finally line 23 will set the \textbf{used} member with respect to the digit actually set. This function 
will always make the integer positive.

One important limitation of this function is that it will only set one digit.  The size of a digit is not fixed, meaning source that uses 
this function should take that into account.  The define \textbf{DIGIT\_BIT} in ``tommath.h'' 
defines how many bits per digit are available.  Generally at least seven bits are guaranteed to be available per 
digit.  This means that trivially small constants can be set using this function.

\subsection{Setting Large Constants}
To overcome the limitations of the mp\_set algorithm the mp\_set\_int algorithm is provided.  It accepts a ``long''
data type as input and will always treat it as a 32-bit integer.

\begin{figure}[here]
\begin{center}
\begin{tabular}{l}
\hline Algorithm \textbf{mp\_set\_int}. \\
\textbf{Input}.   An mp\_int $a$ and a ``long'' integer $b$ \\
\textbf{Output}.  Make $a$ equivalent to $b$ \\
\hline \\
1.  Zero $a$ (\textit{hint: use mp\_zero}) \\
2.  for $n$ from 0 to 7 do \\
\hspace{3mm}2.1  $a \leftarrow a \cdot 16$ (\textit{hint: use mp\_mul2d}) \\
\hspace{3mm}2.2  $u \leftarrow \lfloor b / 2^{4(7 - n)} \rfloor \mbox{ (mod }16\mbox{)}$\\
\hspace{3mm}2.3  $a_0 \leftarrow a_0 + u$ \\
\hspace{3mm}2.4  $a.used \leftarrow a.used + \lfloor 32 / lg(\beta) \rfloor + 1$ \\
3.  Clamp excess used digits (\textit{hint: use mp\_clamp}) \\
\hline
\end{tabular}
\end{center}
\caption{Algorithm mp\_set\_int}
\end{figure}

\textbf{Algorithm mp\_set\_int.}
The algorithm performs eight iterations of a simple loop where in each iteration four bits from the source are added to the 
mp\_int.  Step 2.1 will multiply the current result by sixteen making room for four more bits.  In step 2.2 the
next four bits from the source are extracted.  The four bits are added to the mp\_int and the \textbf{used} digit count is 
incremented.  The \textbf{used} digit counter is incremented since if any of the leading digits were zero the mp\_int would have
zero digits used and the newly added four bits would be ignored.

Excess zero digits are trimmed in steps 2.1 and 3 by using higher level algorithms mp\_mul2d and mp\_clamp.

\index{bn\_mp\_set\_int.c}
\vspace{+3mm}\begin{small}
\hspace{-5.1mm}{\bf File}: bn\_mp\_set\_int.c
\vspace{-3mm}
\begin{alltt}
016   
017   /* set a 32-bit const */
018   int
019   mp_set_int (mp_int * a, unsigned int b)
020   \{
021     int     x, res;
022   
023     mp_zero (a);
024     /* set four bits at a time */
025     for (x = 0; x < 8; x++) \{
026       /* shift the number up four bits */
027       if ((res = mp_mul_2d (a, 4, a)) != MP_OKAY) \{
028         return res;
029       \}
030   
031       /* OR in the top four bits of the source */
032       a->dp[0] |= (b >> 28) & 15;
033   
034       /* shift the source up to the next four bits */
035       b <<= 4;
036   
037       /* ensure that digits are not clamped off */
038       a->used += 32 / DIGIT_BIT + 2;
039     \}
040     mp_clamp (a);
041     return MP_OKAY;
042   \}
\end{alltt}
\end{small}

This function sets four bits of the number at a time to handle all practical \textbf{DIGIT\_BIT} sizes.  The weird
addition on line 38 ensures that the newly added in bits are added to the number of digits.  While it may not 
seem obvious as to why the digit counter does not grow exceedingly large it is because of the shift on line 27 
as well as the  call to mp\_clamp() on line 40.  Both functions will clamp excess leading digits which keeps 
the number of used digits low.

\section{Comparisons}
\subsection{Unsigned Comparisions}
Comparing a multiple precision integer is performed with the exact same algorithm used to compare two decimal numbers.  For example,
to compare $1,234$ to $1,264$ the digits are extracted by their positions.  That is we compare $1 \cdot 10^3 + 2 \cdot 10^2 + 3 \cdot 10^1 + 4 \cdot 10^0$
to $1 \cdot 10^3 + 2 \cdot 10^2 + 6 \cdot 10^1 + 4 \cdot 10^0$ by comparing single digits at a time starting with the highest magnitude 
positions.  If any leading digit of one integer is greater than a digit in the same position of another integer then obviously it must be greater.  

The first comparision routine that will be developed is the unsigned magnitude compare which will perform a comparison based on the digits of two
mp\_int variables alone.  It will ignore the sign of the two inputs.  Such a function is useful when an absolute comparison is required or if the 
signs are known to agree in advance.

To facilitate working with the results of the comparison functions three constants are required.  

\begin{figure}[here]
\begin{center}
\begin{tabular}{|r|l|}
\hline \textbf{Constant} & \textbf{Meaning} \\
\hline \textbf{MP\_GT} & Greater Than \\
\hline \textbf{MP\_EQ} & Equal To \\
\hline \textbf{MP\_LT} & Less Than \\
\hline
\end{tabular}
\end{center}
\caption{Comparison Return Codes}
\end{figure}

\begin{figure}[here]
\begin{center}
\begin{tabular}{l}
\hline Algorithm \textbf{mp\_cmp\_mag}. \\
\textbf{Input}.   Two mp\_ints $a$ and $b$.  \\
\textbf{Output}.  Unsigned comparison results ($a$ to the left of $b$). \\
\hline \\
1.  If $a.used > b.used$ then return(\textit{MP\_GT}) \\
2.  If $a.used < b.used$ then return(\textit{MP\_LT}) \\
3.  for n from $a.used - 1$ to 0 do \\
\hspace{+3mm}3.1  if $a_n > b_n$ then return(\textit{MP\_GT}) \\
\hspace{+3mm}3.2  if $a_n < b_n$ then return(\textit{MP\_LT}) \\
4.  Return(\textit{MP\_EQ}) \\
\hline
\end{tabular}
\end{center}
\caption{Algorithm mp\_cmp\_mag}
\end{figure}

\textbf{Algorithm mp\_cmp\_mag.}
By saying ``$a$ to the left of $b$'' it is meant that the comparison is with respect to $a$, that is if $a$ is greater than $b$ it will return
\textbf{MP\_GT} and similar with respect to when $a = b$ and $a < b$.  The first two steps compare the number of digits used in both $a$ and $b$.  
Obviously if the digit counts differ there would be an imaginary zero digit in the smaller number where the leading digit of the larger number is.  
If both have the same number of digits than the actual digits themselves must be compared starting at the leading digit.  

By step three both inputs must have the same number of digits so its safe to start from either $a.used - 1$ or $b.used - 1$ and count down to
the zero'th digit.  If after all of the digits have been compared and no difference found the algorithm simply returns \textbf{MP\_EQ}.

\index{bn\_mp\_cmp\_mag.c}
\vspace{+3mm}\begin{small}
\hspace{-5.1mm}{\bf File}: bn\_mp\_cmp\_mag.c
\vspace{-3mm}
\begin{alltt}
016   
017   /* compare maginitude of two ints (unsigned) */
018   int
019   mp_cmp_mag (mp_int * a, mp_int * b)
020   \{
021     int     n;
022   
023     /* compare based on # of non-zero digits */
024     if (a->used > b->used) \{
025       return MP_GT;
026     \} 
027     
028     if (a->used < b->used) \{
029       return MP_LT;
030     \}
031   
032     /* compare based on digits  */
033     for (n = a->used - 1; n >= 0; n--) \{
034       if (a->dp[n] > b->dp[n]) \{
035         return MP_GT;
036       \} 
037       
038       if (a->dp[n] < b->dp[n]) \{
039         return MP_LT;
040       \}
041     \}
042     return MP_EQ;
043   \}
\end{alltt}
\end{small}

The two if statements on lines 24 and 28 compare the number of digits in the two inputs.  These two are performed before all of the digits
are compared since it is a very cheap test to perform and can potentially save considerable time.  The implementation given is also not valid 
without those two statements.  $b.alloc$ may be smaller than $a.used$, meaning that undefined values will be read from $b$ passed the end of the 
array of digits.

\subsection{Signed Comparisons}
Comparing with sign considerations is also fairly critical in several routines (\textit{division for example}).  Based on an unsigned magnitude 
comparison a trivial signed comparison algorithm can be written.

\newpage\begin{figure}[here]
\begin{center}
\begin{tabular}{l}
\hline Algorithm \textbf{mp\_cmp}. \\
\textbf{Input}.   Two mp\_ints $a$ and $b$ \\
\textbf{Output}.  Signed Comparison Results ($a$ to the left of $b$) \\
\hline \\
1.  if $a.sign = MP\_NEG$ and $b.sign = MP\_ZPOS$ then return(\textit{MP\_LT}) \\
2.  if $a.sign = MP\_ZPOS$ and $b.sign = MP\_NEG$ then return(\textit{MP\_GT}) \\
3.  if $a.sign = MP\_NEG$ then \\
\hspace{+3mm}3.1  Return the unsigned comparison of $b$ and $a$ (\textit{hint: use mp\_cmp\_mag}) \\
4   Otherwise \\
\hspace{+3mm}4.1  Return the unsigned comparison of $a$ and $b$ \\
\hline
\end{tabular}
\end{center}
\caption{Algorithm mp\_cmp}
\end{figure}

\textbf{Algorithm mp\_cmp.}
The first two steps compare the signs of the two inputs.  If the signs do not agree then it can return right away with the appropriate 
comparison code.  When the signs are equal the digits of the inputs must be compared to determine the correct result.  In step 
three the unsigned comparision flips the order of the arguments since they are both negative.  For instance, if $-a > -b$ then 
$\vert a \vert < \vert b \vert$.  Step number four will compare the two when they are both positive.

\index{bn\_mp\_cmp.c}
\vspace{+3mm}\begin{small}
\hspace{-5.1mm}{\bf File}: bn\_mp\_cmp.c
\vspace{-3mm}
\begin{alltt}
016   
017   /* compare two ints (signed)*/
018   int
019   mp_cmp (mp_int * a, mp_int * b)
020   \{
021     /* compare based on sign */
022     if (a->sign == MP_NEG && b->sign == MP_ZPOS) \{
023       return MP_LT;
024     \} 
025     
026     if (a->sign == MP_ZPOS && b->sign == MP_NEG) \{
027       return MP_GT;
028     \}
029     
030     /* compare digits */
031     if (a->sign == MP_NEG) \{
032        /* if negative compare opposite direction */
033        return mp_cmp_mag(b, a);
034     \} else \{
035        return mp_cmp_mag(a, b);
036     \}
037   \}
\end{alltt}
\end{small}

The two if statements on lines 22 and 26 perform the initial sign comparison.  If the signs are not the equal then which ever
has the positive sign is larger.   At line 31, the inputs are compared based on magnitudes.  If the signs were both negative then 
the unsigned comparison is performed in the opposite direction (\textit{line 33}).  Otherwise, the signs are assumed to 
be both positive and a forward direction unsigned comparison is performed.

\section*{Exercises}
\begin{tabular}{cl}
$\left [ 2 \right ]$ & Modify algorithm mp\_set\_int to accept as input a variable length array of bits. \\
                     & \\
$\left [ 3 \right ]$ & Give the probability that algorithm mp\_cmp\_mag will have to compare $k$ digits  \\
                     & of two random digits (of equal magnitude) before a difference is found. \\
                     & \\
$\left [ 1 \right ]$ & Suggest a simple method to speed up the implementation of mp\_cmp\_mag based  \\
                     & on the observations made in the previous problem. \\
                     &
\end{tabular}

\chapter{Basic Arithmetic}
\section{Building Blocks}
At this point algorithms for initialization, de-initialization, zeroing, copying, comparing and setting small constants have been 
established.  The next logical set of algorithms to develop are the addition, subtraction and digit movement algorithms.  These 
algorithms make use of the lower level algorithms and are the cruicial building block for the multipliers.  It is very important that these 
algorithms are highly optimized.  On their own they are simple $O(n)$ algorithms but they can be called from higher level algorithms 
which easily places them at $O(n^2)$ or even $O(n^3)$ work levels.  

All nine algorithms within this chapter make use of the logical bit shift operations denoted by $<<$ and $>>$ for left and right 
logical shifts respectively.  A logical shift is analogous to sliding the decimal point of radix-10 representations.  For example, the real 
number $0.9345$ is equivalent to $93.45\%$ which is found by sliding the the decimal two places to the right (\textit{multiplying by $10^2$}).  
Mathematically a logical shift is equivalent to a division or multiplication by a power of two.  
For example, $a << k = a \cdot 2^k$ while $a >> k = \lfloor a/2^k \rfloor$.

One significant difference between a logical shift and the way decimals are shifted is that digits below the zero'th position are removed
from the number.  For example, consider $1101_2 >> 1$ using decimal notation this would produce $110.1_2$.  However, with a logical shift the 
result is $110_2$.  

\section{Addition and Subtraction}
In normal fixed precision arithmetic negative numbers are easily represented by subtraction from the modulus.  For example, with 32-bit integers
$a - b\mbox{ (mod }2^{32}\mbox{)}$ is the same as $a + (2^{32} - b) \mbox{ (mod }2^{32}\mbox{)}$  since $2^{32} \equiv 0 \mbox{ (mod }2^{32}\mbox{)}$.  
As a result subtraction can be performed with a trivial series of logical operations and an addition.

However, in multiple precision arithmetic negative numbers are not represented in the same way.  Instead a sign flag is used to keep track of the
sign of the integer.  As a result signed addition and subtraction are actually implemented as conditional usage of lower level addition or 
subtraction algorithms with the sign fixed up appropriately.

The lower level algorithms will add or subtract integers without regard to the sign flag.  That is they will add or subtract the magnitude of
the integers respectively.

\subsection{Low Level Addition}
An unsigned addition of multiple precision integers is performed with the same long-hand algorithm used to add decimal numbers.  That is to add the 
trailing digits first and propagate the resulting carry upwards.  Since this is a lower level algorithm the name will have a ``s\_'' prefix.  
Historically that convention stems from the MPI library where ``s\_'' stood for static functions that were hidden from the developer entirely.

\newpage
\begin{figure}[!here]
\begin{center}
\begin{small}
\begin{tabular}{l}
\hline Algorithm \textbf{s\_mp\_add}. \\
\textbf{Input}.   Two mp\_ints $a$ and $b$ \\
\textbf{Output}.  The unsigned addition $c = \vert a \vert + \vert b \vert$. \\
\hline \\
1.  if $a.used > b.used$ then \\
\hspace{+3mm}1.1  $min \leftarrow b.used$ \\
\hspace{+3mm}1.2  $max \leftarrow a.used$ \\
\hspace{+3mm}1.3  $x   \leftarrow a$ \\
2.  else  \\
\hspace{+3mm}2.1  $min \leftarrow a.used$ \\
\hspace{+3mm}2.2  $max \leftarrow b.used$ \\
\hspace{+3mm}2.3  $x   \leftarrow b$ \\
3.  If $c.alloc < max + 1$ then grow $c$ to hold at least $max + 1$ digits (\textit{hint: use mp\_grow}) \\
4.  If failed to grow $c$ return(\textit{MP\_MEM}) \\
5.  $oldused \leftarrow c.used$ \\
6.  $c.used \leftarrow max + 1$ \\
7.  $u \leftarrow 0$ \\
8.  for $n$ from $0$ to $min - 1$ do \\
\hspace{+3mm}8.1  $c_n \leftarrow a_n + b_n + u$ \\
\hspace{+3mm}8.2  $u \leftarrow c_n >> lg(\beta)$ \\
\hspace{+3mm}8.3  $c_n \leftarrow c_n \mbox{ (mod }\beta\mbox{)}$ \\
9.  if $min \ne max$ then do \\
\hspace{+3mm}9.1  for $n$ from $min$ to $max - 1$ do \\
\hspace{+6mm}9.1.1  $c_n \leftarrow x_n + u$ \\
\hspace{+6mm}9.1.2  $u \leftarrow c_n >> lg(\beta)$ \\
\hspace{+6mm}9.1.3  $c_n \leftarrow c_n \mbox{ (mod }\beta\mbox{)}$ \\
10.  $c_{max} \leftarrow u$ \\
11.  if $olduse > max$ then \\
\hspace{+3mm}11.1  for $n$ from $max + 1$ to $olduse - 1$ do \\
\hspace{+6mm}11.1.1  $c_n \leftarrow 0$ \\
12.  Clamp excess digits in $c$.  (\textit{hint: use mp\_clamp}) \\
13.  Return(\textit{MP\_OKAY}) \\
\hline
\end{tabular}
\end{small}
\end{center}
\caption{Algorithm s\_mp\_add}
\end{figure}

\textbf{Algorithm s\_mp\_add.}
This algorithm is loosely based on algorithm 14.7 of \cite[pp. 594]{HAC} but has been extended to allow the inputs to have different magnitudes.  
Coincidentally the description of algorithm A in \cite[pp. 266]{TAOCPV2} shares the same flaw as that from \cite{HAC}.  Even the MIX pseudo 
machine code presented  \cite[pp. 266-267]{TAOCPV2} is incapable of handling inputs which are of different magnitudes.

Steps 1 and 2 will sort the two inputs based on their \textbf{used} digit count.  This allows the inputs to have varying magnitudes which not 
only makes it more efficient than the trivial algorithm presented in the other references but more flexible.  The variable $min$ is given the lowest 
digit count while $max$ is given the highest digit count.  If both inputs have the same \textbf{used} digit count both $min$ and $max$ are 
set to the same.  The variable $x$ is an \textit{alias} for the largest input and not meant to be a copy of it.  After the inputs are sorted steps 
3 and 4 will ensure that the destination $c$ can accommodate the result.  The old \textbf{used} count from $c$ is copied to $oldused$ and the 
new count is set to $max + 1$.  

At step 7 the carry variable $u$ is set to zero and the first leg of the addition loop can begin.  The first step of the loop (\textit{8.1}) adds
digits from the two inputs together along with the carry variable $u$.  The following step extracts the carry bit by shifting the result of the
preceding step right $lg(\beta)$ positions.  The shift to extract the carry is similar to how carry extraction works with decimal addition.

Consider adding $77$ to $65$, the first addition of the first column is $7 + 5$ which produces the result $12$.  The trailing digit of the result
is $2 \equiv 12 \mbox{ (mod }10\mbox{)}$ and the carry is found by dividing (\textit{and ignoring the remainder}) $12$ by the radix or in this case $10$.  The
division and multiplication of $10$ is simply a logical shift right or left respectively of the digits.  In otherwords the carry can be extracted
by shifting one digit to the right.

Note that $lg()$ is simply the base two logarithm such that $lg(2^k) = k$.  This implies that $lg(\beta)$ is the number of bits in a radix-$\beta$ 
digit.  Therefore, a logical shift right of the single digit by $lg(\beta)$ will extract the carry.  The final step of the  loop reduces the digit 
modulo the radix $\beta$ to ensure it is in range.

After step 8 the smallest input (\textit{or both if they are the same magnitude}) has been exhausted.  Step 9 decides whether
the inputs were of equal magnitude.  If not than another loop similar to that in step 8 must be executed.  The loop at step
number 9.1 differs from the previous loop since it only adds the mp\_int $x$ along with the carry.  

Step 10 finishes the addition phase by copying the final carry to the highest location in the result $c_{max}$.  Step 11 ensures that 
leading digits that were originally present in $c$ are cleared.  Finally excess leading digits are clamped and the algorithm returns success.

\index{bn\_s\_mp\_add.c}
\vspace{+3mm}\begin{small}
\hspace{-5.1mm}{\bf File}: bn\_s\_mp\_add.c
\vspace{-3mm}
\begin{alltt}
016   
017   /* low level addition, based on HAC pp.594, Algorithm 14.7 */
018   int
019   s_mp_add (mp_int * a, mp_int * b, mp_int * c)
020   \{
021     mp_int *x;
022     int     olduse, res, min, max;
023   
024     /* find sizes, we let |a| <= |b| which means we have to sort
025      * them.  "x" will point to the input with the most digits
026      */
027     if (a->used > b->used) \{
028       min = b->used;
029       max = a->used;
030       x = a;
031     \} else \{
032       min = a->used;
033       max = b->used;
034       x = b;
035     \}
036   
037     /* init result */
038     if (c->alloc < max + 1) \{
039       if ((res = mp_grow (c, max + 1)) != MP_OKAY) \{
040         return res;
041       \}
042     \}
043   
044     /* get old used digit count and set new one */
045     olduse = c->used;
046     c->used = max + 1;
047   
048     /* set the carry to zero */
049     \{
050       register mp_digit u, *tmpa, *tmpb, *tmpc;
051       register int i;
052   
053       /* alias for digit pointers */
054   
055       /* first input */
056       tmpa = a->dp;
057   
058       /* second input */
059       tmpb = b->dp;
060   
061       /* destination */
062       tmpc = c->dp;
063   
064       /* zero the carry */
065       u = 0;
066       for (i = 0; i < min; i++) \{
067         /* Compute the sum at one digit, T[i] = A[i] + B[i] + U */
068         *tmpc = *tmpa++ + *tmpb++ + u;
069   
070         /* U = carry bit of T[i] */
071         u = *tmpc >> ((mp_digit)DIGIT_BIT);
072   
073         /* take away carry bit from T[i] */
074         *tmpc++ &= MP_MASK;
075       \}
076   
077       /* now copy higher words if any, that is in A+B 
078        * if A or B has more digits add those in 
079        */
080       if (min != max) \{
081         for (; i < max; i++) \{
082           /* T[i] = X[i] + U */
083           *tmpc = x->dp[i] + u;
084   
085           /* U = carry bit of T[i] */
086           u = *tmpc >> ((mp_digit)DIGIT_BIT);
087   
088           /* take away carry bit from T[i] */
089           *tmpc++ &= MP_MASK;
090         \}
091       \}
092   
093       /* add carry */
094       *tmpc++ = u;
095   
096       /* clear digits above oldused */
097       for (i = c->used; i < olduse; i++) \{
098         *tmpc++ = 0;
099       \}
100     \}
101   
102     mp_clamp (c);
103     return MP_OKAY;
104   \}
\end{alltt}
\end{small}

Lines 27 to 35 perform the initial sorting of the inputs and determine the $min$ and $max$ variables.  Note that $x$ is pointer to a 
mp\_int assigned to the largest input, in effect it is a local alias.  Lines 37 to 42 ensure that the destination is grown to 
accomodate the result of the addition. 

Similar to the implementation of mp\_copy this function uses the braced code and local aliases coding style.  The three aliases on 
lines 56, 59 and 62 are the for the two inputs and destination respectively.  These aliases are used to ensure the
compiler does not have to dereference $a$, $b$ or $c$ (respectively) to access the digits of the respective mp\_int.

The initial carry $u$ is cleared on line 65, note that $u$ is of type mp\_digit which ensures type compatibility within the 
implementation.  The initial addition loop begins on line 66 and ends on line 75.  Similarly the conditional addition loop
begins on line 81 and ends on line 90.  The addition is finished with the final carry being stored in $tmpc$ on line 94.  
Note the ``++'' operator on the same line.  After line 94 $tmpc$ will point to the $c.used$'th digit of the mp\_int $c$.  This is useful
for the next loop on lines 97 to 99 which set any old upper digits to zero.

\subsection{Low Level Subtraction}
The low level unsigned subtraction algorithm is very similar to the low level unsigned addition algorithm.  The principle difference is that the
unsigned subtraction algorithm requires the result to be positive.  That is when computing $a - b$ the condition $\vert a \vert \ge \vert b\vert$ must 
be met for this algorithm to function properly.  Keep in mind this low level algorithm is not meant to be used in higher level algorithms directly.  
This algorithm as will be shown can be used to create functional signed addition and subtraction algorithms.


For this algorithm a new variable is required to make the description simpler.  Recall from section 1.3.1 that a mp\_digit must be able to represent
the range $0 \le x < 2\beta$.  It is allowable that a mp\_digit represent a larger range of values.  For this algorithm we will assume that
the variable $\gamma$ represents the number of bits available in a mp\_digit (\textit{this implies $2^{\gamma} > \beta$}).

\newpage\begin{figure}[!here]
\begin{center}
\begin{small}
\begin{tabular}{l}
\hline Algorithm \textbf{s\_mp\_sub}. \\
\textbf{Input}.   Two mp\_ints $a$ and $b$ ($\vert a \vert \ge \vert b \vert$) \\
\textbf{Output}.  The unsigned subtraction $c = \vert a \vert - \vert b \vert$. \\
\hline \\
1.  $min \leftarrow b.used$ \\
2.  $max \leftarrow a.used$ \\
3.  If $c.alloc < max$ then grow $c$ to hold at least $max$ digits.  (\textit{hint: use mp\_grow}) \\
4.  If the reallocation failed return(\textit{MP\_MEM}). \\
5.  $oldused \leftarrow c.used$ \\ 
6.  $c.used \leftarrow max$ \\
7.  $u \leftarrow 0$ \\
8.  for $n$ from $0$ to $min - 1$ do \\
\hspace{3mm}8.1  $c_n \leftarrow a_n - b_n - u$ \\
\hspace{3mm}8.2  $u   \leftarrow c_n >> (\gamma - 1)$ \\
\hspace{3mm}8.3  $c_n \leftarrow c_n \mbox{ (mod }\beta\mbox{)}$ \\
9.  if $min < max$ then do \\
\hspace{3mm}9.1  for $n$ from $min$ to $max - 1$ do \\
\hspace{6mm}9.1.1  $c_n \leftarrow a_n - u$ \\
\hspace{6mm}9.1.2  $u   \leftarrow c_n >> (\gamma - 1)$ \\
\hspace{6mm}9.1.3  $c_n \leftarrow c_n \mbox{ (mod }\beta\mbox{)}$ \\
10. if $oldused > max$ then do \\
\hspace{3mm}10.1  for $n$ from $max$ to $oldused - 1$ do \\
\hspace{6mm}10.1.1  $c_n \leftarrow 0$ \\
11. Clamp excess digits of $c$.  (\textit{hint: use mp\_clamp}). \\
12. Return(\textit{MP\_OKAY}). \\
\hline
\end{tabular}
\end{small}
\end{center}
\caption{Algorithm s\_mp\_sub}
\end{figure}

\textbf{Algorithm s\_mp\_sub.}
This algorithm performs the unsigned subtraction of two mp\_int variables under the restriction that the result must be positive.  That is when
passing variables $a$ and $b$ the condition that $\vert a \vert \ge \vert b \vert$ must be met for the algorithm to function correctly.  This
algorithm is loosely based on algorithm 14.9 \cite[pp. 595]{HAC} and is similar to algorithm S in \cite[pp. 267]{TAOCPV2} as well.  As was the case
of the algorithm s\_mp\_add both other references lack discussion concerning various practical details such as when the inputs differ in magnitude.

The initial sorting of the inputs is trivial in this algorithm since $a$ is guaranteed to have at least the same magnitude of $b$.  Steps 1 and 2 
set the $min$ and $max$ variables.  Unlike the addition routine there is guaranteed to be no carry which means that the final result can be at 
most $max$ digits in length as oppose to $max + 1$.  Similar to the addition algorithm the \textbf{used} count of $c$ is copied locally and 
set to the maximal count for the operation.

The subtraction loop that begins on step 8 is essentially the same as the addition loop of algorithm s\_mp\_add except single precision 
subtraction is used instead.  Note the use of the $\gamma$ variable to extract the carry within the subtraction loops.  Under the assumption
that two's complement single precision arithmetic is used this will successfully extract the carry.  

For example, consider subtracting $0101_2$ from
$0100_2$ where $\gamma = 4$.  The least significant bit will force a carry upwards to the third bit which will be set to zero after the borrow.  After
the very first bit has been subtracted $4 - 1 \equiv 0011_2$ will remain,  When the third bit of $0101_2$ is subtracted from the result it will cause
another carry.  In this case though the carry will be forced to propagate all the way to the most significant bit.  

Recall that $\beta < 2^{\gamma}$.  This means that if a carry does occur it will propagate all the way to the most significant bit.  Therefore a single
logical shift right by $\gamma - 1$ positions is sufficient to extract the carry.  This method of carry extraction may seem awkward but the reason for 
it becomes apparent when the implementation is discussed.  

If $b$ has a smaller magnitude than $a$ then step 9 will force the carry and copy operation to propagate through the larger input $a$ into $c$.  Step
10 will ensure that any leading digits of $c$ above the $max$'th position are zeroed.

\index{bn\_s\_mp\_sub.c}
\vspace{+3mm}\begin{small}
\hspace{-5.1mm}{\bf File}: bn\_s\_mp\_sub.c
\vspace{-3mm}
\begin{alltt}
016   
017   /* low level subtraction (assumes |a| > |b|), HAC pp.595 Algorithm 14.9 */
018   int
019   s_mp_sub (mp_int * a, mp_int * b, mp_int * c)
020   \{
021     int     olduse, res, min, max;
022   
023     /* find sizes */
024     min = b->used;
025     max = a->used;
026   
027     /* init result */
028     if (c->alloc < max) \{
029       if ((res = mp_grow (c, max)) != MP_OKAY) \{
030         return res;
031       \}
032     \}
033     olduse = c->used;
034     c->used = max;
035   
036     /* sub digits from lower part */
037     \{
038       register mp_digit u, *tmpa, *tmpb, *tmpc;
039       register int i;
040   
041       /* alias for digit pointers */
042       tmpa = a->dp;
043       tmpb = b->dp;
044       tmpc = c->dp;
045   
046       /* set carry to zero */
047       u = 0;
048       for (i = 0; i < min; i++) \{
049         /* T[i] = A[i] - B[i] - U */
050         *tmpc = *tmpa++ - *tmpb++ - u;
051   
052         /* U = carry bit of T[i]
053          * Note this saves performing an AND operation since
054          * if a carry does occur it will propagate all the way to the
055          * MSB.  As a result a single shift is required to get the carry
056          */
057         u = *tmpc >> ((mp_digit)(CHAR_BIT * sizeof (mp_digit) - 1));
058   
059         /* Clear carry from T[i] */
060         *tmpc++ &= MP_MASK;
061       \}
062   
063       /* now copy higher words if any, e.g. if A has more digits than B  */
064       for (; i < max; i++) \{
065         /* T[i] = A[i] - U */
066         *tmpc = *tmpa++ - u;
067   
068         /* U = carry bit of T[i] */
069         u = *tmpc >> ((mp_digit)(CHAR_BIT * sizeof (mp_digit) - 1));
070   
071         /* Clear carry from T[i] */
072         *tmpc++ &= MP_MASK;
073       \}
074   
075       /* clear digits above used (since we may not have grown result above) */
      
076       for (i = c->used; i < olduse; i++) \{
077         *tmpc++ = 0;
078       \}
079     \}
080   
081     mp_clamp (c);
082     return MP_OKAY;
083   \}
\end{alltt}
\end{small}

Line 24 and 25 perform the initial hardcoded sorting.  In reality they are only aliases and are only used to make the source easier to 
read.  Again the pointer alias optimization is used within this algorithm.  Lines 42, 43 and 44 initialize the aliases for 
$a$, $b$ and $c$ respectively.

The first subtraction loop occurs on lines 47 through 61.  The theory behind the subtraction loop is exactly the same as that for
the addition loop.  As remarked earlier there is an implementation reason for using the ``awkward'' method of extracting the carry 
(\textit{see line 57}).  The traditional method for extracting the carry would be to shift by $lg(\beta)$ positions and logically AND 
the least significant bit.  The AND operation is required because all of the bits above the $\lg(\beta)$'th bit will be set to one after a carry
occurs from subtraction.  This carry extraction requires two relatively cheap operations to extract the carry.  The other method is to simply 
shift the most significant bit to the least significant bit thus extracting the carry with a single cheap operation.  This optimization only works on
twos compliment machines which is a safe assumption to make.

If $a$ has a higher magnitude than $b$ an additional loop (\textit{see lines 64 through 73}) is required to propagate the carry through
$a$ and copy the result to $c$.  

\subsection{High Level Addition}
Now that both lower level addition and subtraction algorithms have been established an effective high level signed addition algorithm can be
established.  This high level addition algorithm will be what other algorithms and developers will use to perform addition of mp\_int data 
types.  

Recall from section 5.2 that an mp\_int represents an integer with an unsigned mantissa (\textit{the array of digits}) and a \textbf{sign} 
flag.  A high level addition is actually performed as a series of eight seperate cases which can be optimized down to three unique cases.

\newpage\begin{figure}[!here]
\begin{center}
\begin{tabular}{l}
\hline Algorithm \textbf{mp\_add}. \\
\textbf{Input}.   Two mp\_ints $a$ and $b$  \\
\textbf{Output}.  The signed addition $c = a + b$. \\
\hline \\
1.  if $a.sign = b.sign$ then do \\
\hspace{3mm}1.1  $c.sign \leftarrow a.sign$  \\
\hspace{3mm}1.2  $c \leftarrow \vert a \vert + \vert b \vert$ (\textit{hint: use s\_mp\_add})\\
2.  else do \\
\hspace{3mm}2.1  if $\vert a \vert < \vert b \vert$ then do (\textit{hint: use mp\_cmp\_mag})  \\
\hspace{6mm}2.1.1  $c.sign \leftarrow b.sign$ \\
\hspace{6mm}2.1.2  $c \leftarrow \vert b \vert - \vert a \vert$ (\textit{hint: use s\_mp\_sub}) \\
\hspace{3mm}2.2  else do \\
\hspace{6mm}2.2.1  $c.sign \leftarrow a.sign$ \\
\hspace{6mm}2.2.2  $c \leftarrow \vert a \vert - \vert b \vert$ \\
3.  If any of the lower level operations failed return(\textit{MP\_MEM}) \\
4.  Return(\textit{MP\_OKAY}). \\
\hline
\end{tabular}
\end{center}
\caption{Algorithm mp\_add}
\end{figure}

\textbf{Algorithm mp\_add.}
This algorithm performs the signed addition of two mp\_int variables.  There is no reference algorithm to draw upon from either \cite{TAOCPV2} or 
\cite{HAC} since they both only provide unsigned operations.  The algorithm is fairly straightforward but restricted since subtraction can only 
produce positive results.  Consider the following chart of possible inputs.

\begin{figure}[!here]
\begin{small}
\begin{center}
\begin{tabular}{|c|c|c|c|c|}
\hline \textbf{Sign of $a$} & \textbf{Sign of $b$} & \textbf{$\vert a \vert > \vert b \vert $} & \textbf{Unsigned Operation} & \textbf{Result Sign Flag} \\
\hline $+$ & $+$ & Yes & $c = a + b$ & $a.sign$ \\
\hline $+$ & $+$ & No  & $c = a + b$ & $a.sign$ \\
\hline $-$ & $-$ & Yes & $c = a + b$ & $a.sign$ \\
\hline $-$ & $-$ & No  & $c = a + b$ & $a.sign$ \\
\hline &&&&\\

\hline $+$ & $-$ & No  & $c = b - a$ & $b.sign$ \\
\hline $-$ & $+$ & No  & $c = b - a$ & $b.sign$ \\

\hline &&&&\\

\hline $+$ & $-$ & Yes & $c = a - b$ & $a.sign$ \\
\hline $-$ & $+$ & Yes & $c = a - b$ & $a.sign$ \\

\hline
\end{tabular}
\end{center}
\end{small}
\caption{Addition Guide Chart}
\end{figure}

The chart lists all of the eight possible input combinations and is sorted to show that only three specific cases need to be handled.  The 
return code of the unsigned operations at step 1.2, 2.1.2 and 2.2.2 are forwarded to step 3 to check for errors.  This simpliies the description
of the algorithm considerably and best follows how the implementation actually was achieved.

Also note how the \textbf{sign} is set before the unsigned addition or subtraction is performed.  Recall from the descriptions of algorithms
s\_mp\_add and s\_mp\_sub that the mp\_clamp function is used at the end to trim excess digits.  The mp\_clamp algorithm will set the \textbf{sign}
to \textbf{MP\_ZPOS} when the \textbf{used} digit count reaches zero.  

For example, consider performing $-a + a$ with algorithm mp\_add.  By the description of the algorithm the sign is set to \textbf{MP\_NEG} which would
produce a result of $-0$.  However, since the sign is set first then the unsigned addition is performed the subsequent usage of algorithm mp\_clamp 
within algorithm s\_mp\_add will force $-0$ to become $0$.  

\index{bn\_mp\_add.c}
\vspace{+3mm}\begin{small}
\hspace{-5.1mm}{\bf File}: bn\_mp\_add.c
\vspace{-3mm}
\begin{alltt}
016   
017   /* high level addition (handles signs) */
018   int
019   mp_add (mp_int * a, mp_int * b, mp_int * c)
020   \{
021     int     sa, sb, res;
022   
023     /* get sign of both inputs */
024     sa = a->sign;
025     sb = b->sign;
026   
027     /* handle two cases, not four */
028     if (sa == sb) \{
029       /* both positive or both negative */
030       /* add their magnitudes, copy the sign */
031       c->sign = sa;
032       res = s_mp_add (a, b, c);
033     \} else \{
034       /* one positive, the other negative */
035       /* subtract the one with the greater magnitude from */
036       /* the one of the lesser magnitude.  The result gets */
037       /* the sign of the one with the greater magnitude. */
038       if (mp_cmp_mag (a, b) == MP_LT) \{
039         c->sign = sb;
040         res = s_mp_sub (b, a, c);
041       \} else \{
042         c->sign = sa;
043         res = s_mp_sub (a, b, c);
044       \}
045     \}
046     return res;
047   \}
048   
\end{alltt}
\end{small}

The source code follows the algorithm fairly closely.  The most notable new source code addition is the usage of the $res$ integer variable which
is used to pass result of the unsigned operations forward.  Unlike in the algorithm, the variable $res$ is merely returned as is without
explicitly checking it and returning the constant \textbf{MP\_OKAY}.  The observation is this algorithm will succeed or fail only if the lower
level functions do so.  Returning their return code is sufficient.

\subsection{High Level Subtraction}
The high level signed subtraction algorithm is essentially the same as the high level signed addition algorithm.  

\begin{figure}[!here]
\begin{center}
\begin{tabular}{l}
\hline Algorithm \textbf{mp\_sub}. \\
\textbf{Input}.   Two mp\_ints $a$ and $b$  \\
\textbf{Output}.  The signed subtraction $c = a - b$. \\
\hline \\
1.  if $a.sign \ne b.sign$ then do \\
\hspace{3mm}1.1  $c.sign \leftarrow a.sign$ \\
\hspace{3mm}1.2  $c \leftarrow \vert a \vert + \vert b \vert$ (\textit{hint: use s\_mp\_add}) \\
2.  else do \\
\hspace{3mm}2.1  if $\vert a \vert \ge \vert b \vert$ then do (\textit{hint: use mp\_cmp\_mag}) \\
\hspace{6mm}2.1.1  $c.sign \leftarrow a.sign$ \\
\hspace{6mm}2.1.2  $c \leftarrow \vert a \vert  - \vert b \vert$ (\textit{hint: use s\_mp\_sub}) \\
\hspace{3mm}2.2  else do \\
\hspace{6mm}2.2.1  $c.sign \leftarrow  \left \lbrace \begin{array}{ll}
                              MP\_ZPOS &  \mbox{if }a.sign = MP\_NEG \\
                              MP\_NEG  &  \mbox{otherwise} \\
                              \end{array} \right .$ \\
\hspace{6mm}2.2.2  $c \leftarrow \vert b \vert  - \vert a \vert$ \\
3.  If any of the lower level operations failed return(\textit{MP\_MEM}). \\
4.  Return(\textit{MP\_OKAY}). \\
\hline
\end{tabular}
\end{center}
\caption{Algorithm mp\_sub}
\end{figure}

\textbf{Algorithm mp\_sub.}
This algorithm performs the signed subtraction of two inputs.  Similar to algorithm mp\_add there is no reference in either \cite{TAOCPV2} or 
\cite{HAC}.  Also this algorithm is restricted by algorithm s\_mp\_sub.  The following chart lists the eight possible inputs and
the operations required.

\newpage\begin{figure}[!here]
\begin{small}
\begin{center}
\begin{tabular}{|c|c|c|c|c|}
\hline \textbf{Sign of $a$} & \textbf{Sign of $b$} & \textbf{$\vert a \vert \ge \vert b \vert $} & \textbf{Unsigned Operation} & \textbf{Result Sign Flag} \\
\hline $+$ & $-$ & Yes & $c = a + b$ & $a.sign$ \\
\hline $+$ & $-$ & No  & $c = a + b$ & $a.sign$ \\
\hline $-$ & $+$ & Yes & $c = a + b$ & $a.sign$ \\
\hline $-$ & $+$ & No  & $c = a + b$ & $a.sign$ \\
\hline &&&& \\
\hline $+$ & $+$ & Yes & $c = a - b$ & $a.sign$ \\
\hline $-$ & $-$ & Yes & $c = a - b$ & $a.sign$ \\
\hline &&&& \\
\hline $+$ & $+$ & No  & $c = b - a$ & $\mbox{opposite of }a.sign$ \\
\hline $-$ & $-$ & No  & $c = b - a$ & $\mbox{opposite of }a.sign$ \\
\hline
\end{tabular}
\end{center}
\end{small}
\caption{Subtraction Guide Chart}
\end{figure}

Similar to the case of algorithm mp\_add the \textbf{sign} is set first before the unsigned addition or subtraction.  That is to prevent the 
algorithm from producing $-a - -a = -0$ as a result.  

\index{bn\_mp\_sub.c}
\vspace{+3mm}\begin{small}
\hspace{-5.1mm}{\bf File}: bn\_mp\_sub.c
\vspace{-3mm}
\begin{alltt}
016   
017   /* high level subtraction (handles signs) */
018   int
019   mp_sub (mp_int * a, mp_int * b, mp_int * c)
020   \{
021     int     sa, sb, res;
022   
023     sa = a->sign;
024     sb = b->sign;
025   
026     if (sa != sb) \{
027       /* subtract a negative from a positive, OR */
028       /* subtract a positive from a negative. */
029       /* In either case, ADD their magnitudes, */
030       /* and use the sign of the first number. */
031       c->sign = sa;
032       res = s_mp_add (a, b, c);
033     \} else \{
034       /* subtract a positive from a positive, OR */
035       /* subtract a negative from a negative. */
036       /* First, take the difference between their */
037       /* magnitudes, then... */
038       if (mp_cmp_mag (a, b) != MP_LT) \{
039         /* Copy the sign from the first */
040         c->sign = sa;
041         /* The first has a larger or equal magnitude */
042         res = s_mp_sub (a, b, c);
043       \} else \{
044         /* The result has the *opposite* sign from */
045         /* the first number. */
046         c->sign = (sa == MP_ZPOS) ? MP_NEG : MP_ZPOS;
047         /* The second has a larger magnitude */
048         res = s_mp_sub (b, a, c);
049       \}
050     \}
051     return res;
052   \}
053   
\end{alltt}
\end{small}

Much like the implementation of algorithm mp\_add the variable $res$ is used to catch the return code of the unsigned addition or subtraction operations
and forward it to the end of the function.  On line 38 the ``not equal to'' \textbf{MP\_LT} expression is used to emulate a 
``greater than or equal to'' comparison.  

\section{Bit and Digit Shifting}
It is quite common to think of a multiple precision integer as a polynomial in $x$, that is $y = f(\beta)$ where $f(x) = \sum_{i=0}^{n-1} a_i x^i$.  
This notation arises within discussion of Montgomery and Diminished Radix Reduction as well as Karatsuba multiplication and squaring.  

In order to facilitate operations on polynomials in $x$ as above a series of simple ``digit'' algorithms have to be established.  That is to shift
the digits left or right as well to shift individual bits of the digits left and right.  It is important to note that not all ``shift'' operations
are on radix-$\beta$ digits.  

\subsection{Multiplication by Two}

In a binary system where the radix is a power of two multiplication by two not only arises often in other algorithms it is a fairly efficient 
operation to perform.  A single precision logical shift left is sufficient to multiply a single digit by two.  

\newpage\begin{figure}[!here]
\begin{small}
\begin{center}
\begin{tabular}{l}
\hline Algorithm \textbf{mp\_mul\_2}. \\
\textbf{Input}.   One mp\_int $a$ \\
\textbf{Output}.  $b = 2a$. \\
\hline \\
1.  If $b.alloc < a.used + 1$ then grow $b$ to hold $a.used + 1$ digits.  (\textit{hint: use mp\_grow}) \\
2.  If the reallocation failed return(\textit{MP\_MEM}). \\
3.  $oldused \leftarrow b.used$ \\
4.  $b.used \leftarrow a.used$ \\
5.  $r \leftarrow 0$ \\
6.  for $n$ from 0 to $a.used - 1$ do \\
\hspace{3mm}6.1  $rr \leftarrow a_n >> (lg(\beta) - 1)$ \\
\hspace{3mm}6.2  $b_n \leftarrow (a_n << 1) + r \mbox{ (mod }\beta\mbox{)}$ \\
\hspace{3mm}6.3  $r \leftarrow rr$ \\
7.  If $r \ne 0$ then do \\
\hspace{3mm}7.1  $b_{a.used} = 1$ \\
\hspace{3mm}7.2  $b.used \leftarrow b.used + 1$ \\
8.  If $b.used < oldused - 1$ then do \\
\hspace{3mm}8.1  for $n$ from $b.used$ to $oldused - 1$ do \\
\hspace{6mm}8.1.1  $b_n \leftarrow 0$ \\
9.  $b.sign \leftarrow a.sign$ \\
10.  Return(\textit{MP\_OKAY}).\\
\hline
\end{tabular}
\end{center}
\end{small}
\caption{Algorithm mp\_mul\_2}
\end{figure}

\textbf{Algorithm mp\_mul\_2.}
This algorithm will quickly multiply a mp\_int by two provided $\beta$ is a power of two.  Neither \cite{TAOCPV2} nor \cite{HAC} describe such 
an algorithm despite the fact it arises often in other algorithms.  The algorithm is setup much like the lower level algorithm s\_mp\_add since 
it is for all intents and purposes equivalent to the operation $b = \vert a \vert + \vert a \vert$.  

Step 1 and 2 grow the input as required to accomodate the maximum number of \textbf{used} digits in the result.  The initial \textbf{used} count
is set to $a.used$ at step 4.  Only if there is a final carry will the \textbf{used} count require adjustment.

Step 6 is an optimization implementation of the addition loop for this specific case.  That is since the two values being added together 
are the same there is no need to perform two reads from the digits of $a$.  Step 6.1 performs a single precision shift on the current digit $a_n$ to
obtain what will be the carry for the next iteration.  Step 6.2 calculates the $n$'th digit of the result as single precision shift of $a_n$ plus
the previous carry.  Recall from section 5.1 that $a_n << 1$ is equivalent to $a_n \cdot 2$.  An iteration of the addition loop is finished with 
forwarding the carry to the next iteration.

Step 7 takes care of any final carry by setting the $a.used$'th digit of the result to one and augmenting the \textbf{used} count.  Step 8 clears
any original leading digits of $b$.

\index{bn\_mp\_mul\_2.c}
\vspace{+3mm}\begin{small}
\hspace{-5.1mm}{\bf File}: bn\_mp\_mul\_2.c
\vspace{-3mm}
\begin{alltt}
016   
017   /* b = a*2 */
018   int
019   mp_mul_2 (mp_int * a, mp_int * b)
020   \{
021     int     x, res, oldused;
022   
023     /* grow to accomodate result */
024     if (b->alloc < a->used + 1) \{
025       if ((res = mp_grow (b, a->used + 1)) != MP_OKAY) \{
026         return res;
027       \}
028     \}
029   
030     oldused = b->used;
031     b->used = a->used;
032   
033     \{
034       register mp_digit r, rr, *tmpa, *tmpb;
035   
036       /* alias for source */
037       tmpa = a->dp;
038       
039       /* alias for dest */
040       tmpb = b->dp;
041   
042       /* carry */
043       r = 0;
044       for (x = 0; x < a->used; x++) \{
045       
046         /* get what will be the *next* carry bit from the 
047          * MSB of the current digit 
048          */
049         rr = *tmpa >> ((mp_digit)(DIGIT_BIT - 1));
050         
051         /* now shift up this digit, add in the carry [from the previous] */
052         *tmpb++ = ((*tmpa++ << ((mp_digit)1)) | r) & MP_MASK;
053         
054         /* copy the carry that would be from the source 
055          * digit into the next iteration 
056          */
057         r = rr;
058       \}
059   
060       /* new leading digit? */
061       if (r != 0) \{
062         /* add a MSB which is always 1 at this point */
063         *tmpb = 1;
064         ++b->used;
065       \}
066   
067       /* now zero any excess digits on the destination 
068        * that we didn't write to 
069        */
070       tmpb = b->dp + b->used;
071       for (x = b->used; x < oldused; x++) \{
072         *tmpb++ = 0;
073       \}
074     \}
075     b->sign = a->sign;
076     return MP_OKAY;
077   \}
\end{alltt}
\end{small}

This implementation is essentially an optimized implementation of s\_mp\_add for the case of doubling an input.  The only noteworthy difference
is the use of the logical shift operator on line 52 to perform a single precision doubling.  

\subsection{Division by Two}
A division by two can just as easily be accomplished with a logical shift right as multiplication by two can be with a logical shift left.

\newpage\begin{figure}[!here]
\begin{small}
\begin{center}
\begin{tabular}{l}
\hline Algorithm \textbf{mp\_div\_2}. \\
\textbf{Input}.   One mp\_int $a$ \\
\textbf{Output}.  $b = a/2$. \\
\hline \\
1.  If $b.alloc < a.used$ then grow $b$ to hold $a.used$ digits.  (\textit{hint: use mp\_grow}) \\
2.  If the reallocation failed return(\textit{MP\_MEM}). \\
3.  $oldused \leftarrow b.used$ \\
4.  $b.used \leftarrow a.used$ \\
5.  $r \leftarrow 0$ \\
6.  for $n$ from $b.used - 1$ to $0$ do \\
\hspace{3mm}6.1  $rr \leftarrow a_n \mbox{ (mod }2\mbox{)}$\\
\hspace{3mm}6.2  $b_n \leftarrow (a_n >> 1) + (r << (lg(\beta) - 1)) \mbox{ (mod }\beta\mbox{)}$ \\
\hspace{3mm}6.3  $r \leftarrow rr$ \\
7.  If $b.used < oldused - 1$ then do \\
\hspace{3mm}7.1  for $n$ from $b.used$ to $oldused - 1$ do \\
\hspace{6mm}7.1.1  $b_n \leftarrow 0$ \\
8.  $b.sign \leftarrow a.sign$ \\
9.  Return(\textit{MP\_OKAY}).\\
\hline
\end{tabular}
\end{center}
\end{small}
\caption{Algorithm mp\_div\_2}
\end{figure}

\textbf{Algorithm mp\_div\_2.}
This algorithm will divide an mp\_int by two using logical shifts to the right.  Like mp\_mul\_2 it uses a modified low level addition
core as the basis of the algorithm.  Unlike mp\_mul\_2 the shift operations work from the leading digit to the trailing digit.  The algorithm
could be written to work from the trailing digit to the leading digit however, it would have to stop one short of $a.used - 1$ digits to prevent
reading passed the end of the array of digits.

Essentially the loop at step 6 is similar to that of mp\_mul\_2 except the logical shifts go in the opposite direction and the carry is at the 
least significant bit not the most significant bit.  

\index{bn\_mp\_div\_2.c}
\vspace{+3mm}\begin{small}
\hspace{-5.1mm}{\bf File}: bn\_mp\_div\_2.c
\vspace{-3mm}
\begin{alltt}
016   
017   /* b = a/2 */
018   int
019   mp_div_2 (mp_int * a, mp_int * b)
020   \{
021     int     x, res, oldused;
022   
023     /* copy */
024     if (b->alloc < a->used) \{
025       if ((res = mp_grow (b, a->used)) != MP_OKAY) \{
026         return res;
027       \}
028     \}
029   
030     oldused = b->used;
031     b->used = a->used;
032     \{
033       register mp_digit r, rr, *tmpa, *tmpb;
034   
035       /* source alias */
036       tmpa = a->dp + b->used - 1;
037   
038       /* dest alias */
039       tmpb = b->dp + b->used - 1;
040   
041       /* carry */
042       r = 0;
043       for (x = b->used - 1; x >= 0; x--) \{
044         /* get the carry for the next iteration */
045         rr = *tmpa & 1;
046   
047         /* shift the current digit, add in carry and store */
048         *tmpb-- = (*tmpa-- >> 1) | (r << (DIGIT_BIT - 1));
049   
050         /* forward carry to next iteration */
051         r = rr;
052       \}
053   
054       /* zero excess digits */
055       tmpb = b->dp + b->used;
056       for (x = b->used; x < oldused; x++) \{
057         *tmpb++ = 0;
058       \}
059     \}
060     b->sign = a->sign;
061     mp_clamp (b);
062     return MP_OKAY;
063   \}
\end{alltt}
\end{small}

\section{Polynomial Basis Operations}
Recall from section 5.3 that any integer can be represented as a polynomial in $x$ as $y = f(\beta)$.  Such a representation is also known as
the polynomial basis \cite[pp. 48]{ROSE}. Given such a notation a multiplication or division by $x$ amounts to shifting whole digits a single 
place.  The need for such operations arises in several other higher level algorithms such as Barrett and Montgomery reduction, integer
division and Karatsuba multiplication.  

Converting from an array of digits to polynomial basis is very simple.  Consider the integer $y \equiv (a_2, a_1, a_0)_{\beta}$ and recall that
$y = \sum_{i=0}^{2} a_i \beta^i$.  Simply replace $\beta$ with $x$ and the expression is in polynomial basis.  For example, $f(x) = 8x + 9$ is the
polynomial basis representation for $89$ using radix ten.  That is, $f(10) = 8(10) + 9 = 89$.  

\subsection{Multiplication by $x$}

Given a polynomial in $x$ such as $f(x) = a_n x^n + a_{n-1} x^{n-1} + ... + a_0$ multiplying by $x$ amounts to shifting the coefficients up one 
degree.  In this case $f(x) \cdot x = a_n x^{n+1} + a_{n-1} x^n + ... + a_0 x$.  From a scalar basis point of view multiplying by $x$ is equivalent to
multiplying by the integer $\beta$.  

\newpage\begin{figure}[!here]
\begin{small}
\begin{center}
\begin{tabular}{l}
\hline Algorithm \textbf{mp\_lshd}. \\
\textbf{Input}.   One mp\_int $a$ and an integer $b$ \\
\textbf{Output}.  $a \leftarrow a \cdot \beta^b$ (Multiply by $x^b$). \\
\hline \\
1.  If $b \le 0$ then return(\textit{MP\_OKAY}). \\
2.  If $a.alloc < a.used + b$ then grow $a$ to at least $a.used + b$ digits.  (\textit{hint: use mp\_grow}). \\
3.  If the reallocation failed return(\textit{MP\_MEM}). \\
4.  $a.used \leftarrow a.used + b$ \\
5.  $i \leftarrow a.used - 1$ \\
6.  $j \leftarrow a.used - 1 - b$ \\
7.  for $n$ from $a.used - 1$ to $b$ do \\
\hspace{3mm}7.1  $a_{i} \leftarrow a_{j}$ \\
\hspace{3mm}7.2  $i \leftarrow i - 1$ \\
\hspace{3mm}7.3  $j \leftarrow j - 1$ \\
8.  for $n$ from 0 to $b - 1$ do \\
\hspace{3mm}8.1  $a_n \leftarrow 0$ \\
9.  Return(\textit{MP\_OKAY}). \\
\hline
\end{tabular}
\end{center}
\end{small}
\caption{Algorithm mp\_lshd}
\end{figure}

\textbf{Algorithm mp\_lshd.}
This algorithm multiplies an mp\_int by the $b$'th power of $x$.  This is equivalent to multiplying by $\beta^b$.  The algorithm differs 
from the other algorithms presented so far as it performs the operation in place instead storing the result in a seperate location.  The algorithm
will return success immediately if $b \le 0$ since the rest of algorithm is only valid when $b > 0$.  

First the destination $a$ is grown as required to accomodate the result.  The counters $i$ and $j$ are used to form a \textit{sliding window} over
the digits of $a$ of length $b$.  The head of the sliding window is at $i$ (\textit{the leading digit}) and the tail at $j$ (\textit{the trailing digit}).  
The loop on step 7 copies the digit from the tail to the head.  In each iteration the window is moved down one digit.   The last loop on 
step 8 sets the lower $b$ digits to zero.

\newpage
\begin{center}
\begin{figure}[here]
\includegraphics{pics/sliding_window.ps}
\caption{Sliding Window Movement}
\end{figure}
\end{center}

\index{bn\_mp\_lshd.c}
\vspace{+3mm}\begin{small}
\hspace{-5.1mm}{\bf File}: bn\_mp\_lshd.c
\vspace{-3mm}
\begin{alltt}
016   
017   /* shift left a certain amount of digits */
018   int
019   mp_lshd (mp_int * a, int b)
020   \{
021     int     x, res;
022   
023     /* if its less than zero return */
024     if (b <= 0) \{
025       return MP_OKAY;
026     \}
027   
028     /* grow to fit the new digits */
029     if (a->alloc < a->used + b) \{
030        if ((res = mp_grow (a, a->used + b)) != MP_OKAY) \{
031          return res;
032        \}
033     \}
034   
035     \{
036       register mp_digit *tmpa, *tmpaa;
037   
038       /* increment the used by the shift amount than copy upwards */
039       a->used += b;
040   
041       /* top */
042       tmpa = a->dp + a->used - 1;
043   
044       /* base */
045       tmpaa = a->dp + a->used - 1 - b;
046   
047       /* much like mp_rshd this is implemented using a sliding window
048        * except the window goes the otherway around.  Copying from
049        * the bottom to the top.  see bn_mp_rshd.c for more info.
050        */
051       for (x = a->used - 1; x >= b; x--) \{
052         *tmpa-- = *tmpaa--;
053       \}
054   
055       /* zero the lower digits */
056       tmpa = a->dp;
057       for (x = 0; x < b; x++) \{
058         *tmpa++ = 0;
059       \}
060     \}
061     return MP_OKAY;
062   \}
\end{alltt}
\end{small}

The if statement on line 24 ensures that the $b$ variable is greater than zero.  The \textbf{used} count is incremented by $b$ before
the copy loop begins.  This elminates the need for an additional variable in the for loop.  The variable $tmpa$ on line 42 is an alias
for the leading digit while $tmpaa$ on line 45 is an alias for the trailing edge.  The aliases form a window of exactly $b$ digits
over the input.  

\subsection{Division by $x$}

Division by powers of $x$ is easily achieved by shifting the digits right and removing any that will end up to the right of the zero'th digit.  

\newpage\begin{figure}[!here]
\begin{small}
\begin{center}
\begin{tabular}{l}
\hline Algorithm \textbf{mp\_rshd}. \\
\textbf{Input}.   One mp\_int $a$ and an integer $b$ \\
\textbf{Output}.  $a \leftarrow a / \beta^b$ (Divide by $x^b$). \\
\hline \\
1.  If $b \le 0$ then return. \\
2.  If $a.used \le b$ then do \\
\hspace{3mm}2.1  Zero $a$.  (\textit{hint: use mp\_zero}). \\
\hspace{3mm}2.2  Return. \\
3.  $i \leftarrow 0$ \\
4.  $j \leftarrow b$ \\
5.  for $n$ from 0 to $a.used - b - 1$ do \\
\hspace{3mm}5.1  $a_i \leftarrow a_j$ \\
\hspace{3mm}5.2  $i \leftarrow i + 1$ \\
\hspace{3mm}5.3  $j \leftarrow j + 1$ \\
6.  for $n$ from $a.used - b$ to $a.used - 1$ do \\
\hspace{3mm}6.1  $a_n \leftarrow 0$ \\
7.  Clamp excess digits.  (\textit{hint: use mp\_clamp}). \\
8.  Return. \\
\hline
\end{tabular}
\end{center}
\end{small}
\caption{Algorithm mp\_rshd}
\end{figure}

\textbf{Algorithm mp\_rshd.}
This algorithm divides the input in place by the $b$'th power of $x$.  It is analogous to dividing by a $\beta^b$ but much quicker since
it does not require single precision division.  This algorithm does not actually return an error code as it cannot fail.  

If the input $b$ is less than one the algorithm quickly returns without performing any work.  If the \textbf{used} count is less than or equal
to the shift count $b$ then it will simply zero the input and return.

After the trivial cases of inputs have been handled the sliding window is setup.  Much like the case of algorithm mp\_lshd a sliding window that
is $b$ digits wide is used to copy the digits.  Unlike mp\_lshd the window slides in the opposite direction from the trailing to the leading digit.  
Also the digits are copied from the leading to the trailing edge.

Once the window copy is complete the upper digits must be zeroed.  Finally algorithm mp\_clamp is used to trim excess digits.

\index{bn\_mp\_rshd.c}
\vspace{+3mm}\begin{small}
\hspace{-5.1mm}{\bf File}: bn\_mp\_rshd.c
\vspace{-3mm}
\begin{alltt}
016   
017   /* shift right a certain amount of digits */
018   void
019   mp_rshd (mp_int * a, int b)
020   \{
021     int     x;
022   
023     /* if b <= 0 then ignore it */
024     if (b <= 0) \{
025       return;
026     \}
027   
028     /* if b > used then simply zero it and return */
029     if (a->used <= b) \{
030       mp_zero (a);
031       return;
032     \}
033   
034     \{
035       register mp_digit *tmpa, *tmpaa;
036   
037       /* shift the digits down */
038   
039       /* base */
040       tmpa = a->dp;
041   
042       /* offset into digits */
043       tmpaa = a->dp + b;
044   
045       /* this is implemented as a sliding window where 
046        * the window is b-digits long and digits from 
047        * the top of the window are copied to the bottom
048        *
049        * e.g.
050   
051        b-2 | b-1 | b0 | b1 | b2 | ... | bb |   ---->
052                    /\symbol{92}                   |      ---->
053                     \symbol{92}-------------------/      ---->
054        */
055       for (x = 0; x < (a->used - b); x++) \{
056         *tmpa++ = *tmpaa++;
057       \}
058   
059       /* zero the top digits */
060       for (; x < a->used; x++) \{
061         *tmpa++ = 0;
062       \}
063     \}
064     mp_clamp (a);
065   \}
\end{alltt}
\end{small}

The only noteworthy element of this routine is the lack of a return type.  This function cannot fail and as such it is more optimal to not
return anything.

\section{Powers of Two}

Now that algorithms for moving single bits as well as whole digits exist algorithms for moving the ``in between'' distances are required.  For 
example, to quickly multiply by $2^k$ for any $k$ without using a full multiplier algorithm would prove useful.  Instead of performing single
shifts $k$ times to achieve a multiplication by $2^{\pm k}$ a mixture of whole digit shifting and partial digit shifting is employed.  

\subsection{Multiplication by Power of Two}

\newpage\begin{figure}[!here]
\begin{small}
\begin{center}
\begin{tabular}{l}
\hline Algorithm \textbf{mp\_mul\_2d}. \\
\textbf{Input}.   One mp\_int $a$ and an integer $b$ \\
\textbf{Output}.  $c \leftarrow a \cdot 2^b$. \\
\hline \\
1.  $c \leftarrow a$.  (\textit{hint: use mp\_copy}) \\
2.  If $c.alloc < c.used + \lfloor b / lg(\beta) \rfloor + 2$ then grow $c$ accordingly. \\
3.  If the reallocation failed return(\textit{MP\_MEM}). \\
4.  If $b \ge lg(\beta)$ then \\
\hspace{3mm}4.1  $c \leftarrow c \cdot \beta^{\lfloor b / lg(\beta) \rfloor}$ (\textit{hint: use mp\_lshd}). \\
\hspace{3mm}4.2  If step 4.1 failed return(\textit{MP\_MEM}). \\
5.  $d \leftarrow b \mbox{ (mod }lg(\beta)\mbox{)}$ \\
6.  If $d \ne 0$ then do \\
\hspace{3mm}6.1  $mask \leftarrow 2^d$ \\
\hspace{3mm}6.2  $r \leftarrow 0$ \\
\hspace{3mm}6.3  for $n$ from $0$ to $c.used - 1$ do \\
\hspace{6mm}6.3.1  $rr \leftarrow c_n >> (lg(\beta) - d) \mbox{ (mod }mask\mbox{)}$ \\
\hspace{6mm}6.3.2  $c_n \leftarrow (c_n << d) + r \mbox{ (mod }\beta\mbox{)}$ \\
\hspace{6mm}6.3.3  $r \leftarrow rr$ \\
\hspace{3mm}6.4  If $r > 0$ then do \\
\hspace{6mm}6.4.1  $c_{c.used} \leftarrow r$ \\
\hspace{6mm}6.4.2  $c.used \leftarrow c.used + 1$ \\
7.  Return(\textit{MP\_OKAY}). \\
\hline
\end{tabular}
\end{center}
\end{small}
\caption{Algorithm mp\_mul\_2d}
\end{figure}

\textbf{Algorithm mp\_mul\_2d.}
This algorithm multiplies $a$ by $2^b$ and stores the result in $c$.  The algorithm uses algorithm mp\_lshd and a derivative of algorithm mp\_mul\_2 to
quickly compute the product.

First the algorithm will multiply $a$ by $x^{\lfloor b / lg(\beta) \rfloor}$ which will ensure that the remainder multiplicand is less than 
$\beta$.  For example, if $b = 37$ and $\beta = 2^{28}$ then this step will multiply by $x$ leaving a multiplication by $2^{37 - 28} = 2^{9}$ 
left.

The logarithm of the residue is calculated on step 5.  If it is non-zero a modified shift loop is used to calculate the remaining product.  
Essentially the loop is a generic version of algorith mp\_mul2 designed to handle any shift count in the range $1 \le x < lg(\beta)$.  The $mask$
variable is used to extract the upper $d$ bits to form the carry for the next iteration.  

This algorithm is loosely measured as a $O(2n)$ algorithm which means that if the input is $n$-digits that it takes $2n$ ``time'' to 
complete.  It is possible to optimize this algorithm down to a $O(n)$ algorithm at a cost of making the algorithm slightly harder to follow.

\index{bn\_mp\_mul\_2d.c}
\vspace{+3mm}\begin{small}
\hspace{-5.1mm}{\bf File}: bn\_mp\_mul\_2d.c
\vspace{-3mm}
\begin{alltt}
016   
017   /* NOTE:  This routine requires updating.  For instance the c->used = c->all
      oc bit
018      is wrong.  We should just shift c->used digits then set the carry as c->d
      p[c->used] = carry
019    
020      To be fixed for LTM 0.18
021    */
022   
023   /* shift left by a certain bit count */
024   int
025   mp_mul_2d (mp_int * a, int b, mp_int * c)
026   \{
027     mp_digit d;
028     int      res;
029   
030     /* copy */
031     if (a != c) \{
032        if ((res = mp_copy (a, c)) != MP_OKAY) \{
033          return res;
034        \}
035     \}
036   
037     if (c->alloc < (int)(c->used + b/DIGIT_BIT + 2)) \{
038        if ((res = mp_grow (c, c->used + b / DIGIT_BIT + 2)) != MP_OKAY) \{
039          return res;
040        \}
041     \}
042   
043     /* shift by as many digits in the bit count */
044     if (b >= (int)DIGIT_BIT) \{
045       if ((res = mp_lshd (c, b / DIGIT_BIT)) != MP_OKAY) \{
046         return res;
047       \}
048     \}
049     c->used = c->alloc;
050   
051     /* shift any bit count < DIGIT_BIT */
052     d = (mp_digit) (b % DIGIT_BIT);
053     if (d != 0) \{
054       register mp_digit *tmpc, mask, r, rr;
055       register int x;
056   
057       /* bitmask for carries */
058       mask = (((mp_digit)1) << d) - 1;
059   
060       /* alias */
061       tmpc = c->dp;
062   
063       /* carry */
064       r    = 0;
065       for (x = 0; x < c->used; x++) \{
066         /* get the higher bits of the current word */
067         rr = (*tmpc >> (DIGIT_BIT - d)) & mask;
068   
069         /* shift the current word and OR in the carry */
070         *tmpc = ((*tmpc << d) | r) & MP_MASK;
071         ++tmpc;
072   
073         /* set the carry to the carry bits of the current word */
074         r = rr;
075       \}
076     \}
077     mp_clamp (c);
078     return MP_OKAY;
079   \}
\end{alltt}
\end{small}

Notes to be revised when code is updated. -- Tom

\subsection{Division by Power of Two}

\newpage\begin{figure}[!here]
\begin{small}
\begin{center}
\begin{tabular}{l}
\hline Algorithm \textbf{mp\_div\_2d}. \\
\textbf{Input}.   One mp\_int $a$ and an integer $b$ \\
\textbf{Output}.  $c \leftarrow \lfloor a / 2^b \rfloor, d \leftarrow a \mbox{ (mod }2^b\mbox{)}$. \\
\hline \\
1.  If $b \le 0$ then do \\
\hspace{3mm}1.1  $c \leftarrow a$ (\textit{hint: use mp\_copy}) \\
\hspace{3mm}1.2  $d \leftarrow 0$ (\textit{hint: use mp\_zero}) \\
\hspace{3mm}1.3  Return(\textit{MP\_OKAY}). \\
2.  $c \leftarrow a$ \\
3.  $d \leftarrow a \mbox{ (mod }2^b\mbox{)}$ (\textit{hint: use mp\_mod\_2d}) \\
4.  If $b \ge lg(\beta)$ then do \\
\hspace{3mm}4.1  $c \leftarrow \lfloor c/\beta^{\lfloor b/lg(\beta) \rfloor} \rfloor$ (\textit{hint: use mp\_rshd}). \\
5.  $k \leftarrow b \mbox{ (mod }lg(\beta)\mbox{)}$ \\
6.  If $k \ne 0$ then do \\
\hspace{3mm}6.1  $mask \leftarrow 2^k$ \\
\hspace{3mm}6.2  $r \leftarrow 0$ \\
\hspace{3mm}6.3  for $n$ from $c.used - 1$ to $0$ do \\
\hspace{6mm}6.3.1  $rr \leftarrow c_n \mbox{ (mod }mask\mbox{)}$ \\
\hspace{6mm}6.3.2  $c_n \leftarrow (c_n >> k) + (r << (lg(\beta) - k))$ \\
\hspace{6mm}6.3.3  $r \leftarrow rr$ \\
7.  Clamp excess digits of $c$.  (\textit{hint: use mp\_clamp}) \\
8.  Return(\textit{MP\_OKAY}). \\
\hline
\end{tabular}
\end{center}
\end{small}
\caption{Algorithm mp\_div\_2d}
\end{figure}

\textbf{Algorithm mp\_div\_2d.}
This algorithm will divide an input $a$ by $2^b$ and produce the quotient and remainder.  The algorithm is designed much like algorithm 
mp\_mul\_2d by first using whole digit shifts then single precision shifts.  This algorithm will also produce the remainder of the division
by using algorithm mp\_mod\_2d.

\index{bn\_mp\_div\_2d.c}
\vspace{+3mm}\begin{small}
\hspace{-5.1mm}{\bf File}: bn\_mp\_div\_2d.c
\vspace{-3mm}
\begin{alltt}
016   
017   /* shift right by a certain bit count (store quotient in c, remainder in d) 
      */
018   int
019   mp_div_2d (mp_int * a, int b, mp_int * c, mp_int * d)
020   \{
021     mp_digit D, r, rr;
022     int     x, res;
023     mp_int  t;
024   
025   
026     /* if the shift count is <= 0 then we do no work */
027     if (b <= 0) \{
028       res = mp_copy (a, c);
029       if (d != NULL) \{
030         mp_zero (d);
031       \}
032       return res;
033     \}
034   
035     if ((res = mp_init (&t)) != MP_OKAY) \{
036       return res;
037     \}
038   
039     /* get the remainder */
040     if (d != NULL) \{
041       if ((res = mp_mod_2d (a, b, &t)) != MP_OKAY) \{
042         mp_clear (&t);
043         return res;
044       \}
045     \}
046   
047     /* copy */
048     if ((res = mp_copy (a, c)) != MP_OKAY) \{
049       mp_clear (&t);
050       return res;
051     \}
052   
053     /* shift by as many digits in the bit count */
054     if (b >= (int)DIGIT_BIT) \{
055       mp_rshd (c, b / DIGIT_BIT);
056     \}
057   
058     /* shift any bit count < DIGIT_BIT */
059     D = (mp_digit) (b % DIGIT_BIT);
060     if (D != 0) \{
061       register mp_digit *tmpc, mask;
062   
063       /* mask */
064       mask = (((mp_digit)1) << D) - 1;
065   
066       /* alias */
067       tmpc = c->dp + (c->used - 1);
068   
069       /* carry */
070       r = 0;
071       for (x = c->used - 1; x >= 0; x--) \{
072         /* get the lower  bits of this word in a temp */
073         rr = *tmpc & mask;
074   
075         /* shift the current word and mix in the carry bits from the previous 
      word */
076         *tmpc = (*tmpc >> D) | (r << (DIGIT_BIT - D));
077         --tmpc;
078   
079         /* set the carry to the carry bits of the current word found above */
080         r = rr;
081       \}
082     \}
083     mp_clamp (c);
084     res = MP_OKAY;
085     if (d != NULL) \{
086       mp_exch (&t, d);
087     \}
088     mp_clear (&t);
089     return MP_OKAY;
090   \}
\end{alltt}
\end{small}

The implementation of algorithm mp\_div\_2d is slightly different than the algorithm specifies.  The remainder $d$ may be optionally 
ignored by passing \textbf{NULL} as the pointer to the mp\_int variable.    The temporary mp\_int variable $t$ is used to hold the 
result of the remainder operation until the end.  This allows $d = a$ to be true without overwriting the input before they are no longer required.  

The remainder of the source code is essentially the same as the source code for mp\_mul\_2d.  (-- Fix this paragraph up later, Tom).

\subsection{Remainder of Division by Power of Two}

The last algorithm in the series of polynomial basis power of two algorithms is calculating the remainder of division by $2^b$.  This
algorithm benefits from the fact that in twos complement arithmetic $a \mbox{ (mod }2^b\mbox{)}$ is the same as $a$ AND $2^b - 1$.  

\begin{figure}[!here]
\begin{small}
\begin{center}
\begin{tabular}{l}
\hline Algorithm \textbf{mp\_mod\_2d}. \\
\textbf{Input}.   One mp\_int $a$ and an integer $b$ \\
\textbf{Output}.  $c \leftarrow a \mbox{ (mod }2^b\mbox{)}$. \\
\hline \\
1.  If $b \le 0$ then do \\
\hspace{3mm}1.1  $c \leftarrow 0$ (\textit{hint: use mp\_zero}) \\
\hspace{3mm}1.2  Return(\textit{MP\_OKAY}). \\
2.  If $b > a.used \cdot lg(\beta)$ then do \\
\hspace{3mm}2.1  $c \leftarrow a$ (\textit{hint: use mp\_copy}) \\
\hspace{3mm}2.2  Return the result of step 2.1. \\
3.  $c \leftarrow a$ \\
4.  If step 3 failed return(\textit{MP\_MEM}). \\
5.  for $n$ from $\lceil b / lg(\beta) \rceil$ to $c.used$ do \\
\hspace{3mm}5.1  $c_n \leftarrow 0$ \\
6.  $k \leftarrow b \mbox{ (mod }lg(\beta)\mbox{)}$ \\
7.  $c_{\lfloor b / lg(\beta) \rfloor} \leftarrow c_{\lfloor b / lg(\beta) \rfloor} \mbox{ (mod }2^{k}\mbox{)}$. \\
8.  Return(\textit{MP\_OKAY}). \\
\hline
\end{tabular}
\end{center}
\end{small}
\caption{Algorithm mp\_mod\_2d}
\end{figure}

\textbf{Algorithm mp\_mod\_2d.}
This algorithm will quickly calculate the value of $a \mbox{ (mod }2^b\mbox{)}$.  First if $b$ is less than or equal to zero the 
result is set to zero.  If $b$ is greater than the number of bits in $a$ then it simply copies $a$ to $c$ and returns.  Otherwise, $a$ 
is copied to $b$, leading digits are removed and the remaining leading digit is trimed to the exact bit count.

\index{bn\_mp\_mod\_2d.c}
\vspace{+3mm}\begin{small}
\hspace{-5.1mm}{\bf File}: bn\_mp\_mod\_2d.c
\vspace{-3mm}
\begin{alltt}
016   
017   /* calc a value mod 2\b */
018   int
019   mp_mod_2d (mp_int * a, int b, mp_int * c)
020   \{
021     int     x, res;
022   
023   
024     /* if b is <= 0 then zero the int */
025     if (b <= 0) \{
026       mp_zero (c);
027       return MP_OKAY;
028     \}
029   
030     /* if the modulus is larger than the value than return */
031     if (b > (int) (a->used * DIGIT_BIT)) \{
032       res = mp_copy (a, c);
033       return res;
034     \}
035   
036     /* copy */
037     if ((res = mp_copy (a, c)) != MP_OKAY) \{
038       return res;
039     \}
040   
041     /* zero digits above the last digit of the modulus */
042     for (x = (b / DIGIT_BIT) + ((b % DIGIT_BIT) == 0 ? 0 : 1); x < c->used; x+
      +) \{
043       c->dp[x] = 0;
044     \}
045     /* clear the digit that is not completely outside/inside the modulus */
046     c->dp[b / DIGIT_BIT] &=
047       (mp_digit) ((((mp_digit) 1) << (((mp_digit) b) % DIGIT_BIT)) - ((mp_digi
      t) 1));
048     mp_clamp (c);
049     return MP_OKAY;
050   \}
\end{alltt}
\end{small}

-- Add comments later, Tom.

\section*{Exercises}
\begin{tabular}{cl}
$\left [ 3 \right ] $ & Devise an algorithm that performs $a \cdot 2^b$ for generic values of $b$ \\
                      & in $O(n)$ time. \\
                      &\\
$\left [ 3 \right ] $ & Devise an efficient algorithm to multiply by small low hamming  \\
                      & weight values such as $3$, $5$ and $9$.  Extend it to handle all values \\
                      & upto $64$ with a hamming weight less than three. \\
                      &\\
$\left [ 2 \right ] $ & Modify the preceding algorithm to handle values of the form \\
                      & $2^k - 1$ as well. \\
                      &\\
$\left [ 3 \right ] $ & Using only algorithms mp\_mul\_2, mp\_div\_2 and mp\_add create an \\
                      & algorithm to multiply two integers in roughly $O(2n^2)$ time for \\
                      & any $n$-bit input.  Note that the time of addition is ignored in the \\
                      & calculation.  \\
                      & \\
$\left [ 5 \right ] $ & Improve the previous algorithm to have a working time of at most \\
                      & $O \left (2^{(k-1)}n + \left ({2n^2 \over k} \right ) \right )$ for an appropriate choice of $k$.  Again ignore \\
                      & the cost of addition. \\
                      & \\
$\left [ 1 \right ] $ & There exists an improvement on the previous algorithm to \\
                      & slightly reduce the number of additions required.  Modify the \\
                      & previous algorithm to include this improvement. \\
                      & \\
$\left [ 2 \right ] $ & Devise a chart to find optimal values of $k$ for the previous problem \\
                      & for $n = 64 \ldots 1024$ in steps of $64$. \\
                      & \\
$\left [ 2 \right ] $ & Using only algorithms mp\_abs and mp\_sub devise another method for \\
                      & calculating the result of a signed comparison. \\
                      &
\end{tabular}

\chapter{Multiplication and Squaring}
\section{The Multipliers}
For most number theoretic systems including public key cryptographic algorithms the set of algorithms collectively known as the
``multipliers'' form the most important subset of algorithms of any multiple precision integer package.  The set of multipliers 
include multiplication, squaring and modular reduction algorithms.  

The importance of these algorithms is driven by the fact that most popular public key algorithms are based on modular 
exponentiation.  That is performing $d \equiv a^b \mbox{ (mod }c\mbox{)}$ for some arbitrary choice of $a$, $b$, $c$ and $d$.  Roughly
speaking the a modular exponentiation will spend about 40\% of the time in modular reductions, 35\% of the time in squaring and 25\% of
the time in multiplications.  Only a small trivial amount of time is spent on lower level algorithms such as mp\_clamp, mp\_init, etc...

This chapter will discuss only two of the multipliers algorithms, multiplication and squaring.  As will be discussed shortly very efficient
multiplier algorithms are not always straightforward and deserve a lot of attention.

\section{Multiplication}
\subsection{The Baseline Multiplication}
\index{baseline multiplication}
Computing the product of two integers in software can be achieved using a trivial adaptation of the standard $O(n^2)$ long-hand multiplication
algorithm school children are taught.  The ``baseline multiplication'' algorithm is designed to act as the ``catch-all'' algorithm only called
when the faster algorithms cannot be used.  This algorithm does not use any particularly interesting optimizations.

The first algorithm to review is the unsigned multiplication algorithm from which a signed multiplication algorithm can be established.  One important 
facet of this algorithm to note is that it has been modified to only produce a certain amount of output digits as resolution.  Recall that for
a $n$ and $m$ digit input the product will be at most $n + m + 1$ digits.  Therefore, this algorithm can be reduced to a full multiplier by
telling it to produce $n + m + 1$ digits.  

Recall from sub-section 5.2.2 the definition of $\gamma$ as the number of bits in the type \textbf{mp\_digit}.  We shall now extend this variable set to 
include $\alpha$ which shall represent the number of bits in the type \textbf{mp\_word}.  This implies that $2^{\alpha} > 2 \cdot \beta^2$.  The 
constant $\delta = 2^{\alpha - 2lg(\beta)}$ will represent the maximal weight of any column in a product (\textit{see sub-section 6.2.2 for more information}).

\newpage\begin{figure}[!here]
\begin{small}
\begin{center}
\begin{tabular}{l}
\hline Algorithm \textbf{s\_mp\_mul\_digs}. \\
\textbf{Input}.   mp\_int $a$, mp\_int $b$ and an integer $digs$ \\
\textbf{Output}.  $c \leftarrow \vert a \vert \cdot \vert b \vert \mbox{ (mod }\beta^{digs}\mbox{)}$. \\
\hline \\
1.  If min$(a.used, b.used) < \delta$ then do \\
\hspace{3mm}1.1  Calculate $c = \vert a \vert \cdot \vert b \vert$ by the Comba method.  \\
\hspace{3mm}1.2  Return the result of step 1.1 \\
\\
Allocate and initialize a temporary mp\_int. \\
2.  Init $t$ to be of size $digs$ \\
3.  If step 2 failed return(\textit{MP\_MEM}). \\
4.  $t.used \leftarrow digs$ \\
\\
Compute the product. \\
5.  for $ix$ from $0$ to $a.used - 1$ do \\
\hspace{3mm}5.1  $u \leftarrow 0$ \\
\hspace{3mm}5.2  $pb \leftarrow \mbox{min}(b.used, digs - ix)$ \\
\hspace{3mm}5.3  If $pb < 1$ then goto step 6. \\
\hspace{3mm}5.4  for $iy$ from $0$ to $pb - 1$ do \\
\hspace{6mm}5.4.1  $\hat r \leftarrow t_{iy + ix} + a_{ix} \cdot b_{iy} + u$ \\
\hspace{6mm}5.4.2  $t_{iy + ix} \leftarrow \hat r \mbox{ (mod }\beta\mbox{)}$ \\
\hspace{6mm}5.4.3  $u \leftarrow \lfloor \hat r / \beta \rfloor$ \\
\hspace{3mm}5.5  if $ix + iy < digs$ then do \\
\hspace{6mm}5.5.1  $t_{ix + pb} \leftarrow u$ \\
6.  Clamp excess digits of $t$. \\
7.  Swap $c$ with $t$ \\
8.  Clear $t$ \\
9.  Return(\textit{MP\_OKAY}). \\
\hline
\end{tabular}
\end{center}
\end{small}
\caption{Algorithm s\_mp\_mul\_digs}
\end{figure}

\textbf{Algorithm s\_mp\_mul\_digs.}
This algorithm computes the unsigned product of two inputs $a$ and $c$ limited to an output precision of $digs$ digits.  While it may seem
a bit awkward to modify the function from its simple $O(n^2)$ description the usefulness of partial multipliers will arise in a future 
algorithm.  The algorithm is loosely based on algorithm 14.12 from \cite[pp. 595]{HAC} and is similar to Algorithm M \cite[pp. 268]{TAOCPV2}.  The
algorithm differs from those cited references because it can produce a variable output precision regardless of the precision of the inputs.

The first thing this algorithm checks for is whether a Comba multiplier can be used instead.   That is if the minimal digit count of either
input is less than $\delta$ the Comba method is used.    After the Comba method is ruled out the baseline algorithm begins.  A 
temporary mp\_int variable $t$ is used to hold the intermediate result of the product.  This allows the algorithm to be used to 
compute products when either $a = c$ or $b = c$ without overwriting the inputs.  

All of step 5 is the infamous $O(n^2)$ multiplication loop slightly modified to only produce upto $digs$ digits of output.  The $pb$ variable
is given the count of digits to read from $b$ inside the nested loop.  If $pb < 0$ then no more output digits can be produced and the algorithm
will exit the loop.  The best way to think of the loops are as a series of $pb \times 1$ multiplication.    That is, in each pass of the 
innermost loop $a_{ix}$ is multiplied against $b$ and the result is added (\textit{with an appropriate shift}) to $t$.  

For example, consider multiplying $576$ by $241$.  That is equivalent to computing $10^0(1)(576) + 10^1(4)(576) + 10^2(2)(576)$ which is best
visualized as the following table.

\begin{figure}[here]
\begin{center}
\begin{tabular}{|c|c|c|c|c|c|c|}
\hline   &&          & 5 & 7 & 6 & \\
\hline   $\times$&&  & 2 & 4 & 1 & \\
\hline &&&&&&\\
  &&          & 5 & 7 & 6 & $10^0(1)(576)$ \\
  &2 &   3    & 0 & 4 & 0 & $10^1(4)(576)$ \\
  1 & 1 & 5 & 2 & 0 & 0 &  $10^2(2)(576)$ \\
\hline  
\end{tabular}
\end{center}
\caption{Long-Hand Multiplication Diagram}
\end{figure}

Each row of the product is added to the result after being shifted to the left (\textit{multiplied by a power of the radix}) by the appropriate 
count.  That is in pass $ix$ of the inner loop the product is added starting at the $ix$'th digit of the reult.

Step 5.4.1 introduces the hat symbol (\textit{e.g. $\hat x$}) which represents a double precision variable.  The multiplication on that step
is assumed to be a double wide output single precision multiplication.  That is, two single precision variables are multiplied to produce a
double precision result.  The step is somewhat optimized from a long-hand multiplication algorithm because the carry from the addition in step
5.4.1 is forwarded through the nested loop.  If the carry was ignored it would overflow the single precision digit $t_{ix+iy}$ and the result
would be lost.  

At step 5.5 the nested loop is finished and any carry that was left over should be forwarded.  That is provided $ix + iy < digs$ otherwise the
carry is ignored since it will not be part of the result anyways.  

\index{bn\_s\_mp\_mul\_digs.c}
\vspace{+3mm}\begin{small}
\hspace{-5.1mm}{\bf File}: bn\_s\_mp\_mul\_digs.c
\vspace{-3mm}
\begin{alltt}
016   
017   /* multiplies |a| * |b| and only computes upto digs digits of result
018    * HAC pp. 595, Algorithm 14.12  Modified so you can control how 
019    * many digits of output are created.
020    */
021   int
022   s_mp_mul_digs (mp_int * a, mp_int * b, mp_int * c, int digs)
023   \{
024     mp_int  t;
025     int     res, pa, pb, ix, iy;
026     mp_digit u;
027     mp_word r;
028     mp_digit tmpx, *tmpt, *tmpy;
029   
030     /* can we use the fast multiplier? */
031     if (((digs) < MP_WARRAY) &&
032         MIN (a->used, b->used) < 
033             (1 << ((CHAR_BIT * sizeof (mp_word)) - (2 * DIGIT_BIT)))) \{
034       return fast_s_mp_mul_digs (a, b, c, digs);
035     \}
036   
037     if ((res = mp_init_size (&t, digs)) != MP_OKAY) \{
038       return res;
039     \}
040     t.used = digs;
041   
042     /* compute the digits of the product directly */
043     pa = a->used;
044     for (ix = 0; ix < pa; ix++) \{
045       /* set the carry to zero */
046       u = 0;
047   
048       /* limit ourselves to making digs digits of output */
049       pb = MIN (b->used, digs - ix);
050   
051       /* setup some aliases */
052       /* copy of the digit from a used within the nested loop */
053       tmpx = a->dp[ix];
054       
055       /* an alias for the destination shifted ix places */
056       tmpt = t.dp + ix;
057       
058       /* an alias for the digits of b */
059       tmpy = b->dp;
060   
061       /* compute the columns of the output and propagate the carry */
062       for (iy = 0; iy < pb; iy++) \{
063         /* compute the column as a mp_word */
064         r = ((mp_word) *tmpt) + 
065             ((mp_word) tmpx) * ((mp_word) * tmpy++) + 
066             ((mp_word) u);
067   
068         /* the new column is the lower part of the result */
069         *tmpt++ = (mp_digit) (r & ((mp_word) MP_MASK));
070   
071         /* get the carry word from the result */
072         u = (mp_digit) (r >> ((mp_word) DIGIT_BIT));
073       \}
074       /* set carry if it is placed below digs */
075       if (ix + iy < digs) \{
076         *tmpt = u;
077       \}
078     \}
079   
080     mp_clamp (&t);
081     mp_exch (&t, c);
082   
083     mp_clear (&t);
084     return MP_OKAY;
085   \}
\end{alltt}
\end{small}

Lines 31 to 35 determine if the Comba method can be used first.  The conditions for using the Comba routine are that min$(a.used, b.used) < \delta$ and
the number of digits of output is less than \textbf{MP\_WARRAY}.  This new constant is used to control the stack usage in the Comba routines.  By
default it is set to $\delta$ but can be reduced when memory is at a premium.

Of particular importance is the calculation of the $ix+iy$'th column on lines 64, 65 and 66.  Note how all of the
variables are cast to the type \textbf{mp\_word}.  That is to ensure that double precision operations are used instead of single precision.  The
multiplication on line 65 is a bit of a GCC optimization.  On the outset it looks like the compiler will have to use a double precision
multiplication to produce the result required.  Such an operation would be horribly slow on most processors and drag this to a crawl.  However,
GCC is smart enough to realize that double wide output single precision multipliers can be used.  For example, the instruction ``MUL'' on the
x86 processor can multiply two 32-bit values and produce a 64-bit result.  

\subsection{Faster Multiplication by the ``Comba'' Method}

One of the huge drawbacks of the ``baseline'' algorithms is that at the $O(n^2)$ level the carry must be computed and propagated upwards.  This
makes the nested loop very sequential and hard to unroll and implement in parallel.  The ``Comba'' method is named after little known 
(\textit{in cryptographic venues}) Paul G. Comba where in \cite{COMBA} a method of implementing fast multipliers that do not require nested 
carry fixup operations was presented.

At the heart of algorithm is once again the long-hand algorithm for multiplication.  Except in this case a slight twist is placed on how
the columns of the result are produced.  In the standard long-hand algorithm rows of products are produced then added together to form the 
final result.  In the baseline algorithm the columns are added together to get the result instantaneously.  

In the Comba algorithm however, the columns of the result are produced entirely independently of each other.  That is at the $O(n^2)$ level a 
simple multiplication and addition step is performed.  Or more succintly that 

\begin{equation}
x_n = \sum_{i+j = n} a_ib_j
\end{equation}

Where $x_n$ is the $n'th$ column of the output vector.  To see how this works consider once again multiplying $576$ by $241$.  

\begin{figure}[here]
\begin{small}
\begin{center}
\begin{tabular}{|c|c|c|c|c|c|}
  \hline &          & 5 & 7 & 6 & First Input\\
  \hline $\times$ & & 2 & 4 & 1 & Second Input\\
\hline            &                        & $1 \cdot 5 = 5$   & $1 \cdot 7 = 7$   & $1 \cdot 6 = 6$ & First pass \\
                  &  $4 \cdot 5 = 20$      & $4 \cdot 7+5=33$  & $4 \cdot 6+7=31$  & 6               & Second pass \\
   $2 \cdot 5 = 10$ &  $2 \cdot 7 + 20 = 34$ & $2 \cdot 6+33=45$ & 31                & 6             & Third pass \\
\hline 10 & 34 & 45 & 31 & 6 & Final Result \\   
\hline   
\end{tabular}
\end{center}
\end{small}
\caption{Comba Multiplication Diagram}
\end{figure}

At this point the vector $x = \left < 10, 34, 45, 31, 6 \right >$ is the result of the first step of the Comba multipler.  
Now the columns must be fixed by propagating the carry upwards.  The following trivial algorithm will accomplish this.

\begin{enumerate}
    \item for $n$ from 0 to $k - 1$ do
    \item \hspace{3mm} $x_{n+1} \leftarrow x_{n+1} + \lfloor x_{n}/\beta \rfloor$ 
    \item \hspace{3mm} $x_{n} \leftarrow x_{n} \mbox{ (mod }\beta\mbox{)}$
\end{enumerate}

With that algorithm and $k = 5$ and $\beta = 10$ the following vector is produced $y = \left < 1, 3, 8, 8, 1, 6 \right >$.  In this case 
$241 \cdot 576$ is in fact $138816$ and the procedure succeeded.  If the algorithm is correct and as will be demonstrated shortly more
efficient than the baseline algorithm why not simply always use this algorithm?

\subsubsection{Column Weight.}
At the nested $O(n^2)$ level the Comba method adds the product of two single precision variables to a each column of the output 
independently.  A serious obstacle is if the carry is lost due to lack of precision before the algorithm has a chance to fix
the carries.  For example, in the multiplication of two three-digit numbers the third column of output will be the sum of
three single precision multiplications.  If the precision of the accumulator for the output digits is less then $3 \cdot (\beta - 1)^2$ then
an overflow can occur and the carry information will be lost.  For any $m$ and $n$ digit input the maximal weight of any column is 
min$(m, n)$ which is fairly obvious.

The maximal number of terms in any column of a product is known as the ``column weight'' and strictly governs when the algorithm can be used.  Recall
from earlier that a double precision type has $\alpha$ bits of resolution and a single precision digit has $lg(\beta)$ bits of precision.  Given these
two quantities we may not violate the following

\begin{equation}
k \cdot \left (\beta - 1 \right )^2 < 2^{\alpha}
\end{equation}

Which reduces to 

\begin{equation}
k \cdot \left ( \beta^2 - 2\beta + 1 \right ) < 2^{\alpha}
\end{equation}

Let $\rho = lg(\beta)$ represent the number of bits in a single precision digit.  By further re-arrangement of the equation the final solution is
found.

\begin{equation}
k \cdot \left (2^{2\rho} - 2^{\rho + 1} + 1 \right ) < 2^{\alpha}
\end{equation}

The defaults for LibTomMath are $\beta = 2^{28}, \alpha = 2^{64}$ which simplies to $72057593501057025 \cdot k < 2^{64}$ which when divided out
result in $k < 257$.  This implies that the smallest input may not have more than $256$ digits if the Comba method is to be used in
this configuration.  This is quite satisfactory for most applications since $256$ digits would be allow for numbers in the range of $2^{7168}$ 
which is much larger than the typical $2^{100}$ to $2^{4000}$ range most public key cryptographic algorithms use.  

\newpage\begin{figure}[!here]
\begin{small}
\begin{center}
\begin{tabular}{l}
\hline Algorithm \textbf{fast\_s\_mp\_mul\_digs}. \\
\textbf{Input}.   mp\_int $a$, mp\_int $b$ and an integer $digs$ \\
\textbf{Output}.  $c \leftarrow \vert a \vert \cdot \vert b \vert \mbox{ (mod }\beta^{digs}\mbox{)}$. \\
\hline \\
Place an array of \textbf{MP\_WARRAY} double precision digits named $\hat W$ on the stack. \\
1.  If $c.alloc < digs$ then grow $c$ to $digs$ digits. (\textit{hint: use mp\_grow}) \\
2.  If step 1 failed return(\textit{MP\_MEM}).\\
\\
Zero the temporary array $\hat W$. \\
3.  for $n$ from $0$ to $digs - 1$ do \\
\hspace{3mm}3.1  $\hat W_n \leftarrow 0$ \\
\\
Compute the columns. \\
4.  for $ix$ from $0$ to $a.used - 1$ do \\
\hspace{3mm}4.1  $pb \leftarrow \mbox{min}(b.used, digs - ix)$ \\
\hspace{3mm}4.2  If $pb < 1$ then goto step 5. \\
\hspace{3mm}4.3  for $iy$ from $0$ to $pb - 1$ do \\
\hspace{6mm}4.3.1  $\hat W_{ix+iy} \leftarrow \hat W_{ix+iy} + a_{ix}b_{iy}$ \\
\\
Propagate the carries upwards. \\
5.  $oldused \leftarrow c.used$ \\
6.  $c.used \leftarrow digs$ \\
7.  If $digs > 1$ then do \\
\hspace{3mm}7.1.  for $ix$ from $1$ to $digs - 1$ do \\
\hspace{6mm}7.1.1  $\hat W_{ix} \leftarrow \hat W_{ix} + \lfloor \hat W_{ix-1} / \beta \rfloor$ \\
\hspace{6mm}7.1.2  $c_{ix - 1} \leftarrow \hat W_{ix - 1} \mbox{ (mod }\beta\mbox{)}$ \\
8.  else do \\
\hspace{3mm}8.1  $ix \leftarrow 0$ \\
9.  $c_{ix} \leftarrow \hat W_{ix} \mbox{ (mod }\beta\mbox{)}$ \\
\\
Zero excess digits. \\
10.  If $digs < oldused$ then do \\
\hspace{3mm}10.1  for $n$ from $digs$ to $oldused - 1$ do \\
\hspace{6mm}10.1.1  $c_n \leftarrow 0$ \\
11.  Clamp excessive digits of $c$.  (\textit{hint: use mp\_clamp}) \\
12.  Return(\textit{MP\_OKAY}). \\
\hline
\end{tabular}
\end{center}
\end{small}
\caption{Algorithm fast\_s\_mp\_mul\_digs}
\end{figure}

\textbf{Algorithm fast\_s\_mp\_mul\_digs.}
This algorithm performs the unsigned multiplication of $a$ and $b$ using the Comba method limited to $digs$ digits of precision.  The algorithm
essentially peforms the same calculation as algorithm s\_mp\_mul\_digs but much faster.

The array $\hat W$ is meant to be on the stack when the algorithm is used.  The size of the array does not change which is ideal.  Note also that 
unlike algorithm s\_mp\_mul\_digs no temporary mp\_int is required since the result is calculated in place in $\hat W$.  

The $O(n^2)$ loop on step four is where the Comba method starts to show through.  First there is no carry variable in the loop.  Second the
double precision multiply and add step does not have a carry fixup of any sort.  In fact the nested loop is very simple and can be implemented
in parallel.  

What makes the Comba method so attractive is that the carry propagation only takes place outside the $O(n^2)$ nested loop.  For example, if the 
cost in terms of time of a multiply and add is $p$ and the cost of a carry propagation is $q$ then a baseline multiplication would require 
$O \left ((p + q)n^2 \right )$ time to multiply two $n$-digit numbers.  The Comba method only requires $pn^2 + qn$ time, however, in practice 
the speed increase is actually much more.  With $O(n)$ space the algorithm can be reduced to $O(pn + qn)$ time by implementing the $n$ multiply
and add operations in the nested loop in parallel.  

The carry propagation loop on step 7 is fairly straightforward.  It could have been written phased the other direction, that is, to assign
to $c_{ix}$ instead of $c_{ix-1}$ in each iteration.  However, it would still require pre-caution to make sure that $\hat W_{ix+1}$ is not beyond
the \textbf{MP\_WARRAY} words set aside.  

\index{bn\_fast\_s\_mp\_mul\_digs.c}
\vspace{+3mm}\begin{small}
\hspace{-5.1mm}{\bf File}: bn\_fast\_s\_mp\_mul\_digs.c
\vspace{-3mm}
\begin{alltt}
016   
017   /* Fast (comba) multiplier
018    *
019    * This is the fast column-array [comba] multiplier.  It is 
020    * designed to compute the columns of the product first 
021    * then handle the carries afterwards.  This has the effect 
022    * of making the nested loops that compute the columns very
023    * simple and schedulable on super-scalar processors.
024    *
025    * This has been modified to produce a variable number of 
026    * digits of output so if say only a half-product is required 
027    * you don't have to compute the upper half (a feature 
028    * required for fast Barrett reduction).
029    *
030    * Based on Algorithm 14.12 on pp.595 of HAC.
031    *
032    */
033   int
034   fast_s_mp_mul_digs (mp_int * a, mp_int * b, mp_int * c, int digs)
035   \{
036     int     olduse, res, pa, ix;
037     mp_word W[MP_WARRAY];
038   
039     /* grow the destination as required */
040     if (c->alloc < digs) \{
041       if ((res = mp_grow (c, digs)) != MP_OKAY) \{
042         return res;
043       \}
044     \}
045   
046     /* clear temp buf (the columns) */
047     memset (W, 0, sizeof (mp_word) * digs);
048   
049     /* calculate the columns */
050     pa = a->used;
051     for (ix = 0; ix < pa; ix++) \{
052       /* this multiplier has been modified to allow you to 
053        * control how many digits of output are produced.  
054        * So at most we want to make upto "digs" digits of output.
055        *
056        * this adds products to distinct columns (at ix+iy) of W
057        * note that each step through the loop is not dependent on
058        * the previous which means the compiler can easily unroll
059        * the loop without scheduling problems
060        */
061       \{
062         register mp_digit tmpx, *tmpy;
063         register mp_word *_W;
064         register int iy, pb;
065   
066         /* alias for the the word on the left e.g. A[ix] * A[iy] */
067         tmpx = a->dp[ix];
068   
069         /* alias for the right side */
070         tmpy = b->dp;
071   
072         /* alias for the columns, each step through the loop adds a new
073            term to each column
074          */
075         _W = W + ix;
076   
077         /* the number of digits is limited by their placement.  E.g.
078            we avoid multiplying digits that will end up above the # of
079            digits of precision requested
080          */
081         pb = MIN (b->used, digs - ix);
082   
083         for (iy = 0; iy < pb; iy++) \{
084           *_W++ += ((mp_word) tmpx) * ((mp_word) * tmpy++);
085         \}
086       \}
087   
088     \}
089   
090     /* setup dest */
091     olduse = c->used;
092     c->used = digs;
093   
094     \{
095       register mp_digit *tmpc;
096   
097       /* At this point W[] contains the sums of each column.  To get the
098        * correct result we must take the extra bits from each column and
099        * carry them down
100        *
101        * Note that while this adds extra code to the multiplier it 
102        * saves time since the carry propagation is removed from the 
103        * above nested loop.This has the effect of reducing the work 
104        * from N*(N+N*c)==N**2 + c*N**2 to N**2 + N*c where c is the 
105        * cost of the shifting.  On very small numbers this is slower 
106        * but on most cryptographic size numbers it is faster.
107        */
108       tmpc = c->dp;
109       for (ix = 1; ix < digs; ix++) \{
110         W[ix] += (W[ix - 1] >> ((mp_word) DIGIT_BIT));
111         *tmpc++ = (mp_digit) (W[ix - 1] & ((mp_word) MP_MASK));
112       \}
113       *tmpc++ = (mp_digit) (W[digs - 1] & ((mp_word) MP_MASK));
114   
115       /* clear unused */
116       for (; ix < olduse; ix++) \{
117         *tmpc++ = 0;
118       \}
119     \}
120   
121     mp_clamp (c);
122     return MP_OKAY;
123   \}
\end{alltt}
\end{small}

The memset on line 47 clears the initial $\hat W$ array to zero in a single step. Like the slower baseline multiplication
implementation a series of aliases (\textit{lines 67, 70 and 75}) are used to simplify the inner $O(n^2)$ loop.  
In this case a new alias $\_\hat W$ has been added which refers to the double precision columns offset by $ix$ in each pass.  

The inner loop on line 84 is where the algorithm will spend the majority of the time.  Which is why it has been stripped to the 
bones of any extra baggage\footnote{Hence the pointer aliases.}.  On x86 processors the multiply and add amounts to at the very least five
instructions (\textit{two loads, two additions, one multiply}) while on the ARMv4 processors it amounts to only three (\textit{one load, one store,
one multiply-add}).   On both the x86 and ARMv4 processors GCC v3.2 does a very good job at unrolling the loop and scheduling it so there 
are very few dependency stalls.

In theory the difference between the baseline and comba algorithms is a mere $O(qn)$ time difference.  However, in the $O(n^2)$ nested loop of the
baseline method there are dependency stalls as the algorithm must wait for the multiplier to finish before propagating the carry to the next 
digit.  As a result fewer of the often multiple execution units\footnote{The AMD Athlon has three execution units and the Intel P4 has four.} can
be simultaneously used.  

\subsection{Multiplication at New Bounds by Karatsuba Method}
So far two methods of multiplication have been presented.  Both of the algorithms require asymptotically $O(n^2)$ time to multiply two $n$-digit 
numbers together.  While the Comba method is much faster than the baseline algorithm it still requires far too much time to multiply 
large inputs together.  In fact it was not until \cite{KARA} in 1962 that a faster algorithm had been proposed at all.

The idea behind Karatsubas method is that an input can be represented in polynomial basis as two halves then multiplied.  For example, if 
$f(x) = ax + b$ and $g(x) = cx + b$ then the product of the two polynomials $h(x) = f(x)g(x)$ will allow $h(\beta) = (f(\beta))(g(\beta))$.  

So how does this help?  First expand the product $h(x)$.

\begin{center}
\begin{tabular}{rcl}
$h(x)$ & $=$ & $f(x)g(x)$ \\
       & $=$ & $(ax + b)(cx + d)$ \\
       & $=$ & $acx^2 + adx + bcx + bd$ \\
\end{tabular}
\end{center}

The next equation is a bit of genius on the part of Karatsuba.  He proved that the previous equation is equivalent to 

\begin{equation}
h(x) = acx^2 + ((a - c)(b - d) + bd + ac)x + bd
\end{equation}

Essentially the proof lies in some fairly light algebraic number theory (\textit{see \cite{KARAP} for details}) that is not important for
the discussion.  At first glance it appears that the Karatsuba method is actually harder than the straight $O(n^2)$ approach.  
However, further investigation will prove otherwise.  

The first important observation is that both $f(x)$ and $g(x)$ are the polynomial basis representation of two-digit numbers.  This means that 
$\left < a, b, c, d \right >$ are single digit values.  Using either the baseline or straight polynomial multiplication the old method requires
$O \left (4(n/2)^2 \right ) = O(n^2)$ single precision multiplications.  Looking closer at Karatsubas equation there are only three unique multiplications 
required which are $ac$, $bd$ and $(a - c)(b - d)$.  As a result only $O \left (3 \cdot (n/2)^2 \right ) = O \left ( {3 \over 4}n^2 \right )$ 
multiplications are required.  

So far the algorithm has been discussed from the point of view of ``two-digit'' numbers.  However, there is no reason why two digits implies a range of 
$\beta^2$.  It could just as easily represent a range of $\left (\beta^k \right)^2$ as well.  For example, the polynomial 
$f(x) = a_3x^3 + a_2x^2 + a_1x + a_0$ could also be written as $f'(x) = a'_1x + a'_0$ where $f(\beta) = f'(\beta^2)$.  Fortunately representing an
integer which is already in an array of radix-$\beta$ digits in polynomial basis in terms of a power of $\beta$ is very simple.  

\subsubsection{Recursion}
The Karatsuba multiplication algorithm can be applied to practically any size of input.  Therefore, it is possible that the Karatsuba method itself
be used for the three multiplications required.  For example, when multiplying two four-digit numbers there will be three multiplications of two-digit
numbers.  In this case the smaller multiplication requires $p(n) = {3 \over 4}n^2$ time to complete while the larger multiplication requires
$q(n) = 3 \cdot p(n/2)$ multiplications.  

By expanding $q(n)$ the following equation is achieved. 

\begin{center}
\begin{tabular}{rcl}
$q(n)$ & $=$ & $3 \cdot p(n/2)$ \\
       & $=$ & $3 \cdot (3 \cdot ((n/2)/2)^2)$ \\
       & $=$ & $9 \cdot (n/4)^2$ \\
       & $=$ & ${9 \over 16}n^2$ \\
\end{tabular}
\end{center}

The generic expression for the multiplicand is simply $\left ( {3 \over 4} \right )^k$ for $k \ge 1$ recurisions.  The maximal number of recursions
is approximately $lg(n)$.  Putting this all in terms of a base $n$ logarithm the asymptotic running time can be deduced.

\begin{center}
\begin{tabular}{rcl}
$lg_n \left ( \left ( {3 \over 4} \right )^{lg_2 n} \cdot n^2 \right )$ & $=$ & $lg_2 n \cdot lg_n \left ( { 3 \over 4 } \right ) + 2$ \\
                                                                        & $=$ & $\left ( {log N \over log 2} \right ) \cdot \left ( {log \left ( {3 \over 4} \right ) \over log N } \right ) + 2$ \\
                                                                        & $=$ & ${ log 3 - log 2^2 + 2 \cdot log 2} \over log 2$ \\
                                                                        & $=$ & $log 3 \over log 2$ \\
\end{tabular}
\end{center}

Which leads to a running time of $O \left ( n^{lg(3)} \right )$ which is approximately $O(n^{1.584})$.  This can lead to 
impressive savings with fairly moderate sized numbers.  For example, when multiplying two 128-digit numbers the Karatsuba 
method saves $14,197$ (\textit{or $86\%$ of the total}) single precision multiplications.  

The immediate question becomes why not simply use Karatsuba multiplication all the time and forget about the baseline and Comba algorithms? 

\subsubsection{Overhead}
While the Karatsuba method saves on the number of single precision multiplications required this savings is not entirely free.  The product
of three half size products must be stored somewhere as well as four additions and two subtractions performed.  These operations incur sufficient
overhead that often for fairly trivial sized inputs the Karatsuba method is slower.

\index{cutoff point}
The \textit{cutoff point} for Karatsuba multiplication is the point at which the Karatsuba multiplication and baseline (\textit{or Comba}) meet.  
For the purposes of this discussion call this value $x$.  For any input with $n$ digits such that $n < x$ Karatsuba multiplication will be slower 
and for $n > x$ it will be faster.  Often the break between the two algorithms is not so clean cut in reality.  The cleaner the cut the more 
efficient multiplication will be which is why tuning the multiplication is a very important process.  For example, a properly tuned Karatsuba 
multiplication algorithm can multiply two $4,096$ bit numbers up to five times faster on an Athlon processor compared to the standard baseline
algorithm.  

The exact placement of the value of $x$ depends on several key factors.   The cost of allocating storage for the temporary variables, the cost of 
performing the additions and most importantly the cost of performing a single precision multiplication.  With a processor where single precision 
multiplication is fast\footnote{The AMD Athlon for instance has a six cycle multiplier compared to the Intel P4 which has a 15 cycle multiplier.} the 
cutoff point will move upwards.  Similarly with a slower processor the cutoff point will move downwards.  

\newpage\begin{figure}[!here]
\begin{small}
\begin{center}
\begin{tabular}{l}
\hline Algorithm \textbf{mp\_karatsuba\_mul}. \\
\textbf{Input}.   mp\_int $a$ and mp\_int $b$ \\
\textbf{Output}.  $c \leftarrow \vert a \vert \cdot \vert b \vert$ \\
\hline \\
1.  $B \leftarrow \mbox{min}(a.used, b.used)/2$ \\
2.  Init the following mp\_int variables: $x0$, $x1$, $y0$, $y1$, $t1$, $x0y0$, $x1y1$.\\
3.  If step 2 failed then return(\textit{MP\_MEM}). \\
\\
Split the input.  e.g. $a = x1 \cdot \beta^B + x0$ \\
4.  $x0 \leftarrow a \mbox{ (mod }\beta^B\mbox{)}$ (\textit{hint: use mp\_mod\_2d}) \\
5.  $y0 \leftarrow b \mbox{ (mod }\beta^B\mbox{)}$ \\
6.  $x1 \leftarrow \lfloor a / \beta^B \rfloor$ (\textit{hint: use mp\_rshd}) \\
7.  $y1 \leftarrow \lfloor b / \beta^B \rfloor$ \\
\\
Calculate the three products. \\
8.  $x0y0 \leftarrow x0 \cdot y0$ (\textit{hint: use mp\_mul}) \\
9.  $x1y1 \leftarrow x1 \cdot y1$ \\
10.  $t1 \leftarrow x1 - x0$ (\textit{hint: use mp\_sub}) \\
11.  $x0 \leftarrow y1 - y0$ \\
12.  $t1 \leftarrow t1 \cdot x0$ \\
\\
Calculate the middle term. \\
13.  $x0 \leftarrow x0y0 + x1y1$ \\
14.  $t1 \leftarrow x0 - t1$ \\
\\
Calculate the final product. \\
15.  $t1 \leftarrow t1 \cdot \beta^B$ (\textit{hint: use mp\_lshd}) \\
16.  $x1y1 \leftarrow x1y1 \cdot \beta^{2B}$ \\
17.  $t1 \leftarrow x0y0 + t1$ \\
18.  $c \leftarrow t1 + x1y1$ \\
19.  Clear all of the temporary variables. \\
20.  Return(\textit{MP\_OKAY}).\\
\hline 
\end{tabular}
\end{center}
\end{small}
\caption{Algorithm mp\_karatsuba\_mul}
\end{figure}

\textbf{Algorithm mp\_karatsuba\_mul.}


\section{Squaring}
\subsection{The Baseline Squaring Algorithm}
\subsection{Faster Squaring by the ``Comba'' Method}
\subsection{Karatsuba Squaring}
\section{Tuning Algorithms}
\subsection{How to Tune Karatsuba Algorithms}

\chapter{Modular Reductions}
\section{Basics of Modular Reduction}
\section{The Barrett Reduction}
\section{The Montgomery Reduction}
\subsection{Faster ``Comba'' Montgomery Reduction}
\subsection{Example Montgomery Algorithms}
\section{The Diminished Radix Algorithm}
\section{Algorithm Comparison}

\chapter{Exponentiation}
\section{Single Digit Exponentiation}
\section{Modular Exponentiation}
\subsection{General Case}
\subsection{Odd or Diminished Radix Moduli}
\section{Quick Power of Two}

\chapter{Higher Level Algorithms}
\section{Integer Division with Remainder}
\section{Single Digit Helpers}
\subsection{Single Digit Addition}
\subsection{Single Digit Subtraction}
\subsection{Single Digit Multiplication}
\subsection{Single Digit Division}
\subsection{Single Digit Modulo}
\subsection{Single Digit Root Extraction}
\section{Random Number Generation}
\section{Formatted Output}
\subsection{Getting The Output Size}
\subsection{Generating Radix-n Output}
\subsection{Reading Radix-n Input}
\section{Unformatted Output}
\subsection{Getting The Output Size}
\subsection{Generating Output}
\subsection{Reading Input}

\chapter{Number Theoretic Algorithms}
\section{Greatest Common Divisor}
\section{Least Common Multiple}
\section{Jacobi Symbol Computation}
\section{Modular Inverse}
\subsection{General Case}
\subsection{Odd Moduli}
\section{Primality Tests}
\subsection{Trial Division}
\subsection{The Fermat Test}
\subsection{The Miller-Rabin Test}
\subsection{Primality Test in a Bottle}
\subsection{The Next Prime}
\section{Root Extraction}

\backmatter
\appendix
\begin{thebibliography}{ABCDEF}
\bibitem[1]{TAOCPV2}
Donald Knuth, \textit{The Art of Computer Programming}, Third Edition, Volume Two, Seminumerical Algorithms, Addison-Wesley, 1998

\bibitem[2]{HAC}
A. Menezes, P. van Oorschot, S. Vanstone, \textit{Handbook of Applied Cryptography}, CRC Press, 1996

\bibitem[3]{ROSE}
Michael Rosing, \textit{Implementing Elliptic Curve Cryptography}, Manning Publications, 1999

\bibitem[4]{COMBA}
Paul G. Comba, \textit{Exponentiation Cryptosystems on the IBM PC}. IBM Systems Journal 29(4): 526-538 (1990)

\bibitem[5]{KARA}
A. Karatsuba, Doklay Akad. Nauk SSSR 145 (1962), pp.293-294

\bibitem[6]{KARAP}
Andre Weimerskirch and Christof Paar, \textit{Generalizations of the Karatsuba Algorithm for Polynomial Multiplication}, Submitted to Design, Codes and Cryptography, March 2002

\end{thebibliography}

\documentclass[b5paper]{book}
\usepackage{makeidx}
\usepackage{amssymb}
\usepackage{color}
\usepackage{alltt}
\usepackage{graphicx}
\usepackage{layout}
\def\union{\cup}
\def\intersect{\cap}
\def\getsrandom{\stackrel{\rm R}{\gets}}
\def\cross{\times}
\def\cat{\hspace{0.5em} \| \hspace{0.5em}}
\def\catn{$\|$}
\def\divides{\hspace{0.3em} | \hspace{0.3em}}
\def\nequiv{\not\equiv}
\def\approx{\raisebox{0.2ex}{\mbox{\small $\sim$}}}
\def\lcm{{\rm lcm}}
\def\gcd{{\rm gcd}}
\def\log{{\rm log}}
\def\ord{{\rm ord}}
\def\abs{{\mathit abs}}
\def\rep{{\mathit rep}}
\def\mod{{\mathit\ mod\ }}
\renewcommand{\pmod}[1]{\ ({\rm mod\ }{#1})}
\newcommand{\floor}[1]{\left\lfloor{#1}\right\rfloor}
\newcommand{\ceil}[1]{\left\lceil{#1}\right\rceil}
\def\Or{{\rm\ or\ }}
\def\And{{\rm\ and\ }}
\def\iff{\hspace{1em}\Longleftrightarrow\hspace{1em}}
\def\implies{\Rightarrow}
\def\undefined{{\rm ``undefined"}}
\def\Proof{\vspace{1ex}\noindent {\bf Proof:}\hspace{1em}}
\let\oldphi\phi
\def\phi{\varphi}
\def\Pr{{\rm Pr}}
\newcommand{\str}[1]{{\mathbf{#1}}}
\def\F{{\mathbb F}}
\def\N{{\mathbb N}}
\def\Z{{\mathbb Z}}
\def\R{{\mathbb R}}
\def\C{{\mathbb C}}
\def\Q{{\mathbb Q}}
\definecolor{DGray}{gray}{0.5}
\newcommand{\url}[1]{\mbox{$<${#1}$>$}}
\newcommand{\emailaddr}[1]{\mbox{$<${#1}$>$}}
\def\twiddle{\raisebox{0.3ex}{\mbox{\tiny $\sim$}}}
\def\gap{\vspace{0.5ex}}
\makeindex
\begin{document}
\frontmatter
\pagestyle{empty}
\title{Multiple-Precision Integer Arithmetic, \\ A Case Study Involving the LibTomMath Project \\ - DRAFT - }
\author{\mbox{
%\begin{small}
\begin{tabular}{c}
Tom St Denis \\
Algonquin College \\
\\
Mads Rasmussen \\
Open Communications Security \\
\\
Gregory Rose \\
Qualcomm \\
\end{tabular}
%\end{small}
}
}
\maketitle
This text in its entirety is copyrighted \copyright{}2003 by Tom St Denis.  It may not be redistributed 
electronically or otherwise without the sole permission of the author.  The text is freely re distributable as long as
it is packaged along with the LibTomMath project in a non-commercial project.  Contact the
author for other redistribution rights.

This text corresponds to the v0.17 release of the LibTomMath project.

\begin{alltt}
Tom St Denis
111 Banning Rd
Ottawa, Ontario
K2L 1C3
Canada

Phone: 1-613-836-3160
Email: tomstdenis@iahu.ca
\end{alltt}

This text is formatted to the international B5 paper size of 176mm wide by 250mm tall using the \LaTeX{} 
{\em book} macro package and the Perl {\em booker} package.

\tableofcontents
\listoffigures
\chapter*{Preface}
Blah.

\mainmatter
\pagestyle{headings}
\chapter{Introduction}
\section{Multiple Precision Arithmetic}
\subsection{The Need for Multiple Precision Arithmetic}
The most prevalent use for multiple precision arithmetic (\textit{often referred to as bignum math}) is within public
key cryptography.   Algorithms such as RSA, Diffie-Hellman and Elliptic Curve Cryptography require large integers in order to 
resist known cryptanalytic attacks.  Typical modern programming languages such as C and Java only provide small 
single-precision data types which are incapable of precisely representing integers which are often hundreds of bits long.

For example, consider multiplying $1,234,567$ by $9,876,543$ in C with an ``unsigned long'' data type.  With an 
x86 machine the result is $4,136,875,833$ while the true result is $12,193,254,061,881$.  The original inputs 
were approximately $21$ and $24$ bits respectively.  If the C language cannot multiply two relatively small values 
together precisely how does anyone expect it to multiply two values which are considerably larger?

Most advancements in fast multiple precision arithmetic stems from the desire for faster cryptographic primitives.  However, cryptography
is not the only field of study that can benefit fast large integer routines.  Another auxiliary use for multiple precision integers is 
high precision floating point data types.  The basic IEEE standard floating point type is made up of an integer mantissa $q$ and an exponent $e$.  
Numbers are given in the form $n = q \cdot b^e$ where $b = 2$ is convention.  Since IEEE is meant to be implemented in 
hardware the precision of the mantissa is often fairly small (\textit{roughly 23 bits}).  Since the mantissa is merely an 
integer a large multiple precision integer could be used.  In effect very high precision floating point arithmetic 
could be performed.  This would be useful where scientific applications must minimize the total output error over long simulations.  

\subsection{Multiple Precision Arithmetic}
\index{multiple precision}
Multiple precision arithmetic attempts to the solve the shortcomings of single precision data types such as those from
the C and Java programming languages.  In essence multiple precision arithmetic is a set of operations that can be 
performed on members of an algebraic group whose precision is not fixed.  The algorithms when implemented to be multiple
precision can allow a developer to work with any practical precision required.

Typically the arithmetic is performed over the ring of integers denoted by a $\Z$ and referred to casually as ``bignum'' 
routines.  However, it is possible to have rings of polynomials as well typically denoted by $\Z/p\Z \left [ X \right ]$ 
which could have variable precision (\textit{or degree}).  This text will discuss implementation of the former, however,
implementing polynomial basis routines should be relatively easy after reading this text.

\subsection{Benefits of Multiple Precision Arithmetic}
\index{precision} \index{accuracy}
Precision is defined loosely as the proximity to the real value a given representation is.  Accuracy is defined as the 
reproducibility of the result.  For example, the calculation $1/3 = 0.25$ is imprecise but can be accurate provided 
it is reproducible.

The benefit of multiple precision representations over single precision representations is that 
often no precision is lost while representing the result of an operation which requires excess precision.  For example, 
the multiplication of two $n$-bit integers requires at least $2n$ bits to represent the result.  A multiple precision 
system would augment the precision of the destination to accomodate the result while a single precision system would
truncate excess bits to maintain a fixed level of precision.

Multiple precision representations allow for the precision to be very high (\textit{if not exacting}) but at a cost of
modest computer resources.  The only reasonable case where a multiple precision system will lose precision is when
emulating a floating point data type.  However, with multiple precision integer arithmetic no precision is lost.

\subsection{Basis of Operations}
At the heart of all multiple precision integer operations are the ``long-hand'' algorithms we all learnt as children 
in grade school.  For example, to multiply $1,234$ by $981$ the student is not taught to memorize the times table for 
$1,234$ instead they are taught how to long-multiply.  That is to multiply each column using simple single digit 
multiplications and add the resulting products by column.  The representation that most are familiar with is known as 
decimal or formally as radix-10. A radix-$n$ representation simply means there are $n$ possible values per digit.  
For example, binary would be a radix-2 representation.

In essence computer based multiple precision arithmetic is very much the same.  The most notable difference is the usage
of a binary friendly radix.  That is to use a radix of the form $2^k$ where $k$ is typically the size of a machine 
register.  Also occasionally more optimal algorithms are used to perform certain operations such as multiplication and 
squaring instead of traditional long-hand algorithms.

\section{Purpose of This Text}
The purpose of this text is to instruct the reader regarding how to implement multiple precision algorithms.  That is 
to not only explain the core theoretical algorithms but also the various ``house keeping'' tasks that are neglected by
authors of other texts on the subject.  Texts such as Knuths' ``The Art of Computer Programming, vol 2.'' and the 
Handbook of Applied Cryptography (\textit{HAC}) give considerably detailed explanations of the theoretical aspects of 
the algorithms and very little regarding the practical aspects.  

That is how an algorithm is explained and how it is actually implemented are two very different 
realities.  For example, algorithm 14.7 on page 594 of HAC lists a relatively simple algorithm for performing multiple 
precision integer addition.  However, what the description lacks is any discussion concerning the fact that the two 
integer inputs may be of differing magnitudes.  Similarly the division routine (\textit{Algorithm 14.20, pp. 598}) 
does not discuss how to handle sign or handle the dividends decreasing magnitude in the main loop (\textit{Step \#3}).

As well as the numerous practical oversights both of the texts do not discuss several key optimal algorithms required 
such as ``Comba'' and Karatsuba multipliers and fast modular inversion.  These optimal algorithms are considerably
vital to achieve any form of useful performance in non-trivial applications.  

To solve this problem the focus of this text is on the practical aspects of implementing the algorithms that 
constitute a multiple precision integer package with light cursory discussions on the theoretical aspects.  As a case 
study the ``LibTomMath''\footnote{Available freely at http://math.libtomcrypt.org} package is used to demonstrate 
algorithms with implementations that have been field tested and work very well.

\section{Discussion and Notation}
\subsection{Notation}
A multiple precision integer of $n$-digits shall be denoted as $x = (x_n ... x_1 x_0)_{ \beta }$ to be the 
multiple precision notation for the integer $x \equiv \sum_{i=0}^{n} x_i\beta^i$.  The elements of the array $x$ are
said to be the radix $\beta$ digits of the integer.  For example, $x = (15,0,7)_{\beta}$ would represent the 
integer $15\cdot\beta^2 + 0\cdot\beta^1 + 7\cdot\beta^0$.  

A ``mp\_int'' shall refer to a composite structure which contains the digits of the integer as well as auxilary data
required to manipulate the data.  These additional members are discussed in chapter three.  For the purposes of this text
a ``multiple precision integer'' and a ``mp\_int'' are synonymous.

\index{single-precision} \index{double-precision} \index{mp\_digit} \index{mp\_word}
For the purposes of this text a single-precision variable must be able to represent integers in the range $0 \le x < 2 \beta$ while
a double-precision variable must be able to represent integers in the range $0 \le x < 2 \beta^2$.  Within the source code that will be
presented the data type \textbf{mp\_digit} will represent a single-precision type while \textbf{mp\_word} will represent a 
double-precision type.  In several algorithms (\textit{notably the Comba routines}) temporary results 
will be stored in a double-precision arrays.  For the purposes of this text $x_j$ will refer to the 
$j$'th digit of a single-precision array and $\hat x_j$ will refer to the $j$'th digit of a double-precision
array.

\subsection{Work Effort}
\index{big-O}
To measure the efficiency of various algorithms a modified big-O notation is used.  In this system all 
single precision operations are considered to have the same cost\footnote{Except where explicitly noted.}.  
That is a single precision addition, multiplication and division are assumed to take the same time to 
complete.  While this is generally not true in practice it will simplify the discussions considerably.

Some algorithms have slight advantages over others which is why some constants will not be removed in 
the notation.  For example, a normal multiplication requires $O(n^2)$ work while a squaring requires 
$O({{n^2 + n}\over 2})$ work.  In standard big-O notation these would be said to be equivalent.  However, in the 
context of the this text the magnitude of the inputs will not approach an infinite size.  This means the conventional limit 
notation wisdom does not apply to the cancellation of constants.

Throughout the discussions various ``work levels'' will be discussed.  These levels are the $O(1)$,
$O(n)$, $O(n^2)$, ..., $O(n^k)$ work efforts.  For example, operations at the $O(n^k)$ ``level'' are said to be
executed more frequently than operations at the $O(n^m)$ ``level'' when $k > m$.  Obviously most optimizations will pay
off the most at the higher levels since they represent the bulk of the effort required.  

\section{Exercises}
Within the more advanced chapters a section will be set aside to give the reader some challenging exercises.  These exercises are not 
designed to be prize winning problems yet instead to be thought provoking.  Wherever possible the problems are foreward minded stating 
problems that will be answered in subsequent chapters.  The reader is encouraged to finish the exercises as they appear to get a 
better understanding of the subject material.  

Similar to the exercises of \cite{TAOCPV2} as explained on pp.\textit{ix} these exercises are given a scoring system.  However, unlike 
\cite{TAOCPV2} the problems do not get nearly as hard as often.  The scoring of these exercises ranges from one (\textit{the easiest}) to
five (\textit{the hardest}).  The following table sumarizes the scoring.

\vspace{5mm}
\begin{tabular}{cl}
$\left [ 1 \right ]$ & An easy problem that should only take the reader a manner of \\
                     & minutes to solve.  Usually does not involve much computer time. \\
                     & \\
$\left [ 2 \right ]$ & An easy problem that involves a marginal amount of computer \\
                     & time usage.  Usually requires a program to be written to \\
                     & solve the problem. \\
                     & \\
$\left [ 3 \right ]$ & A moderately hard problem that requires a non-trivial amount \\
                     & of work.  Usually involves trivial research and development of \\
                     & new theory from the perspective of a student. \\
                     & \\
$\left [ 4 \right ]$ & A moderately hard problem that involves a non-trivial amount \\
                     & of work and research.  The solution to which will demonstrate \\
                     & a higher mastery of the subject matter. \\
                     & \\
$\left [ 5 \right ]$ & A hard problem that involves concepts that are non-trivial.  \\
                     & Solutions to these problems will demonstrate a complete mastery \\
                     & of the given subject. \\
                     & \\
\end{tabular}

Essentially problems at the first level are meant to be simple questions that the reader can answer quickly without programming a solution or
devising new theory.  These problems are quick tests to see if the material is understood.  Problems at the second level are also
designed to be easy but will require a program or algorithm to be implemented to arrive at the answer.  

Problems at the third level are meant to be a bit more difficult.  Often the answer is fairly obvious but arriving at an exacting solution
requires some thought and skill.  These problems will almost always involve devising a new algorithm or implementing a variation of
another algorithm.

Problems at the fourth level are meant to be even more difficult as well as involve some research.  The reader will most likely not know
the answer right away nor will this text provide the exact details of the answer (\textit{or at least not until a subsequent chapter}).  Problems
at the fifth level are meant to be the hardest problems relative to all the other problems in the chapter.  People who can correctly 
answer fifth level problems have a mastery of the subject matter at hand.

Often problems will be tied together.  The purpose of this is to start a chain of thought that will be discussed in future chapters.  The reader
is encouraged to answer the follow-up problems and try to draw the relevence of problems.

\chapter{Introduction to LibTomMath}

\section{What is the LibTomMath?}
LibTomMath is a free and open source multiple precision number theoretic library written in portable ISO C
source code.  By portable it is meant that the library does not contain any code that is platform dependent or otherwise
problematic to use on any given platform.  The library has been successfully tested under numerous operating systems 
including Solaris, MacOS, Windows, Linux, PalmOS and on standalone hardware such as the Gameboy Advance.  The 
library is designed to contain enough functionality to be able to develop number theoretic applications such as public 
key cryptosystems.

\section{Goals of the LibTomMath}

Even though the library is written entirely in portable ISO C considerable care has been taken to 
optimize the algorithm implementations within the library.  Specifically the code has been written to work well with
the GNU C Compiler (\textit{GCC}) on both x86 and ARMv4 processors.  Wherever possible optimal 
algorithms (\textit{such as Karatsuba multiplication, sliding window exponentiation and Montgomery reduction.}) have 
been provided to make the library as efficient as possible.  Even with the optimal and sometimes specialized 
algorithms that have been included the API has been kept as simple as possible.  Often generic place holder routines 
will make use of specialized algorithms automatically without the developers attention.  One such example
is the generic multiplication algorithm \textbf{mp\_mul()} which will automatically use Karatsuba multiplication if the 
inputs are of a specific size.

Making LibTomMath as efficient as possible is not the only goal of the LibTomMath project.  Ideally the library should 
be source compatible with another popular library which makes it more attractive for developers to use.  In this case the
MPI library was used as a API template for all the basic functions.

The project is also meant to act as a learning tool for students.  The logic being that no easy to follow ``bignum'' 
library exists which can be used to teach computer science students how to perform fast and reliable multiple precision 
arithmetic.  To this end the source code has been given quite a few comments and algorithm discussion points.  Often 
where applicable routines have more comments than lines of code.

\section{Choice of LibTomMath}
LibTomMath was chosen as the case study of this text not only because the author of both projects is one and the same but
for more worthy reasons.  Other libraries such as GMP, MPI, LIP and OpenSSL have multiple precision 
integer arithmetic routines but would not be ideal for this text for numerous reasons as will be explained in the 
following sub-sections.

\subsection{Code Base}
The LibTomMath code base is all portable ISO C source code.  This means that there are no platform dependent conditional
segments of code littered throughout the source.  This clean and uncluttered approach to the library means that a
developer can more readily ascertain the true intent of a given section of source code without trying to keep track of
what conditional code will be used.

The code base of LibTomMath is also exceptionally well organized.  Each function is in its own separate source code file 
which allows the reader to find a given function very fast.  When compiled with GCC for the x86 processor the entire 
library is a mere 87,760 bytes (\textit{$116,182$ bytes for ARMv4 processors}).  This includes every single function 
LibTomMath provides from basic arithmetic to various number theoretic functions such as modular exponentiation, various 
reduction algorithms and Jacobi symbol computation.  

By comparison MPI which has fewer number theoretic functions than LibTomMath compiled with the same conditions is 
45,429 bytes (\textit{$54,536$ for ARMv4}).  GMP which has rather large collection of functions with the default 
configuration on an x86 Athlon is 2,950,688 bytes.  Note that while LibTomMath has fewer functions than GMP it has been
been used as the sole basis for several public key cryptosystems without having to seek additional outside functions
to supplement the library.

\subsection{API Simplicity}
LibTomMath is designed after the MPI library and shares the API design.  Quite often programs that use MPI will build 
with LibTomMath without change. The function names are relatively straight forward as to what they perform.  Almost all of the 
functions except for a few minor exceptions which as will be discussed are for good reasons share the same parameter passing 
convention.  The learning curve is fairly shallow with the API provided which is an extremely valuable benefit for the 
student and developer alike.  

The LIP library is an example of a library with an API that is awkward to work with.  LIP uses function names that are often ``compressed'' to 
illegible short hand.  LibTomMath does not share this fault.

\subsection{Optimizations}
While LibTomMath is certainly not the fastest library (\textit{GMP often beats LibTomMath by a factor of two}) it does
feature a set of optimal algorithms for tasks ranging from modular reduction to squaring.  GMP and LIP also feature
such optimizations while MPI only uses baseline algorithms with no optimizations.

LibTomMath is almost always a magnitude faster than the MPI library at computationally expensive tasks such as modular
exponentiation.  In the grand scheme of ``bignum'' libraries LibTomMath is faster than the average library and usually  
slower than the best libraries such as GMP and OpenSSL by a small factor.

\subsection{Portability and Stability}
LibTomMath will build ``out of the box'' on any platform equipped with a modern version of the GNU C Compiler 
(\textit{GCC}).  This means that without changes the library will build without configuration or setting up any 
variables.  LIP and MPI will build ``out of the box'' as well but have numerous known bugs.  Most notably the author of 
MPI is not working on his library anymore.  

GMP requires a configuration script to run and will not build out of the box.   GMP and LibTomMath are still in active
development and are very stable across a variety of platforms.

\subsection{Choice}
LibTomMath is a relatively compact, well documented, highly optimized and portable library which seems only natural for
the case study of this text.  Various source files from the LibTomMath project will be included within the text.  However, the 
reader is encouraged to download their own copy of the library to actually be able to work with the library.  

\chapter{Getting Started}
\section{Library Basics}
To get the ``ball rolling'' so to speak a primitive data type and a series of primitive algorithms must be established.  First a data
type that will hold the information required to maintain a multiple precision integer must be designed.  With this basic data type of a series
of low level algorithms for initializing, clearing, growing and clamping integers can be developed to form the basis of the entire
package of algorithms.

\section{The mp\_int structure}
First the data type for storing multiple precision integers must be designed.  This data type must be able to hold information to 
maintain an array of digits, how many are actually used in the representation and the sign.  The ISO C standard does not provide for 
any such data type but it does provide for making composite data types known as structures.  The following is the structure definition 
used within LibTomMath.

\index{mp\_int}
\begin{verbatim}
typedef struct  {
    int used, alloc, sign;
    mp_digit *dp;
} mp_int;
\end{verbatim}

The \textbf{used} parameter denotes how many digits of the array \textbf{dp} are actually being used.  The array 
\textbf{dp} holds the digits that represent the integer desired.  The \textbf{alloc} parameter denotes how 
many digits are available in the array to use by functions before it has to increase in size.  When the \textbf{used} count 
of a result would exceed the \textbf{alloc} count all LibTomMath routines will automatically increase the size of the 
array to accommodate the precision of the result.  The \textbf{sign} parameter denotes the sign as either zero/positive 
(\textbf{MP\_ZPOS}) or negative (\textbf{MP\_NEG}).  

\section{Argument Passing}
A convention of arugment passing must be adopted early on in the development of any library.  Making the function prototypes
consistent will help eliminate many headaches in the future as the library grows to significant complexity.  In LibTomMath the multiple precision 
integer functions accept parameters from left to right as pointers to mp\_int structures.  That means that the source operands are 
placed on the left and the destination on the right.   Consider the following examples.

\begin{verbatim}
   mp_mul(&a, &b, &c);   /* c = a * b */
   mp_add(&a, &b, &a);   /* a = a + b */
   mp_sqr(&a, &b);       /* b = a * a */
\end{verbatim}

The left to right order is a fairly natural way to implement the functions since it lets the developer read aloud the
functions and make sense of them.  For example, the first function would read ``multiply a and b and store in c''.

Certain libraries (\textit{LIP by Lenstra for instance}) accept parameters the other way around.  That is the destination
on the left and arguments on the right.  In truth it is entirely a matter of preference.  

Another very useful design consideration is whether to allow argument sources to also be a destination.  For example, the
second example (\textit{mp\_add}) adds $a$ to $b$ and stores in $a$.  This is an important feature to implement since it
allows the higher up functions to cut down on the number of variables.  However, to implement this feature specific
care has to be given to ensure the destination is not written before the source is fully read.

\section{Return Values}
A well implemented library, no matter what its purpose, should trap as many runtime errors as possible and return them to the 
caller.  By catching runtime errors a library can be guaranteed to prevent undefined behaviour within reason.  In a multiple precision 
library the only errors that are bound to occur are related to inappropriate inputs (\textit{division by zero for instance}) or 
memory allocation errors.

In LibTomMath any function that can cause a runtime error will return an error as an \textbf{int} data type with one of the 
following values.

\index{MP\_OKAY} \index{MP\_VAL} \index{MP\_MEM}
\begin{center}
\begin{tabular}{|l|l|}
\hline \textbf{Value} & \textbf{Meaning} \\
\hline \textbf{MP\_OKAY} & The function was successful \\
\hline \textbf{MP\_VAL}  & One of the input value(s) was invalid \\
\hline \textbf{MP\_MEM}  & The function ran out of heap memory \\
\hline
\end{tabular}
\end{center}

When an error is detected within a function it should free any memory they allocated and return as soon as possible.  The goal
is to leave the system in the same state the system was when the function was called.  Error checking with this style of API is fairly simple.

\begin{verbatim}
   int err;
   if ((err = mp_add(&a, &b, &c)) != MP_OKAY) {
      printf("Error: %d\n", err);
      exit(EXIT_FAILURE);
   }
\end{verbatim}

The GMP library uses C style \textit{signals} to flag errors which is of questionable use.  Not all errors are fatal 
and it is not ideal to force developers to have signal handlers for such cases.

\section{Initialization and Clearing}
The logical starting point when actually writing multiple precision integer functions is the initialization and 
clearing of the integers.  These two functions will be used by far the most throughout the algorithms whenever 
temporary integers are required.

Given the basic mp\_int structure an initialization routine must first allocate memory to hold the digits of
the integer.  Often it is optimal to allocate a sufficiently large pre-set number of digits even considering
the initial integer will represent zero.  If only a single digit were allocated quite a few re-allocations
would occur for the majority of inputs.  There exists a tradeoff between how many default digits to allocate
and how many re-allocations are tolerable.  

If the memory for the digits has been successfully allocated then the rest of the members of the structure must
be initialized.  Since the initial state is to represent a zero integer the digits allocated must all be zeroed.  The
\textbf{used} count set to zero and \textbf{sign} set to \textbf{MP\_ZPOS}.

\subsection{Initializing an mp\_int}
To initialize an mp\_int the mp\_init algorithm shall be used.  The purpose of this algorithm is to allocate 
the memory required and initialize the integer to a default representation of zero.

\begin{figure}[here]
\begin{center}
\begin{tabular}{l}
\hline Algorithm \textbf{mp\_init}. \\
\textbf{Input}.   An mp\_int $a$ \\
\textbf{Output}.  Allocate memory for the digits and set to a zero state. \\
\hline \\
1.  Allocate memory for \textbf{MP\_PREC} digits. \\
2.  If the allocation failed then return(\textit{MP\_MEM}) \\
3.  for $n$ from $0$ to $MP\_PREC - 1$ do  \\
\hspace{3mm}3.1  $a_n \leftarrow 0$\\
4.  $a.sign \leftarrow MP\_ZPOS$\\
5.  $a.used \leftarrow 0$\\
6.  $a.alloc \leftarrow MP\_PREC$\\
7.  Return(\textit{MP\_OKAY})\\
\hline
\end{tabular}
\end{center}
\caption{Algorithm mp\_init}
\end{figure}

\textbf{Algorithm mp\_init.}
The \textbf{MP\_PREC} variable is a simple constant used to dictate minimal precision of allocated integers.  It is ideally at least equal to $32$ but 
can be any reasonable power of two.  Step one and two allocate the memory and account for it.  If the allocation fails the algorithm returns
immediately to signal the failure.  Step three will ensure that all the digits are in the default state of zero.  Finally steps 
four through six set the default settings of the \textbf{sign}, \textbf{used} and \textbf{alloc} members of the mp\_int structure.

\index{bn\_mp\_init.c}
\vspace{+3mm}\begin{small}
\hspace{-5.1mm}{\bf File}: bn\_mp\_init.c
\vspace{-3mm}
\begin{alltt}
016   
017   /* init a new bigint */
018   int
019   mp_init (mp_int * a)
020   \{
021     /* allocate ram required and clear it */
022     a->dp = OPT_CAST calloc (sizeof (mp_digit), MP_PREC);
023     if (a->dp == NULL) \{
024       return MP_MEM;
025     \}
026   
027     /* set the used to zero, allocated digit to the default precision
028      * and sign to positive */
029     a->used  = 0;
030     a->alloc = MP_PREC;
031     a->sign  = MP_ZPOS;
032   
033     return MP_OKAY;
034   \}
\end{alltt}
\end{small}

The \textbf{OPT\_CAST} type cast on line 22 is designed to allow C++ compilers to build the code out of
the box.  Microsoft C V5.00 is known to cause problems without the cast.  Also note that if the memory
allocation fails the other members of the mp\_int will be in an undefined state.  The code from 
line 29 to line 31 sets the default state for a mp\_int which is zero, positive and no used digits.

\subsection{Clearing an mp\_int}
When an mp\_int is no longer required the memory allocated for it can be cleared from the heap with 
the mp\_clear algorithm.

\begin{figure}[here]
\begin{center}
\begin{tabular}{l}
\hline Algorithm \textbf{mp\_clear}. \\
\textbf{Input}.   An mp\_int $a$ \\
\textbf{Output}.  The memory for $a$ is cleared. \\
\hline \\
1.  If $a$ has been previously freed then return(\textit{MP\_OKAY}). \\
2.  Free the digits of $a$ and mark $a$ as freed. \\
3.  $a.used \leftarrow 0$ \\
4.  $a.alloc \leftarrow 0$ \\
5.  Return(\textit{MP\_OKAY}). \\
\hline
\end{tabular}
\end{center}
\caption{Algorithm mp\_clear}
\end{figure}

\textbf{Algorithm mp\_clear.}
In steps one and two the memory for the digits are only free'd if they had not been previously released before.  
This is more of concern for the implementation since it is used to prevent ``double-free'' errors.  It also helps catch
code errors where mp\_ints are used after being cleared.  Simiarly steps three and four set the 
\textbf{used} and \textbf{alloc} to known values which would be easy to spot during debugging.  For example, if an mp\_int is expected
to be non-zero and its \textbf{used} member observed to be zero (\textit{due to being cleared}) then an obvious bug in the code has been
spotted.

\index{bn\_mp\_clear.c}
\vspace{+3mm}\begin{small}
\hspace{-5.1mm}{\bf File}: bn\_mp\_clear.c
\vspace{-3mm}
\begin{alltt}
016   
017   /* clear one (frees)  */
018   void
019   mp_clear (mp_int * a)
020   \{
021     if (a->dp != NULL) \{
022   
023       /* first zero the digits */
024       memset (a->dp, 0, sizeof (mp_digit) * a->used);
025   
026       /* free ram */
027       free (a->dp);
028   
029       /* reset members to make debugging easier */
030       a->dp = NULL;
031       a->alloc = a->used = 0;
032     \}
033   \}
\end{alltt}
\end{small}

The \textbf{if} statement on line 21 prevents the heap from being corrupted if a user double-frees an 
mp\_int.  For example, a trivial case of this bug would be as follows.

\begin{verbatim}
mp_int a;
mp_init(&a);
mp_clear(&a);
mp_clear(&a);
\end{verbatim}

Without that check the code would try to free the memory allocated for the digits twice which will cause most standard C
libraries to cause a fault.  Also by setting the pointer to \textbf{NULL} it helps debug code that may inadvertently 
free the mp\_int before it is truly not needed.  The allocated digits are set to zero before being freed on line 24.  
This is ideal for cryptographic situations where the mp\_int is a secret parameter.

The following snippet is an example of using both the init and clear functions.  

\begin{small}
\begin{verbatim}
#include <tommath.h>
#include <stdio.h>
#include <stdlib.h>
int main(void)
{
   mp_int num;
   int err;
   
   /* init the bignum */
   if ((err = mp_init(&num)) != MP_OKAY) {
      printf("Error: %d\n", err);
      return EXIT_FAILURE;
   }
   
   /* do work with it ... */
   
   /* clear up */
   mp_clear(&num);
   
   return EXIT_SUCCESS;
}
\end{verbatim}
\end{small}

\section{Other Initialization Routines}

It is often helpful to have specialized initialization algorithms to simplify the design of other algorithms.  For example, an 
initialization followed by a copy is a common operation when temporary copies of integers are required.  It is quite
beneficial to have a series of simple helper functions available.

\subsection{Initializing Variable Sized mp\_int Structures}
Occasionally the number of digits required will be known in advance of an initialization.  In these
cases the mp\_init\_size algorithm can be of use.  The purpose of this algorithm is similar to mp\_init except that 
it will allocate \textit{at least} a specified number of digits.  This is ideal to prevent re-allocations when the 
input size is known.

\newpage\begin{figure}[here]
\begin{center}
\begin{tabular}{l}
\hline Algorithm \textbf{mp\_init\_size}. \\
\textbf{Input}.   An mp\_int $a$ and the requested number of digits $b$\\
\textbf{Output}.  $a$ is initialized to hold at least $b$ digits. \\
\hline \\
1.  $u \leftarrow b\mbox{ (mod }MP\_PREC\mbox{)}$ \\
2.  $v \leftarrow b + 2 \cdot MP\_PREC - u$ \\
3.  Allocate $v$ digits. \\
4.  If the allocation failed then return(\textit{MP\_MEM}). \\
5.  for $n$ from $0$ to $v - 1$ do \\
\hspace{3mm}5.1  $a_n \leftarrow 0$ \\
6.  $a.sign \leftarrow MP\_ZPOS$\\
7.  $a.used \leftarrow 0$\\
8.  $a.alloc \leftarrow v$\\
9.  Return(\textit{MP\_OKAY})\\
\hline
\end{tabular}
\end{center}
\caption{Algorithm mp\_init\_size}
\end{figure}

\textbf{Algorithm mp\_init\_size.}
The value of $v$ is calculated to be at least the requested amount of digits $b$ plus additional padding.  The padding is calculated
to be at least \textbf{MP\_PREC} digits plus enough digits to make the digit count a multiple of \textbf{MP\_PREC}.  This padding is used to 
prevent trivial allocations from becomming a bottleneck in the rest of the algorithms that depend on this.

\index{bn\_mp\_init\_size.c}
\vspace{+3mm}\begin{small}
\hspace{-5.1mm}{\bf File}: bn\_mp\_init\_size.c
\vspace{-3mm}
\begin{alltt}
016   
017   /* init a mp_init and grow it to a given size */
018   int
019   mp_init_size (mp_int * a, int size)
020   \{
021   
022     /* pad size so there are always extra digits */
023     size += (MP_PREC * 2) - (size & (MP_PREC - 1));    
024     
025     /* alloc mem */
026     a->dp = OPT_CAST calloc (sizeof (mp_digit), size);
027     if (a->dp == NULL) \{
028       return MP_MEM;
029     \}
030     a->used = 0;
031     a->alloc = size;
032     a->sign = MP_ZPOS;
033   
034     return MP_OKAY;
035   \}
\end{alltt}
\end{small}

Line 23 will ensure that the number of digits actually allocated is padded up to the next multiple of 
\textbf{MP\_PREC} plus an additional \textbf{MP\_PREC}.  This ensures that the number of allocated digit is 
always greater than the amount requested.  As a result it prevents many trivial memory allocations.  The value of 
\textbf{MP\_PREC} is defined in ``tommath.h'' and must be a power of two.

\subsection{Creating a Clone}
Another common sequence of operations is to make a local temporary copy of an argument.  To initialize then copy a mp\_int will be known as 
creating a clone.  This is useful within functions that need to modify an integer argument but do not wish to actually modify the original copy.  
The mp\_init\_copy algorithm will perform this very task.

\begin{figure}[here]
\begin{center}
\begin{tabular}{l}
\hline Algorithm \textbf{mp\_init\_copy}. \\
\textbf{Input}.   An mp\_int $a$ and $b$\\
\textbf{Output}.  $a$ is initialized to be a copy of $b$. \\
\hline \\
1.  Init $a$.  (\textit{hint: use mp\_init}) \\
2.  If the init of $a$ was unsuccessful return(\textit{MP\_MEM}) \\
3.  Copy $b$ to $a$.  (\textit{hint: use mp\_copy}) \\
4.  Return the status of the copy operation. \\
\hline
\end{tabular}
\end{center}
\caption{Algorithm mp\_init\_copy}
\end{figure}

\textbf{Algorithm mp\_init\_copy.}
This algorithm will initialize a mp\_int variable and copy another previously initialized mp\_int variable into it.  The algorithm will
detect when the initialization fails and returns the error to the calling algorithm.  As such this algorithm will perform two operations
in one step.  

\index{bn\_mp\_init\_copy.c}
\vspace{+3mm}\begin{small}
\hspace{-5.1mm}{\bf File}: bn\_mp\_init\_copy.c
\vspace{-3mm}
\begin{alltt}
016   
017   /* creates "a" then copies b into it */
018   int
019   mp_init_copy (mp_int * a, mp_int * b)
020   \{
021     int     res;
022   
023     if ((res = mp_init (a)) != MP_OKAY) \{
024       return res;
025     \}
026     return mp_copy (b, a);
027   \}
\end{alltt}
\end{small}

This will initialize \textbf{a} and make it a verbatim copy of the contents of \textbf{b}.  Note that 
\textbf{a} will have its own memory allocated which means that \textbf{b} may be cleared after the call
and \textbf{a} will be left intact.  

\subsection{Multiple Integer Initializations}
Occasionally a function will require a series of mp\_int data types to be made available.  The mp\_init\_multi algorithm
is provided to simplify such cases.  The purpose of this algorithm is to initialize a variable length array of mp\_int 
structures at once.  As a result algorithms that require multiple integers only has to use 
one algorithm to initialize all the mp\_int variables.

\begin{figure}[here]
\begin{center}
\begin{tabular}{l}
\hline Algorithm \textbf{mp\_init\_multi}. \\
\textbf{Input}.   Variable length array of mp\_int variables of length $k$. \\
\textbf{Output}.  The array is initialized such that each each mp\_int is ready to use. \\
\hline \\
1.  for $n$ from 0 to $k - 1$ do \\
\hspace{+3mm}1.1.  Initialize the $n$'th mp\_int (\textit{hint: use mp\_init}) \\
\hspace{+3mm}1.2.  If initialization failed then do \\
\hspace{+6mm}1.2.1.  for $j$ from $0$ to $n$ do \\
\hspace{+9mm}1.2.1.1.  Free the $j$'th mp\_int (\textit{hint: use mp\_clear}) \\
\hspace{+6mm}1.2.2.   Return(\textit{MP\_MEM}) \\
2.  Return(\textit{MP\_OKAY}) \\
\hline
\end{tabular}
\end{center}
\caption{Algorithm mp\_init\_multi}
\end{figure}

\textbf{Algorithm mp\_init\_multi.}
The algorithm will initialize the array of mp\_int variables one at a time.  As soon as an runtime error is detected (\textit{step 1.2}) all of
the previously initialized variables are cleared.  The goal is an ``all or nothing'' initialization which allows for quick recovery from runtime 
errors.

\subsection{Multiple Integer Clearing}
Similarly to clear a variable length list of mp\_int structures the mp\_clear\_multi algorithm will be used.

\index{bn\_mp\_multi.c}
\vspace{+3mm}\begin{small}
\hspace{-5.1mm}{\bf File}: bn\_mp\_multi.c
\vspace{-3mm}
\begin{alltt}
016   #include <stdarg.h>
017   
018   int mp_init_multi(mp_int *mp, ...) 
019   \{
020       mp_err res = MP_OKAY;      /* Assume ok until proven otherwise */
021       int n = 0;                 /* Number of ok inits */
022       mp_int* cur_arg = mp;
023       va_list args;
024   
025       va_start(args, mp);        /* init args to next argument from caller */
026       while (cur_arg != NULL) \{
027           if (mp_init(cur_arg) != MP_OKAY) \{
028               /* Oops - error! Back-track and mp_clear what we already
029                  succeeded in init-ing, then return error.
030               */
031               va_list clean_args;
032               
033               /* end the current list */
034               va_end(args);
035               
036               /* now start cleaning up */            
037               cur_arg = mp;
038               va_start(clean_args, mp);
039               while (n--) \{
040                   mp_clear(cur_arg);
041                   cur_arg = va_arg(clean_args, mp_int*);
042               \}
043               va_end(clean_args);
044               res = MP_MEM;
045               break;
046           \}
047           n++;
048           cur_arg = va_arg(args, mp_int*);
049       \}
050       va_end(args);
051       return res;                /* Assumed ok, if error flagged above. */
052   \}
053   
054   void mp_clear_multi(mp_int *mp, ...) 
055   \{
056       mp_int* next_mp = mp;
057       va_list args;
058       va_start(args, mp);
059       while (next_mp != NULL) \{
060           mp_clear(next_mp);
061           next_mp = va_arg(args, mp_int*);
062       \}
063       va_end(args);
064   \}
\end{alltt}
\end{small}

Consider the following snippet which demonstrates how to use both routines.
\begin{small}
\begin{verbatim}
#include <tommath.h>
#include <stdio.h>
#include <stdlib.h>
int main(void)
{
   mp_int num1, num2, num3;
   int err;
   
   if ((err = mp_init_multi(&num1, &num2, &num3, NULL)) !- MP_OKAY) {
      printf("Error: %d\n", err);
      return EXIT_FAILURE;
   }
   
   /* at this point num1/num2/num3 are ready */
   
   /* free them */
   mp_clear_multi(&num1, &num2, &num3, NULL);
   
   return EXIT_SUCCESS;
}
\end{verbatim}
\end{small}

\section{Maintenance}
A small useful collection of mp\_int maintenance functions will also prove useful.  

\subsection{Augmenting Integer Precision}
When storing a value in an mp\_int sufficient digits must be available to accomodate the entire value without
loss of precision.  Quite often the size of the array given by the \textbf{alloc} member is large enough to simply
increase the \textbf{used} digit count.  However, when the size of the array is too small it must be re-sized 
appropriately to accomodate the result.  The mp\_grow algorithm will provide this functionality.

\begin{figure}[here]
\begin{center}
\begin{tabular}{l}
\hline Algorithm \textbf{mp\_grow}. \\
\textbf{Input}.   An mp\_int $a$ and an integer $b$. \\
\textbf{Output}.  $a$ is expanded to accomodate $b$ digits. \\
\hline \\
1.  if $a.alloc \ge b$ then return(\textit{MP\_OKAY}) \\
2.  $u \leftarrow b\mbox{ (mod }MP\_PREC\mbox{)}$ \\
3.  $v \leftarrow b + 2 \cdot MP\_PREC - u$ \\
4.  Re-Allocate the array of digits $a$ to size $v$ \\
5.  If the allocation failed then return(\textit{MP\_MEM}). \\
6.  for n from a.alloc to $v - 1$ do  \\
\hspace{+3mm}6.1  $a_n \leftarrow 0$ \\
7.  $a.alloc \leftarrow v$ \\
8.  Return(\textit{MP\_OKAY}) \\
\hline
\end{tabular}
\end{center}
\caption{Algorithm mp\_grow}
\end{figure}

\textbf{Algorithm mp\_grow.}
Step one will prevent a re-allocation from being performed if it was not required.  This is useful to prevent mp\_ints
from growing excessively in code that erroneously calls mp\_grow.  Similar to mp\_init\_size the requested digit count
is padded to provide more digits than requested.  

In step four it is assumed that the reallocation leaves the lower $a.alloc$ digits intact.  Much akin to how the 
\textit{realloc} function from the standard C library works.  Since the newly allocated digits are assumed to contain
undefined values they are also initially zeroed.

\index{bn\_mp\_grow.c}
\vspace{+3mm}\begin{small}
\hspace{-5.1mm}{\bf File}: bn\_mp\_grow.c
\vspace{-3mm}
\begin{alltt}
016   
017   /* grow as required */
018   int
019   mp_grow (mp_int * a, int size)
020   \{
021     int     i;
022   
023     /* if the alloc size is smaller alloc more ram */
024     if (a->alloc < size) \{
025       /* ensure there are always at least MP_PREC digits extra on top */
026       size += (MP_PREC * 2) - (size & (MP_PREC - 1));     
027   
028       a->dp = OPT_CAST realloc (a->dp, sizeof (mp_digit) * size);
029       if (a->dp == NULL) \{
030         return MP_MEM;
031       \}
032   
033       /* zero excess digits */
034       i        = a->alloc;
035       a->alloc = size;
036       for (; i < a->alloc; i++) \{
037         a->dp[i] = 0;
038       \}
039     \}
040     return MP_OKAY;
041   \}
\end{alltt}
\end{small}

The first step is to see if we actually need to perform a re-allocation at all.  This is tested for on line 
24.  Similar to mp\_init\_size the same code on line 26 was used to resize the 
digits requested.  A simple for loop from line 34 to line 38 will zero all digits that were above the 
old \textbf{alloc} limit to make sure the integer is in a known state.

\subsection{Clamping Excess Digits}
When a function anticipates a result will be $n$ digits it is simpler to assume this is true within the body of 
the function.  For example, a multiplication of a $i$ digit number by a $j$ digit produces a result of at most 
$i + j + 1$ digits.  It is entirely possible that the result is $i + j$ though, with no final carry into the last 
position.  However, suppose the destination had to be first expanded (\textit{via mp\_grow}) to accomodate $i + j$
digits than further expanded to accomodate the final carry.  That would be a considerable waste of time since heap
operations are relatively slow.

The ideal solution is to always assume the result is $i + j + 1$ and fix up the \textbf{used} count after the function
terminates.  This way a single heap operation (\textit{at most}) is required.  However, if the result was not checked
there would be an excess high order zero digit.  

For example, suppose the product of two integers was $x_n = (0x_{n-1}x_{n-2}...x_0)_{\beta}$.  The leading zero digit 
will not contribute to the precision of the result.  In fact, through subsequent operations more leading zero digits would
accumulate to the point the size of the integer would be prohibitive.  As a result even though the precision is very 
low the representation is excessively large.  

The mp\_clamp algorithm is designed to solve this very problem.  It will trim leading zeros by decrementing the 
\textbf{used} count until a non-zero leading digit is found.  Also in this system, zero is considered to be a positive 
number which means that if the \textbf{used} count is decremented to zero the sign must be set to \textbf{MP\_ZPOS}.

\begin{figure}[here]
\begin{center}
\begin{tabular}{l}
\hline Algorithm \textbf{mp\_clamp}. \\
\textbf{Input}.   An mp\_int $a$ \\
\textbf{Output}.  Any excess leading zero digits of $a$ are removed \\
\hline \\
1.  while $a.used > 0$ and $a_{a.used - 1} = 0$ do \\
\hspace{+3mm}1.1  $a.used \leftarrow a.used - 1$ \\
2.  if $a.used = 0$ then do \\
\hspace{+3mm}2.1  $a.sign \leftarrow MP\_ZPOS$ \\
\hline \\
\end{tabular}
\end{center}
\caption{Algorithm mp\_clamp}
\end{figure}

\textbf{Algorithm mp\_clamp.}
As can be expected this algorithm is very simple.  The loop on step one is indended to be iterate only once or twice at
the most.  For example, for cases where there is not a carry to fill the last position.  Step two fixes the sign for 
when all of the digits are zero to ensure that the mp\_int is valid at all times.

\index{bn\_mp\_clamp.c}
\vspace{+3mm}\begin{small}
\hspace{-5.1mm}{\bf File}: bn\_mp\_clamp.c
\vspace{-3mm}
\begin{alltt}
016   
017   /* trim unused digits 
018    *
019    * This is used to ensure that leading zero digits are
020    * trimed and the leading "used" digit will be non-zero
021    * Typically very fast.  Also fixes the sign if there
022    * are no more leading digits
023    */
024   void
025   mp_clamp (mp_int * a)
026   \{
027     while (a->used > 0 && a->dp[a->used - 1] == 0) \{
028       --(a->used);
029     \}
030     if (a->used == 0) \{
031       a->sign = MP_ZPOS;
032     \}
033   \}
\end{alltt}
\end{small}

Note on line 27 how to test for the \textbf{used} count is made on the left of the \&\& operator.  In the C programming
language the terms to \&\& are evaluated left to right with a boolean short-circuit if any condition fails.  This is 
important since if the \textbf{used} is zero the test on the right would fetch below the array.  That is obviously 
undesirable.  The parenthesis on line 28 is used to make sure the \textbf{used} count is decremented and not
the pointer ``a''.  

\section*{Exercises}
\begin{tabular}{cl}
$\left [ 1 \right ]$ & Discuss the relevance of the \textbf{used} member of the mp\_int structure. \\
                     & \\
$\left [ 1 \right ]$ & Discuss the consequences of not using padding when performing allocations.  \\
                     & \\
$\left [ 2 \right ]$ & Estimate an ideal value for \textbf{MP\_PREC} when performing 1024-bit RSA \\
                     & encryption when $\beta = 2^{28}$.  \\
                     & \\
$\left [ 1 \right ]$ & Discuss the relevance of the algorithm mp\_clamp.  What does it prevent? \\
                     & \\
$\left [ 1 \right ]$ & Give an example of when the algorithm  mp\_init\_copy might be useful. \\
                     & \\
\end{tabular}


\chapter{Basic Operations}
\section{Copying an Integer}
After the various house-keeping routines are in place, simpl algorithms can be designed to take advantage of them.  Being able
to make a verbatim copy of an integer is a very useful function to have.  To copy an integer the mp\_copy algorithm will be used.

\newpage\begin{figure}[here]
\begin{center}
\begin{tabular}{l}
\hline Algorithm \textbf{mp\_copy}. \\
\textbf{Input}.  An mp\_int $a$ and $b$. \\
\textbf{Output}.  Store a copy of $a$ in $b$. \\
\hline \\
1.  Check if $a$ and $b$ point to the same location in memory. \\
2.  If true then return(\textit{MP\_OKAY}). \\
3.  If $b.alloc < a.used$ then grow $b$ to $a.used$ digits.  (\textit{hint: use mp\_grow}) \\
4.  If failed to grow then return(\textit{MP\_MEM}). \\
5.  for $n$ from 0 to $a.used - 1$ do \\
\hspace{3mm}5.1  $b_{n} \leftarrow a_{n}$ \\
6.  if $a.used < b.used - 1$ then \\ 
\hspace{3mm}6.1.  for $n$ from $a.used$ to $b.used - 1$ do \\
\hspace{6mm}6.1.1  $b_{n} \leftarrow 0$ \\
7.  $b.used \leftarrow a.used$ \\
8.  $b.sign \leftarrow a.sign$ \\
9.  return(\textit{MP\_OKAY}) \\
\hline
\end{tabular}
\end{center}
\caption{Algorithm mp\_copy}
\end{figure}

\textbf{Algorithm mp\_copy.}
Step 1 and 2 make sure that the two mp\_ints are unique.  This allows the user to call the copy function with
potentially the same input and not waste time.  Step 3 and 4 ensure that the destination is large enough to
hold a copy of the input $a$.  Note that the \textbf{used} member of $b$ may be smaller than the \textbf{used}
member of $a$ but a memory re-allocation is only required if the \textbf{alloc} member of $b$ is smaller.  This
prevents trivial memory reallocations.

Step 5 copies the digits from $a$ to $b$ while step 6 ensures that if initially $\vert b \vert > \vert a \vert$,
the leading digits of $b$ will be zeroed.  Finally steps 7 and 8 copies the \textbf{used} and \textbf{sign} members over 
which completes the copy operation.

\index{bn\_mp\_copy.c}
\vspace{+3mm}\begin{small}
\hspace{-5.1mm}{\bf File}: bn\_mp\_copy.c
\vspace{-3mm}
\begin{alltt}
016   
017   /* copy, b = a */
018   int
019   mp_copy (mp_int * a, mp_int * b)
020   \{
021     int     res, n;
022   
023     /* if dst == src do nothing */
024     if (a == b || a->dp == b->dp) \{
025       return MP_OKAY;
026     \}
027   
028     /* grow dest */
029     if ((res = mp_grow (b, a->used)) != MP_OKAY) \{
030       return res;
031     \}
032   
033     /* zero b and copy the parameters over */
034     \{
035       register mp_digit *tmpa, *tmpb;
036   
037       /* pointer aliases */
038       tmpa = a->dp;
039       tmpb = b->dp;
040   
041       /* copy all the digits */
042       for (n = 0; n < a->used; n++) \{
043         *tmpb++ = *tmpa++;
044       \}
045   
046       /* clear high digits */
047       for (; n < b->used; n++) \{
048         *tmpb++ = 0;
049       \}
050     \}
051     b->used = a->used;
052     b->sign = a->sign;
053     return MP_OKAY;
054   \}
\end{alltt}
\end{small}

Source lines 23-31 do the initial house keeping.  That is to see if the input is unique and if so to 
make sure there is enough room.  If not enough space is available it returns the error and leaves the destination variable
intact.

The inner loop of the copy operation is contained between lines 34 and 50.  Many LibTomMath routines are designed with this source code style
in mind, making aliases to shorten lengthy pointers (\textit{see line 38 and 39}) for rapid to use.  Also the
use of nested braces creates a simple way to denote various portions of code that reside on various work levels.  Here, the copy loop is at the 
$O(n)$ level.  

\section{Zeroing an Integer}
Reseting an mp\_int to the default state is a common step in many algorithms.  The mp\_zero algorithm will be the algorithm used to
perform this task.

\begin{figure}[here]
\begin{center}
\begin{tabular}{l}
\hline Algorithm \textbf{mp\_zero}. \\
\textbf{Input}.   An mp\_int $a$ \\
\textbf{Output}.  Zero the contents of $a$ \\
\hline \\
1.  $a.used \leftarrow 0$ \\
2.  $a.sign \leftarrow$ MP\_ZPOS \\
3.  for $n$ from 0 to $a.alloc - 1$ do \\
\hspace{3mm}3.1  $a_n \leftarrow 0$ \\
\hline
\end{tabular}
\end{center}
\caption{Algorithm mp\_zero}
\end{figure}

\textbf{Algorithm mp\_zero.}
This algorithm simply resets a mp\_int to the default state.  

\index{bn\_mp\_zero.c}
\vspace{+3mm}\begin{small}
\hspace{-5.1mm}{\bf File}: bn\_mp\_zero.c
\vspace{-3mm}
\begin{alltt}
016   
017   /* set to zero */
018   void
019   mp_zero (mp_int * a)
020   \{
021     a->sign = MP_ZPOS;
022     a->used = 0;
023     memset (a->dp, 0, sizeof (mp_digit) * a->alloc);
024   \}
\end{alltt}
\end{small}

After the function is completed, all of the digits are zeroed, the \textbf{used} count is zeroed and the 
\textbf{sign} variable is set to \textbf{MP\_ZPOS}.

\section{Sign Manipulation}
\subsection{Absolute Value}
With the mp\_int representation of an integer, calculating the absolute value is trivial.  The mp\_abs algorithm will compute
the absolute value of an mp\_int.

\begin{figure}[here]
\begin{center}
\begin{tabular}{l}
\hline Algorithm \textbf{mp\_abs}. \\
\textbf{Input}.   An mp\_int $a$ \\
\textbf{Output}.  Computes $b = \vert a \vert$ \\
\hline \\
1.  Copy $a$ to $b$.  (\textit{hint: use mp\_copy}) \\
2.  If the copy failed return(\textit{MP\_MEM}). \\
3.  $b.sign \leftarrow MP\_ZPOS$ \\
4.  Return(\textit{MP\_OKAY}) \\
\hline
\end{tabular}
\end{center}
\caption{Algorithm mp\_abs}
\end{figure}

\textbf{Algorithm mp\_abs.}
This algorithm computes the absolute of an mp\_int input.  As can be expected the algorithm is very trivial.

\index{bn\_mp\_abs.c}
\vspace{+3mm}\begin{small}
\hspace{-5.1mm}{\bf File}: bn\_mp\_abs.c
\vspace{-3mm}
\begin{alltt}
016   
017   /* b = |a| 
018    *
019    * Simple function copies the input and fixes the sign to positive
020    */
021   int
022   mp_abs (mp_int * a, mp_int * b)
023   \{
024     int     res;
025     if ((res = mp_copy (a, b)) != MP_OKAY) \{
026       return res;
027     \}
028     b->sign = MP_ZPOS;
029     return MP_OKAY;
030   \}
\end{alltt}
\end{small}

\subsection{Integer Negation}
With the mp\_int representation of an integer, calculating the negation is also trivial.  The mp\_neg algorithm will compute
the negative of an mp\_int input.

\newpage\begin{figure}[here]
\begin{center}
\begin{tabular}{l}
\hline Algorithm \textbf{mp\_neg}. \\
\textbf{Input}.   An mp\_int $a$ \\
\textbf{Output}.  Computes $b = -a$ \\
\hline \\
1.  Copy $a$ to $b$.  (\textit{hint: use mp\_copy}) \\
2.  If the copy failed return(\textit{MP\_MEM}). \\
3.  If $a.sign = MP\_ZPOS$ then do \\
\hspace{3mm}3.1  $b.sign = MP\_NEG$. \\
4.  else do \\
\hspace{3mm}4.1  $b.sign = MP\_ZPOS$. \\
5.  Return(\textit{MP\_OKAY}) \\
\hline
\end{tabular}
\end{center}
\caption{Algorithm mp\_neg}
\end{figure}

\textbf{Algorithm mp\_neg.}
This algorithm computes the negation of an input.  

\index{bn\_mp\_neg.c}
\vspace{+3mm}\begin{small}
\hspace{-5.1mm}{\bf File}: bn\_mp\_neg.c
\vspace{-3mm}
\begin{alltt}
016   
017   /* b = -a */
018   int
019   mp_neg (mp_int * a, mp_int * b)
020   \{
021     int     res;
022     if ((res = mp_copy (a, b)) != MP_OKAY) \{
023       return res;
024     \}
025     b->sign = (a->sign == MP_ZPOS) ? MP_NEG : MP_ZPOS;
026     return MP_OKAY;
027   \}
\end{alltt}
\end{small}

\section{Small Constants}
\subsection{Setting Small Constants}
Often a mp\_int must be set to a relatively small value such as $1$ or $2$.  For these cases the mp\_set algorithm is useful.

\newpage\begin{figure}
\begin{center}
\begin{tabular}{l}
\hline Algorithm \textbf{mp\_set}. \\
\textbf{Input}.   An mp\_int $a$ and a digit $b$ \\
\textbf{Output}.  Make $a$ equivalent to $b$ \\
\hline \\
1.  Zero $a$ (\textit{hint: use mp\_zero}). \\
2.  $a_0 \leftarrow b \mbox{ (mod }\beta\mbox{)}$ \\
3.  $a.used \leftarrow  \left \lbrace \begin{array}{ll}
                              1 &  \mbox{if }a_0 > 0 \\
                              0 &  \mbox{if }a_0 = 0 
                              \end{array} \right .$ \\
\hline                              
\end{tabular}
\end{center}
\caption{Algorithm mp\_set}
\end{figure}

\textbf{Algorithm mp\_set.}
This algorithm sets a mp\_int to a small single digit value.  Step number 1 ensures that the integer is reset to the default state.  The
single digit is set (\textit{modulo $\beta$}) and the \textbf{used} count is adjusted accordingly.

\index{bn\_mp\_set.c}
\vspace{+3mm}\begin{small}
\hspace{-5.1mm}{\bf File}: bn\_mp\_set.c
\vspace{-3mm}
\begin{alltt}
016   
017   /* set to a digit */
018   void
019   mp_set (mp_int * a, mp_digit b)
020   \{
021     mp_zero (a);
022     a->dp[0] = b & MP_MASK;
023     a->used = (a->dp[0] != 0) ? 1 : 0;
024   \}
\end{alltt}
\end{small}

Line 21 calls mp\_zero() to clear the mp\_int and reset the sign.  Line 22 actually copies digit 
into the least significant location.  Note the usage of a new constant \textbf{MP\_MASK}.  This constant is used to quickly
reduce an integer modulo $\beta$.  Since $\beta = 2^k$ it suffices to perform a binary AND with $MP\_MASK = 2^k - 1$ to perform
the reduction.  Finally line 23 will set the \textbf{used} member with respect to the digit actually set. This function 
will always make the integer positive.

One important limitation of this function is that it will only set one digit.  The size of a digit is not fixed, meaning source that uses 
this function should take that into account.  The define \textbf{DIGIT\_BIT} in ``tommath.h'' 
defines how many bits per digit are available.  Generally at least seven bits are guaranteed to be available per 
digit.  This means that trivially small constants can be set using this function.

\subsection{Setting Large Constants}
To overcome the limitations of the mp\_set algorithm the mp\_set\_int algorithm is provided.  It accepts a ``long''
data type as input and will always treat it as a 32-bit integer.

\begin{figure}[here]
\begin{center}
\begin{tabular}{l}
\hline Algorithm \textbf{mp\_set\_int}. \\
\textbf{Input}.   An mp\_int $a$ and a ``long'' integer $b$ \\
\textbf{Output}.  Make $a$ equivalent to $b$ \\
\hline \\
1.  Zero $a$ (\textit{hint: use mp\_zero}) \\
2.  for $n$ from 0 to 7 do \\
\hspace{3mm}2.1  $a \leftarrow a \cdot 16$ (\textit{hint: use mp\_mul2d}) \\
\hspace{3mm}2.2  $u \leftarrow \lfloor b / 2^{4(7 - n)} \rfloor \mbox{ (mod }16\mbox{)}$\\
\hspace{3mm}2.3  $a_0 \leftarrow a_0 + u$ \\
\hspace{3mm}2.4  $a.used \leftarrow a.used + \lfloor 32 / lg(\beta) \rfloor + 1$ \\
3.  Clamp excess used digits (\textit{hint: use mp\_clamp}) \\
\hline
\end{tabular}
\end{center}
\caption{Algorithm mp\_set\_int}
\end{figure}

\textbf{Algorithm mp\_set\_int.}
The algorithm performs eight iterations of a simple loop where in each iteration four bits from the source are added to the 
mp\_int.  Step 2.1 will multiply the current result by sixteen making room for four more bits.  In step 2.2 the
next four bits from the source are extracted.  The four bits are added to the mp\_int and the \textbf{used} digit count is 
incremented.  The \textbf{used} digit counter is incremented since if any of the leading digits were zero the mp\_int would have
zero digits used and the newly added four bits would be ignored.

Excess zero digits are trimmed in steps 2.1 and 3 by using higher level algorithms mp\_mul2d and mp\_clamp.

\index{bn\_mp\_set\_int.c}
\vspace{+3mm}\begin{small}
\hspace{-5.1mm}{\bf File}: bn\_mp\_set\_int.c
\vspace{-3mm}
\begin{alltt}
016   
017   /* set a 32-bit const */
018   int
019   mp_set_int (mp_int * a, unsigned int b)
020   \{
021     int     x, res;
022   
023     mp_zero (a);
024     /* set four bits at a time */
025     for (x = 0; x < 8; x++) \{
026       /* shift the number up four bits */
027       if ((res = mp_mul_2d (a, 4, a)) != MP_OKAY) \{
028         return res;
029       \}
030   
031       /* OR in the top four bits of the source */
032       a->dp[0] |= (b >> 28) & 15;
033   
034       /* shift the source up to the next four bits */
035       b <<= 4;
036   
037       /* ensure that digits are not clamped off */
038       a->used += 32 / DIGIT_BIT + 2;
039     \}
040     mp_clamp (a);
041     return MP_OKAY;
042   \}
\end{alltt}
\end{small}

This function sets four bits of the number at a time to handle all practical \textbf{DIGIT\_BIT} sizes.  The weird
addition on line 38 ensures that the newly added in bits are added to the number of digits.  While it may not 
seem obvious as to why the digit counter does not grow exceedingly large it is because of the shift on line 27 
as well as the  call to mp\_clamp() on line 40.  Both functions will clamp excess leading digits which keeps 
the number of used digits low.

\section{Comparisons}
\subsection{Unsigned Comparisions}
Comparing a multiple precision integer is performed with the exact same algorithm used to compare two decimal numbers.  For example,
to compare $1,234$ to $1,264$ the digits are extracted by their positions.  That is we compare $1 \cdot 10^3 + 2 \cdot 10^2 + 3 \cdot 10^1 + 4 \cdot 10^0$
to $1 \cdot 10^3 + 2 \cdot 10^2 + 6 \cdot 10^1 + 4 \cdot 10^0$ by comparing single digits at a time starting with the highest magnitude 
positions.  If any leading digit of one integer is greater than a digit in the same position of another integer then obviously it must be greater.  

The first comparision routine that will be developed is the unsigned magnitude compare which will perform a comparison based on the digits of two
mp\_int variables alone.  It will ignore the sign of the two inputs.  Such a function is useful when an absolute comparison is required or if the 
signs are known to agree in advance.

To facilitate working with the results of the comparison functions three constants are required.  

\begin{figure}[here]
\begin{center}
\begin{tabular}{|r|l|}
\hline \textbf{Constant} & \textbf{Meaning} \\
\hline \textbf{MP\_GT} & Greater Than \\
\hline \textbf{MP\_EQ} & Equal To \\
\hline \textbf{MP\_LT} & Less Than \\
\hline
\end{tabular}
\end{center}
\caption{Comparison Return Codes}
\end{figure}

\begin{figure}[here]
\begin{center}
\begin{tabular}{l}
\hline Algorithm \textbf{mp\_cmp\_mag}. \\
\textbf{Input}.   Two mp\_ints $a$ and $b$.  \\
\textbf{Output}.  Unsigned comparison results ($a$ to the left of $b$). \\
\hline \\
1.  If $a.used > b.used$ then return(\textit{MP\_GT}) \\
2.  If $a.used < b.used$ then return(\textit{MP\_LT}) \\
3.  for n from $a.used - 1$ to 0 do \\
\hspace{+3mm}3.1  if $a_n > b_n$ then return(\textit{MP\_GT}) \\
\hspace{+3mm}3.2  if $a_n < b_n$ then return(\textit{MP\_LT}) \\
4.  Return(\textit{MP\_EQ}) \\
\hline
\end{tabular}
\end{center}
\caption{Algorithm mp\_cmp\_mag}
\end{figure}

\textbf{Algorithm mp\_cmp\_mag.}
By saying ``$a$ to the left of $b$'' it is meant that the comparison is with respect to $a$, that is if $a$ is greater than $b$ it will return
\textbf{MP\_GT} and similar with respect to when $a = b$ and $a < b$.  The first two steps compare the number of digits used in both $a$ and $b$.  
Obviously if the digit counts differ there would be an imaginary zero digit in the smaller number where the leading digit of the larger number is.  
If both have the same number of digits than the actual digits themselves must be compared starting at the leading digit.  

By step three both inputs must have the same number of digits so its safe to start from either $a.used - 1$ or $b.used - 1$ and count down to
the zero'th digit.  If after all of the digits have been compared and no difference found the algorithm simply returns \textbf{MP\_EQ}.

\index{bn\_mp\_cmp\_mag.c}
\vspace{+3mm}\begin{small}
\hspace{-5.1mm}{\bf File}: bn\_mp\_cmp\_mag.c
\vspace{-3mm}
\begin{alltt}
016   
017   /* compare maginitude of two ints (unsigned) */
018   int
019   mp_cmp_mag (mp_int * a, mp_int * b)
020   \{
021     int     n;
022   
023     /* compare based on # of non-zero digits */
024     if (a->used > b->used) \{
025       return MP_GT;
026     \} 
027     
028     if (a->used < b->used) \{
029       return MP_LT;
030     \}
031   
032     /* compare based on digits  */
033     for (n = a->used - 1; n >= 0; n--) \{
034       if (a->dp[n] > b->dp[n]) \{
035         return MP_GT;
036       \} 
037       
038       if (a->dp[n] < b->dp[n]) \{
039         return MP_LT;
040       \}
041     \}
042     return MP_EQ;
043   \}
\end{alltt}
\end{small}

The two if statements on lines 24 and 28 compare the number of digits in the two inputs.  These two are performed before all of the digits
are compared since it is a very cheap test to perform and can potentially save considerable time.  The implementation given is also not valid 
without those two statements.  $b.alloc$ may be smaller than $a.used$, meaning that undefined values will be read from $b$ passed the end of the 
array of digits.

\subsection{Signed Comparisons}
Comparing with sign considerations is also fairly critical in several routines (\textit{division for example}).  Based on an unsigned magnitude 
comparison a trivial signed comparison algorithm can be written.

\newpage\begin{figure}[here]
\begin{center}
\begin{tabular}{l}
\hline Algorithm \textbf{mp\_cmp}. \\
\textbf{Input}.   Two mp\_ints $a$ and $b$ \\
\textbf{Output}.  Signed Comparison Results ($a$ to the left of $b$) \\
\hline \\
1.  if $a.sign = MP\_NEG$ and $b.sign = MP\_ZPOS$ then return(\textit{MP\_LT}) \\
2.  if $a.sign = MP\_ZPOS$ and $b.sign = MP\_NEG$ then return(\textit{MP\_GT}) \\
3.  if $a.sign = MP\_NEG$ then \\
\hspace{+3mm}3.1  Return the unsigned comparison of $b$ and $a$ (\textit{hint: use mp\_cmp\_mag}) \\
4   Otherwise \\
\hspace{+3mm}4.1  Return the unsigned comparison of $a$ and $b$ \\
\hline
\end{tabular}
\end{center}
\caption{Algorithm mp\_cmp}
\end{figure}

\textbf{Algorithm mp\_cmp.}
The first two steps compare the signs of the two inputs.  If the signs do not agree then it can return right away with the appropriate 
comparison code.  When the signs are equal the digits of the inputs must be compared to determine the correct result.  In step 
three the unsigned comparision flips the order of the arguments since they are both negative.  For instance, if $-a > -b$ then 
$\vert a \vert < \vert b \vert$.  Step number four will compare the two when they are both positive.

\index{bn\_mp\_cmp.c}
\vspace{+3mm}\begin{small}
\hspace{-5.1mm}{\bf File}: bn\_mp\_cmp.c
\vspace{-3mm}
\begin{alltt}
016   
017   /* compare two ints (signed)*/
018   int
019   mp_cmp (mp_int * a, mp_int * b)
020   \{
021     /* compare based on sign */
022     if (a->sign == MP_NEG && b->sign == MP_ZPOS) \{
023       return MP_LT;
024     \} 
025     
026     if (a->sign == MP_ZPOS && b->sign == MP_NEG) \{
027       return MP_GT;
028     \}
029     
030     /* compare digits */
031     if (a->sign == MP_NEG) \{
032        /* if negative compare opposite direction */
033        return mp_cmp_mag(b, a);
034     \} else \{
035        return mp_cmp_mag(a, b);
036     \}
037   \}
\end{alltt}
\end{small}

The two if statements on lines 22 and 26 perform the initial sign comparison.  If the signs are not the equal then which ever
has the positive sign is larger.   At line 31, the inputs are compared based on magnitudes.  If the signs were both negative then 
the unsigned comparison is performed in the opposite direction (\textit{line 33}).  Otherwise, the signs are assumed to 
be both positive and a forward direction unsigned comparison is performed.

\section*{Exercises}
\begin{tabular}{cl}
$\left [ 2 \right ]$ & Modify algorithm mp\_set\_int to accept as input a variable length array of bits. \\
                     & \\
$\left [ 3 \right ]$ & Give the probability that algorithm mp\_cmp\_mag will have to compare $k$ digits  \\
                     & of two random digits (of equal magnitude) before a difference is found. \\
                     & \\
$\left [ 1 \right ]$ & Suggest a simple method to speed up the implementation of mp\_cmp\_mag based  \\
                     & on the observations made in the previous problem. \\
                     &
\end{tabular}

\chapter{Basic Arithmetic}
\section{Building Blocks}
At this point algorithms for initialization, de-initialization, zeroing, copying, comparing and setting small constants have been 
established.  The next logical set of algorithms to develop are the addition, subtraction and digit movement algorithms.  These 
algorithms make use of the lower level algorithms and are the cruicial building block for the multipliers.  It is very important that these 
algorithms are highly optimized.  On their own they are simple $O(n)$ algorithms but they can be called from higher level algorithms 
which easily places them at $O(n^2)$ or even $O(n^3)$ work levels.  

All nine algorithms within this chapter make use of the logical bit shift operations denoted by $<<$ and $>>$ for left and right 
logical shifts respectively.  A logical shift is analogous to sliding the decimal point of radix-10 representations.  For example, the real 
number $0.9345$ is equivalent to $93.45\%$ which is found by sliding the the decimal two places to the right (\textit{multiplying by $10^2$}).  
Mathematically a logical shift is equivalent to a division or multiplication by a power of two.  
For example, $a << k = a \cdot 2^k$ while $a >> k = \lfloor a/2^k \rfloor$.

One significant difference between a logical shift and the way decimals are shifted is that digits below the zero'th position are removed
from the number.  For example, consider $1101_2 >> 1$ using decimal notation this would produce $110.1_2$.  However, with a logical shift the 
result is $110_2$.  

\section{Addition and Subtraction}
In normal fixed precision arithmetic negative numbers are easily represented by subtraction from the modulus.  For example, with 32-bit integers
$a - b\mbox{ (mod }2^{32}\mbox{)}$ is the same as $a + (2^{32} - b) \mbox{ (mod }2^{32}\mbox{)}$  since $2^{32} \equiv 0 \mbox{ (mod }2^{32}\mbox{)}$.  
As a result subtraction can be performed with a trivial series of logical operations and an addition.

However, in multiple precision arithmetic negative numbers are not represented in the same way.  Instead a sign flag is used to keep track of the
sign of the integer.  As a result signed addition and subtraction are actually implemented as conditional usage of lower level addition or 
subtraction algorithms with the sign fixed up appropriately.

The lower level algorithms will add or subtract integers without regard to the sign flag.  That is they will add or subtract the magnitude of
the integers respectively.

\subsection{Low Level Addition}
An unsigned addition of multiple precision integers is performed with the same long-hand algorithm used to add decimal numbers.  That is to add the 
trailing digits first and propagate the resulting carry upwards.  Since this is a lower level algorithm the name will have a ``s\_'' prefix.  
Historically that convention stems from the MPI library where ``s\_'' stood for static functions that were hidden from the developer entirely.

\newpage
\begin{figure}[!here]
\begin{center}
\begin{small}
\begin{tabular}{l}
\hline Algorithm \textbf{s\_mp\_add}. \\
\textbf{Input}.   Two mp\_ints $a$ and $b$ \\
\textbf{Output}.  The unsigned addition $c = \vert a \vert + \vert b \vert$. \\
\hline \\
1.  if $a.used > b.used$ then \\
\hspace{+3mm}1.1  $min \leftarrow b.used$ \\
\hspace{+3mm}1.2  $max \leftarrow a.used$ \\
\hspace{+3mm}1.3  $x   \leftarrow a$ \\
2.  else  \\
\hspace{+3mm}2.1  $min \leftarrow a.used$ \\
\hspace{+3mm}2.2  $max \leftarrow b.used$ \\
\hspace{+3mm}2.3  $x   \leftarrow b$ \\
3.  If $c.alloc < max + 1$ then grow $c$ to hold at least $max + 1$ digits (\textit{hint: use mp\_grow}) \\
4.  If failed to grow $c$ return(\textit{MP\_MEM}) \\
5.  $oldused \leftarrow c.used$ \\
6.  $c.used \leftarrow max + 1$ \\
7.  $u \leftarrow 0$ \\
8.  for $n$ from $0$ to $min - 1$ do \\
\hspace{+3mm}8.1  $c_n \leftarrow a_n + b_n + u$ \\
\hspace{+3mm}8.2  $u \leftarrow c_n >> lg(\beta)$ \\
\hspace{+3mm}8.3  $c_n \leftarrow c_n \mbox{ (mod }\beta\mbox{)}$ \\
9.  if $min \ne max$ then do \\
\hspace{+3mm}9.1  for $n$ from $min$ to $max - 1$ do \\
\hspace{+6mm}9.1.1  $c_n \leftarrow x_n + u$ \\
\hspace{+6mm}9.1.2  $u \leftarrow c_n >> lg(\beta)$ \\
\hspace{+6mm}9.1.3  $c_n \leftarrow c_n \mbox{ (mod }\beta\mbox{)}$ \\
10.  $c_{max} \leftarrow u$ \\
11.  if $olduse > max$ then \\
\hspace{+3mm}11.1  for $n$ from $max + 1$ to $olduse - 1$ do \\
\hspace{+6mm}11.1.1  $c_n \leftarrow 0$ \\
12.  Clamp excess digits in $c$.  (\textit{hint: use mp\_clamp}) \\
13.  Return(\textit{MP\_OKAY}) \\
\hline
\end{tabular}
\end{small}
\end{center}
\caption{Algorithm s\_mp\_add}
\end{figure}

\textbf{Algorithm s\_mp\_add.}
This algorithm is loosely based on algorithm 14.7 of \cite[pp. 594]{HAC} but has been extended to allow the inputs to have different magnitudes.  
Coincidentally the description of algorithm A in \cite[pp. 266]{TAOCPV2} shares the same flaw as that from \cite{HAC}.  Even the MIX pseudo 
machine code presented  \cite[pp. 266-267]{TAOCPV2} is incapable of handling inputs which are of different magnitudes.

Steps 1 and 2 will sort the two inputs based on their \textbf{used} digit count.  This allows the inputs to have varying magnitudes which not 
only makes it more efficient than the trivial algorithm presented in the other references but more flexible.  The variable $min$ is given the lowest 
digit count while $max$ is given the highest digit count.  If both inputs have the same \textbf{used} digit count both $min$ and $max$ are 
set to the same.  The variable $x$ is an \textit{alias} for the largest input and not meant to be a copy of it.  After the inputs are sorted steps 
3 and 4 will ensure that the destination $c$ can accommodate the result.  The old \textbf{used} count from $c$ is copied to $oldused$ and the 
new count is set to $max + 1$.  

At step 7 the carry variable $u$ is set to zero and the first leg of the addition loop can begin.  The first step of the loop (\textit{8.1}) adds
digits from the two inputs together along with the carry variable $u$.  The following step extracts the carry bit by shifting the result of the
preceding step right $lg(\beta)$ positions.  The shift to extract the carry is similar to how carry extraction works with decimal addition.

Consider adding $77$ to $65$, the first addition of the first column is $7 + 5$ which produces the result $12$.  The trailing digit of the result
is $2 \equiv 12 \mbox{ (mod }10\mbox{)}$ and the carry is found by dividing (\textit{and ignoring the remainder}) $12$ by the radix or in this case $10$.  The
division and multiplication of $10$ is simply a logical shift right or left respectively of the digits.  In otherwords the carry can be extracted
by shifting one digit to the right.

Note that $lg()$ is simply the base two logarithm such that $lg(2^k) = k$.  This implies that $lg(\beta)$ is the number of bits in a radix-$\beta$ 
digit.  Therefore, a logical shift right of the single digit by $lg(\beta)$ will extract the carry.  The final step of the  loop reduces the digit 
modulo the radix $\beta$ to ensure it is in range.

After step 8 the smallest input (\textit{or both if they are the same magnitude}) has been exhausted.  Step 9 decides whether
the inputs were of equal magnitude.  If not than another loop similar to that in step 8 must be executed.  The loop at step
number 9.1 differs from the previous loop since it only adds the mp\_int $x$ along with the carry.  

Step 10 finishes the addition phase by copying the final carry to the highest location in the result $c_{max}$.  Step 11 ensures that 
leading digits that were originally present in $c$ are cleared.  Finally excess leading digits are clamped and the algorithm returns success.

\index{bn\_s\_mp\_add.c}
\vspace{+3mm}\begin{small}
\hspace{-5.1mm}{\bf File}: bn\_s\_mp\_add.c
\vspace{-3mm}
\begin{alltt}
016   
017   /* low level addition, based on HAC pp.594, Algorithm 14.7 */
018   int
019   s_mp_add (mp_int * a, mp_int * b, mp_int * c)
020   \{
021     mp_int *x;
022     int     olduse, res, min, max;
023   
024     /* find sizes, we let |a| <= |b| which means we have to sort
025      * them.  "x" will point to the input with the most digits
026      */
027     if (a->used > b->used) \{
028       min = b->used;
029       max = a->used;
030       x = a;
031     \} else \{
032       min = a->used;
033       max = b->used;
034       x = b;
035     \}
036   
037     /* init result */
038     if (c->alloc < max + 1) \{
039       if ((res = mp_grow (c, max + 1)) != MP_OKAY) \{
040         return res;
041       \}
042     \}
043   
044     /* get old used digit count and set new one */
045     olduse = c->used;
046     c->used = max + 1;
047   
048     /* set the carry to zero */
049     \{
050       register mp_digit u, *tmpa, *tmpb, *tmpc;
051       register int i;
052   
053       /* alias for digit pointers */
054   
055       /* first input */
056       tmpa = a->dp;
057   
058       /* second input */
059       tmpb = b->dp;
060   
061       /* destination */
062       tmpc = c->dp;
063   
064       /* zero the carry */
065       u = 0;
066       for (i = 0; i < min; i++) \{
067         /* Compute the sum at one digit, T[i] = A[i] + B[i] + U */
068         *tmpc = *tmpa++ + *tmpb++ + u;
069   
070         /* U = carry bit of T[i] */
071         u = *tmpc >> ((mp_digit)DIGIT_BIT);
072   
073         /* take away carry bit from T[i] */
074         *tmpc++ &= MP_MASK;
075       \}
076   
077       /* now copy higher words if any, that is in A+B 
078        * if A or B has more digits add those in 
079        */
080       if (min != max) \{
081         for (; i < max; i++) \{
082           /* T[i] = X[i] + U */
083           *tmpc = x->dp[i] + u;
084   
085           /* U = carry bit of T[i] */
086           u = *tmpc >> ((mp_digit)DIGIT_BIT);
087   
088           /* take away carry bit from T[i] */
089           *tmpc++ &= MP_MASK;
090         \}
091       \}
092   
093       /* add carry */
094       *tmpc++ = u;
095   
096       /* clear digits above oldused */
097       for (i = c->used; i < olduse; i++) \{
098         *tmpc++ = 0;
099       \}
100     \}
101   
102     mp_clamp (c);
103     return MP_OKAY;
104   \}
\end{alltt}
\end{small}

Lines 27 to 35 perform the initial sorting of the inputs and determine the $min$ and $max$ variables.  Note that $x$ is pointer to a 
mp\_int assigned to the largest input, in effect it is a local alias.  Lines 37 to 42 ensure that the destination is grown to 
accomodate the result of the addition. 

Similar to the implementation of mp\_copy this function uses the braced code and local aliases coding style.  The three aliases on 
lines 56, 59 and 62 are the for the two inputs and destination respectively.  These aliases are used to ensure the
compiler does not have to dereference $a$, $b$ or $c$ (respectively) to access the digits of the respective mp\_int.

The initial carry $u$ is cleared on line 65, note that $u$ is of type mp\_digit which ensures type compatibility within the 
implementation.  The initial addition loop begins on line 66 and ends on line 75.  Similarly the conditional addition loop
begins on line 81 and ends on line 90.  The addition is finished with the final carry being stored in $tmpc$ on line 94.  
Note the ``++'' operator on the same line.  After line 94 $tmpc$ will point to the $c.used$'th digit of the mp\_int $c$.  This is useful
for the next loop on lines 97 to 99 which set any old upper digits to zero.

\subsection{Low Level Subtraction}
The low level unsigned subtraction algorithm is very similar to the low level unsigned addition algorithm.  The principle difference is that the
unsigned subtraction algorithm requires the result to be positive.  That is when computing $a - b$ the condition $\vert a \vert \ge \vert b\vert$ must 
be met for this algorithm to function properly.  Keep in mind this low level algorithm is not meant to be used in higher level algorithms directly.  
This algorithm as will be shown can be used to create functional signed addition and subtraction algorithms.


For this algorithm a new variable is required to make the description simpler.  Recall from section 1.3.1 that a mp\_digit must be able to represent
the range $0 \le x < 2\beta$.  It is allowable that a mp\_digit represent a larger range of values.  For this algorithm we will assume that
the variable $\gamma$ represents the number of bits available in a mp\_digit (\textit{this implies $2^{\gamma} > \beta$}).

\newpage\begin{figure}[!here]
\begin{center}
\begin{small}
\begin{tabular}{l}
\hline Algorithm \textbf{s\_mp\_sub}. \\
\textbf{Input}.   Two mp\_ints $a$ and $b$ ($\vert a \vert \ge \vert b \vert$) \\
\textbf{Output}.  The unsigned subtraction $c = \vert a \vert - \vert b \vert$. \\
\hline \\
1.  $min \leftarrow b.used$ \\
2.  $max \leftarrow a.used$ \\
3.  If $c.alloc < max$ then grow $c$ to hold at least $max$ digits.  (\textit{hint: use mp\_grow}) \\
4.  If the reallocation failed return(\textit{MP\_MEM}). \\
5.  $oldused \leftarrow c.used$ \\ 
6.  $c.used \leftarrow max$ \\
7.  $u \leftarrow 0$ \\
8.  for $n$ from $0$ to $min - 1$ do \\
\hspace{3mm}8.1  $c_n \leftarrow a_n - b_n - u$ \\
\hspace{3mm}8.2  $u   \leftarrow c_n >> (\gamma - 1)$ \\
\hspace{3mm}8.3  $c_n \leftarrow c_n \mbox{ (mod }\beta\mbox{)}$ \\
9.  if $min < max$ then do \\
\hspace{3mm}9.1  for $n$ from $min$ to $max - 1$ do \\
\hspace{6mm}9.1.1  $c_n \leftarrow a_n - u$ \\
\hspace{6mm}9.1.2  $u   \leftarrow c_n >> (\gamma - 1)$ \\
\hspace{6mm}9.1.3  $c_n \leftarrow c_n \mbox{ (mod }\beta\mbox{)}$ \\
10. if $oldused > max$ then do \\
\hspace{3mm}10.1  for $n$ from $max$ to $oldused - 1$ do \\
\hspace{6mm}10.1.1  $c_n \leftarrow 0$ \\
11. Clamp excess digits of $c$.  (\textit{hint: use mp\_clamp}). \\
12. Return(\textit{MP\_OKAY}). \\
\hline
\end{tabular}
\end{small}
\end{center}
\caption{Algorithm s\_mp\_sub}
\end{figure}

\textbf{Algorithm s\_mp\_sub.}
This algorithm performs the unsigned subtraction of two mp\_int variables under the restriction that the result must be positive.  That is when
passing variables $a$ and $b$ the condition that $\vert a \vert \ge \vert b \vert$ must be met for the algorithm to function correctly.  This
algorithm is loosely based on algorithm 14.9 \cite[pp. 595]{HAC} and is similar to algorithm S in \cite[pp. 267]{TAOCPV2} as well.  As was the case
of the algorithm s\_mp\_add both other references lack discussion concerning various practical details such as when the inputs differ in magnitude.

The initial sorting of the inputs is trivial in this algorithm since $a$ is guaranteed to have at least the same magnitude of $b$.  Steps 1 and 2 
set the $min$ and $max$ variables.  Unlike the addition routine there is guaranteed to be no carry which means that the final result can be at 
most $max$ digits in length as oppose to $max + 1$.  Similar to the addition algorithm the \textbf{used} count of $c$ is copied locally and 
set to the maximal count for the operation.

The subtraction loop that begins on step 8 is essentially the same as the addition loop of algorithm s\_mp\_add except single precision 
subtraction is used instead.  Note the use of the $\gamma$ variable to extract the carry within the subtraction loops.  Under the assumption
that two's complement single precision arithmetic is used this will successfully extract the carry.  

For example, consider subtracting $0101_2$ from
$0100_2$ where $\gamma = 4$.  The least significant bit will force a carry upwards to the third bit which will be set to zero after the borrow.  After
the very first bit has been subtracted $4 - 1 \equiv 0011_2$ will remain,  When the third bit of $0101_2$ is subtracted from the result it will cause
another carry.  In this case though the carry will be forced to propagate all the way to the most significant bit.  

Recall that $\beta < 2^{\gamma}$.  This means that if a carry does occur it will propagate all the way to the most significant bit.  Therefore a single
logical shift right by $\gamma - 1$ positions is sufficient to extract the carry.  This method of carry extraction may seem awkward but the reason for 
it becomes apparent when the implementation is discussed.  

If $b$ has a smaller magnitude than $a$ then step 9 will force the carry and copy operation to propagate through the larger input $a$ into $c$.  Step
10 will ensure that any leading digits of $c$ above the $max$'th position are zeroed.

\index{bn\_s\_mp\_sub.c}
\vspace{+3mm}\begin{small}
\hspace{-5.1mm}{\bf File}: bn\_s\_mp\_sub.c
\vspace{-3mm}
\begin{alltt}
016   
017   /* low level subtraction (assumes |a| > |b|), HAC pp.595 Algorithm 14.9 */
018   int
019   s_mp_sub (mp_int * a, mp_int * b, mp_int * c)
020   \{
021     int     olduse, res, min, max;
022   
023     /* find sizes */
024     min = b->used;
025     max = a->used;
026   
027     /* init result */
028     if (c->alloc < max) \{
029       if ((res = mp_grow (c, max)) != MP_OKAY) \{
030         return res;
031       \}
032     \}
033     olduse = c->used;
034     c->used = max;
035   
036     /* sub digits from lower part */
037     \{
038       register mp_digit u, *tmpa, *tmpb, *tmpc;
039       register int i;
040   
041       /* alias for digit pointers */
042       tmpa = a->dp;
043       tmpb = b->dp;
044       tmpc = c->dp;
045   
046       /* set carry to zero */
047       u = 0;
048       for (i = 0; i < min; i++) \{
049         /* T[i] = A[i] - B[i] - U */
050         *tmpc = *tmpa++ - *tmpb++ - u;
051   
052         /* U = carry bit of T[i]
053          * Note this saves performing an AND operation since
054          * if a carry does occur it will propagate all the way to the
055          * MSB.  As a result a single shift is required to get the carry
056          */
057         u = *tmpc >> ((mp_digit)(CHAR_BIT * sizeof (mp_digit) - 1));
058   
059         /* Clear carry from T[i] */
060         *tmpc++ &= MP_MASK;
061       \}
062   
063       /* now copy higher words if any, e.g. if A has more digits than B  */
064       for (; i < max; i++) \{
065         /* T[i] = A[i] - U */
066         *tmpc = *tmpa++ - u;
067   
068         /* U = carry bit of T[i] */
069         u = *tmpc >> ((mp_digit)(CHAR_BIT * sizeof (mp_digit) - 1));
070   
071         /* Clear carry from T[i] */
072         *tmpc++ &= MP_MASK;
073       \}
074   
075       /* clear digits above used (since we may not have grown result above) */
      
076       for (i = c->used; i < olduse; i++) \{
077         *tmpc++ = 0;
078       \}
079     \}
080   
081     mp_clamp (c);
082     return MP_OKAY;
083   \}
\end{alltt}
\end{small}

Line 24 and 25 perform the initial hardcoded sorting.  In reality they are only aliases and are only used to make the source easier to 
read.  Again the pointer alias optimization is used within this algorithm.  Lines 42, 43 and 44 initialize the aliases for 
$a$, $b$ and $c$ respectively.

The first subtraction loop occurs on lines 47 through 61.  The theory behind the subtraction loop is exactly the same as that for
the addition loop.  As remarked earlier there is an implementation reason for using the ``awkward'' method of extracting the carry 
(\textit{see line 57}).  The traditional method for extracting the carry would be to shift by $lg(\beta)$ positions and logically AND 
the least significant bit.  The AND operation is required because all of the bits above the $\lg(\beta)$'th bit will be set to one after a carry
occurs from subtraction.  This carry extraction requires two relatively cheap operations to extract the carry.  The other method is to simply 
shift the most significant bit to the least significant bit thus extracting the carry with a single cheap operation.  This optimization only works on
twos compliment machines which is a safe assumption to make.

If $a$ has a higher magnitude than $b$ an additional loop (\textit{see lines 64 through 73}) is required to propagate the carry through
$a$ and copy the result to $c$.  

\subsection{High Level Addition}
Now that both lower level addition and subtraction algorithms have been established an effective high level signed addition algorithm can be
established.  This high level addition algorithm will be what other algorithms and developers will use to perform addition of mp\_int data 
types.  

Recall from section 5.2 that an mp\_int represents an integer with an unsigned mantissa (\textit{the array of digits}) and a \textbf{sign} 
flag.  A high level addition is actually performed as a series of eight seperate cases which can be optimized down to three unique cases.

\newpage\begin{figure}[!here]
\begin{center}
\begin{tabular}{l}
\hline Algorithm \textbf{mp\_add}. \\
\textbf{Input}.   Two mp\_ints $a$ and $b$  \\
\textbf{Output}.  The signed addition $c = a + b$. \\
\hline \\
1.  if $a.sign = b.sign$ then do \\
\hspace{3mm}1.1  $c.sign \leftarrow a.sign$  \\
\hspace{3mm}1.2  $c \leftarrow \vert a \vert + \vert b \vert$ (\textit{hint: use s\_mp\_add})\\
2.  else do \\
\hspace{3mm}2.1  if $\vert a \vert < \vert b \vert$ then do (\textit{hint: use mp\_cmp\_mag})  \\
\hspace{6mm}2.1.1  $c.sign \leftarrow b.sign$ \\
\hspace{6mm}2.1.2  $c \leftarrow \vert b \vert - \vert a \vert$ (\textit{hint: use s\_mp\_sub}) \\
\hspace{3mm}2.2  else do \\
\hspace{6mm}2.2.1  $c.sign \leftarrow a.sign$ \\
\hspace{6mm}2.2.2  $c \leftarrow \vert a \vert - \vert b \vert$ \\
3.  If any of the lower level operations failed return(\textit{MP\_MEM}) \\
4.  Return(\textit{MP\_OKAY}). \\
\hline
\end{tabular}
\end{center}
\caption{Algorithm mp\_add}
\end{figure}

\textbf{Algorithm mp\_add.}
This algorithm performs the signed addition of two mp\_int variables.  There is no reference algorithm to draw upon from either \cite{TAOCPV2} or 
\cite{HAC} since they both only provide unsigned operations.  The algorithm is fairly straightforward but restricted since subtraction can only 
produce positive results.  Consider the following chart of possible inputs.

\begin{figure}[!here]
\begin{small}
\begin{center}
\begin{tabular}{|c|c|c|c|c|}
\hline \textbf{Sign of $a$} & \textbf{Sign of $b$} & \textbf{$\vert a \vert > \vert b \vert $} & \textbf{Unsigned Operation} & \textbf{Result Sign Flag} \\
\hline $+$ & $+$ & Yes & $c = a + b$ & $a.sign$ \\
\hline $+$ & $+$ & No  & $c = a + b$ & $a.sign$ \\
\hline $-$ & $-$ & Yes & $c = a + b$ & $a.sign$ \\
\hline $-$ & $-$ & No  & $c = a + b$ & $a.sign$ \\
\hline &&&&\\

\hline $+$ & $-$ & No  & $c = b - a$ & $b.sign$ \\
\hline $-$ & $+$ & No  & $c = b - a$ & $b.sign$ \\

\hline &&&&\\

\hline $+$ & $-$ & Yes & $c = a - b$ & $a.sign$ \\
\hline $-$ & $+$ & Yes & $c = a - b$ & $a.sign$ \\

\hline
\end{tabular}
\end{center}
\end{small}
\caption{Addition Guide Chart}
\end{figure}

The chart lists all of the eight possible input combinations and is sorted to show that only three specific cases need to be handled.  The 
return code of the unsigned operations at step 1.2, 2.1.2 and 2.2.2 are forwarded to step 3 to check for errors.  This simpliies the description
of the algorithm considerably and best follows how the implementation actually was achieved.

Also note how the \textbf{sign} is set before the unsigned addition or subtraction is performed.  Recall from the descriptions of algorithms
s\_mp\_add and s\_mp\_sub that the mp\_clamp function is used at the end to trim excess digits.  The mp\_clamp algorithm will set the \textbf{sign}
to \textbf{MP\_ZPOS} when the \textbf{used} digit count reaches zero.  

For example, consider performing $-a + a$ with algorithm mp\_add.  By the description of the algorithm the sign is set to \textbf{MP\_NEG} which would
produce a result of $-0$.  However, since the sign is set first then the unsigned addition is performed the subsequent usage of algorithm mp\_clamp 
within algorithm s\_mp\_add will force $-0$ to become $0$.  

\index{bn\_mp\_add.c}
\vspace{+3mm}\begin{small}
\hspace{-5.1mm}{\bf File}: bn\_mp\_add.c
\vspace{-3mm}
\begin{alltt}
016   
017   /* high level addition (handles signs) */
018   int
019   mp_add (mp_int * a, mp_int * b, mp_int * c)
020   \{
021     int     sa, sb, res;
022   
023     /* get sign of both inputs */
024     sa = a->sign;
025     sb = b->sign;
026   
027     /* handle two cases, not four */
028     if (sa == sb) \{
029       /* both positive or both negative */
030       /* add their magnitudes, copy the sign */
031       c->sign = sa;
032       res = s_mp_add (a, b, c);
033     \} else \{
034       /* one positive, the other negative */
035       /* subtract the one with the greater magnitude from */
036       /* the one of the lesser magnitude.  The result gets */
037       /* the sign of the one with the greater magnitude. */
038       if (mp_cmp_mag (a, b) == MP_LT) \{
039         c->sign = sb;
040         res = s_mp_sub (b, a, c);
041       \} else \{
042         c->sign = sa;
043         res = s_mp_sub (a, b, c);
044       \}
045     \}
046     return res;
047   \}
048   
\end{alltt}
\end{small}

The source code follows the algorithm fairly closely.  The most notable new source code addition is the usage of the $res$ integer variable which
is used to pass result of the unsigned operations forward.  Unlike in the algorithm, the variable $res$ is merely returned as is without
explicitly checking it and returning the constant \textbf{MP\_OKAY}.  The observation is this algorithm will succeed or fail only if the lower
level functions do so.  Returning their return code is sufficient.

\subsection{High Level Subtraction}
The high level signed subtraction algorithm is essentially the same as the high level signed addition algorithm.  

\begin{figure}[!here]
\begin{center}
\begin{tabular}{l}
\hline Algorithm \textbf{mp\_sub}. \\
\textbf{Input}.   Two mp\_ints $a$ and $b$  \\
\textbf{Output}.  The signed subtraction $c = a - b$. \\
\hline \\
1.  if $a.sign \ne b.sign$ then do \\
\hspace{3mm}1.1  $c.sign \leftarrow a.sign$ \\
\hspace{3mm}1.2  $c \leftarrow \vert a \vert + \vert b \vert$ (\textit{hint: use s\_mp\_add}) \\
2.  else do \\
\hspace{3mm}2.1  if $\vert a \vert \ge \vert b \vert$ then do (\textit{hint: use mp\_cmp\_mag}) \\
\hspace{6mm}2.1.1  $c.sign \leftarrow a.sign$ \\
\hspace{6mm}2.1.2  $c \leftarrow \vert a \vert  - \vert b \vert$ (\textit{hint: use s\_mp\_sub}) \\
\hspace{3mm}2.2  else do \\
\hspace{6mm}2.2.1  $c.sign \leftarrow  \left \lbrace \begin{array}{ll}
                              MP\_ZPOS &  \mbox{if }a.sign = MP\_NEG \\
                              MP\_NEG  &  \mbox{otherwise} \\
                              \end{array} \right .$ \\
\hspace{6mm}2.2.2  $c \leftarrow \vert b \vert  - \vert a \vert$ \\
3.  If any of the lower level operations failed return(\textit{MP\_MEM}). \\
4.  Return(\textit{MP\_OKAY}). \\
\hline
\end{tabular}
\end{center}
\caption{Algorithm mp\_sub}
\end{figure}

\textbf{Algorithm mp\_sub.}
This algorithm performs the signed subtraction of two inputs.  Similar to algorithm mp\_add there is no reference in either \cite{TAOCPV2} or 
\cite{HAC}.  Also this algorithm is restricted by algorithm s\_mp\_sub.  The following chart lists the eight possible inputs and
the operations required.

\newpage\begin{figure}[!here]
\begin{small}
\begin{center}
\begin{tabular}{|c|c|c|c|c|}
\hline \textbf{Sign of $a$} & \textbf{Sign of $b$} & \textbf{$\vert a \vert \ge \vert b \vert $} & \textbf{Unsigned Operation} & \textbf{Result Sign Flag} \\
\hline $+$ & $-$ & Yes & $c = a + b$ & $a.sign$ \\
\hline $+$ & $-$ & No  & $c = a + b$ & $a.sign$ \\
\hline $-$ & $+$ & Yes & $c = a + b$ & $a.sign$ \\
\hline $-$ & $+$ & No  & $c = a + b$ & $a.sign$ \\
\hline &&&& \\
\hline $+$ & $+$ & Yes & $c = a - b$ & $a.sign$ \\
\hline $-$ & $-$ & Yes & $c = a - b$ & $a.sign$ \\
\hline &&&& \\
\hline $+$ & $+$ & No  & $c = b - a$ & $\mbox{opposite of }a.sign$ \\
\hline $-$ & $-$ & No  & $c = b - a$ & $\mbox{opposite of }a.sign$ \\
\hline
\end{tabular}
\end{center}
\end{small}
\caption{Subtraction Guide Chart}
\end{figure}

Similar to the case of algorithm mp\_add the \textbf{sign} is set first before the unsigned addition or subtraction.  That is to prevent the 
algorithm from producing $-a - -a = -0$ as a result.  

\index{bn\_mp\_sub.c}
\vspace{+3mm}\begin{small}
\hspace{-5.1mm}{\bf File}: bn\_mp\_sub.c
\vspace{-3mm}
\begin{alltt}
016   
017   /* high level subtraction (handles signs) */
018   int
019   mp_sub (mp_int * a, mp_int * b, mp_int * c)
020   \{
021     int     sa, sb, res;
022   
023     sa = a->sign;
024     sb = b->sign;
025   
026     if (sa != sb) \{
027       /* subtract a negative from a positive, OR */
028       /* subtract a positive from a negative. */
029       /* In either case, ADD their magnitudes, */
030       /* and use the sign of the first number. */
031       c->sign = sa;
032       res = s_mp_add (a, b, c);
033     \} else \{
034       /* subtract a positive from a positive, OR */
035       /* subtract a negative from a negative. */
036       /* First, take the difference between their */
037       /* magnitudes, then... */
038       if (mp_cmp_mag (a, b) != MP_LT) \{
039         /* Copy the sign from the first */
040         c->sign = sa;
041         /* The first has a larger or equal magnitude */
042         res = s_mp_sub (a, b, c);
043       \} else \{
044         /* The result has the *opposite* sign from */
045         /* the first number. */
046         c->sign = (sa == MP_ZPOS) ? MP_NEG : MP_ZPOS;
047         /* The second has a larger magnitude */
048         res = s_mp_sub (b, a, c);
049       \}
050     \}
051     return res;
052   \}
053   
\end{alltt}
\end{small}

Much like the implementation of algorithm mp\_add the variable $res$ is used to catch the return code of the unsigned addition or subtraction operations
and forward it to the end of the function.  On line 38 the ``not equal to'' \textbf{MP\_LT} expression is used to emulate a 
``greater than or equal to'' comparison.  

\section{Bit and Digit Shifting}
It is quite common to think of a multiple precision integer as a polynomial in $x$, that is $y = f(\beta)$ where $f(x) = \sum_{i=0}^{n-1} a_i x^i$.  
This notation arises within discussion of Montgomery and Diminished Radix Reduction as well as Karatsuba multiplication and squaring.  

In order to facilitate operations on polynomials in $x$ as above a series of simple ``digit'' algorithms have to be established.  That is to shift
the digits left or right as well to shift individual bits of the digits left and right.  It is important to note that not all ``shift'' operations
are on radix-$\beta$ digits.  

\subsection{Multiplication by Two}

In a binary system where the radix is a power of two multiplication by two not only arises often in other algorithms it is a fairly efficient 
operation to perform.  A single precision logical shift left is sufficient to multiply a single digit by two.  

\newpage\begin{figure}[!here]
\begin{small}
\begin{center}
\begin{tabular}{l}
\hline Algorithm \textbf{mp\_mul\_2}. \\
\textbf{Input}.   One mp\_int $a$ \\
\textbf{Output}.  $b = 2a$. \\
\hline \\
1.  If $b.alloc < a.used + 1$ then grow $b$ to hold $a.used + 1$ digits.  (\textit{hint: use mp\_grow}) \\
2.  If the reallocation failed return(\textit{MP\_MEM}). \\
3.  $oldused \leftarrow b.used$ \\
4.  $b.used \leftarrow a.used$ \\
5.  $r \leftarrow 0$ \\
6.  for $n$ from 0 to $a.used - 1$ do \\
\hspace{3mm}6.1  $rr \leftarrow a_n >> (lg(\beta) - 1)$ \\
\hspace{3mm}6.2  $b_n \leftarrow (a_n << 1) + r \mbox{ (mod }\beta\mbox{)}$ \\
\hspace{3mm}6.3  $r \leftarrow rr$ \\
7.  If $r \ne 0$ then do \\
\hspace{3mm}7.1  $b_{a.used} = 1$ \\
\hspace{3mm}7.2  $b.used \leftarrow b.used + 1$ \\
8.  If $b.used < oldused - 1$ then do \\
\hspace{3mm}8.1  for $n$ from $b.used$ to $oldused - 1$ do \\
\hspace{6mm}8.1.1  $b_n \leftarrow 0$ \\
9.  $b.sign \leftarrow a.sign$ \\
10.  Return(\textit{MP\_OKAY}).\\
\hline
\end{tabular}
\end{center}
\end{small}
\caption{Algorithm mp\_mul\_2}
\end{figure}

\textbf{Algorithm mp\_mul\_2.}
This algorithm will quickly multiply a mp\_int by two provided $\beta$ is a power of two.  Neither \cite{TAOCPV2} nor \cite{HAC} describe such 
an algorithm despite the fact it arises often in other algorithms.  The algorithm is setup much like the lower level algorithm s\_mp\_add since 
it is for all intents and purposes equivalent to the operation $b = \vert a \vert + \vert a \vert$.  

Step 1 and 2 grow the input as required to accomodate the maximum number of \textbf{used} digits in the result.  The initial \textbf{used} count
is set to $a.used$ at step 4.  Only if there is a final carry will the \textbf{used} count require adjustment.

Step 6 is an optimization implementation of the addition loop for this specific case.  That is since the two values being added together 
are the same there is no need to perform two reads from the digits of $a$.  Step 6.1 performs a single precision shift on the current digit $a_n$ to
obtain what will be the carry for the next iteration.  Step 6.2 calculates the $n$'th digit of the result as single precision shift of $a_n$ plus
the previous carry.  Recall from section 5.1 that $a_n << 1$ is equivalent to $a_n \cdot 2$.  An iteration of the addition loop is finished with 
forwarding the carry to the next iteration.

Step 7 takes care of any final carry by setting the $a.used$'th digit of the result to one and augmenting the \textbf{used} count.  Step 8 clears
any original leading digits of $b$.

\index{bn\_mp\_mul\_2.c}
\vspace{+3mm}\begin{small}
\hspace{-5.1mm}{\bf File}: bn\_mp\_mul\_2.c
\vspace{-3mm}
\begin{alltt}
016   
017   /* b = a*2 */
018   int
019   mp_mul_2 (mp_int * a, mp_int * b)
020   \{
021     int     x, res, oldused;
022   
023     /* grow to accomodate result */
024     if (b->alloc < a->used + 1) \{
025       if ((res = mp_grow (b, a->used + 1)) != MP_OKAY) \{
026         return res;
027       \}
028     \}
029   
030     oldused = b->used;
031     b->used = a->used;
032   
033     \{
034       register mp_digit r, rr, *tmpa, *tmpb;
035   
036       /* alias for source */
037       tmpa = a->dp;
038       
039       /* alias for dest */
040       tmpb = b->dp;
041   
042       /* carry */
043       r = 0;
044       for (x = 0; x < a->used; x++) \{
045       
046         /* get what will be the *next* carry bit from the 
047          * MSB of the current digit 
048          */
049         rr = *tmpa >> ((mp_digit)(DIGIT_BIT - 1));
050         
051         /* now shift up this digit, add in the carry [from the previous] */
052         *tmpb++ = ((*tmpa++ << ((mp_digit)1)) | r) & MP_MASK;
053         
054         /* copy the carry that would be from the source 
055          * digit into the next iteration 
056          */
057         r = rr;
058       \}
059   
060       /* new leading digit? */
061       if (r != 0) \{
062         /* add a MSB which is always 1 at this point */
063         *tmpb = 1;
064         ++b->used;
065       \}
066   
067       /* now zero any excess digits on the destination 
068        * that we didn't write to 
069        */
070       tmpb = b->dp + b->used;
071       for (x = b->used; x < oldused; x++) \{
072         *tmpb++ = 0;
073       \}
074     \}
075     b->sign = a->sign;
076     return MP_OKAY;
077   \}
\end{alltt}
\end{small}

This implementation is essentially an optimized implementation of s\_mp\_add for the case of doubling an input.  The only noteworthy difference
is the use of the logical shift operator on line 52 to perform a single precision doubling.  

\subsection{Division by Two}
A division by two can just as easily be accomplished with a logical shift right as multiplication by two can be with a logical shift left.

\newpage\begin{figure}[!here]
\begin{small}
\begin{center}
\begin{tabular}{l}
\hline Algorithm \textbf{mp\_div\_2}. \\
\textbf{Input}.   One mp\_int $a$ \\
\textbf{Output}.  $b = a/2$. \\
\hline \\
1.  If $b.alloc < a.used$ then grow $b$ to hold $a.used$ digits.  (\textit{hint: use mp\_grow}) \\
2.  If the reallocation failed return(\textit{MP\_MEM}). \\
3.  $oldused \leftarrow b.used$ \\
4.  $b.used \leftarrow a.used$ \\
5.  $r \leftarrow 0$ \\
6.  for $n$ from $b.used - 1$ to $0$ do \\
\hspace{3mm}6.1  $rr \leftarrow a_n \mbox{ (mod }2\mbox{)}$\\
\hspace{3mm}6.2  $b_n \leftarrow (a_n >> 1) + (r << (lg(\beta) - 1)) \mbox{ (mod }\beta\mbox{)}$ \\
\hspace{3mm}6.3  $r \leftarrow rr$ \\
7.  If $b.used < oldused - 1$ then do \\
\hspace{3mm}7.1  for $n$ from $b.used$ to $oldused - 1$ do \\
\hspace{6mm}7.1.1  $b_n \leftarrow 0$ \\
8.  $b.sign \leftarrow a.sign$ \\
9.  Return(\textit{MP\_OKAY}).\\
\hline
\end{tabular}
\end{center}
\end{small}
\caption{Algorithm mp\_div\_2}
\end{figure}

\textbf{Algorithm mp\_div\_2.}
This algorithm will divide an mp\_int by two using logical shifts to the right.  Like mp\_mul\_2 it uses a modified low level addition
core as the basis of the algorithm.  Unlike mp\_mul\_2 the shift operations work from the leading digit to the trailing digit.  The algorithm
could be written to work from the trailing digit to the leading digit however, it would have to stop one short of $a.used - 1$ digits to prevent
reading passed the end of the array of digits.

Essentially the loop at step 6 is similar to that of mp\_mul\_2 except the logical shifts go in the opposite direction and the carry is at the 
least significant bit not the most significant bit.  

\index{bn\_mp\_div\_2.c}
\vspace{+3mm}\begin{small}
\hspace{-5.1mm}{\bf File}: bn\_mp\_div\_2.c
\vspace{-3mm}
\begin{alltt}
016   
017   /* b = a/2 */
018   int
019   mp_div_2 (mp_int * a, mp_int * b)
020   \{
021     int     x, res, oldused;
022   
023     /* copy */
024     if (b->alloc < a->used) \{
025       if ((res = mp_grow (b, a->used)) != MP_OKAY) \{
026         return res;
027       \}
028     \}
029   
030     oldused = b->used;
031     b->used = a->used;
032     \{
033       register mp_digit r, rr, *tmpa, *tmpb;
034   
035       /* source alias */
036       tmpa = a->dp + b->used - 1;
037   
038       /* dest alias */
039       tmpb = b->dp + b->used - 1;
040   
041       /* carry */
042       r = 0;
043       for (x = b->used - 1; x >= 0; x--) \{
044         /* get the carry for the next iteration */
045         rr = *tmpa & 1;
046   
047         /* shift the current digit, add in carry and store */
048         *tmpb-- = (*tmpa-- >> 1) | (r << (DIGIT_BIT - 1));
049   
050         /* forward carry to next iteration */
051         r = rr;
052       \}
053   
054       /* zero excess digits */
055       tmpb = b->dp + b->used;
056       for (x = b->used; x < oldused; x++) \{
057         *tmpb++ = 0;
058       \}
059     \}
060     b->sign = a->sign;
061     mp_clamp (b);
062     return MP_OKAY;
063   \}
\end{alltt}
\end{small}

\section{Polynomial Basis Operations}
Recall from section 5.3 that any integer can be represented as a polynomial in $x$ as $y = f(\beta)$.  Such a representation is also known as
the polynomial basis \cite[pp. 48]{ROSE}. Given such a notation a multiplication or division by $x$ amounts to shifting whole digits a single 
place.  The need for such operations arises in several other higher level algorithms such as Barrett and Montgomery reduction, integer
division and Karatsuba multiplication.  

Converting from an array of digits to polynomial basis is very simple.  Consider the integer $y \equiv (a_2, a_1, a_0)_{\beta}$ and recall that
$y = \sum_{i=0}^{2} a_i \beta^i$.  Simply replace $\beta$ with $x$ and the expression is in polynomial basis.  For example, $f(x) = 8x + 9$ is the
polynomial basis representation for $89$ using radix ten.  That is, $f(10) = 8(10) + 9 = 89$.  

\subsection{Multiplication by $x$}

Given a polynomial in $x$ such as $f(x) = a_n x^n + a_{n-1} x^{n-1} + ... + a_0$ multiplying by $x$ amounts to shifting the coefficients up one 
degree.  In this case $f(x) \cdot x = a_n x^{n+1} + a_{n-1} x^n + ... + a_0 x$.  From a scalar basis point of view multiplying by $x$ is equivalent to
multiplying by the integer $\beta$.  

\newpage\begin{figure}[!here]
\begin{small}
\begin{center}
\begin{tabular}{l}
\hline Algorithm \textbf{mp\_lshd}. \\
\textbf{Input}.   One mp\_int $a$ and an integer $b$ \\
\textbf{Output}.  $a \leftarrow a \cdot \beta^b$ (Multiply by $x^b$). \\
\hline \\
1.  If $b \le 0$ then return(\textit{MP\_OKAY}). \\
2.  If $a.alloc < a.used + b$ then grow $a$ to at least $a.used + b$ digits.  (\textit{hint: use mp\_grow}). \\
3.  If the reallocation failed return(\textit{MP\_MEM}). \\
4.  $a.used \leftarrow a.used + b$ \\
5.  $i \leftarrow a.used - 1$ \\
6.  $j \leftarrow a.used - 1 - b$ \\
7.  for $n$ from $a.used - 1$ to $b$ do \\
\hspace{3mm}7.1  $a_{i} \leftarrow a_{j}$ \\
\hspace{3mm}7.2  $i \leftarrow i - 1$ \\
\hspace{3mm}7.3  $j \leftarrow j - 1$ \\
8.  for $n$ from 0 to $b - 1$ do \\
\hspace{3mm}8.1  $a_n \leftarrow 0$ \\
9.  Return(\textit{MP\_OKAY}). \\
\hline
\end{tabular}
\end{center}
\end{small}
\caption{Algorithm mp\_lshd}
\end{figure}

\textbf{Algorithm mp\_lshd.}
This algorithm multiplies an mp\_int by the $b$'th power of $x$.  This is equivalent to multiplying by $\beta^b$.  The algorithm differs 
from the other algorithms presented so far as it performs the operation in place instead storing the result in a seperate location.  The algorithm
will return success immediately if $b \le 0$ since the rest of algorithm is only valid when $b > 0$.  

First the destination $a$ is grown as required to accomodate the result.  The counters $i$ and $j$ are used to form a \textit{sliding window} over
the digits of $a$ of length $b$.  The head of the sliding window is at $i$ (\textit{the leading digit}) and the tail at $j$ (\textit{the trailing digit}).  
The loop on step 7 copies the digit from the tail to the head.  In each iteration the window is moved down one digit.   The last loop on 
step 8 sets the lower $b$ digits to zero.

\newpage
\begin{center}
\begin{figure}[here]
\includegraphics{pics/sliding_window.ps}
\caption{Sliding Window Movement}
\end{figure}
\end{center}

\index{bn\_mp\_lshd.c}
\vspace{+3mm}\begin{small}
\hspace{-5.1mm}{\bf File}: bn\_mp\_lshd.c
\vspace{-3mm}
\begin{alltt}
016   
017   /* shift left a certain amount of digits */
018   int
019   mp_lshd (mp_int * a, int b)
020   \{
021     int     x, res;
022   
023     /* if its less than zero return */
024     if (b <= 0) \{
025       return MP_OKAY;
026     \}
027   
028     /* grow to fit the new digits */
029     if (a->alloc < a->used + b) \{
030        if ((res = mp_grow (a, a->used + b)) != MP_OKAY) \{
031          return res;
032        \}
033     \}
034   
035     \{
036       register mp_digit *tmpa, *tmpaa;
037   
038       /* increment the used by the shift amount than copy upwards */
039       a->used += b;
040   
041       /* top */
042       tmpa = a->dp + a->used - 1;
043   
044       /* base */
045       tmpaa = a->dp + a->used - 1 - b;
046   
047       /* much like mp_rshd this is implemented using a sliding window
048        * except the window goes the otherway around.  Copying from
049        * the bottom to the top.  see bn_mp_rshd.c for more info.
050        */
051       for (x = a->used - 1; x >= b; x--) \{
052         *tmpa-- = *tmpaa--;
053       \}
054   
055       /* zero the lower digits */
056       tmpa = a->dp;
057       for (x = 0; x < b; x++) \{
058         *tmpa++ = 0;
059       \}
060     \}
061     return MP_OKAY;
062   \}
\end{alltt}
\end{small}

The if statement on line 24 ensures that the $b$ variable is greater than zero.  The \textbf{used} count is incremented by $b$ before
the copy loop begins.  This elminates the need for an additional variable in the for loop.  The variable $tmpa$ on line 42 is an alias
for the leading digit while $tmpaa$ on line 45 is an alias for the trailing edge.  The aliases form a window of exactly $b$ digits
over the input.  

\subsection{Division by $x$}

Division by powers of $x$ is easily achieved by shifting the digits right and removing any that will end up to the right of the zero'th digit.  

\newpage\begin{figure}[!here]
\begin{small}
\begin{center}
\begin{tabular}{l}
\hline Algorithm \textbf{mp\_rshd}. \\
\textbf{Input}.   One mp\_int $a$ and an integer $b$ \\
\textbf{Output}.  $a \leftarrow a / \beta^b$ (Divide by $x^b$). \\
\hline \\
1.  If $b \le 0$ then return. \\
2.  If $a.used \le b$ then do \\
\hspace{3mm}2.1  Zero $a$.  (\textit{hint: use mp\_zero}). \\
\hspace{3mm}2.2  Return. \\
3.  $i \leftarrow 0$ \\
4.  $j \leftarrow b$ \\
5.  for $n$ from 0 to $a.used - b - 1$ do \\
\hspace{3mm}5.1  $a_i \leftarrow a_j$ \\
\hspace{3mm}5.2  $i \leftarrow i + 1$ \\
\hspace{3mm}5.3  $j \leftarrow j + 1$ \\
6.  for $n$ from $a.used - b$ to $a.used - 1$ do \\
\hspace{3mm}6.1  $a_n \leftarrow 0$ \\
7.  Clamp excess digits.  (\textit{hint: use mp\_clamp}). \\
8.  Return. \\
\hline
\end{tabular}
\end{center}
\end{small}
\caption{Algorithm mp\_rshd}
\end{figure}

\textbf{Algorithm mp\_rshd.}
This algorithm divides the input in place by the $b$'th power of $x$.  It is analogous to dividing by a $\beta^b$ but much quicker since
it does not require single precision division.  This algorithm does not actually return an error code as it cannot fail.  

If the input $b$ is less than one the algorithm quickly returns without performing any work.  If the \textbf{used} count is less than or equal
to the shift count $b$ then it will simply zero the input and return.

After the trivial cases of inputs have been handled the sliding window is setup.  Much like the case of algorithm mp\_lshd a sliding window that
is $b$ digits wide is used to copy the digits.  Unlike mp\_lshd the window slides in the opposite direction from the trailing to the leading digit.  
Also the digits are copied from the leading to the trailing edge.

Once the window copy is complete the upper digits must be zeroed.  Finally algorithm mp\_clamp is used to trim excess digits.

\index{bn\_mp\_rshd.c}
\vspace{+3mm}\begin{small}
\hspace{-5.1mm}{\bf File}: bn\_mp\_rshd.c
\vspace{-3mm}
\begin{alltt}
016   
017   /* shift right a certain amount of digits */
018   void
019   mp_rshd (mp_int * a, int b)
020   \{
021     int     x;
022   
023     /* if b <= 0 then ignore it */
024     if (b <= 0) \{
025       return;
026     \}
027   
028     /* if b > used then simply zero it and return */
029     if (a->used <= b) \{
030       mp_zero (a);
031       return;
032     \}
033   
034     \{
035       register mp_digit *tmpa, *tmpaa;
036   
037       /* shift the digits down */
038   
039       /* base */
040       tmpa = a->dp;
041   
042       /* offset into digits */
043       tmpaa = a->dp + b;
044   
045       /* this is implemented as a sliding window where 
046        * the window is b-digits long and digits from 
047        * the top of the window are copied to the bottom
048        *
049        * e.g.
050   
051        b-2 | b-1 | b0 | b1 | b2 | ... | bb |   ---->
052                    /\symbol{92}                   |      ---->
053                     \symbol{92}-------------------/      ---->
054        */
055       for (x = 0; x < (a->used - b); x++) \{
056         *tmpa++ = *tmpaa++;
057       \}
058   
059       /* zero the top digits */
060       for (; x < a->used; x++) \{
061         *tmpa++ = 0;
062       \}
063     \}
064     mp_clamp (a);
065   \}
\end{alltt}
\end{small}

The only noteworthy element of this routine is the lack of a return type.  This function cannot fail and as such it is more optimal to not
return anything.

\section{Powers of Two}

Now that algorithms for moving single bits as well as whole digits exist algorithms for moving the ``in between'' distances are required.  For 
example, to quickly multiply by $2^k$ for any $k$ without using a full multiplier algorithm would prove useful.  Instead of performing single
shifts $k$ times to achieve a multiplication by $2^{\pm k}$ a mixture of whole digit shifting and partial digit shifting is employed.  

\subsection{Multiplication by Power of Two}

\newpage\begin{figure}[!here]
\begin{small}
\begin{center}
\begin{tabular}{l}
\hline Algorithm \textbf{mp\_mul\_2d}. \\
\textbf{Input}.   One mp\_int $a$ and an integer $b$ \\
\textbf{Output}.  $c \leftarrow a \cdot 2^b$. \\
\hline \\
1.  $c \leftarrow a$.  (\textit{hint: use mp\_copy}) \\
2.  If $c.alloc < c.used + \lfloor b / lg(\beta) \rfloor + 2$ then grow $c$ accordingly. \\
3.  If the reallocation failed return(\textit{MP\_MEM}). \\
4.  If $b \ge lg(\beta)$ then \\
\hspace{3mm}4.1  $c \leftarrow c \cdot \beta^{\lfloor b / lg(\beta) \rfloor}$ (\textit{hint: use mp\_lshd}). \\
\hspace{3mm}4.2  If step 4.1 failed return(\textit{MP\_MEM}). \\
5.  $d \leftarrow b \mbox{ (mod }lg(\beta)\mbox{)}$ \\
6.  If $d \ne 0$ then do \\
\hspace{3mm}6.1  $mask \leftarrow 2^d$ \\
\hspace{3mm}6.2  $r \leftarrow 0$ \\
\hspace{3mm}6.3  for $n$ from $0$ to $c.used - 1$ do \\
\hspace{6mm}6.3.1  $rr \leftarrow c_n >> (lg(\beta) - d) \mbox{ (mod }mask\mbox{)}$ \\
\hspace{6mm}6.3.2  $c_n \leftarrow (c_n << d) + r \mbox{ (mod }\beta\mbox{)}$ \\
\hspace{6mm}6.3.3  $r \leftarrow rr$ \\
\hspace{3mm}6.4  If $r > 0$ then do \\
\hspace{6mm}6.4.1  $c_{c.used} \leftarrow r$ \\
\hspace{6mm}6.4.2  $c.used \leftarrow c.used + 1$ \\
7.  Return(\textit{MP\_OKAY}). \\
\hline
\end{tabular}
\end{center}
\end{small}
\caption{Algorithm mp\_mul\_2d}
\end{figure}

\textbf{Algorithm mp\_mul\_2d.}
This algorithm multiplies $a$ by $2^b$ and stores the result in $c$.  The algorithm uses algorithm mp\_lshd and a derivative of algorithm mp\_mul\_2 to
quickly compute the product.

First the algorithm will multiply $a$ by $x^{\lfloor b / lg(\beta) \rfloor}$ which will ensure that the remainder multiplicand is less than 
$\beta$.  For example, if $b = 37$ and $\beta = 2^{28}$ then this step will multiply by $x$ leaving a multiplication by $2^{37 - 28} = 2^{9}$ 
left.

The logarithm of the residue is calculated on step 5.  If it is non-zero a modified shift loop is used to calculate the remaining product.  
Essentially the loop is a generic version of algorith mp\_mul2 designed to handle any shift count in the range $1 \le x < lg(\beta)$.  The $mask$
variable is used to extract the upper $d$ bits to form the carry for the next iteration.  

This algorithm is loosely measured as a $O(2n)$ algorithm which means that if the input is $n$-digits that it takes $2n$ ``time'' to 
complete.  It is possible to optimize this algorithm down to a $O(n)$ algorithm at a cost of making the algorithm slightly harder to follow.

\index{bn\_mp\_mul\_2d.c}
\vspace{+3mm}\begin{small}
\hspace{-5.1mm}{\bf File}: bn\_mp\_mul\_2d.c
\vspace{-3mm}
\begin{alltt}
016   
017   /* NOTE:  This routine requires updating.  For instance the c->used = c->all
      oc bit
018      is wrong.  We should just shift c->used digits then set the carry as c->d
      p[c->used] = carry
019    
020      To be fixed for LTM 0.18
021    */
022   
023   /* shift left by a certain bit count */
024   int
025   mp_mul_2d (mp_int * a, int b, mp_int * c)
026   \{
027     mp_digit d;
028     int      res;
029   
030     /* copy */
031     if (a != c) \{
032        if ((res = mp_copy (a, c)) != MP_OKAY) \{
033          return res;
034        \}
035     \}
036   
037     if (c->alloc < (int)(c->used + b/DIGIT_BIT + 2)) \{
038        if ((res = mp_grow (c, c->used + b / DIGIT_BIT + 2)) != MP_OKAY) \{
039          return res;
040        \}
041     \}
042   
043     /* shift by as many digits in the bit count */
044     if (b >= (int)DIGIT_BIT) \{
045       if ((res = mp_lshd (c, b / DIGIT_BIT)) != MP_OKAY) \{
046         return res;
047       \}
048     \}
049     c->used = c->alloc;
050   
051     /* shift any bit count < DIGIT_BIT */
052     d = (mp_digit) (b % DIGIT_BIT);
053     if (d != 0) \{
054       register mp_digit *tmpc, mask, r, rr;
055       register int x;
056   
057       /* bitmask for carries */
058       mask = (((mp_digit)1) << d) - 1;
059   
060       /* alias */
061       tmpc = c->dp;
062   
063       /* carry */
064       r    = 0;
065       for (x = 0; x < c->used; x++) \{
066         /* get the higher bits of the current word */
067         rr = (*tmpc >> (DIGIT_BIT - d)) & mask;
068   
069         /* shift the current word and OR in the carry */
070         *tmpc = ((*tmpc << d) | r) & MP_MASK;
071         ++tmpc;
072   
073         /* set the carry to the carry bits of the current word */
074         r = rr;
075       \}
076     \}
077     mp_clamp (c);
078     return MP_OKAY;
079   \}
\end{alltt}
\end{small}

Notes to be revised when code is updated. -- Tom

\subsection{Division by Power of Two}

\newpage\begin{figure}[!here]
\begin{small}
\begin{center}
\begin{tabular}{l}
\hline Algorithm \textbf{mp\_div\_2d}. \\
\textbf{Input}.   One mp\_int $a$ and an integer $b$ \\
\textbf{Output}.  $c \leftarrow \lfloor a / 2^b \rfloor, d \leftarrow a \mbox{ (mod }2^b\mbox{)}$. \\
\hline \\
1.  If $b \le 0$ then do \\
\hspace{3mm}1.1  $c \leftarrow a$ (\textit{hint: use mp\_copy}) \\
\hspace{3mm}1.2  $d \leftarrow 0$ (\textit{hint: use mp\_zero}) \\
\hspace{3mm}1.3  Return(\textit{MP\_OKAY}). \\
2.  $c \leftarrow a$ \\
3.  $d \leftarrow a \mbox{ (mod }2^b\mbox{)}$ (\textit{hint: use mp\_mod\_2d}) \\
4.  If $b \ge lg(\beta)$ then do \\
\hspace{3mm}4.1  $c \leftarrow \lfloor c/\beta^{\lfloor b/lg(\beta) \rfloor} \rfloor$ (\textit{hint: use mp\_rshd}). \\
5.  $k \leftarrow b \mbox{ (mod }lg(\beta)\mbox{)}$ \\
6.  If $k \ne 0$ then do \\
\hspace{3mm}6.1  $mask \leftarrow 2^k$ \\
\hspace{3mm}6.2  $r \leftarrow 0$ \\
\hspace{3mm}6.3  for $n$ from $c.used - 1$ to $0$ do \\
\hspace{6mm}6.3.1  $rr \leftarrow c_n \mbox{ (mod }mask\mbox{)}$ \\
\hspace{6mm}6.3.2  $c_n \leftarrow (c_n >> k) + (r << (lg(\beta) - k))$ \\
\hspace{6mm}6.3.3  $r \leftarrow rr$ \\
7.  Clamp excess digits of $c$.  (\textit{hint: use mp\_clamp}) \\
8.  Return(\textit{MP\_OKAY}). \\
\hline
\end{tabular}
\end{center}
\end{small}
\caption{Algorithm mp\_div\_2d}
\end{figure}

\textbf{Algorithm mp\_div\_2d.}
This algorithm will divide an input $a$ by $2^b$ and produce the quotient and remainder.  The algorithm is designed much like algorithm 
mp\_mul\_2d by first using whole digit shifts then single precision shifts.  This algorithm will also produce the remainder of the division
by using algorithm mp\_mod\_2d.

\index{bn\_mp\_div\_2d.c}
\vspace{+3mm}\begin{small}
\hspace{-5.1mm}{\bf File}: bn\_mp\_div\_2d.c
\vspace{-3mm}
\begin{alltt}
016   
017   /* shift right by a certain bit count (store quotient in c, remainder in d) 
      */
018   int
019   mp_div_2d (mp_int * a, int b, mp_int * c, mp_int * d)
020   \{
021     mp_digit D, r, rr;
022     int     x, res;
023     mp_int  t;
024   
025   
026     /* if the shift count is <= 0 then we do no work */
027     if (b <= 0) \{
028       res = mp_copy (a, c);
029       if (d != NULL) \{
030         mp_zero (d);
031       \}
032       return res;
033     \}
034   
035     if ((res = mp_init (&t)) != MP_OKAY) \{
036       return res;
037     \}
038   
039     /* get the remainder */
040     if (d != NULL) \{
041       if ((res = mp_mod_2d (a, b, &t)) != MP_OKAY) \{
042         mp_clear (&t);
043         return res;
044       \}
045     \}
046   
047     /* copy */
048     if ((res = mp_copy (a, c)) != MP_OKAY) \{
049       mp_clear (&t);
050       return res;
051     \}
052   
053     /* shift by as many digits in the bit count */
054     if (b >= (int)DIGIT_BIT) \{
055       mp_rshd (c, b / DIGIT_BIT);
056     \}
057   
058     /* shift any bit count < DIGIT_BIT */
059     D = (mp_digit) (b % DIGIT_BIT);
060     if (D != 0) \{
061       register mp_digit *tmpc, mask;
062   
063       /* mask */
064       mask = (((mp_digit)1) << D) - 1;
065   
066       /* alias */
067       tmpc = c->dp + (c->used - 1);
068   
069       /* carry */
070       r = 0;
071       for (x = c->used - 1; x >= 0; x--) \{
072         /* get the lower  bits of this word in a temp */
073         rr = *tmpc & mask;
074   
075         /* shift the current word and mix in the carry bits from the previous 
      word */
076         *tmpc = (*tmpc >> D) | (r << (DIGIT_BIT - D));
077         --tmpc;
078   
079         /* set the carry to the carry bits of the current word found above */
080         r = rr;
081       \}
082     \}
083     mp_clamp (c);
084     res = MP_OKAY;
085     if (d != NULL) \{
086       mp_exch (&t, d);
087     \}
088     mp_clear (&t);
089     return MP_OKAY;
090   \}
\end{alltt}
\end{small}

The implementation of algorithm mp\_div\_2d is slightly different than the algorithm specifies.  The remainder $d$ may be optionally 
ignored by passing \textbf{NULL} as the pointer to the mp\_int variable.    The temporary mp\_int variable $t$ is used to hold the 
result of the remainder operation until the end.  This allows $d = a$ to be true without overwriting the input before they are no longer required.  

The remainder of the source code is essentially the same as the source code for mp\_mul\_2d.  (-- Fix this paragraph up later, Tom).

\subsection{Remainder of Division by Power of Two}

The last algorithm in the series of polynomial basis power of two algorithms is calculating the remainder of division by $2^b$.  This
algorithm benefits from the fact that in twos complement arithmetic $a \mbox{ (mod }2^b\mbox{)}$ is the same as $a$ AND $2^b - 1$.  

\begin{figure}[!here]
\begin{small}
\begin{center}
\begin{tabular}{l}
\hline Algorithm \textbf{mp\_mod\_2d}. \\
\textbf{Input}.   One mp\_int $a$ and an integer $b$ \\
\textbf{Output}.  $c \leftarrow a \mbox{ (mod }2^b\mbox{)}$. \\
\hline \\
1.  If $b \le 0$ then do \\
\hspace{3mm}1.1  $c \leftarrow 0$ (\textit{hint: use mp\_zero}) \\
\hspace{3mm}1.2  Return(\textit{MP\_OKAY}). \\
2.  If $b > a.used \cdot lg(\beta)$ then do \\
\hspace{3mm}2.1  $c \leftarrow a$ (\textit{hint: use mp\_copy}) \\
\hspace{3mm}2.2  Return the result of step 2.1. \\
3.  $c \leftarrow a$ \\
4.  If step 3 failed return(\textit{MP\_MEM}). \\
5.  for $n$ from $\lceil b / lg(\beta) \rceil$ to $c.used$ do \\
\hspace{3mm}5.1  $c_n \leftarrow 0$ \\
6.  $k \leftarrow b \mbox{ (mod }lg(\beta)\mbox{)}$ \\
7.  $c_{\lfloor b / lg(\beta) \rfloor} \leftarrow c_{\lfloor b / lg(\beta) \rfloor} \mbox{ (mod }2^{k}\mbox{)}$. \\
8.  Return(\textit{MP\_OKAY}). \\
\hline
\end{tabular}
\end{center}
\end{small}
\caption{Algorithm mp\_mod\_2d}
\end{figure}

\textbf{Algorithm mp\_mod\_2d.}
This algorithm will quickly calculate the value of $a \mbox{ (mod }2^b\mbox{)}$.  First if $b$ is less than or equal to zero the 
result is set to zero.  If $b$ is greater than the number of bits in $a$ then it simply copies $a$ to $c$ and returns.  Otherwise, $a$ 
is copied to $b$, leading digits are removed and the remaining leading digit is trimed to the exact bit count.

\index{bn\_mp\_mod\_2d.c}
\vspace{+3mm}\begin{small}
\hspace{-5.1mm}{\bf File}: bn\_mp\_mod\_2d.c
\vspace{-3mm}
\begin{alltt}
016   
017   /* calc a value mod 2\b */
018   int
019   mp_mod_2d (mp_int * a, int b, mp_int * c)
020   \{
021     int     x, res;
022   
023   
024     /* if b is <= 0 then zero the int */
025     if (b <= 0) \{
026       mp_zero (c);
027       return MP_OKAY;
028     \}
029   
030     /* if the modulus is larger than the value than return */
031     if (b > (int) (a->used * DIGIT_BIT)) \{
032       res = mp_copy (a, c);
033       return res;
034     \}
035   
036     /* copy */
037     if ((res = mp_copy (a, c)) != MP_OKAY) \{
038       return res;
039     \}
040   
041     /* zero digits above the last digit of the modulus */
042     for (x = (b / DIGIT_BIT) + ((b % DIGIT_BIT) == 0 ? 0 : 1); x < c->used; x+
      +) \{
043       c->dp[x] = 0;
044     \}
045     /* clear the digit that is not completely outside/inside the modulus */
046     c->dp[b / DIGIT_BIT] &=
047       (mp_digit) ((((mp_digit) 1) << (((mp_digit) b) % DIGIT_BIT)) - ((mp_digi
      t) 1));
048     mp_clamp (c);
049     return MP_OKAY;
050   \}
\end{alltt}
\end{small}

-- Add comments later, Tom.

\section*{Exercises}
\begin{tabular}{cl}
$\left [ 3 \right ] $ & Devise an algorithm that performs $a \cdot 2^b$ for generic values of $b$ \\
                      & in $O(n)$ time. \\
                      &\\
$\left [ 3 \right ] $ & Devise an efficient algorithm to multiply by small low hamming  \\
                      & weight values such as $3$, $5$ and $9$.  Extend it to handle all values \\
                      & upto $64$ with a hamming weight less than three. \\
                      &\\
$\left [ 2 \right ] $ & Modify the preceding algorithm to handle values of the form \\
                      & $2^k - 1$ as well. \\
                      &\\
$\left [ 3 \right ] $ & Using only algorithms mp\_mul\_2, mp\_div\_2 and mp\_add create an \\
                      & algorithm to multiply two integers in roughly $O(2n^2)$ time for \\
                      & any $n$-bit input.  Note that the time of addition is ignored in the \\
                      & calculation.  \\
                      & \\
$\left [ 5 \right ] $ & Improve the previous algorithm to have a working time of at most \\
                      & $O \left (2^{(k-1)}n + \left ({2n^2 \over k} \right ) \right )$ for an appropriate choice of $k$.  Again ignore \\
                      & the cost of addition. \\
                      & \\
$\left [ 1 \right ] $ & There exists an improvement on the previous algorithm to \\
                      & slightly reduce the number of additions required.  Modify the \\
                      & previous algorithm to include this improvement. \\
                      & \\
$\left [ 2 \right ] $ & Devise a chart to find optimal values of $k$ for the previous problem \\
                      & for $n = 64 \ldots 1024$ in steps of $64$. \\
                      & \\
$\left [ 2 \right ] $ & Using only algorithms mp\_abs and mp\_sub devise another method for \\
                      & calculating the result of a signed comparison. \\
                      &
\end{tabular}

\chapter{Multiplication and Squaring}
\section{The Multipliers}
For most number theoretic systems including public key cryptographic algorithms the set of algorithms collectively known as the
``multipliers'' form the most important subset of algorithms of any multiple precision integer package.  The set of multipliers 
include multiplication, squaring and modular reduction algorithms.  

The importance of these algorithms is driven by the fact that most popular public key algorithms are based on modular 
exponentiation.  That is performing $d \equiv a^b \mbox{ (mod }c\mbox{)}$ for some arbitrary choice of $a$, $b$, $c$ and $d$.  Roughly
speaking the a modular exponentiation will spend about 40\% of the time in modular reductions, 35\% of the time in squaring and 25\% of
the time in multiplications.  Only a small trivial amount of time is spent on lower level algorithms such as mp\_clamp, mp\_init, etc...

This chapter will discuss only two of the multipliers algorithms, multiplication and squaring.  As will be discussed shortly very efficient
multiplier algorithms are not always straightforward and deserve a lot of attention.

\section{Multiplication}
\subsection{The Baseline Multiplication}
\index{baseline multiplication}
Computing the product of two integers in software can be achieved using a trivial adaptation of the standard $O(n^2)$ long-hand multiplication
algorithm school children are taught.  The ``baseline multiplication'' algorithm is designed to act as the ``catch-all'' algorithm only called
when the faster algorithms cannot be used.  This algorithm does not use any particularly interesting optimizations.

The first algorithm to review is the unsigned multiplication algorithm from which a signed multiplication algorithm can be established.  One important 
facet of this algorithm to note is that it has been modified to only produce a certain amount of output digits as resolution.  Recall that for
a $n$ and $m$ digit input the product will be at most $n + m + 1$ digits.  Therefore, this algorithm can be reduced to a full multiplier by
telling it to produce $n + m + 1$ digits.  

Recall from sub-section 5.2.2 the definition of $\gamma$ as the number of bits in the type \textbf{mp\_digit}.  We shall now extend this variable set to 
include $\alpha$ which shall represent the number of bits in the type \textbf{mp\_word}.  This implies that $2^{\alpha} > 2 \cdot \beta^2$.  The 
constant $\delta = 2^{\alpha - 2lg(\beta)}$ will represent the maximal weight of any column in a product (\textit{see sub-section 6.2.2 for more information}).

\newpage\begin{figure}[!here]
\begin{small}
\begin{center}
\begin{tabular}{l}
\hline Algorithm \textbf{s\_mp\_mul\_digs}. \\
\textbf{Input}.   mp\_int $a$, mp\_int $b$ and an integer $digs$ \\
\textbf{Output}.  $c \leftarrow \vert a \vert \cdot \vert b \vert \mbox{ (mod }\beta^{digs}\mbox{)}$. \\
\hline \\
1.  If min$(a.used, b.used) < \delta$ then do \\
\hspace{3mm}1.1  Calculate $c = \vert a \vert \cdot \vert b \vert$ by the Comba method.  \\
\hspace{3mm}1.2  Return the result of step 1.1 \\
\\
Allocate and initialize a temporary mp\_int. \\
2.  Init $t$ to be of size $digs$ \\
3.  If step 2 failed return(\textit{MP\_MEM}). \\
4.  $t.used \leftarrow digs$ \\
\\
Compute the product. \\
5.  for $ix$ from $0$ to $a.used - 1$ do \\
\hspace{3mm}5.1  $u \leftarrow 0$ \\
\hspace{3mm}5.2  $pb \leftarrow \mbox{min}(b.used, digs - ix)$ \\
\hspace{3mm}5.3  If $pb < 1$ then goto step 6. \\
\hspace{3mm}5.4  for $iy$ from $0$ to $pb - 1$ do \\
\hspace{6mm}5.4.1  $\hat r \leftarrow t_{iy + ix} + a_{ix} \cdot b_{iy} + u$ \\
\hspace{6mm}5.4.2  $t_{iy + ix} \leftarrow \hat r \mbox{ (mod }\beta\mbox{)}$ \\
\hspace{6mm}5.4.3  $u \leftarrow \lfloor \hat r / \beta \rfloor$ \\
\hspace{3mm}5.5  if $ix + iy < digs$ then do \\
\hspace{6mm}5.5.1  $t_{ix + pb} \leftarrow u$ \\
6.  Clamp excess digits of $t$. \\
7.  Swap $c$ with $t$ \\
8.  Clear $t$ \\
9.  Return(\textit{MP\_OKAY}). \\
\hline
\end{tabular}
\end{center}
\end{small}
\caption{Algorithm s\_mp\_mul\_digs}
\end{figure}

\textbf{Algorithm s\_mp\_mul\_digs.}
This algorithm computes the unsigned product of two inputs $a$ and $c$ limited to an output precision of $digs$ digits.  While it may seem
a bit awkward to modify the function from its simple $O(n^2)$ description the usefulness of partial multipliers will arise in a future 
algorithm.  The algorithm is loosely based on algorithm 14.12 from \cite[pp. 595]{HAC} and is similar to Algorithm M \cite[pp. 268]{TAOCPV2}.  The
algorithm differs from those cited references because it can produce a variable output precision regardless of the precision of the inputs.

The first thing this algorithm checks for is whether a Comba multiplier can be used instead.   That is if the minimal digit count of either
input is less than $\delta$ the Comba method is used.    After the Comba method is ruled out the baseline algorithm begins.  A 
temporary mp\_int variable $t$ is used to hold the intermediate result of the product.  This allows the algorithm to be used to 
compute products when either $a = c$ or $b = c$ without overwriting the inputs.  

All of step 5 is the infamous $O(n^2)$ multiplication loop slightly modified to only produce upto $digs$ digits of output.  The $pb$ variable
is given the count of digits to read from $b$ inside the nested loop.  If $pb < 0$ then no more output digits can be produced and the algorithm
will exit the loop.  The best way to think of the loops are as a series of $pb \times 1$ multiplication.    That is, in each pass of the 
innermost loop $a_{ix}$ is multiplied against $b$ and the result is added (\textit{with an appropriate shift}) to $t$.  

For example, consider multiplying $576$ by $241$.  That is equivalent to computing $10^0(1)(576) + 10^1(4)(576) + 10^2(2)(576)$ which is best
visualized as the following table.

\begin{figure}[here]
\begin{center}
\begin{tabular}{|c|c|c|c|c|c|c|}
\hline   &&          & 5 & 7 & 6 & \\
\hline   $\times$&&  & 2 & 4 & 1 & \\
\hline &&&&&&\\
  &&          & 5 & 7 & 6 & $10^0(1)(576)$ \\
  &2 &   3    & 0 & 4 & 0 & $10^1(4)(576)$ \\
  1 & 1 & 5 & 2 & 0 & 0 &  $10^2(2)(576)$ \\
\hline  
\end{tabular}
\end{center}
\caption{Long-Hand Multiplication Diagram}
\end{figure}

Each row of the product is added to the result after being shifted to the left (\textit{multiplied by a power of the radix}) by the appropriate 
count.  That is in pass $ix$ of the inner loop the product is added starting at the $ix$'th digit of the reult.

Step 5.4.1 introduces the hat symbol (\textit{e.g. $\hat x$}) which represents a double precision variable.  The multiplication on that step
is assumed to be a double wide output single precision multiplication.  That is, two single precision variables are multiplied to produce a
double precision result.  The step is somewhat optimized from a long-hand multiplication algorithm because the carry from the addition in step
5.4.1 is forwarded through the nested loop.  If the carry was ignored it would overflow the single precision digit $t_{ix+iy}$ and the result
would be lost.  

At step 5.5 the nested loop is finished and any carry that was left over should be forwarded.  That is provided $ix + iy < digs$ otherwise the
carry is ignored since it will not be part of the result anyways.  

\index{bn\_s\_mp\_mul\_digs.c}
\vspace{+3mm}\begin{small}
\hspace{-5.1mm}{\bf File}: bn\_s\_mp\_mul\_digs.c
\vspace{-3mm}
\begin{alltt}
016   
017   /* multiplies |a| * |b| and only computes upto digs digits of result
018    * HAC pp. 595, Algorithm 14.12  Modified so you can control how 
019    * many digits of output are created.
020    */
021   int
022   s_mp_mul_digs (mp_int * a, mp_int * b, mp_int * c, int digs)
023   \{
024     mp_int  t;
025     int     res, pa, pb, ix, iy;
026     mp_digit u;
027     mp_word r;
028     mp_digit tmpx, *tmpt, *tmpy;
029   
030     /* can we use the fast multiplier? */
031     if (((digs) < MP_WARRAY) &&
032         MIN (a->used, b->used) < 
033             (1 << ((CHAR_BIT * sizeof (mp_word)) - (2 * DIGIT_BIT)))) \{
034       return fast_s_mp_mul_digs (a, b, c, digs);
035     \}
036   
037     if ((res = mp_init_size (&t, digs)) != MP_OKAY) \{
038       return res;
039     \}
040     t.used = digs;
041   
042     /* compute the digits of the product directly */
043     pa = a->used;
044     for (ix = 0; ix < pa; ix++) \{
045       /* set the carry to zero */
046       u = 0;
047   
048       /* limit ourselves to making digs digits of output */
049       pb = MIN (b->used, digs - ix);
050   
051       /* setup some aliases */
052       /* copy of the digit from a used within the nested loop */
053       tmpx = a->dp[ix];
054       
055       /* an alias for the destination shifted ix places */
056       tmpt = t.dp + ix;
057       
058       /* an alias for the digits of b */
059       tmpy = b->dp;
060   
061       /* compute the columns of the output and propagate the carry */
062       for (iy = 0; iy < pb; iy++) \{
063         /* compute the column as a mp_word */
064         r = ((mp_word) *tmpt) + 
065             ((mp_word) tmpx) * ((mp_word) * tmpy++) + 
066             ((mp_word) u);
067   
068         /* the new column is the lower part of the result */
069         *tmpt++ = (mp_digit) (r & ((mp_word) MP_MASK));
070   
071         /* get the carry word from the result */
072         u = (mp_digit) (r >> ((mp_word) DIGIT_BIT));
073       \}
074       /* set carry if it is placed below digs */
075       if (ix + iy < digs) \{
076         *tmpt = u;
077       \}
078     \}
079   
080     mp_clamp (&t);
081     mp_exch (&t, c);
082   
083     mp_clear (&t);
084     return MP_OKAY;
085   \}
\end{alltt}
\end{small}

Lines 31 to 35 determine if the Comba method can be used first.  The conditions for using the Comba routine are that min$(a.used, b.used) < \delta$ and
the number of digits of output is less than \textbf{MP\_WARRAY}.  This new constant is used to control the stack usage in the Comba routines.  By
default it is set to $\delta$ but can be reduced when memory is at a premium.

Of particular importance is the calculation of the $ix+iy$'th column on lines 64, 65 and 66.  Note how all of the
variables are cast to the type \textbf{mp\_word}.  That is to ensure that double precision operations are used instead of single precision.  The
multiplication on line 65 is a bit of a GCC optimization.  On the outset it looks like the compiler will have to use a double precision
multiplication to produce the result required.  Such an operation would be horribly slow on most processors and drag this to a crawl.  However,
GCC is smart enough to realize that double wide output single precision multipliers can be used.  For example, the instruction ``MUL'' on the
x86 processor can multiply two 32-bit values and produce a 64-bit result.  

\subsection{Faster Multiplication by the ``Comba'' Method}

One of the huge drawbacks of the ``baseline'' algorithms is that at the $O(n^2)$ level the carry must be computed and propagated upwards.  This
makes the nested loop very sequential and hard to unroll and implement in parallel.  The ``Comba'' method is named after little known 
(\textit{in cryptographic venues}) Paul G. Comba where in \cite{COMBA} a method of implementing fast multipliers that do not require nested 
carry fixup operations was presented.

At the heart of algorithm is once again the long-hand algorithm for multiplication.  Except in this case a slight twist is placed on how
the columns of the result are produced.  In the standard long-hand algorithm rows of products are produced then added together to form the 
final result.  In the baseline algorithm the columns are added together to get the result instantaneously.  

In the Comba algorithm however, the columns of the result are produced entirely independently of each other.  That is at the $O(n^2)$ level a 
simple multiplication and addition step is performed.  Or more succintly that 

\begin{equation}
x_n = \sum_{i+j = n} a_ib_j
\end{equation}

Where $x_n$ is the $n'th$ column of the output vector.  To see how this works consider once again multiplying $576$ by $241$.  

\begin{figure}[here]
\begin{small}
\begin{center}
\begin{tabular}{|c|c|c|c|c|c|}
  \hline &          & 5 & 7 & 6 & First Input\\
  \hline $\times$ & & 2 & 4 & 1 & Second Input\\
\hline            &                        & $1 \cdot 5 = 5$   & $1 \cdot 7 = 7$   & $1 \cdot 6 = 6$ & First pass \\
                  &  $4 \cdot 5 = 20$      & $4 \cdot 7+5=33$  & $4 \cdot 6+7=31$  & 6               & Second pass \\
   $2 \cdot 5 = 10$ &  $2 \cdot 7 + 20 = 34$ & $2 \cdot 6+33=45$ & 31                & 6             & Third pass \\
\hline 10 & 34 & 45 & 31 & 6 & Final Result \\   
\hline   
\end{tabular}
\end{center}
\end{small}
\caption{Comba Multiplication Diagram}
\end{figure}

At this point the vector $x = \left < 10, 34, 45, 31, 6 \right >$ is the result of the first step of the Comba multipler.  
Now the columns must be fixed by propagating the carry upwards.  The following trivial algorithm will accomplish this.

\begin{enumerate}
    \item for $n$ from 0 to $k - 1$ do
    \item \hspace{3mm} $x_{n+1} \leftarrow x_{n+1} + \lfloor x_{n}/\beta \rfloor$ 
    \item \hspace{3mm} $x_{n} \leftarrow x_{n} \mbox{ (mod }\beta\mbox{)}$
\end{enumerate}

With that algorithm and $k = 5$ and $\beta = 10$ the following vector is produced $y = \left < 1, 3, 8, 8, 1, 6 \right >$.  In this case 
$241 \cdot 576$ is in fact $138816$ and the procedure succeeded.  If the algorithm is correct and as will be demonstrated shortly more
efficient than the baseline algorithm why not simply always use this algorithm?

\subsubsection{Column Weight.}
At the nested $O(n^2)$ level the Comba method adds the product of two single precision variables to a each column of the output 
independently.  A serious obstacle is if the carry is lost due to lack of precision before the algorithm has a chance to fix
the carries.  For example, in the multiplication of two three-digit numbers the third column of output will be the sum of
three single precision multiplications.  If the precision of the accumulator for the output digits is less then $3 \cdot (\beta - 1)^2$ then
an overflow can occur and the carry information will be lost.  For any $m$ and $n$ digit input the maximal weight of any column is 
min$(m, n)$ which is fairly obvious.

The maximal number of terms in any column of a product is known as the ``column weight'' and strictly governs when the algorithm can be used.  Recall
from earlier that a double precision type has $\alpha$ bits of resolution and a single precision digit has $lg(\beta)$ bits of precision.  Given these
two quantities we may not violate the following

\begin{equation}
k \cdot \left (\beta - 1 \right )^2 < 2^{\alpha}
\end{equation}

Which reduces to 

\begin{equation}
k \cdot \left ( \beta^2 - 2\beta + 1 \right ) < 2^{\alpha}
\end{equation}

Let $\rho = lg(\beta)$ represent the number of bits in a single precision digit.  By further re-arrangement of the equation the final solution is
found.

\begin{equation}
k \cdot \left (2^{2\rho} - 2^{\rho + 1} + 1 \right ) < 2^{\alpha}
\end{equation}

The defaults for LibTomMath are $\beta = 2^{28}, \alpha = 2^{64}$ which simplies to $72057593501057025 \cdot k < 2^{64}$ which when divided out
result in $k < 257$.  This implies that the smallest input may not have more than $256$ digits if the Comba method is to be used in
this configuration.  This is quite satisfactory for most applications since $256$ digits would be allow for numbers in the range of $2^{7168}$ 
which is much larger than the typical $2^{100}$ to $2^{4000}$ range most public key cryptographic algorithms use.  

\newpage\begin{figure}[!here]
\begin{small}
\begin{center}
\begin{tabular}{l}
\hline Algorithm \textbf{fast\_s\_mp\_mul\_digs}. \\
\textbf{Input}.   mp\_int $a$, mp\_int $b$ and an integer $digs$ \\
\textbf{Output}.  $c \leftarrow \vert a \vert \cdot \vert b \vert \mbox{ (mod }\beta^{digs}\mbox{)}$. \\
\hline \\
Place an array of \textbf{MP\_WARRAY} double precision digits named $\hat W$ on the stack. \\
1.  If $c.alloc < digs$ then grow $c$ to $digs$ digits. (\textit{hint: use mp\_grow}) \\
2.  If step 1 failed return(\textit{MP\_MEM}).\\
\\
Zero the temporary array $\hat W$. \\
3.  for $n$ from $0$ to $digs - 1$ do \\
\hspace{3mm}3.1  $\hat W_n \leftarrow 0$ \\
\\
Compute the columns. \\
4.  for $ix$ from $0$ to $a.used - 1$ do \\
\hspace{3mm}4.1  $pb \leftarrow \mbox{min}(b.used, digs - ix)$ \\
\hspace{3mm}4.2  If $pb < 1$ then goto step 5. \\
\hspace{3mm}4.3  for $iy$ from $0$ to $pb - 1$ do \\
\hspace{6mm}4.3.1  $\hat W_{ix+iy} \leftarrow \hat W_{ix+iy} + a_{ix}b_{iy}$ \\
\\
Propagate the carries upwards. \\
5.  $oldused \leftarrow c.used$ \\
6.  $c.used \leftarrow digs$ \\
7.  If $digs > 1$ then do \\
\hspace{3mm}7.1.  for $ix$ from $1$ to $digs - 1$ do \\
\hspace{6mm}7.1.1  $\hat W_{ix} \leftarrow \hat W_{ix} + \lfloor \hat W_{ix-1} / \beta \rfloor$ \\
\hspace{6mm}7.1.2  $c_{ix - 1} \leftarrow \hat W_{ix - 1} \mbox{ (mod }\beta\mbox{)}$ \\
8.  else do \\
\hspace{3mm}8.1  $ix \leftarrow 0$ \\
9.  $c_{ix} \leftarrow \hat W_{ix} \mbox{ (mod }\beta\mbox{)}$ \\
\\
Zero excess digits. \\
10.  If $digs < oldused$ then do \\
\hspace{3mm}10.1  for $n$ from $digs$ to $oldused - 1$ do \\
\hspace{6mm}10.1.1  $c_n \leftarrow 0$ \\
11.  Clamp excessive digits of $c$.  (\textit{hint: use mp\_clamp}) \\
12.  Return(\textit{MP\_OKAY}). \\
\hline
\end{tabular}
\end{center}
\end{small}
\caption{Algorithm fast\_s\_mp\_mul\_digs}
\end{figure}

\textbf{Algorithm fast\_s\_mp\_mul\_digs.}
This algorithm performs the unsigned multiplication of $a$ and $b$ using the Comba method limited to $digs$ digits of precision.  The algorithm
essentially peforms the same calculation as algorithm s\_mp\_mul\_digs but much faster.

The array $\hat W$ is meant to be on the stack when the algorithm is used.  The size of the array does not change which is ideal.  Note also that 
unlike algorithm s\_mp\_mul\_digs no temporary mp\_int is required since the result is calculated in place in $\hat W$.  

The $O(n^2)$ loop on step four is where the Comba method starts to show through.  First there is no carry variable in the loop.  Second the
double precision multiply and add step does not have a carry fixup of any sort.  In fact the nested loop is very simple and can be implemented
in parallel.  

What makes the Comba method so attractive is that the carry propagation only takes place outside the $O(n^2)$ nested loop.  For example, if the 
cost in terms of time of a multiply and add is $p$ and the cost of a carry propagation is $q$ then a baseline multiplication would require 
$O \left ((p + q)n^2 \right )$ time to multiply two $n$-digit numbers.  The Comba method only requires $pn^2 + qn$ time, however, in practice 
the speed increase is actually much more.  With $O(n)$ space the algorithm can be reduced to $O(pn + qn)$ time by implementing the $n$ multiply
and add operations in the nested loop in parallel.  

The carry propagation loop on step 7 is fairly straightforward.  It could have been written phased the other direction, that is, to assign
to $c_{ix}$ instead of $c_{ix-1}$ in each iteration.  However, it would still require pre-caution to make sure that $\hat W_{ix+1}$ is not beyond
the \textbf{MP\_WARRAY} words set aside.  

\index{bn\_fast\_s\_mp\_mul\_digs.c}
\vspace{+3mm}\begin{small}
\hspace{-5.1mm}{\bf File}: bn\_fast\_s\_mp\_mul\_digs.c
\vspace{-3mm}
\begin{alltt}
016   
017   /* Fast (comba) multiplier
018    *
019    * This is the fast column-array [comba] multiplier.  It is 
020    * designed to compute the columns of the product first 
021    * then handle the carries afterwards.  This has the effect 
022    * of making the nested loops that compute the columns very
023    * simple and schedulable on super-scalar processors.
024    *
025    * This has been modified to produce a variable number of 
026    * digits of output so if say only a half-product is required 
027    * you don't have to compute the upper half (a feature 
028    * required for fast Barrett reduction).
029    *
030    * Based on Algorithm 14.12 on pp.595 of HAC.
031    *
032    */
033   int
034   fast_s_mp_mul_digs (mp_int * a, mp_int * b, mp_int * c, int digs)
035   \{
036     int     olduse, res, pa, ix;
037     mp_word W[MP_WARRAY];
038   
039     /* grow the destination as required */
040     if (c->alloc < digs) \{
041       if ((res = mp_grow (c, digs)) != MP_OKAY) \{
042         return res;
043       \}
044     \}
045   
046     /* clear temp buf (the columns) */
047     memset (W, 0, sizeof (mp_word) * digs);
048   
049     /* calculate the columns */
050     pa = a->used;
051     for (ix = 0; ix < pa; ix++) \{
052       /* this multiplier has been modified to allow you to 
053        * control how many digits of output are produced.  
054        * So at most we want to make upto "digs" digits of output.
055        *
056        * this adds products to distinct columns (at ix+iy) of W
057        * note that each step through the loop is not dependent on
058        * the previous which means the compiler can easily unroll
059        * the loop without scheduling problems
060        */
061       \{
062         register mp_digit tmpx, *tmpy;
063         register mp_word *_W;
064         register int iy, pb;
065   
066         /* alias for the the word on the left e.g. A[ix] * A[iy] */
067         tmpx = a->dp[ix];
068   
069         /* alias for the right side */
070         tmpy = b->dp;
071   
072         /* alias for the columns, each step through the loop adds a new
073            term to each column
074          */
075         _W = W + ix;
076   
077         /* the number of digits is limited by their placement.  E.g.
078            we avoid multiplying digits that will end up above the # of
079            digits of precision requested
080          */
081         pb = MIN (b->used, digs - ix);
082   
083         for (iy = 0; iy < pb; iy++) \{
084           *_W++ += ((mp_word) tmpx) * ((mp_word) * tmpy++);
085         \}
086       \}
087   
088     \}
089   
090     /* setup dest */
091     olduse = c->used;
092     c->used = digs;
093   
094     \{
095       register mp_digit *tmpc;
096   
097       /* At this point W[] contains the sums of each column.  To get the
098        * correct result we must take the extra bits from each column and
099        * carry them down
100        *
101        * Note that while this adds extra code to the multiplier it 
102        * saves time since the carry propagation is removed from the 
103        * above nested loop.This has the effect of reducing the work 
104        * from N*(N+N*c)==N**2 + c*N**2 to N**2 + N*c where c is the 
105        * cost of the shifting.  On very small numbers this is slower 
106        * but on most cryptographic size numbers it is faster.
107        */
108       tmpc = c->dp;
109       for (ix = 1; ix < digs; ix++) \{
110         W[ix] += (W[ix - 1] >> ((mp_word) DIGIT_BIT));
111         *tmpc++ = (mp_digit) (W[ix - 1] & ((mp_word) MP_MASK));
112       \}
113       *tmpc++ = (mp_digit) (W[digs - 1] & ((mp_word) MP_MASK));
114   
115       /* clear unused */
116       for (; ix < olduse; ix++) \{
117         *tmpc++ = 0;
118       \}
119     \}
120   
121     mp_clamp (c);
122     return MP_OKAY;
123   \}
\end{alltt}
\end{small}

The memset on line 47 clears the initial $\hat W$ array to zero in a single step. Like the slower baseline multiplication
implementation a series of aliases (\textit{lines 67, 70 and 75}) are used to simplify the inner $O(n^2)$ loop.  
In this case a new alias $\_\hat W$ has been added which refers to the double precision columns offset by $ix$ in each pass.  

The inner loop on line 84 is where the algorithm will spend the majority of the time.  Which is why it has been stripped to the 
bones of any extra baggage\footnote{Hence the pointer aliases.}.  On x86 processors the multiply and add amounts to at the very least five
instructions (\textit{two loads, two additions, one multiply}) while on the ARMv4 processors it amounts to only three (\textit{one load, one store,
one multiply-add}).   On both the x86 and ARMv4 processors GCC v3.2 does a very good job at unrolling the loop and scheduling it so there 
are very few dependency stalls.

In theory the difference between the baseline and comba algorithms is a mere $O(qn)$ time difference.  However, in the $O(n^2)$ nested loop of the
baseline method there are dependency stalls as the algorithm must wait for the multiplier to finish before propagating the carry to the next 
digit.  As a result fewer of the often multiple execution units\footnote{The AMD Athlon has three execution units and the Intel P4 has four.} can
be simultaneously used.  

\subsection{Multiplication at New Bounds by Karatsuba Method}
So far two methods of multiplication have been presented.  Both of the algorithms require asymptotically $O(n^2)$ time to multiply two $n$-digit 
numbers together.  While the Comba method is much faster than the baseline algorithm it still requires far too much time to multiply 
large inputs together.  In fact it was not until \cite{KARA} in 1962 that a faster algorithm had been proposed at all.

The idea behind Karatsubas method is that an input can be represented in polynomial basis as two halves then multiplied.  For example, if 
$f(x) = ax + b$ and $g(x) = cx + b$ then the product of the two polynomials $h(x) = f(x)g(x)$ will allow $h(\beta) = (f(\beta))(g(\beta))$.  

So how does this help?  First expand the product $h(x)$.

\begin{center}
\begin{tabular}{rcl}
$h(x)$ & $=$ & $f(x)g(x)$ \\
       & $=$ & $(ax + b)(cx + d)$ \\
       & $=$ & $acx^2 + adx + bcx + bd$ \\
\end{tabular}
\end{center}

The next equation is a bit of genius on the part of Karatsuba.  He proved that the previous equation is equivalent to 

\begin{equation}
h(x) = acx^2 + ((a - c)(b - d) + bd + ac)x + bd
\end{equation}

Essentially the proof lies in some fairly light algebraic number theory (\textit{see \cite{KARAP} for details}) that is not important for
the discussion.  At first glance it appears that the Karatsuba method is actually harder than the straight $O(n^2)$ approach.  
However, further investigation will prove otherwise.  

The first important observation is that both $f(x)$ and $g(x)$ are the polynomial basis representation of two-digit numbers.  This means that 
$\left < a, b, c, d \right >$ are single digit values.  Using either the baseline or straight polynomial multiplication the old method requires
$O \left (4(n/2)^2 \right ) = O(n^2)$ single precision multiplications.  Looking closer at Karatsubas equation there are only three unique multiplications 
required which are $ac$, $bd$ and $(a - c)(b - d)$.  As a result only $O \left (3 \cdot (n/2)^2 \right ) = O \left ( {3 \over 4}n^2 \right )$ 
multiplications are required.  

So far the algorithm has been discussed from the point of view of ``two-digit'' numbers.  However, there is no reason why two digits implies a range of 
$\beta^2$.  It could just as easily represent a range of $\left (\beta^k \right)^2$ as well.  For example, the polynomial 
$f(x) = a_3x^3 + a_2x^2 + a_1x + a_0$ could also be written as $f'(x) = a'_1x + a'_0$ where $f(\beta) = f'(\beta^2)$.  Fortunately representing an
integer which is already in an array of radix-$\beta$ digits in polynomial basis in terms of a power of $\beta$ is very simple.  

\subsubsection{Recursion}
The Karatsuba multiplication algorithm can be applied to practically any size of input.  Therefore, it is possible that the Karatsuba method itself
be used for the three multiplications required.  For example, when multiplying two four-digit numbers there will be three multiplications of two-digit
numbers.  In this case the smaller multiplication requires $p(n) = {3 \over 4}n^2$ time to complete while the larger multiplication requires
$q(n) = 3 \cdot p(n/2)$ multiplications.  

By expanding $q(n)$ the following equation is achieved. 

\begin{center}
\begin{tabular}{rcl}
$q(n)$ & $=$ & $3 \cdot p(n/2)$ \\
       & $=$ & $3 \cdot (3 \cdot ((n/2)/2)^2)$ \\
       & $=$ & $9 \cdot (n/4)^2$ \\
       & $=$ & ${9 \over 16}n^2$ \\
\end{tabular}
\end{center}

The generic expression for the multiplicand is simply $\left ( {3 \over 4} \right )^k$ for $k \ge 1$ recurisions.  The maximal number of recursions
is approximately $lg(n)$.  Putting this all in terms of a base $n$ logarithm the asymptotic running time can be deduced.

\begin{center}
\begin{tabular}{rcl}
$lg_n \left ( \left ( {3 \over 4} \right )^{lg_2 n} \cdot n^2 \right )$ & $=$ & $lg_2 n \cdot lg_n \left ( { 3 \over 4 } \right ) + 2$ \\
                                                                        & $=$ & $\left ( {log N \over log 2} \right ) \cdot \left ( {log \left ( {3 \over 4} \right ) \over log N } \right ) + 2$ \\
                                                                        & $=$ & ${ log 3 - log 2^2 + 2 \cdot log 2} \over log 2$ \\
                                                                        & $=$ & $log 3 \over log 2$ \\
\end{tabular}
\end{center}

Which leads to a running time of $O \left ( n^{lg(3)} \right )$ which is approximately $O(n^{1.584})$.  This can lead to 
impressive savings with fairly moderate sized numbers.  For example, when multiplying two 128-digit numbers the Karatsuba 
method saves $14,197$ (\textit{or $86\%$ of the total}) single precision multiplications.  

The immediate question becomes why not simply use Karatsuba multiplication all the time and forget about the baseline and Comba algorithms? 

\subsubsection{Overhead}
While the Karatsuba method saves on the number of single precision multiplications required this savings is not entirely free.  The product
of three half size products must be stored somewhere as well as four additions and two subtractions performed.  These operations incur sufficient
overhead that often for fairly trivial sized inputs the Karatsuba method is slower.

\index{cutoff point}
The \textit{cutoff point} for Karatsuba multiplication is the point at which the Karatsuba multiplication and baseline (\textit{or Comba}) meet.  
For the purposes of this discussion call this value $x$.  For any input with $n$ digits such that $n < x$ Karatsuba multiplication will be slower 
and for $n > x$ it will be faster.  Often the break between the two algorithms is not so clean cut in reality.  The cleaner the cut the more 
efficient multiplication will be which is why tuning the multiplication is a very important process.  For example, a properly tuned Karatsuba 
multiplication algorithm can multiply two $4,096$ bit numbers up to five times faster on an Athlon processor compared to the standard baseline
algorithm.  

The exact placement of the value of $x$ depends on several key factors.   The cost of allocating storage for the temporary variables, the cost of 
performing the additions and most importantly the cost of performing a single precision multiplication.  With a processor where single precision 
multiplication is fast\footnote{The AMD Athlon for instance has a six cycle multiplier compared to the Intel P4 which has a 15 cycle multiplier.} the 
cutoff point will move upwards.  Similarly with a slower processor the cutoff point will move downwards.  

\newpage\begin{figure}[!here]
\begin{small}
\begin{center}
\begin{tabular}{l}
\hline Algorithm \textbf{mp\_karatsuba\_mul}. \\
\textbf{Input}.   mp\_int $a$ and mp\_int $b$ \\
\textbf{Output}.  $c \leftarrow \vert a \vert \cdot \vert b \vert$ \\
\hline \\
1.  $B \leftarrow \mbox{min}(a.used, b.used)/2$ \\
2.  Init the following mp\_int variables: $x0$, $x1$, $y0$, $y1$, $t1$, $x0y0$, $x1y1$.\\
3.  If step 2 failed then return(\textit{MP\_MEM}). \\
\\
Split the input.  e.g. $a = x1 \cdot \beta^B + x0$ \\
4.  $x0 \leftarrow a \mbox{ (mod }\beta^B\mbox{)}$ (\textit{hint: use mp\_mod\_2d}) \\
5.  $y0 \leftarrow b \mbox{ (mod }\beta^B\mbox{)}$ \\
6.  $x1 \leftarrow \lfloor a / \beta^B \rfloor$ (\textit{hint: use mp\_rshd}) \\
7.  $y1 \leftarrow \lfloor b / \beta^B \rfloor$ \\
\\
Calculate the three products. \\
8.  $x0y0 \leftarrow x0 \cdot y0$ (\textit{hint: use mp\_mul}) \\
9.  $x1y1 \leftarrow x1 \cdot y1$ \\
10.  $t1 \leftarrow x1 - x0$ (\textit{hint: use mp\_sub}) \\
11.  $x0 \leftarrow y1 - y0$ \\
12.  $t1 \leftarrow t1 \cdot x0$ \\
\\
Calculate the middle term. \\
13.  $x0 \leftarrow x0y0 + x1y1$ \\
14.  $t1 \leftarrow x0 - t1$ \\
\\
Calculate the final product. \\
15.  $t1 \leftarrow t1 \cdot \beta^B$ (\textit{hint: use mp\_lshd}) \\
16.  $x1y1 \leftarrow x1y1 \cdot \beta^{2B}$ \\
17.  $t1 \leftarrow x0y0 + t1$ \\
18.  $c \leftarrow t1 + x1y1$ \\
19.  Clear all of the temporary variables. \\
20.  Return(\textit{MP\_OKAY}).\\
\hline 
\end{tabular}
\end{center}
\end{small}
\caption{Algorithm mp\_karatsuba\_mul}
\end{figure}

\textbf{Algorithm mp\_karatsuba\_mul.}


\section{Squaring}
\subsection{The Baseline Squaring Algorithm}
\subsection{Faster Squaring by the ``Comba'' Method}
\subsection{Karatsuba Squaring}
\section{Tuning Algorithms}
\subsection{How to Tune Karatsuba Algorithms}

\chapter{Modular Reductions}
\section{Basics of Modular Reduction}
\section{The Barrett Reduction}
\section{The Montgomery Reduction}
\subsection{Faster ``Comba'' Montgomery Reduction}
\subsection{Example Montgomery Algorithms}
\section{The Diminished Radix Algorithm}
\section{Algorithm Comparison}

\chapter{Exponentiation}
\section{Single Digit Exponentiation}
\section{Modular Exponentiation}
\subsection{General Case}
\subsection{Odd or Diminished Radix Moduli}
\section{Quick Power of Two}

\chapter{Higher Level Algorithms}
\section{Integer Division with Remainder}
\section{Single Digit Helpers}
\subsection{Single Digit Addition}
\subsection{Single Digit Subtraction}
\subsection{Single Digit Multiplication}
\subsection{Single Digit Division}
\subsection{Single Digit Modulo}
\subsection{Single Digit Root Extraction}
\section{Random Number Generation}
\section{Formatted Output}
\subsection{Getting The Output Size}
\subsection{Generating Radix-n Output}
\subsection{Reading Radix-n Input}
\section{Unformatted Output}
\subsection{Getting The Output Size}
\subsection{Generating Output}
\subsection{Reading Input}

\chapter{Number Theoretic Algorithms}
\section{Greatest Common Divisor}
\section{Least Common Multiple}
\section{Jacobi Symbol Computation}
\section{Modular Inverse}
\subsection{General Case}
\subsection{Odd Moduli}
\section{Primality Tests}
\subsection{Trial Division}
\subsection{The Fermat Test}
\subsection{The Miller-Rabin Test}
\subsection{Primality Test in a Bottle}
\subsection{The Next Prime}
\section{Root Extraction}

\backmatter
\appendix
\begin{thebibliography}{ABCDEF}
\bibitem[1]{TAOCPV2}
Donald Knuth, \textit{The Art of Computer Programming}, Third Edition, Volume Two, Seminumerical Algorithms, Addison-Wesley, 1998

\bibitem[2]{HAC}
A. Menezes, P. van Oorschot, S. Vanstone, \textit{Handbook of Applied Cryptography}, CRC Press, 1996

\bibitem[3]{ROSE}
Michael Rosing, \textit{Implementing Elliptic Curve Cryptography}, Manning Publications, 1999

\bibitem[4]{COMBA}
Paul G. Comba, \textit{Exponentiation Cryptosystems on the IBM PC}. IBM Systems Journal 29(4): 526-538 (1990)

\bibitem[5]{KARA}
A. Karatsuba, Doklay Akad. Nauk SSSR 145 (1962), pp.293-294

\bibitem[6]{KARAP}
Andre Weimerskirch and Christof Paar, \textit{Generalizations of the Karatsuba Algorithm for Polynomial Multiplication}, Submitted to Design, Codes and Cryptography, March 2002

\end{thebibliography}

\documentclass[b5paper]{book}
\usepackage{makeidx}
\usepackage{amssymb}
\usepackage{color}
\usepackage{alltt}
\usepackage{graphicx}
\usepackage{layout}
\def\union{\cup}
\def\intersect{\cap}
\def\getsrandom{\stackrel{\rm R}{\gets}}
\def\cross{\times}
\def\cat{\hspace{0.5em} \| \hspace{0.5em}}
\def\catn{$\|$}
\def\divides{\hspace{0.3em} | \hspace{0.3em}}
\def\nequiv{\not\equiv}
\def\approx{\raisebox{0.2ex}{\mbox{\small $\sim$}}}
\def\lcm{{\rm lcm}}
\def\gcd{{\rm gcd}}
\def\log{{\rm log}}
\def\ord{{\rm ord}}
\def\abs{{\mathit abs}}
\def\rep{{\mathit rep}}
\def\mod{{\mathit\ mod\ }}
\renewcommand{\pmod}[1]{\ ({\rm mod\ }{#1})}
\newcommand{\floor}[1]{\left\lfloor{#1}\right\rfloor}
\newcommand{\ceil}[1]{\left\lceil{#1}\right\rceil}
\def\Or{{\rm\ or\ }}
\def\And{{\rm\ and\ }}
\def\iff{\hspace{1em}\Longleftrightarrow\hspace{1em}}
\def\implies{\Rightarrow}
\def\undefined{{\rm ``undefined"}}
\def\Proof{\vspace{1ex}\noindent {\bf Proof:}\hspace{1em}}
\let\oldphi\phi
\def\phi{\varphi}
\def\Pr{{\rm Pr}}
\newcommand{\str}[1]{{\mathbf{#1}}}
\def\F{{\mathbb F}}
\def\N{{\mathbb N}}
\def\Z{{\mathbb Z}}
\def\R{{\mathbb R}}
\def\C{{\mathbb C}}
\def\Q{{\mathbb Q}}
\definecolor{DGray}{gray}{0.5}
\newcommand{\url}[1]{\mbox{$<${#1}$>$}}
\newcommand{\emailaddr}[1]{\mbox{$<${#1}$>$}}
\def\twiddle{\raisebox{0.3ex}{\mbox{\tiny $\sim$}}}
\def\gap{\vspace{0.5ex}}
\makeindex
\begin{document}
\frontmatter
\pagestyle{empty}
\title{Multiple-Precision Integer Arithmetic, \\ A Case Study Involving the LibTomMath Project \\ - DRAFT - }
\author{\mbox{
%\begin{small}
\begin{tabular}{c}
Tom St Denis \\
Algonquin College \\
\\
Mads Rasmussen \\
Open Communications Security \\
\\
Gregory Rose \\
Qualcomm \\
\end{tabular}
%\end{small}
}
}
\maketitle
This text in its entirety is copyrighted \copyright{}2003 by Tom St Denis.  It may not be redistributed 
electronically or otherwise without the sole permission of the author.  The text is freely re distributable as long as
it is packaged along with the LibTomMath project in a non-commercial project.  Contact the
author for other redistribution rights.

This text corresponds to the v0.17 release of the LibTomMath project.

\begin{alltt}
Tom St Denis
111 Banning Rd
Ottawa, Ontario
K2L 1C3
Canada

Phone: 1-613-836-3160
Email: tomstdenis@iahu.ca
\end{alltt}

This text is formatted to the international B5 paper size of 176mm wide by 250mm tall using the \LaTeX{} 
{\em book} macro package and the Perl {\em booker} package.

\tableofcontents
\listoffigures
\chapter*{Preface}
Blah.

\mainmatter
\pagestyle{headings}
\chapter{Introduction}
\section{Multiple Precision Arithmetic}
\subsection{The Need for Multiple Precision Arithmetic}
The most prevalent use for multiple precision arithmetic (\textit{often referred to as bignum math}) is within public
key cryptography.   Algorithms such as RSA, Diffie-Hellman and Elliptic Curve Cryptography require large integers in order to 
resist known cryptanalytic attacks.  Typical modern programming languages such as C and Java only provide small 
single-precision data types which are incapable of precisely representing integers which are often hundreds of bits long.

For example, consider multiplying $1,234,567$ by $9,876,543$ in C with an ``unsigned long'' data type.  With an 
x86 machine the result is $4,136,875,833$ while the true result is $12,193,254,061,881$.  The original inputs 
were approximately $21$ and $24$ bits respectively.  If the C language cannot multiply two relatively small values 
together precisely how does anyone expect it to multiply two values which are considerably larger?

Most advancements in fast multiple precision arithmetic stems from the desire for faster cryptographic primitives.  However, cryptography
is not the only field of study that can benefit fast large integer routines.  Another auxiliary use for multiple precision integers is 
high precision floating point data types.  The basic IEEE standard floating point type is made up of an integer mantissa $q$ and an exponent $e$.  
Numbers are given in the form $n = q \cdot b^e$ where $b = 2$ is convention.  Since IEEE is meant to be implemented in 
hardware the precision of the mantissa is often fairly small (\textit{roughly 23 bits}).  Since the mantissa is merely an 
integer a large multiple precision integer could be used.  In effect very high precision floating point arithmetic 
could be performed.  This would be useful where scientific applications must minimize the total output error over long simulations.  

\subsection{Multiple Precision Arithmetic}
\index{multiple precision}
Multiple precision arithmetic attempts to the solve the shortcomings of single precision data types such as those from
the C and Java programming languages.  In essence multiple precision arithmetic is a set of operations that can be 
performed on members of an algebraic group whose precision is not fixed.  The algorithms when implemented to be multiple
precision can allow a developer to work with any practical precision required.

Typically the arithmetic is performed over the ring of integers denoted by a $\Z$ and referred to casually as ``bignum'' 
routines.  However, it is possible to have rings of polynomials as well typically denoted by $\Z/p\Z \left [ X \right ]$ 
which could have variable precision (\textit{or degree}).  This text will discuss implementation of the former, however,
implementing polynomial basis routines should be relatively easy after reading this text.

\subsection{Benefits of Multiple Precision Arithmetic}
\index{precision} \index{accuracy}
Precision is defined loosely as the proximity to the real value a given representation is.  Accuracy is defined as the 
reproducibility of the result.  For example, the calculation $1/3 = 0.25$ is imprecise but can be accurate provided 
it is reproducible.

The benefit of multiple precision representations over single precision representations is that 
often no precision is lost while representing the result of an operation which requires excess precision.  For example, 
the multiplication of two $n$-bit integers requires at least $2n$ bits to represent the result.  A multiple precision 
system would augment the precision of the destination to accomodate the result while a single precision system would
truncate excess bits to maintain a fixed level of precision.

Multiple precision representations allow for the precision to be very high (\textit{if not exacting}) but at a cost of
modest computer resources.  The only reasonable case where a multiple precision system will lose precision is when
emulating a floating point data type.  However, with multiple precision integer arithmetic no precision is lost.

\subsection{Basis of Operations}
At the heart of all multiple precision integer operations are the ``long-hand'' algorithms we all learnt as children 
in grade school.  For example, to multiply $1,234$ by $981$ the student is not taught to memorize the times table for 
$1,234$ instead they are taught how to long-multiply.  That is to multiply each column using simple single digit 
multiplications and add the resulting products by column.  The representation that most are familiar with is known as 
decimal or formally as radix-10. A radix-$n$ representation simply means there are $n$ possible values per digit.  
For example, binary would be a radix-2 representation.

In essence computer based multiple precision arithmetic is very much the same.  The most notable difference is the usage
of a binary friendly radix.  That is to use a radix of the form $2^k$ where $k$ is typically the size of a machine 
register.  Also occasionally more optimal algorithms are used to perform certain operations such as multiplication and 
squaring instead of traditional long-hand algorithms.

\section{Purpose of This Text}
The purpose of this text is to instruct the reader regarding how to implement multiple precision algorithms.  That is 
to not only explain the core theoretical algorithms but also the various ``house keeping'' tasks that are neglected by
authors of other texts on the subject.  Texts such as Knuths' ``The Art of Computer Programming, vol 2.'' and the 
Handbook of Applied Cryptography (\textit{HAC}) give considerably detailed explanations of the theoretical aspects of 
the algorithms and very little regarding the practical aspects.  

That is how an algorithm is explained and how it is actually implemented are two very different 
realities.  For example, algorithm 14.7 on page 594 of HAC lists a relatively simple algorithm for performing multiple 
precision integer addition.  However, what the description lacks is any discussion concerning the fact that the two 
integer inputs may be of differing magnitudes.  Similarly the division routine (\textit{Algorithm 14.20, pp. 598}) 
does not discuss how to handle sign or handle the dividends decreasing magnitude in the main loop (\textit{Step \#3}).

As well as the numerous practical oversights both of the texts do not discuss several key optimal algorithms required 
such as ``Comba'' and Karatsuba multipliers and fast modular inversion.  These optimal algorithms are considerably
vital to achieve any form of useful performance in non-trivial applications.  

To solve this problem the focus of this text is on the practical aspects of implementing the algorithms that 
constitute a multiple precision integer package with light cursory discussions on the theoretical aspects.  As a case 
study the ``LibTomMath''\footnote{Available freely at http://math.libtomcrypt.org} package is used to demonstrate 
algorithms with implementations that have been field tested and work very well.

\section{Discussion and Notation}
\subsection{Notation}
A multiple precision integer of $n$-digits shall be denoted as $x = (x_n ... x_1 x_0)_{ \beta }$ to be the 
multiple precision notation for the integer $x \equiv \sum_{i=0}^{n} x_i\beta^i$.  The elements of the array $x$ are
said to be the radix $\beta$ digits of the integer.  For example, $x = (15,0,7)_{\beta}$ would represent the 
integer $15\cdot\beta^2 + 0\cdot\beta^1 + 7\cdot\beta^0$.  

A ``mp\_int'' shall refer to a composite structure which contains the digits of the integer as well as auxilary data
required to manipulate the data.  These additional members are discussed in chapter three.  For the purposes of this text
a ``multiple precision integer'' and a ``mp\_int'' are synonymous.

\index{single-precision} \index{double-precision} \index{mp\_digit} \index{mp\_word}
For the purposes of this text a single-precision variable must be able to represent integers in the range $0 \le x < 2 \beta$ while
a double-precision variable must be able to represent integers in the range $0 \le x < 2 \beta^2$.  Within the source code that will be
presented the data type \textbf{mp\_digit} will represent a single-precision type while \textbf{mp\_word} will represent a 
double-precision type.  In several algorithms (\textit{notably the Comba routines}) temporary results 
will be stored in a double-precision arrays.  For the purposes of this text $x_j$ will refer to the 
$j$'th digit of a single-precision array and $\hat x_j$ will refer to the $j$'th digit of a double-precision
array.

\subsection{Work Effort}
\index{big-O}
To measure the efficiency of various algorithms a modified big-O notation is used.  In this system all 
single precision operations are considered to have the same cost\footnote{Except where explicitly noted.}.  
That is a single precision addition, multiplication and division are assumed to take the same time to 
complete.  While this is generally not true in practice it will simplify the discussions considerably.

Some algorithms have slight advantages over others which is why some constants will not be removed in 
the notation.  For example, a normal multiplication requires $O(n^2)$ work while a squaring requires 
$O({{n^2 + n}\over 2})$ work.  In standard big-O notation these would be said to be equivalent.  However, in the 
context of the this text the magnitude of the inputs will not approach an infinite size.  This means the conventional limit 
notation wisdom does not apply to the cancellation of constants.

Throughout the discussions various ``work levels'' will be discussed.  These levels are the $O(1)$,
$O(n)$, $O(n^2)$, ..., $O(n^k)$ work efforts.  For example, operations at the $O(n^k)$ ``level'' are said to be
executed more frequently than operations at the $O(n^m)$ ``level'' when $k > m$.  Obviously most optimizations will pay
off the most at the higher levels since they represent the bulk of the effort required.  

\section{Exercises}
Within the more advanced chapters a section will be set aside to give the reader some challenging exercises.  These exercises are not 
designed to be prize winning problems yet instead to be thought provoking.  Wherever possible the problems are foreward minded stating 
problems that will be answered in subsequent chapters.  The reader is encouraged to finish the exercises as they appear to get a 
better understanding of the subject material.  

Similar to the exercises of \cite{TAOCPV2} as explained on pp.\textit{ix} these exercises are given a scoring system.  However, unlike 
\cite{TAOCPV2} the problems do not get nearly as hard as often.  The scoring of these exercises ranges from one (\textit{the easiest}) to
five (\textit{the hardest}).  The following table sumarizes the scoring.

\vspace{5mm}
\begin{tabular}{cl}
$\left [ 1 \right ]$ & An easy problem that should only take the reader a manner of \\
                     & minutes to solve.  Usually does not involve much computer time. \\
                     & \\
$\left [ 2 \right ]$ & An easy problem that involves a marginal amount of computer \\
                     & time usage.  Usually requires a program to be written to \\
                     & solve the problem. \\
                     & \\
$\left [ 3 \right ]$ & A moderately hard problem that requires a non-trivial amount \\
                     & of work.  Usually involves trivial research and development of \\
                     & new theory from the perspective of a student. \\
                     & \\
$\left [ 4 \right ]$ & A moderately hard problem that involves a non-trivial amount \\
                     & of work and research.  The solution to which will demonstrate \\
                     & a higher mastery of the subject matter. \\
                     & \\
$\left [ 5 \right ]$ & A hard problem that involves concepts that are non-trivial.  \\
                     & Solutions to these problems will demonstrate a complete mastery \\
                     & of the given subject. \\
                     & \\
\end{tabular}

Essentially problems at the first level are meant to be simple questions that the reader can answer quickly without programming a solution or
devising new theory.  These problems are quick tests to see if the material is understood.  Problems at the second level are also
designed to be easy but will require a program or algorithm to be implemented to arrive at the answer.  

Problems at the third level are meant to be a bit more difficult.  Often the answer is fairly obvious but arriving at an exacting solution
requires some thought and skill.  These problems will almost always involve devising a new algorithm or implementing a variation of
another algorithm.

Problems at the fourth level are meant to be even more difficult as well as involve some research.  The reader will most likely not know
the answer right away nor will this text provide the exact details of the answer (\textit{or at least not until a subsequent chapter}).  Problems
at the fifth level are meant to be the hardest problems relative to all the other problems in the chapter.  People who can correctly 
answer fifth level problems have a mastery of the subject matter at hand.

Often problems will be tied together.  The purpose of this is to start a chain of thought that will be discussed in future chapters.  The reader
is encouraged to answer the follow-up problems and try to draw the relevence of problems.

\chapter{Introduction to LibTomMath}

\section{What is the LibTomMath?}
LibTomMath is a free and open source multiple precision number theoretic library written in portable ISO C
source code.  By portable it is meant that the library does not contain any code that is platform dependent or otherwise
problematic to use on any given platform.  The library has been successfully tested under numerous operating systems 
including Solaris, MacOS, Windows, Linux, PalmOS and on standalone hardware such as the Gameboy Advance.  The 
library is designed to contain enough functionality to be able to develop number theoretic applications such as public 
key cryptosystems.

\section{Goals of the LibTomMath}

Even though the library is written entirely in portable ISO C considerable care has been taken to 
optimize the algorithm implementations within the library.  Specifically the code has been written to work well with
the GNU C Compiler (\textit{GCC}) on both x86 and ARMv4 processors.  Wherever possible optimal 
algorithms (\textit{such as Karatsuba multiplication, sliding window exponentiation and Montgomery reduction.}) have 
been provided to make the library as efficient as possible.  Even with the optimal and sometimes specialized 
algorithms that have been included the API has been kept as simple as possible.  Often generic place holder routines 
will make use of specialized algorithms automatically without the developers attention.  One such example
is the generic multiplication algorithm \textbf{mp\_mul()} which will automatically use Karatsuba multiplication if the 
inputs are of a specific size.

Making LibTomMath as efficient as possible is not the only goal of the LibTomMath project.  Ideally the library should 
be source compatible with another popular library which makes it more attractive for developers to use.  In this case the
MPI library was used as a API template for all the basic functions.

The project is also meant to act as a learning tool for students.  The logic being that no easy to follow ``bignum'' 
library exists which can be used to teach computer science students how to perform fast and reliable multiple precision 
arithmetic.  To this end the source code has been given quite a few comments and algorithm discussion points.  Often 
where applicable routines have more comments than lines of code.

\section{Choice of LibTomMath}
LibTomMath was chosen as the case study of this text not only because the author of both projects is one and the same but
for more worthy reasons.  Other libraries such as GMP, MPI, LIP and OpenSSL have multiple precision 
integer arithmetic routines but would not be ideal for this text for numerous reasons as will be explained in the 
following sub-sections.

\subsection{Code Base}
The LibTomMath code base is all portable ISO C source code.  This means that there are no platform dependent conditional
segments of code littered throughout the source.  This clean and uncluttered approach to the library means that a
developer can more readily ascertain the true intent of a given section of source code without trying to keep track of
what conditional code will be used.

The code base of LibTomMath is also exceptionally well organized.  Each function is in its own separate source code file 
which allows the reader to find a given function very fast.  When compiled with GCC for the x86 processor the entire 
library is a mere 87,760 bytes (\textit{$116,182$ bytes for ARMv4 processors}).  This includes every single function 
LibTomMath provides from basic arithmetic to various number theoretic functions such as modular exponentiation, various 
reduction algorithms and Jacobi symbol computation.  

By comparison MPI which has fewer number theoretic functions than LibTomMath compiled with the same conditions is 
45,429 bytes (\textit{$54,536$ for ARMv4}).  GMP which has rather large collection of functions with the default 
configuration on an x86 Athlon is 2,950,688 bytes.  Note that while LibTomMath has fewer functions than GMP it has been
been used as the sole basis for several public key cryptosystems without having to seek additional outside functions
to supplement the library.

\subsection{API Simplicity}
LibTomMath is designed after the MPI library and shares the API design.  Quite often programs that use MPI will build 
with LibTomMath without change. The function names are relatively straight forward as to what they perform.  Almost all of the 
functions except for a few minor exceptions which as will be discussed are for good reasons share the same parameter passing 
convention.  The learning curve is fairly shallow with the API provided which is an extremely valuable benefit for the 
student and developer alike.  

The LIP library is an example of a library with an API that is awkward to work with.  LIP uses function names that are often ``compressed'' to 
illegible short hand.  LibTomMath does not share this fault.

\subsection{Optimizations}
While LibTomMath is certainly not the fastest library (\textit{GMP often beats LibTomMath by a factor of two}) it does
feature a set of optimal algorithms for tasks ranging from modular reduction to squaring.  GMP and LIP also feature
such optimizations while MPI only uses baseline algorithms with no optimizations.

LibTomMath is almost always a magnitude faster than the MPI library at computationally expensive tasks such as modular
exponentiation.  In the grand scheme of ``bignum'' libraries LibTomMath is faster than the average library and usually  
slower than the best libraries such as GMP and OpenSSL by a small factor.

\subsection{Portability and Stability}
LibTomMath will build ``out of the box'' on any platform equipped with a modern version of the GNU C Compiler 
(\textit{GCC}).  This means that without changes the library will build without configuration or setting up any 
variables.  LIP and MPI will build ``out of the box'' as well but have numerous known bugs.  Most notably the author of 
MPI is not working on his library anymore.  

GMP requires a configuration script to run and will not build out of the box.   GMP and LibTomMath are still in active
development and are very stable across a variety of platforms.

\subsection{Choice}
LibTomMath is a relatively compact, well documented, highly optimized and portable library which seems only natural for
the case study of this text.  Various source files from the LibTomMath project will be included within the text.  However, the 
reader is encouraged to download their own copy of the library to actually be able to work with the library.  

\chapter{Getting Started}
\section{Library Basics}
To get the ``ball rolling'' so to speak a primitive data type and a series of primitive algorithms must be established.  First a data
type that will hold the information required to maintain a multiple precision integer must be designed.  With this basic data type of a series
of low level algorithms for initializing, clearing, growing and clamping integers can be developed to form the basis of the entire
package of algorithms.

\section{The mp\_int structure}
First the data type for storing multiple precision integers must be designed.  This data type must be able to hold information to 
maintain an array of digits, how many are actually used in the representation and the sign.  The ISO C standard does not provide for 
any such data type but it does provide for making composite data types known as structures.  The following is the structure definition 
used within LibTomMath.

\index{mp\_int}
\begin{verbatim}
typedef struct  {
    int used, alloc, sign;
    mp_digit *dp;
} mp_int;
\end{verbatim}

The \textbf{used} parameter denotes how many digits of the array \textbf{dp} are actually being used.  The array 
\textbf{dp} holds the digits that represent the integer desired.  The \textbf{alloc} parameter denotes how 
many digits are available in the array to use by functions before it has to increase in size.  When the \textbf{used} count 
of a result would exceed the \textbf{alloc} count all LibTomMath routines will automatically increase the size of the 
array to accommodate the precision of the result.  The \textbf{sign} parameter denotes the sign as either zero/positive 
(\textbf{MP\_ZPOS}) or negative (\textbf{MP\_NEG}).  

\section{Argument Passing}
A convention of arugment passing must be adopted early on in the development of any library.  Making the function prototypes
consistent will help eliminate many headaches in the future as the library grows to significant complexity.  In LibTomMath the multiple precision 
integer functions accept parameters from left to right as pointers to mp\_int structures.  That means that the source operands are 
placed on the left and the destination on the right.   Consider the following examples.

\begin{verbatim}
   mp_mul(&a, &b, &c);   /* c = a * b */
   mp_add(&a, &b, &a);   /* a = a + b */
   mp_sqr(&a, &b);       /* b = a * a */
\end{verbatim}

The left to right order is a fairly natural way to implement the functions since it lets the developer read aloud the
functions and make sense of them.  For example, the first function would read ``multiply a and b and store in c''.

Certain libraries (\textit{LIP by Lenstra for instance}) accept parameters the other way around.  That is the destination
on the left and arguments on the right.  In truth it is entirely a matter of preference.  

Another very useful design consideration is whether to allow argument sources to also be a destination.  For example, the
second example (\textit{mp\_add}) adds $a$ to $b$ and stores in $a$.  This is an important feature to implement since it
allows the higher up functions to cut down on the number of variables.  However, to implement this feature specific
care has to be given to ensure the destination is not written before the source is fully read.

\section{Return Values}
A well implemented library, no matter what its purpose, should trap as many runtime errors as possible and return them to the 
caller.  By catching runtime errors a library can be guaranteed to prevent undefined behaviour within reason.  In a multiple precision 
library the only errors that are bound to occur are related to inappropriate inputs (\textit{division by zero for instance}) or 
memory allocation errors.

In LibTomMath any function that can cause a runtime error will return an error as an \textbf{int} data type with one of the 
following values.

\index{MP\_OKAY} \index{MP\_VAL} \index{MP\_MEM}
\begin{center}
\begin{tabular}{|l|l|}
\hline \textbf{Value} & \textbf{Meaning} \\
\hline \textbf{MP\_OKAY} & The function was successful \\
\hline \textbf{MP\_VAL}  & One of the input value(s) was invalid \\
\hline \textbf{MP\_MEM}  & The function ran out of heap memory \\
\hline
\end{tabular}
\end{center}

When an error is detected within a function it should free any memory they allocated and return as soon as possible.  The goal
is to leave the system in the same state the system was when the function was called.  Error checking with this style of API is fairly simple.

\begin{verbatim}
   int err;
   if ((err = mp_add(&a, &b, &c)) != MP_OKAY) {
      printf("Error: %d\n", err);
      exit(EXIT_FAILURE);
   }
\end{verbatim}

The GMP library uses C style \textit{signals} to flag errors which is of questionable use.  Not all errors are fatal 
and it is not ideal to force developers to have signal handlers for such cases.

\section{Initialization and Clearing}
The logical starting point when actually writing multiple precision integer functions is the initialization and 
clearing of the integers.  These two functions will be used by far the most throughout the algorithms whenever 
temporary integers are required.

Given the basic mp\_int structure an initialization routine must first allocate memory to hold the digits of
the integer.  Often it is optimal to allocate a sufficiently large pre-set number of digits even considering
the initial integer will represent zero.  If only a single digit were allocated quite a few re-allocations
would occur for the majority of inputs.  There exists a tradeoff between how many default digits to allocate
and how many re-allocations are tolerable.  

If the memory for the digits has been successfully allocated then the rest of the members of the structure must
be initialized.  Since the initial state is to represent a zero integer the digits allocated must all be zeroed.  The
\textbf{used} count set to zero and \textbf{sign} set to \textbf{MP\_ZPOS}.

\subsection{Initializing an mp\_int}
To initialize an mp\_int the mp\_init algorithm shall be used.  The purpose of this algorithm is to allocate 
the memory required and initialize the integer to a default representation of zero.

\begin{figure}[here]
\begin{center}
\begin{tabular}{l}
\hline Algorithm \textbf{mp\_init}. \\
\textbf{Input}.   An mp\_int $a$ \\
\textbf{Output}.  Allocate memory for the digits and set to a zero state. \\
\hline \\
1.  Allocate memory for \textbf{MP\_PREC} digits. \\
2.  If the allocation failed then return(\textit{MP\_MEM}) \\
3.  for $n$ from $0$ to $MP\_PREC - 1$ do  \\
\hspace{3mm}3.1  $a_n \leftarrow 0$\\
4.  $a.sign \leftarrow MP\_ZPOS$\\
5.  $a.used \leftarrow 0$\\
6.  $a.alloc \leftarrow MP\_PREC$\\
7.  Return(\textit{MP\_OKAY})\\
\hline
\end{tabular}
\end{center}
\caption{Algorithm mp\_init}
\end{figure}

\textbf{Algorithm mp\_init.}
The \textbf{MP\_PREC} variable is a simple constant used to dictate minimal precision of allocated integers.  It is ideally at least equal to $32$ but 
can be any reasonable power of two.  Step one and two allocate the memory and account for it.  If the allocation fails the algorithm returns
immediately to signal the failure.  Step three will ensure that all the digits are in the default state of zero.  Finally steps 
four through six set the default settings of the \textbf{sign}, \textbf{used} and \textbf{alloc} members of the mp\_int structure.

\index{bn\_mp\_init.c}
\vspace{+3mm}\begin{small}
\hspace{-5.1mm}{\bf File}: bn\_mp\_init.c
\vspace{-3mm}
\begin{alltt}
016   
017   /* init a new bigint */
018   int
019   mp_init (mp_int * a)
020   \{
021     /* allocate ram required and clear it */
022     a->dp = OPT_CAST calloc (sizeof (mp_digit), MP_PREC);
023     if (a->dp == NULL) \{
024       return MP_MEM;
025     \}
026   
027     /* set the used to zero, allocated digit to the default precision
028      * and sign to positive */
029     a->used  = 0;
030     a->alloc = MP_PREC;
031     a->sign  = MP_ZPOS;
032   
033     return MP_OKAY;
034   \}
\end{alltt}
\end{small}

The \textbf{OPT\_CAST} type cast on line 22 is designed to allow C++ compilers to build the code out of
the box.  Microsoft C V5.00 is known to cause problems without the cast.  Also note that if the memory
allocation fails the other members of the mp\_int will be in an undefined state.  The code from 
line 29 to line 31 sets the default state for a mp\_int which is zero, positive and no used digits.

\subsection{Clearing an mp\_int}
When an mp\_int is no longer required the memory allocated for it can be cleared from the heap with 
the mp\_clear algorithm.

\begin{figure}[here]
\begin{center}
\begin{tabular}{l}
\hline Algorithm \textbf{mp\_clear}. \\
\textbf{Input}.   An mp\_int $a$ \\
\textbf{Output}.  The memory for $a$ is cleared. \\
\hline \\
1.  If $a$ has been previously freed then return(\textit{MP\_OKAY}). \\
2.  Free the digits of $a$ and mark $a$ as freed. \\
3.  $a.used \leftarrow 0$ \\
4.  $a.alloc \leftarrow 0$ \\
5.  Return(\textit{MP\_OKAY}). \\
\hline
\end{tabular}
\end{center}
\caption{Algorithm mp\_clear}
\end{figure}

\textbf{Algorithm mp\_clear.}
In steps one and two the memory for the digits are only free'd if they had not been previously released before.  
This is more of concern for the implementation since it is used to prevent ``double-free'' errors.  It also helps catch
code errors where mp\_ints are used after being cleared.  Simiarly steps three and four set the 
\textbf{used} and \textbf{alloc} to known values which would be easy to spot during debugging.  For example, if an mp\_int is expected
to be non-zero and its \textbf{used} member observed to be zero (\textit{due to being cleared}) then an obvious bug in the code has been
spotted.

\index{bn\_mp\_clear.c}
\vspace{+3mm}\begin{small}
\hspace{-5.1mm}{\bf File}: bn\_mp\_clear.c
\vspace{-3mm}
\begin{alltt}
016   
017   /* clear one (frees)  */
018   void
019   mp_clear (mp_int * a)
020   \{
021     if (a->dp != NULL) \{
022   
023       /* first zero the digits */
024       memset (a->dp, 0, sizeof (mp_digit) * a->used);
025   
026       /* free ram */
027       free (a->dp);
028   
029       /* reset members to make debugging easier */
030       a->dp = NULL;
031       a->alloc = a->used = 0;
032     \}
033   \}
\end{alltt}
\end{small}

The \textbf{if} statement on line 21 prevents the heap from being corrupted if a user double-frees an 
mp\_int.  For example, a trivial case of this bug would be as follows.

\begin{verbatim}
mp_int a;
mp_init(&a);
mp_clear(&a);
mp_clear(&a);
\end{verbatim}

Without that check the code would try to free the memory allocated for the digits twice which will cause most standard C
libraries to cause a fault.  Also by setting the pointer to \textbf{NULL} it helps debug code that may inadvertently 
free the mp\_int before it is truly not needed.  The allocated digits are set to zero before being freed on line 24.  
This is ideal for cryptographic situations where the mp\_int is a secret parameter.

The following snippet is an example of using both the init and clear functions.  

\begin{small}
\begin{verbatim}
#include <tommath.h>
#include <stdio.h>
#include <stdlib.h>
int main(void)
{
   mp_int num;
   int err;
   
   /* init the bignum */
   if ((err = mp_init(&num)) != MP_OKAY) {
      printf("Error: %d\n", err);
      return EXIT_FAILURE;
   }
   
   /* do work with it ... */
   
   /* clear up */
   mp_clear(&num);
   
   return EXIT_SUCCESS;
}
\end{verbatim}
\end{small}

\section{Other Initialization Routines}

It is often helpful to have specialized initialization algorithms to simplify the design of other algorithms.  For example, an 
initialization followed by a copy is a common operation when temporary copies of integers are required.  It is quite
beneficial to have a series of simple helper functions available.

\subsection{Initializing Variable Sized mp\_int Structures}
Occasionally the number of digits required will be known in advance of an initialization.  In these
cases the mp\_init\_size algorithm can be of use.  The purpose of this algorithm is similar to mp\_init except that 
it will allocate \textit{at least} a specified number of digits.  This is ideal to prevent re-allocations when the 
input size is known.

\newpage\begin{figure}[here]
\begin{center}
\begin{tabular}{l}
\hline Algorithm \textbf{mp\_init\_size}. \\
\textbf{Input}.   An mp\_int $a$ and the requested number of digits $b$\\
\textbf{Output}.  $a$ is initialized to hold at least $b$ digits. \\
\hline \\
1.  $u \leftarrow b\mbox{ (mod }MP\_PREC\mbox{)}$ \\
2.  $v \leftarrow b + 2 \cdot MP\_PREC - u$ \\
3.  Allocate $v$ digits. \\
4.  If the allocation failed then return(\textit{MP\_MEM}). \\
5.  for $n$ from $0$ to $v - 1$ do \\
\hspace{3mm}5.1  $a_n \leftarrow 0$ \\
6.  $a.sign \leftarrow MP\_ZPOS$\\
7.  $a.used \leftarrow 0$\\
8.  $a.alloc \leftarrow v$\\
9.  Return(\textit{MP\_OKAY})\\
\hline
\end{tabular}
\end{center}
\caption{Algorithm mp\_init\_size}
\end{figure}

\textbf{Algorithm mp\_init\_size.}
The value of $v$ is calculated to be at least the requested amount of digits $b$ plus additional padding.  The padding is calculated
to be at least \textbf{MP\_PREC} digits plus enough digits to make the digit count a multiple of \textbf{MP\_PREC}.  This padding is used to 
prevent trivial allocations from becomming a bottleneck in the rest of the algorithms that depend on this.

\index{bn\_mp\_init\_size.c}
\vspace{+3mm}\begin{small}
\hspace{-5.1mm}{\bf File}: bn\_mp\_init\_size.c
\vspace{-3mm}
\begin{alltt}
016   
017   /* init a mp_init and grow it to a given size */
018   int
019   mp_init_size (mp_int * a, int size)
020   \{
021   
022     /* pad size so there are always extra digits */
023     size += (MP_PREC * 2) - (size & (MP_PREC - 1));    
024     
025     /* alloc mem */
026     a->dp = OPT_CAST calloc (sizeof (mp_digit), size);
027     if (a->dp == NULL) \{
028       return MP_MEM;
029     \}
030     a->used = 0;
031     a->alloc = size;
032     a->sign = MP_ZPOS;
033   
034     return MP_OKAY;
035   \}
\end{alltt}
\end{small}

Line 23 will ensure that the number of digits actually allocated is padded up to the next multiple of 
\textbf{MP\_PREC} plus an additional \textbf{MP\_PREC}.  This ensures that the number of allocated digit is 
always greater than the amount requested.  As a result it prevents many trivial memory allocations.  The value of 
\textbf{MP\_PREC} is defined in ``tommath.h'' and must be a power of two.

\subsection{Creating a Clone}
Another common sequence of operations is to make a local temporary copy of an argument.  To initialize then copy a mp\_int will be known as 
creating a clone.  This is useful within functions that need to modify an integer argument but do not wish to actually modify the original copy.  
The mp\_init\_copy algorithm will perform this very task.

\begin{figure}[here]
\begin{center}
\begin{tabular}{l}
\hline Algorithm \textbf{mp\_init\_copy}. \\
\textbf{Input}.   An mp\_int $a$ and $b$\\
\textbf{Output}.  $a$ is initialized to be a copy of $b$. \\
\hline \\
1.  Init $a$.  (\textit{hint: use mp\_init}) \\
2.  If the init of $a$ was unsuccessful return(\textit{MP\_MEM}) \\
3.  Copy $b$ to $a$.  (\textit{hint: use mp\_copy}) \\
4.  Return the status of the copy operation. \\
\hline
\end{tabular}
\end{center}
\caption{Algorithm mp\_init\_copy}
\end{figure}

\textbf{Algorithm mp\_init\_copy.}
This algorithm will initialize a mp\_int variable and copy another previously initialized mp\_int variable into it.  The algorithm will
detect when the initialization fails and returns the error to the calling algorithm.  As such this algorithm will perform two operations
in one step.  

\index{bn\_mp\_init\_copy.c}
\vspace{+3mm}\begin{small}
\hspace{-5.1mm}{\bf File}: bn\_mp\_init\_copy.c
\vspace{-3mm}
\begin{alltt}
016   
017   /* creates "a" then copies b into it */
018   int
019   mp_init_copy (mp_int * a, mp_int * b)
020   \{
021     int     res;
022   
023     if ((res = mp_init (a)) != MP_OKAY) \{
024       return res;
025     \}
026     return mp_copy (b, a);
027   \}
\end{alltt}
\end{small}

This will initialize \textbf{a} and make it a verbatim copy of the contents of \textbf{b}.  Note that 
\textbf{a} will have its own memory allocated which means that \textbf{b} may be cleared after the call
and \textbf{a} will be left intact.  

\subsection{Multiple Integer Initializations}
Occasionally a function will require a series of mp\_int data types to be made available.  The mp\_init\_multi algorithm
is provided to simplify such cases.  The purpose of this algorithm is to initialize a variable length array of mp\_int 
structures at once.  As a result algorithms that require multiple integers only has to use 
one algorithm to initialize all the mp\_int variables.

\begin{figure}[here]
\begin{center}
\begin{tabular}{l}
\hline Algorithm \textbf{mp\_init\_multi}. \\
\textbf{Input}.   Variable length array of mp\_int variables of length $k$. \\
\textbf{Output}.  The array is initialized such that each each mp\_int is ready to use. \\
\hline \\
1.  for $n$ from 0 to $k - 1$ do \\
\hspace{+3mm}1.1.  Initialize the $n$'th mp\_int (\textit{hint: use mp\_init}) \\
\hspace{+3mm}1.2.  If initialization failed then do \\
\hspace{+6mm}1.2.1.  for $j$ from $0$ to $n$ do \\
\hspace{+9mm}1.2.1.1.  Free the $j$'th mp\_int (\textit{hint: use mp\_clear}) \\
\hspace{+6mm}1.2.2.   Return(\textit{MP\_MEM}) \\
2.  Return(\textit{MP\_OKAY}) \\
\hline
\end{tabular}
\end{center}
\caption{Algorithm mp\_init\_multi}
\end{figure}

\textbf{Algorithm mp\_init\_multi.}
The algorithm will initialize the array of mp\_int variables one at a time.  As soon as an runtime error is detected (\textit{step 1.2}) all of
the previously initialized variables are cleared.  The goal is an ``all or nothing'' initialization which allows for quick recovery from runtime 
errors.

\subsection{Multiple Integer Clearing}
Similarly to clear a variable length list of mp\_int structures the mp\_clear\_multi algorithm will be used.

\index{bn\_mp\_multi.c}
\vspace{+3mm}\begin{small}
\hspace{-5.1mm}{\bf File}: bn\_mp\_multi.c
\vspace{-3mm}
\begin{alltt}
016   #include <stdarg.h>
017   
018   int mp_init_multi(mp_int *mp, ...) 
019   \{
020       mp_err res = MP_OKAY;      /* Assume ok until proven otherwise */
021       int n = 0;                 /* Number of ok inits */
022       mp_int* cur_arg = mp;
023       va_list args;
024   
025       va_start(args, mp);        /* init args to next argument from caller */
026       while (cur_arg != NULL) \{
027           if (mp_init(cur_arg) != MP_OKAY) \{
028               /* Oops - error! Back-track and mp_clear what we already
029                  succeeded in init-ing, then return error.
030               */
031               va_list clean_args;
032               
033               /* end the current list */
034               va_end(args);
035               
036               /* now start cleaning up */            
037               cur_arg = mp;
038               va_start(clean_args, mp);
039               while (n--) \{
040                   mp_clear(cur_arg);
041                   cur_arg = va_arg(clean_args, mp_int*);
042               \}
043               va_end(clean_args);
044               res = MP_MEM;
045               break;
046           \}
047           n++;
048           cur_arg = va_arg(args, mp_int*);
049       \}
050       va_end(args);
051       return res;                /* Assumed ok, if error flagged above. */
052   \}
053   
054   void mp_clear_multi(mp_int *mp, ...) 
055   \{
056       mp_int* next_mp = mp;
057       va_list args;
058       va_start(args, mp);
059       while (next_mp != NULL) \{
060           mp_clear(next_mp);
061           next_mp = va_arg(args, mp_int*);
062       \}
063       va_end(args);
064   \}
\end{alltt}
\end{small}

Consider the following snippet which demonstrates how to use both routines.
\begin{small}
\begin{verbatim}
#include <tommath.h>
#include <stdio.h>
#include <stdlib.h>
int main(void)
{
   mp_int num1, num2, num3;
   int err;
   
   if ((err = mp_init_multi(&num1, &num2, &num3, NULL)) !- MP_OKAY) {
      printf("Error: %d\n", err);
      return EXIT_FAILURE;
   }
   
   /* at this point num1/num2/num3 are ready */
   
   /* free them */
   mp_clear_multi(&num1, &num2, &num3, NULL);
   
   return EXIT_SUCCESS;
}
\end{verbatim}
\end{small}

\section{Maintenance}
A small useful collection of mp\_int maintenance functions will also prove useful.  

\subsection{Augmenting Integer Precision}
When storing a value in an mp\_int sufficient digits must be available to accomodate the entire value without
loss of precision.  Quite often the size of the array given by the \textbf{alloc} member is large enough to simply
increase the \textbf{used} digit count.  However, when the size of the array is too small it must be re-sized 
appropriately to accomodate the result.  The mp\_grow algorithm will provide this functionality.

\begin{figure}[here]
\begin{center}
\begin{tabular}{l}
\hline Algorithm \textbf{mp\_grow}. \\
\textbf{Input}.   An mp\_int $a$ and an integer $b$. \\
\textbf{Output}.  $a$ is expanded to accomodate $b$ digits. \\
\hline \\
1.  if $a.alloc \ge b$ then return(\textit{MP\_OKAY}) \\
2.  $u \leftarrow b\mbox{ (mod }MP\_PREC\mbox{)}$ \\
3.  $v \leftarrow b + 2 \cdot MP\_PREC - u$ \\
4.  Re-Allocate the array of digits $a$ to size $v$ \\
5.  If the allocation failed then return(\textit{MP\_MEM}). \\
6.  for n from a.alloc to $v - 1$ do  \\
\hspace{+3mm}6.1  $a_n \leftarrow 0$ \\
7.  $a.alloc \leftarrow v$ \\
8.  Return(\textit{MP\_OKAY}) \\
\hline
\end{tabular}
\end{center}
\caption{Algorithm mp\_grow}
\end{figure}

\textbf{Algorithm mp\_grow.}
Step one will prevent a re-allocation from being performed if it was not required.  This is useful to prevent mp\_ints
from growing excessively in code that erroneously calls mp\_grow.  Similar to mp\_init\_size the requested digit count
is padded to provide more digits than requested.  

In step four it is assumed that the reallocation leaves the lower $a.alloc$ digits intact.  Much akin to how the 
\textit{realloc} function from the standard C library works.  Since the newly allocated digits are assumed to contain
undefined values they are also initially zeroed.

\index{bn\_mp\_grow.c}
\vspace{+3mm}\begin{small}
\hspace{-5.1mm}{\bf File}: bn\_mp\_grow.c
\vspace{-3mm}
\begin{alltt}
016   
017   /* grow as required */
018   int
019   mp_grow (mp_int * a, int size)
020   \{
021     int     i;
022   
023     /* if the alloc size is smaller alloc more ram */
024     if (a->alloc < size) \{
025       /* ensure there are always at least MP_PREC digits extra on top */
026       size += (MP_PREC * 2) - (size & (MP_PREC - 1));     
027   
028       a->dp = OPT_CAST realloc (a->dp, sizeof (mp_digit) * size);
029       if (a->dp == NULL) \{
030         return MP_MEM;
031       \}
032   
033       /* zero excess digits */
034       i        = a->alloc;
035       a->alloc = size;
036       for (; i < a->alloc; i++) \{
037         a->dp[i] = 0;
038       \}
039     \}
040     return MP_OKAY;
041   \}
\end{alltt}
\end{small}

The first step is to see if we actually need to perform a re-allocation at all.  This is tested for on line 
24.  Similar to mp\_init\_size the same code on line 26 was used to resize the 
digits requested.  A simple for loop from line 34 to line 38 will zero all digits that were above the 
old \textbf{alloc} limit to make sure the integer is in a known state.

\subsection{Clamping Excess Digits}
When a function anticipates a result will be $n$ digits it is simpler to assume this is true within the body of 
the function.  For example, a multiplication of a $i$ digit number by a $j$ digit produces a result of at most 
$i + j + 1$ digits.  It is entirely possible that the result is $i + j$ though, with no final carry into the last 
position.  However, suppose the destination had to be first expanded (\textit{via mp\_grow}) to accomodate $i + j$
digits than further expanded to accomodate the final carry.  That would be a considerable waste of time since heap
operations are relatively slow.

The ideal solution is to always assume the result is $i + j + 1$ and fix up the \textbf{used} count after the function
terminates.  This way a single heap operation (\textit{at most}) is required.  However, if the result was not checked
there would be an excess high order zero digit.  

For example, suppose the product of two integers was $x_n = (0x_{n-1}x_{n-2}...x_0)_{\beta}$.  The leading zero digit 
will not contribute to the precision of the result.  In fact, through subsequent operations more leading zero digits would
accumulate to the point the size of the integer would be prohibitive.  As a result even though the precision is very 
low the representation is excessively large.  

The mp\_clamp algorithm is designed to solve this very problem.  It will trim leading zeros by decrementing the 
\textbf{used} count until a non-zero leading digit is found.  Also in this system, zero is considered to be a positive 
number which means that if the \textbf{used} count is decremented to zero the sign must be set to \textbf{MP\_ZPOS}.

\begin{figure}[here]
\begin{center}
\begin{tabular}{l}
\hline Algorithm \textbf{mp\_clamp}. \\
\textbf{Input}.   An mp\_int $a$ \\
\textbf{Output}.  Any excess leading zero digits of $a$ are removed \\
\hline \\
1.  while $a.used > 0$ and $a_{a.used - 1} = 0$ do \\
\hspace{+3mm}1.1  $a.used \leftarrow a.used - 1$ \\
2.  if $a.used = 0$ then do \\
\hspace{+3mm}2.1  $a.sign \leftarrow MP\_ZPOS$ \\
\hline \\
\end{tabular}
\end{center}
\caption{Algorithm mp\_clamp}
\end{figure}

\textbf{Algorithm mp\_clamp.}
As can be expected this algorithm is very simple.  The loop on step one is indended to be iterate only once or twice at
the most.  For example, for cases where there is not a carry to fill the last position.  Step two fixes the sign for 
when all of the digits are zero to ensure that the mp\_int is valid at all times.

\index{bn\_mp\_clamp.c}
\vspace{+3mm}\begin{small}
\hspace{-5.1mm}{\bf File}: bn\_mp\_clamp.c
\vspace{-3mm}
\begin{alltt}
016   
017   /* trim unused digits 
018    *
019    * This is used to ensure that leading zero digits are
020    * trimed and the leading "used" digit will be non-zero
021    * Typically very fast.  Also fixes the sign if there
022    * are no more leading digits
023    */
024   void
025   mp_clamp (mp_int * a)
026   \{
027     while (a->used > 0 && a->dp[a->used - 1] == 0) \{
028       --(a->used);
029     \}
030     if (a->used == 0) \{
031       a->sign = MP_ZPOS;
032     \}
033   \}
\end{alltt}
\end{small}

Note on line 27 how to test for the \textbf{used} count is made on the left of the \&\& operator.  In the C programming
language the terms to \&\& are evaluated left to right with a boolean short-circuit if any condition fails.  This is 
important since if the \textbf{used} is zero the test on the right would fetch below the array.  That is obviously 
undesirable.  The parenthesis on line 28 is used to make sure the \textbf{used} count is decremented and not
the pointer ``a''.  

\section*{Exercises}
\begin{tabular}{cl}
$\left [ 1 \right ]$ & Discuss the relevance of the \textbf{used} member of the mp\_int structure. \\
                     & \\
$\left [ 1 \right ]$ & Discuss the consequences of not using padding when performing allocations.  \\
                     & \\
$\left [ 2 \right ]$ & Estimate an ideal value for \textbf{MP\_PREC} when performing 1024-bit RSA \\
                     & encryption when $\beta = 2^{28}$.  \\
                     & \\
$\left [ 1 \right ]$ & Discuss the relevance of the algorithm mp\_clamp.  What does it prevent? \\
                     & \\
$\left [ 1 \right ]$ & Give an example of when the algorithm  mp\_init\_copy might be useful. \\
                     & \\
\end{tabular}


\chapter{Basic Operations}
\section{Copying an Integer}
After the various house-keeping routines are in place, simpl algorithms can be designed to take advantage of them.  Being able
to make a verbatim copy of an integer is a very useful function to have.  To copy an integer the mp\_copy algorithm will be used.

\newpage\begin{figure}[here]
\begin{center}
\begin{tabular}{l}
\hline Algorithm \textbf{mp\_copy}. \\
\textbf{Input}.  An mp\_int $a$ and $b$. \\
\textbf{Output}.  Store a copy of $a$ in $b$. \\
\hline \\
1.  Check if $a$ and $b$ point to the same location in memory. \\
2.  If true then return(\textit{MP\_OKAY}). \\
3.  If $b.alloc < a.used$ then grow $b$ to $a.used$ digits.  (\textit{hint: use mp\_grow}) \\
4.  If failed to grow then return(\textit{MP\_MEM}). \\
5.  for $n$ from 0 to $a.used - 1$ do \\
\hspace{3mm}5.1  $b_{n} \leftarrow a_{n}$ \\
6.  if $a.used < b.used - 1$ then \\ 
\hspace{3mm}6.1.  for $n$ from $a.used$ to $b.used - 1$ do \\
\hspace{6mm}6.1.1  $b_{n} \leftarrow 0$ \\
7.  $b.used \leftarrow a.used$ \\
8.  $b.sign \leftarrow a.sign$ \\
9.  return(\textit{MP\_OKAY}) \\
\hline
\end{tabular}
\end{center}
\caption{Algorithm mp\_copy}
\end{figure}

\textbf{Algorithm mp\_copy.}
Step 1 and 2 make sure that the two mp\_ints are unique.  This allows the user to call the copy function with
potentially the same input and not waste time.  Step 3 and 4 ensure that the destination is large enough to
hold a copy of the input $a$.  Note that the \textbf{used} member of $b$ may be smaller than the \textbf{used}
member of $a$ but a memory re-allocation is only required if the \textbf{alloc} member of $b$ is smaller.  This
prevents trivial memory reallocations.

Step 5 copies the digits from $a$ to $b$ while step 6 ensures that if initially $\vert b \vert > \vert a \vert$,
the leading digits of $b$ will be zeroed.  Finally steps 7 and 8 copies the \textbf{used} and \textbf{sign} members over 
which completes the copy operation.

\index{bn\_mp\_copy.c}
\vspace{+3mm}\begin{small}
\hspace{-5.1mm}{\bf File}: bn\_mp\_copy.c
\vspace{-3mm}
\begin{alltt}
016   
017   /* copy, b = a */
018   int
019   mp_copy (mp_int * a, mp_int * b)
020   \{
021     int     res, n;
022   
023     /* if dst == src do nothing */
024     if (a == b || a->dp == b->dp) \{
025       return MP_OKAY;
026     \}
027   
028     /* grow dest */
029     if ((res = mp_grow (b, a->used)) != MP_OKAY) \{
030       return res;
031     \}
032   
033     /* zero b and copy the parameters over */
034     \{
035       register mp_digit *tmpa, *tmpb;
036   
037       /* pointer aliases */
038       tmpa = a->dp;
039       tmpb = b->dp;
040   
041       /* copy all the digits */
042       for (n = 0; n < a->used; n++) \{
043         *tmpb++ = *tmpa++;
044       \}
045   
046       /* clear high digits */
047       for (; n < b->used; n++) \{
048         *tmpb++ = 0;
049       \}
050     \}
051     b->used = a->used;
052     b->sign = a->sign;
053     return MP_OKAY;
054   \}
\end{alltt}
\end{small}

Source lines 23-31 do the initial house keeping.  That is to see if the input is unique and if so to 
make sure there is enough room.  If not enough space is available it returns the error and leaves the destination variable
intact.

The inner loop of the copy operation is contained between lines 34 and 50.  Many LibTomMath routines are designed with this source code style
in mind, making aliases to shorten lengthy pointers (\textit{see line 38 and 39}) for rapid to use.  Also the
use of nested braces creates a simple way to denote various portions of code that reside on various work levels.  Here, the copy loop is at the 
$O(n)$ level.  

\section{Zeroing an Integer}
Reseting an mp\_int to the default state is a common step in many algorithms.  The mp\_zero algorithm will be the algorithm used to
perform this task.

\begin{figure}[here]
\begin{center}
\begin{tabular}{l}
\hline Algorithm \textbf{mp\_zero}. \\
\textbf{Input}.   An mp\_int $a$ \\
\textbf{Output}.  Zero the contents of $a$ \\
\hline \\
1.  $a.used \leftarrow 0$ \\
2.  $a.sign \leftarrow$ MP\_ZPOS \\
3.  for $n$ from 0 to $a.alloc - 1$ do \\
\hspace{3mm}3.1  $a_n \leftarrow 0$ \\
\hline
\end{tabular}
\end{center}
\caption{Algorithm mp\_zero}
\end{figure}

\textbf{Algorithm mp\_zero.}
This algorithm simply resets a mp\_int to the default state.  

\index{bn\_mp\_zero.c}
\vspace{+3mm}\begin{small}
\hspace{-5.1mm}{\bf File}: bn\_mp\_zero.c
\vspace{-3mm}
\begin{alltt}
016   
017   /* set to zero */
018   void
019   mp_zero (mp_int * a)
020   \{
021     a->sign = MP_ZPOS;
022     a->used = 0;
023     memset (a->dp, 0, sizeof (mp_digit) * a->alloc);
024   \}
\end{alltt}
\end{small}

After the function is completed, all of the digits are zeroed, the \textbf{used} count is zeroed and the 
\textbf{sign} variable is set to \textbf{MP\_ZPOS}.

\section{Sign Manipulation}
\subsection{Absolute Value}
With the mp\_int representation of an integer, calculating the absolute value is trivial.  The mp\_abs algorithm will compute
the absolute value of an mp\_int.

\begin{figure}[here]
\begin{center}
\begin{tabular}{l}
\hline Algorithm \textbf{mp\_abs}. \\
\textbf{Input}.   An mp\_int $a$ \\
\textbf{Output}.  Computes $b = \vert a \vert$ \\
\hline \\
1.  Copy $a$ to $b$.  (\textit{hint: use mp\_copy}) \\
2.  If the copy failed return(\textit{MP\_MEM}). \\
3.  $b.sign \leftarrow MP\_ZPOS$ \\
4.  Return(\textit{MP\_OKAY}) \\
\hline
\end{tabular}
\end{center}
\caption{Algorithm mp\_abs}
\end{figure}

\textbf{Algorithm mp\_abs.}
This algorithm computes the absolute of an mp\_int input.  As can be expected the algorithm is very trivial.

\index{bn\_mp\_abs.c}
\vspace{+3mm}\begin{small}
\hspace{-5.1mm}{\bf File}: bn\_mp\_abs.c
\vspace{-3mm}
\begin{alltt}
016   
017   /* b = |a| 
018    *
019    * Simple function copies the input and fixes the sign to positive
020    */
021   int
022   mp_abs (mp_int * a, mp_int * b)
023   \{
024     int     res;
025     if ((res = mp_copy (a, b)) != MP_OKAY) \{
026       return res;
027     \}
028     b->sign = MP_ZPOS;
029     return MP_OKAY;
030   \}
\end{alltt}
\end{small}

\subsection{Integer Negation}
With the mp\_int representation of an integer, calculating the negation is also trivial.  The mp\_neg algorithm will compute
the negative of an mp\_int input.

\newpage\begin{figure}[here]
\begin{center}
\begin{tabular}{l}
\hline Algorithm \textbf{mp\_neg}. \\
\textbf{Input}.   An mp\_int $a$ \\
\textbf{Output}.  Computes $b = -a$ \\
\hline \\
1.  Copy $a$ to $b$.  (\textit{hint: use mp\_copy}) \\
2.  If the copy failed return(\textit{MP\_MEM}). \\
3.  If $a.sign = MP\_ZPOS$ then do \\
\hspace{3mm}3.1  $b.sign = MP\_NEG$. \\
4.  else do \\
\hspace{3mm}4.1  $b.sign = MP\_ZPOS$. \\
5.  Return(\textit{MP\_OKAY}) \\
\hline
\end{tabular}
\end{center}
\caption{Algorithm mp\_neg}
\end{figure}

\textbf{Algorithm mp\_neg.}
This algorithm computes the negation of an input.  

\index{bn\_mp\_neg.c}
\vspace{+3mm}\begin{small}
\hspace{-5.1mm}{\bf File}: bn\_mp\_neg.c
\vspace{-3mm}
\begin{alltt}
016   
017   /* b = -a */
018   int
019   mp_neg (mp_int * a, mp_int * b)
020   \{
021     int     res;
022     if ((res = mp_copy (a, b)) != MP_OKAY) \{
023       return res;
024     \}
025     b->sign = (a->sign == MP_ZPOS) ? MP_NEG : MP_ZPOS;
026     return MP_OKAY;
027   \}
\end{alltt}
\end{small}

\section{Small Constants}
\subsection{Setting Small Constants}
Often a mp\_int must be set to a relatively small value such as $1$ or $2$.  For these cases the mp\_set algorithm is useful.

\newpage\begin{figure}
\begin{center}
\begin{tabular}{l}
\hline Algorithm \textbf{mp\_set}. \\
\textbf{Input}.   An mp\_int $a$ and a digit $b$ \\
\textbf{Output}.  Make $a$ equivalent to $b$ \\
\hline \\
1.  Zero $a$ (\textit{hint: use mp\_zero}). \\
2.  $a_0 \leftarrow b \mbox{ (mod }\beta\mbox{)}$ \\
3.  $a.used \leftarrow  \left \lbrace \begin{array}{ll}
                              1 &  \mbox{if }a_0 > 0 \\
                              0 &  \mbox{if }a_0 = 0 
                              \end{array} \right .$ \\
\hline                              
\end{tabular}
\end{center}
\caption{Algorithm mp\_set}
\end{figure}

\textbf{Algorithm mp\_set.}
This algorithm sets a mp\_int to a small single digit value.  Step number 1 ensures that the integer is reset to the default state.  The
single digit is set (\textit{modulo $\beta$}) and the \textbf{used} count is adjusted accordingly.

\index{bn\_mp\_set.c}
\vspace{+3mm}\begin{small}
\hspace{-5.1mm}{\bf File}: bn\_mp\_set.c
\vspace{-3mm}
\begin{alltt}
016   
017   /* set to a digit */
018   void
019   mp_set (mp_int * a, mp_digit b)
020   \{
021     mp_zero (a);
022     a->dp[0] = b & MP_MASK;
023     a->used = (a->dp[0] != 0) ? 1 : 0;
024   \}
\end{alltt}
\end{small}

Line 21 calls mp\_zero() to clear the mp\_int and reset the sign.  Line 22 actually copies digit 
into the least significant location.  Note the usage of a new constant \textbf{MP\_MASK}.  This constant is used to quickly
reduce an integer modulo $\beta$.  Since $\beta = 2^k$ it suffices to perform a binary AND with $MP\_MASK = 2^k - 1$ to perform
the reduction.  Finally line 23 will set the \textbf{used} member with respect to the digit actually set. This function 
will always make the integer positive.

One important limitation of this function is that it will only set one digit.  The size of a digit is not fixed, meaning source that uses 
this function should take that into account.  The define \textbf{DIGIT\_BIT} in ``tommath.h'' 
defines how many bits per digit are available.  Generally at least seven bits are guaranteed to be available per 
digit.  This means that trivially small constants can be set using this function.

\subsection{Setting Large Constants}
To overcome the limitations of the mp\_set algorithm the mp\_set\_int algorithm is provided.  It accepts a ``long''
data type as input and will always treat it as a 32-bit integer.

\begin{figure}[here]
\begin{center}
\begin{tabular}{l}
\hline Algorithm \textbf{mp\_set\_int}. \\
\textbf{Input}.   An mp\_int $a$ and a ``long'' integer $b$ \\
\textbf{Output}.  Make $a$ equivalent to $b$ \\
\hline \\
1.  Zero $a$ (\textit{hint: use mp\_zero}) \\
2.  for $n$ from 0 to 7 do \\
\hspace{3mm}2.1  $a \leftarrow a \cdot 16$ (\textit{hint: use mp\_mul2d}) \\
\hspace{3mm}2.2  $u \leftarrow \lfloor b / 2^{4(7 - n)} \rfloor \mbox{ (mod }16\mbox{)}$\\
\hspace{3mm}2.3  $a_0 \leftarrow a_0 + u$ \\
\hspace{3mm}2.4  $a.used \leftarrow a.used + \lfloor 32 / lg(\beta) \rfloor + 1$ \\
3.  Clamp excess used digits (\textit{hint: use mp\_clamp}) \\
\hline
\end{tabular}
\end{center}
\caption{Algorithm mp\_set\_int}
\end{figure}

\textbf{Algorithm mp\_set\_int.}
The algorithm performs eight iterations of a simple loop where in each iteration four bits from the source are added to the 
mp\_int.  Step 2.1 will multiply the current result by sixteen making room for four more bits.  In step 2.2 the
next four bits from the source are extracted.  The four bits are added to the mp\_int and the \textbf{used} digit count is 
incremented.  The \textbf{used} digit counter is incremented since if any of the leading digits were zero the mp\_int would have
zero digits used and the newly added four bits would be ignored.

Excess zero digits are trimmed in steps 2.1 and 3 by using higher level algorithms mp\_mul2d and mp\_clamp.

\index{bn\_mp\_set\_int.c}
\vspace{+3mm}\begin{small}
\hspace{-5.1mm}{\bf File}: bn\_mp\_set\_int.c
\vspace{-3mm}
\begin{alltt}
016   
017   /* set a 32-bit const */
018   int
019   mp_set_int (mp_int * a, unsigned int b)
020   \{
021     int     x, res;
022   
023     mp_zero (a);
024     /* set four bits at a time */
025     for (x = 0; x < 8; x++) \{
026       /* shift the number up four bits */
027       if ((res = mp_mul_2d (a, 4, a)) != MP_OKAY) \{
028         return res;
029       \}
030   
031       /* OR in the top four bits of the source */
032       a->dp[0] |= (b >> 28) & 15;
033   
034       /* shift the source up to the next four bits */
035       b <<= 4;
036   
037       /* ensure that digits are not clamped off */
038       a->used += 32 / DIGIT_BIT + 2;
039     \}
040     mp_clamp (a);
041     return MP_OKAY;
042   \}
\end{alltt}
\end{small}

This function sets four bits of the number at a time to handle all practical \textbf{DIGIT\_BIT} sizes.  The weird
addition on line 38 ensures that the newly added in bits are added to the number of digits.  While it may not 
seem obvious as to why the digit counter does not grow exceedingly large it is because of the shift on line 27 
as well as the  call to mp\_clamp() on line 40.  Both functions will clamp excess leading digits which keeps 
the number of used digits low.

\section{Comparisons}
\subsection{Unsigned Comparisions}
Comparing a multiple precision integer is performed with the exact same algorithm used to compare two decimal numbers.  For example,
to compare $1,234$ to $1,264$ the digits are extracted by their positions.  That is we compare $1 \cdot 10^3 + 2 \cdot 10^2 + 3 \cdot 10^1 + 4 \cdot 10^0$
to $1 \cdot 10^3 + 2 \cdot 10^2 + 6 \cdot 10^1 + 4 \cdot 10^0$ by comparing single digits at a time starting with the highest magnitude 
positions.  If any leading digit of one integer is greater than a digit in the same position of another integer then obviously it must be greater.  

The first comparision routine that will be developed is the unsigned magnitude compare which will perform a comparison based on the digits of two
mp\_int variables alone.  It will ignore the sign of the two inputs.  Such a function is useful when an absolute comparison is required or if the 
signs are known to agree in advance.

To facilitate working with the results of the comparison functions three constants are required.  

\begin{figure}[here]
\begin{center}
\begin{tabular}{|r|l|}
\hline \textbf{Constant} & \textbf{Meaning} \\
\hline \textbf{MP\_GT} & Greater Than \\
\hline \textbf{MP\_EQ} & Equal To \\
\hline \textbf{MP\_LT} & Less Than \\
\hline
\end{tabular}
\end{center}
\caption{Comparison Return Codes}
\end{figure}

\begin{figure}[here]
\begin{center}
\begin{tabular}{l}
\hline Algorithm \textbf{mp\_cmp\_mag}. \\
\textbf{Input}.   Two mp\_ints $a$ and $b$.  \\
\textbf{Output}.  Unsigned comparison results ($a$ to the left of $b$). \\
\hline \\
1.  If $a.used > b.used$ then return(\textit{MP\_GT}) \\
2.  If $a.used < b.used$ then return(\textit{MP\_LT}) \\
3.  for n from $a.used - 1$ to 0 do \\
\hspace{+3mm}3.1  if $a_n > b_n$ then return(\textit{MP\_GT}) \\
\hspace{+3mm}3.2  if $a_n < b_n$ then return(\textit{MP\_LT}) \\
4.  Return(\textit{MP\_EQ}) \\
\hline
\end{tabular}
\end{center}
\caption{Algorithm mp\_cmp\_mag}
\end{figure}

\textbf{Algorithm mp\_cmp\_mag.}
By saying ``$a$ to the left of $b$'' it is meant that the comparison is with respect to $a$, that is if $a$ is greater than $b$ it will return
\textbf{MP\_GT} and similar with respect to when $a = b$ and $a < b$.  The first two steps compare the number of digits used in both $a$ and $b$.  
Obviously if the digit counts differ there would be an imaginary zero digit in the smaller number where the leading digit of the larger number is.  
If both have the same number of digits than the actual digits themselves must be compared starting at the leading digit.  

By step three both inputs must have the same number of digits so its safe to start from either $a.used - 1$ or $b.used - 1$ and count down to
the zero'th digit.  If after all of the digits have been compared and no difference found the algorithm simply returns \textbf{MP\_EQ}.

\index{bn\_mp\_cmp\_mag.c}
\vspace{+3mm}\begin{small}
\hspace{-5.1mm}{\bf File}: bn\_mp\_cmp\_mag.c
\vspace{-3mm}
\begin{alltt}
016   
017   /* compare maginitude of two ints (unsigned) */
018   int
019   mp_cmp_mag (mp_int * a, mp_int * b)
020   \{
021     int     n;
022   
023     /* compare based on # of non-zero digits */
024     if (a->used > b->used) \{
025       return MP_GT;
026     \} 
027     
028     if (a->used < b->used) \{
029       return MP_LT;
030     \}
031   
032     /* compare based on digits  */
033     for (n = a->used - 1; n >= 0; n--) \{
034       if (a->dp[n] > b->dp[n]) \{
035         return MP_GT;
036       \} 
037       
038       if (a->dp[n] < b->dp[n]) \{
039         return MP_LT;
040       \}
041     \}
042     return MP_EQ;
043   \}
\end{alltt}
\end{small}

The two if statements on lines 24 and 28 compare the number of digits in the two inputs.  These two are performed before all of the digits
are compared since it is a very cheap test to perform and can potentially save considerable time.  The implementation given is also not valid 
without those two statements.  $b.alloc$ may be smaller than $a.used$, meaning that undefined values will be read from $b$ passed the end of the 
array of digits.

\subsection{Signed Comparisons}
Comparing with sign considerations is also fairly critical in several routines (\textit{division for example}).  Based on an unsigned magnitude 
comparison a trivial signed comparison algorithm can be written.

\newpage\begin{figure}[here]
\begin{center}
\begin{tabular}{l}
\hline Algorithm \textbf{mp\_cmp}. \\
\textbf{Input}.   Two mp\_ints $a$ and $b$ \\
\textbf{Output}.  Signed Comparison Results ($a$ to the left of $b$) \\
\hline \\
1.  if $a.sign = MP\_NEG$ and $b.sign = MP\_ZPOS$ then return(\textit{MP\_LT}) \\
2.  if $a.sign = MP\_ZPOS$ and $b.sign = MP\_NEG$ then return(\textit{MP\_GT}) \\
3.  if $a.sign = MP\_NEG$ then \\
\hspace{+3mm}3.1  Return the unsigned comparison of $b$ and $a$ (\textit{hint: use mp\_cmp\_mag}) \\
4   Otherwise \\
\hspace{+3mm}4.1  Return the unsigned comparison of $a$ and $b$ \\
\hline
\end{tabular}
\end{center}
\caption{Algorithm mp\_cmp}
\end{figure}

\textbf{Algorithm mp\_cmp.}
The first two steps compare the signs of the two inputs.  If the signs do not agree then it can return right away with the appropriate 
comparison code.  When the signs are equal the digits of the inputs must be compared to determine the correct result.  In step 
three the unsigned comparision flips the order of the arguments since they are both negative.  For instance, if $-a > -b$ then 
$\vert a \vert < \vert b \vert$.  Step number four will compare the two when they are both positive.

\index{bn\_mp\_cmp.c}
\vspace{+3mm}\begin{small}
\hspace{-5.1mm}{\bf File}: bn\_mp\_cmp.c
\vspace{-3mm}
\begin{alltt}
016   
017   /* compare two ints (signed)*/
018   int
019   mp_cmp (mp_int * a, mp_int * b)
020   \{
021     /* compare based on sign */
022     if (a->sign == MP_NEG && b->sign == MP_ZPOS) \{
023       return MP_LT;
024     \} 
025     
026     if (a->sign == MP_ZPOS && b->sign == MP_NEG) \{
027       return MP_GT;
028     \}
029     
030     /* compare digits */
031     if (a->sign == MP_NEG) \{
032        /* if negative compare opposite direction */
033        return mp_cmp_mag(b, a);
034     \} else \{
035        return mp_cmp_mag(a, b);
036     \}
037   \}
\end{alltt}
\end{small}

The two if statements on lines 22 and 26 perform the initial sign comparison.  If the signs are not the equal then which ever
has the positive sign is larger.   At line 31, the inputs are compared based on magnitudes.  If the signs were both negative then 
the unsigned comparison is performed in the opposite direction (\textit{line 33}).  Otherwise, the signs are assumed to 
be both positive and a forward direction unsigned comparison is performed.

\section*{Exercises}
\begin{tabular}{cl}
$\left [ 2 \right ]$ & Modify algorithm mp\_set\_int to accept as input a variable length array of bits. \\
                     & \\
$\left [ 3 \right ]$ & Give the probability that algorithm mp\_cmp\_mag will have to compare $k$ digits  \\
                     & of two random digits (of equal magnitude) before a difference is found. \\
                     & \\
$\left [ 1 \right ]$ & Suggest a simple method to speed up the implementation of mp\_cmp\_mag based  \\
                     & on the observations made in the previous problem. \\
                     &
\end{tabular}

\chapter{Basic Arithmetic}
\section{Building Blocks}
At this point algorithms for initialization, de-initialization, zeroing, copying, comparing and setting small constants have been 
established.  The next logical set of algorithms to develop are the addition, subtraction and digit movement algorithms.  These 
algorithms make use of the lower level algorithms and are the cruicial building block for the multipliers.  It is very important that these 
algorithms are highly optimized.  On their own they are simple $O(n)$ algorithms but they can be called from higher level algorithms 
which easily places them at $O(n^2)$ or even $O(n^3)$ work levels.  

All nine algorithms within this chapter make use of the logical bit shift operations denoted by $<<$ and $>>$ for left and right 
logical shifts respectively.  A logical shift is analogous to sliding the decimal point of radix-10 representations.  For example, the real 
number $0.9345$ is equivalent to $93.45\%$ which is found by sliding the the decimal two places to the right (\textit{multiplying by $10^2$}).  
Mathematically a logical shift is equivalent to a division or multiplication by a power of two.  
For example, $a << k = a \cdot 2^k$ while $a >> k = \lfloor a/2^k \rfloor$.

One significant difference between a logical shift and the way decimals are shifted is that digits below the zero'th position are removed
from the number.  For example, consider $1101_2 >> 1$ using decimal notation this would produce $110.1_2$.  However, with a logical shift the 
result is $110_2$.  

\section{Addition and Subtraction}
In normal fixed precision arithmetic negative numbers are easily represented by subtraction from the modulus.  For example, with 32-bit integers
$a - b\mbox{ (mod }2^{32}\mbox{)}$ is the same as $a + (2^{32} - b) \mbox{ (mod }2^{32}\mbox{)}$  since $2^{32} \equiv 0 \mbox{ (mod }2^{32}\mbox{)}$.  
As a result subtraction can be performed with a trivial series of logical operations and an addition.

However, in multiple precision arithmetic negative numbers are not represented in the same way.  Instead a sign flag is used to keep track of the
sign of the integer.  As a result signed addition and subtraction are actually implemented as conditional usage of lower level addition or 
subtraction algorithms with the sign fixed up appropriately.

The lower level algorithms will add or subtract integers without regard to the sign flag.  That is they will add or subtract the magnitude of
the integers respectively.

\subsection{Low Level Addition}
An unsigned addition of multiple precision integers is performed with the same long-hand algorithm used to add decimal numbers.  That is to add the 
trailing digits first and propagate the resulting carry upwards.  Since this is a lower level algorithm the name will have a ``s\_'' prefix.  
Historically that convention stems from the MPI library where ``s\_'' stood for static functions that were hidden from the developer entirely.

\newpage
\begin{figure}[!here]
\begin{center}
\begin{small}
\begin{tabular}{l}
\hline Algorithm \textbf{s\_mp\_add}. \\
\textbf{Input}.   Two mp\_ints $a$ and $b$ \\
\textbf{Output}.  The unsigned addition $c = \vert a \vert + \vert b \vert$. \\
\hline \\
1.  if $a.used > b.used$ then \\
\hspace{+3mm}1.1  $min \leftarrow b.used$ \\
\hspace{+3mm}1.2  $max \leftarrow a.used$ \\
\hspace{+3mm}1.3  $x   \leftarrow a$ \\
2.  else  \\
\hspace{+3mm}2.1  $min \leftarrow a.used$ \\
\hspace{+3mm}2.2  $max \leftarrow b.used$ \\
\hspace{+3mm}2.3  $x   \leftarrow b$ \\
3.  If $c.alloc < max + 1$ then grow $c$ to hold at least $max + 1$ digits (\textit{hint: use mp\_grow}) \\
4.  If failed to grow $c$ return(\textit{MP\_MEM}) \\
5.  $oldused \leftarrow c.used$ \\
6.  $c.used \leftarrow max + 1$ \\
7.  $u \leftarrow 0$ \\
8.  for $n$ from $0$ to $min - 1$ do \\
\hspace{+3mm}8.1  $c_n \leftarrow a_n + b_n + u$ \\
\hspace{+3mm}8.2  $u \leftarrow c_n >> lg(\beta)$ \\
\hspace{+3mm}8.3  $c_n \leftarrow c_n \mbox{ (mod }\beta\mbox{)}$ \\
9.  if $min \ne max$ then do \\
\hspace{+3mm}9.1  for $n$ from $min$ to $max - 1$ do \\
\hspace{+6mm}9.1.1  $c_n \leftarrow x_n + u$ \\
\hspace{+6mm}9.1.2  $u \leftarrow c_n >> lg(\beta)$ \\
\hspace{+6mm}9.1.3  $c_n \leftarrow c_n \mbox{ (mod }\beta\mbox{)}$ \\
10.  $c_{max} \leftarrow u$ \\
11.  if $olduse > max$ then \\
\hspace{+3mm}11.1  for $n$ from $max + 1$ to $olduse - 1$ do \\
\hspace{+6mm}11.1.1  $c_n \leftarrow 0$ \\
12.  Clamp excess digits in $c$.  (\textit{hint: use mp\_clamp}) \\
13.  Return(\textit{MP\_OKAY}) \\
\hline
\end{tabular}
\end{small}
\end{center}
\caption{Algorithm s\_mp\_add}
\end{figure}

\textbf{Algorithm s\_mp\_add.}
This algorithm is loosely based on algorithm 14.7 of \cite[pp. 594]{HAC} but has been extended to allow the inputs to have different magnitudes.  
Coincidentally the description of algorithm A in \cite[pp. 266]{TAOCPV2} shares the same flaw as that from \cite{HAC}.  Even the MIX pseudo 
machine code presented  \cite[pp. 266-267]{TAOCPV2} is incapable of handling inputs which are of different magnitudes.

Steps 1 and 2 will sort the two inputs based on their \textbf{used} digit count.  This allows the inputs to have varying magnitudes which not 
only makes it more efficient than the trivial algorithm presented in the other references but more flexible.  The variable $min$ is given the lowest 
digit count while $max$ is given the highest digit count.  If both inputs have the same \textbf{used} digit count both $min$ and $max$ are 
set to the same.  The variable $x$ is an \textit{alias} for the largest input and not meant to be a copy of it.  After the inputs are sorted steps 
3 and 4 will ensure that the destination $c$ can accommodate the result.  The old \textbf{used} count from $c$ is copied to $oldused$ and the 
new count is set to $max + 1$.  

At step 7 the carry variable $u$ is set to zero and the first leg of the addition loop can begin.  The first step of the loop (\textit{8.1}) adds
digits from the two inputs together along with the carry variable $u$.  The following step extracts the carry bit by shifting the result of the
preceding step right $lg(\beta)$ positions.  The shift to extract the carry is similar to how carry extraction works with decimal addition.

Consider adding $77$ to $65$, the first addition of the first column is $7 + 5$ which produces the result $12$.  The trailing digit of the result
is $2 \equiv 12 \mbox{ (mod }10\mbox{)}$ and the carry is found by dividing (\textit{and ignoring the remainder}) $12$ by the radix or in this case $10$.  The
division and multiplication of $10$ is simply a logical shift right or left respectively of the digits.  In otherwords the carry can be extracted
by shifting one digit to the right.

Note that $lg()$ is simply the base two logarithm such that $lg(2^k) = k$.  This implies that $lg(\beta)$ is the number of bits in a radix-$\beta$ 
digit.  Therefore, a logical shift right of the single digit by $lg(\beta)$ will extract the carry.  The final step of the  loop reduces the digit 
modulo the radix $\beta$ to ensure it is in range.

After step 8 the smallest input (\textit{or both if they are the same magnitude}) has been exhausted.  Step 9 decides whether
the inputs were of equal magnitude.  If not than another loop similar to that in step 8 must be executed.  The loop at step
number 9.1 differs from the previous loop since it only adds the mp\_int $x$ along with the carry.  

Step 10 finishes the addition phase by copying the final carry to the highest location in the result $c_{max}$.  Step 11 ensures that 
leading digits that were originally present in $c$ are cleared.  Finally excess leading digits are clamped and the algorithm returns success.

\index{bn\_s\_mp\_add.c}
\vspace{+3mm}\begin{small}
\hspace{-5.1mm}{\bf File}: bn\_s\_mp\_add.c
\vspace{-3mm}
\begin{alltt}
016   
017   /* low level addition, based on HAC pp.594, Algorithm 14.7 */
018   int
019   s_mp_add (mp_int * a, mp_int * b, mp_int * c)
020   \{
021     mp_int *x;
022     int     olduse, res, min, max;
023   
024     /* find sizes, we let |a| <= |b| which means we have to sort
025      * them.  "x" will point to the input with the most digits
026      */
027     if (a->used > b->used) \{
028       min = b->used;
029       max = a->used;
030       x = a;
031     \} else \{
032       min = a->used;
033       max = b->used;
034       x = b;
035     \}
036   
037     /* init result */
038     if (c->alloc < max + 1) \{
039       if ((res = mp_grow (c, max + 1)) != MP_OKAY) \{
040         return res;
041       \}
042     \}
043   
044     /* get old used digit count and set new one */
045     olduse = c->used;
046     c->used = max + 1;
047   
048     /* set the carry to zero */
049     \{
050       register mp_digit u, *tmpa, *tmpb, *tmpc;
051       register int i;
052   
053       /* alias for digit pointers */
054   
055       /* first input */
056       tmpa = a->dp;
057   
058       /* second input */
059       tmpb = b->dp;
060   
061       /* destination */
062       tmpc = c->dp;
063   
064       /* zero the carry */
065       u = 0;
066       for (i = 0; i < min; i++) \{
067         /* Compute the sum at one digit, T[i] = A[i] + B[i] + U */
068         *tmpc = *tmpa++ + *tmpb++ + u;
069   
070         /* U = carry bit of T[i] */
071         u = *tmpc >> ((mp_digit)DIGIT_BIT);
072   
073         /* take away carry bit from T[i] */
074         *tmpc++ &= MP_MASK;
075       \}
076   
077       /* now copy higher words if any, that is in A+B 
078        * if A or B has more digits add those in 
079        */
080       if (min != max) \{
081         for (; i < max; i++) \{
082           /* T[i] = X[i] + U */
083           *tmpc = x->dp[i] + u;
084   
085           /* U = carry bit of T[i] */
086           u = *tmpc >> ((mp_digit)DIGIT_BIT);
087   
088           /* take away carry bit from T[i] */
089           *tmpc++ &= MP_MASK;
090         \}
091       \}
092   
093       /* add carry */
094       *tmpc++ = u;
095   
096       /* clear digits above oldused */
097       for (i = c->used; i < olduse; i++) \{
098         *tmpc++ = 0;
099       \}
100     \}
101   
102     mp_clamp (c);
103     return MP_OKAY;
104   \}
\end{alltt}
\end{small}

Lines 27 to 35 perform the initial sorting of the inputs and determine the $min$ and $max$ variables.  Note that $x$ is pointer to a 
mp\_int assigned to the largest input, in effect it is a local alias.  Lines 37 to 42 ensure that the destination is grown to 
accomodate the result of the addition. 

Similar to the implementation of mp\_copy this function uses the braced code and local aliases coding style.  The three aliases on 
lines 56, 59 and 62 are the for the two inputs and destination respectively.  These aliases are used to ensure the
compiler does not have to dereference $a$, $b$ or $c$ (respectively) to access the digits of the respective mp\_int.

The initial carry $u$ is cleared on line 65, note that $u$ is of type mp\_digit which ensures type compatibility within the 
implementation.  The initial addition loop begins on line 66 and ends on line 75.  Similarly the conditional addition loop
begins on line 81 and ends on line 90.  The addition is finished with the final carry being stored in $tmpc$ on line 94.  
Note the ``++'' operator on the same line.  After line 94 $tmpc$ will point to the $c.used$'th digit of the mp\_int $c$.  This is useful
for the next loop on lines 97 to 99 which set any old upper digits to zero.

\subsection{Low Level Subtraction}
The low level unsigned subtraction algorithm is very similar to the low level unsigned addition algorithm.  The principle difference is that the
unsigned subtraction algorithm requires the result to be positive.  That is when computing $a - b$ the condition $\vert a \vert \ge \vert b\vert$ must 
be met for this algorithm to function properly.  Keep in mind this low level algorithm is not meant to be used in higher level algorithms directly.  
This algorithm as will be shown can be used to create functional signed addition and subtraction algorithms.


For this algorithm a new variable is required to make the description simpler.  Recall from section 1.3.1 that a mp\_digit must be able to represent
the range $0 \le x < 2\beta$.  It is allowable that a mp\_digit represent a larger range of values.  For this algorithm we will assume that
the variable $\gamma$ represents the number of bits available in a mp\_digit (\textit{this implies $2^{\gamma} > \beta$}).

\newpage\begin{figure}[!here]
\begin{center}
\begin{small}
\begin{tabular}{l}
\hline Algorithm \textbf{s\_mp\_sub}. \\
\textbf{Input}.   Two mp\_ints $a$ and $b$ ($\vert a \vert \ge \vert b \vert$) \\
\textbf{Output}.  The unsigned subtraction $c = \vert a \vert - \vert b \vert$. \\
\hline \\
1.  $min \leftarrow b.used$ \\
2.  $max \leftarrow a.used$ \\
3.  If $c.alloc < max$ then grow $c$ to hold at least $max$ digits.  (\textit{hint: use mp\_grow}) \\
4.  If the reallocation failed return(\textit{MP\_MEM}). \\
5.  $oldused \leftarrow c.used$ \\ 
6.  $c.used \leftarrow max$ \\
7.  $u \leftarrow 0$ \\
8.  for $n$ from $0$ to $min - 1$ do \\
\hspace{3mm}8.1  $c_n \leftarrow a_n - b_n - u$ \\
\hspace{3mm}8.2  $u   \leftarrow c_n >> (\gamma - 1)$ \\
\hspace{3mm}8.3  $c_n \leftarrow c_n \mbox{ (mod }\beta\mbox{)}$ \\
9.  if $min < max$ then do \\
\hspace{3mm}9.1  for $n$ from $min$ to $max - 1$ do \\
\hspace{6mm}9.1.1  $c_n \leftarrow a_n - u$ \\
\hspace{6mm}9.1.2  $u   \leftarrow c_n >> (\gamma - 1)$ \\
\hspace{6mm}9.1.3  $c_n \leftarrow c_n \mbox{ (mod }\beta\mbox{)}$ \\
10. if $oldused > max$ then do \\
\hspace{3mm}10.1  for $n$ from $max$ to $oldused - 1$ do \\
\hspace{6mm}10.1.1  $c_n \leftarrow 0$ \\
11. Clamp excess digits of $c$.  (\textit{hint: use mp\_clamp}). \\
12. Return(\textit{MP\_OKAY}). \\
\hline
\end{tabular}
\end{small}
\end{center}
\caption{Algorithm s\_mp\_sub}
\end{figure}

\textbf{Algorithm s\_mp\_sub.}
This algorithm performs the unsigned subtraction of two mp\_int variables under the restriction that the result must be positive.  That is when
passing variables $a$ and $b$ the condition that $\vert a \vert \ge \vert b \vert$ must be met for the algorithm to function correctly.  This
algorithm is loosely based on algorithm 14.9 \cite[pp. 595]{HAC} and is similar to algorithm S in \cite[pp. 267]{TAOCPV2} as well.  As was the case
of the algorithm s\_mp\_add both other references lack discussion concerning various practical details such as when the inputs differ in magnitude.

The initial sorting of the inputs is trivial in this algorithm since $a$ is guaranteed to have at least the same magnitude of $b$.  Steps 1 and 2 
set the $min$ and $max$ variables.  Unlike the addition routine there is guaranteed to be no carry which means that the final result can be at 
most $max$ digits in length as oppose to $max + 1$.  Similar to the addition algorithm the \textbf{used} count of $c$ is copied locally and 
set to the maximal count for the operation.

The subtraction loop that begins on step 8 is essentially the same as the addition loop of algorithm s\_mp\_add except single precision 
subtraction is used instead.  Note the use of the $\gamma$ variable to extract the carry within the subtraction loops.  Under the assumption
that two's complement single precision arithmetic is used this will successfully extract the carry.  

For example, consider subtracting $0101_2$ from
$0100_2$ where $\gamma = 4$.  The least significant bit will force a carry upwards to the third bit which will be set to zero after the borrow.  After
the very first bit has been subtracted $4 - 1 \equiv 0011_2$ will remain,  When the third bit of $0101_2$ is subtracted from the result it will cause
another carry.  In this case though the carry will be forced to propagate all the way to the most significant bit.  

Recall that $\beta < 2^{\gamma}$.  This means that if a carry does occur it will propagate all the way to the most significant bit.  Therefore a single
logical shift right by $\gamma - 1$ positions is sufficient to extract the carry.  This method of carry extraction may seem awkward but the reason for 
it becomes apparent when the implementation is discussed.  

If $b$ has a smaller magnitude than $a$ then step 9 will force the carry and copy operation to propagate through the larger input $a$ into $c$.  Step
10 will ensure that any leading digits of $c$ above the $max$'th position are zeroed.

\index{bn\_s\_mp\_sub.c}
\vspace{+3mm}\begin{small}
\hspace{-5.1mm}{\bf File}: bn\_s\_mp\_sub.c
\vspace{-3mm}
\begin{alltt}
016   
017   /* low level subtraction (assumes |a| > |b|), HAC pp.595 Algorithm 14.9 */
018   int
019   s_mp_sub (mp_int * a, mp_int * b, mp_int * c)
020   \{
021     int     olduse, res, min, max;
022   
023     /* find sizes */
024     min = b->used;
025     max = a->used;
026   
027     /* init result */
028     if (c->alloc < max) \{
029       if ((res = mp_grow (c, max)) != MP_OKAY) \{
030         return res;
031       \}
032     \}
033     olduse = c->used;
034     c->used = max;
035   
036     /* sub digits from lower part */
037     \{
038       register mp_digit u, *tmpa, *tmpb, *tmpc;
039       register int i;
040   
041       /* alias for digit pointers */
042       tmpa = a->dp;
043       tmpb = b->dp;
044       tmpc = c->dp;
045   
046       /* set carry to zero */
047       u = 0;
048       for (i = 0; i < min; i++) \{
049         /* T[i] = A[i] - B[i] - U */
050         *tmpc = *tmpa++ - *tmpb++ - u;
051   
052         /* U = carry bit of T[i]
053          * Note this saves performing an AND operation since
054          * if a carry does occur it will propagate all the way to the
055          * MSB.  As a result a single shift is required to get the carry
056          */
057         u = *tmpc >> ((mp_digit)(CHAR_BIT * sizeof (mp_digit) - 1));
058   
059         /* Clear carry from T[i] */
060         *tmpc++ &= MP_MASK;
061       \}
062   
063       /* now copy higher words if any, e.g. if A has more digits than B  */
064       for (; i < max; i++) \{
065         /* T[i] = A[i] - U */
066         *tmpc = *tmpa++ - u;
067   
068         /* U = carry bit of T[i] */
069         u = *tmpc >> ((mp_digit)(CHAR_BIT * sizeof (mp_digit) - 1));
070   
071         /* Clear carry from T[i] */
072         *tmpc++ &= MP_MASK;
073       \}
074   
075       /* clear digits above used (since we may not have grown result above) */
      
076       for (i = c->used; i < olduse; i++) \{
077         *tmpc++ = 0;
078       \}
079     \}
080   
081     mp_clamp (c);
082     return MP_OKAY;
083   \}
\end{alltt}
\end{small}

Line 24 and 25 perform the initial hardcoded sorting.  In reality they are only aliases and are only used to make the source easier to 
read.  Again the pointer alias optimization is used within this algorithm.  Lines 42, 43 and 44 initialize the aliases for 
$a$, $b$ and $c$ respectively.

The first subtraction loop occurs on lines 47 through 61.  The theory behind the subtraction loop is exactly the same as that for
the addition loop.  As remarked earlier there is an implementation reason for using the ``awkward'' method of extracting the carry 
(\textit{see line 57}).  The traditional method for extracting the carry would be to shift by $lg(\beta)$ positions and logically AND 
the least significant bit.  The AND operation is required because all of the bits above the $\lg(\beta)$'th bit will be set to one after a carry
occurs from subtraction.  This carry extraction requires two relatively cheap operations to extract the carry.  The other method is to simply 
shift the most significant bit to the least significant bit thus extracting the carry with a single cheap operation.  This optimization only works on
twos compliment machines which is a safe assumption to make.

If $a$ has a higher magnitude than $b$ an additional loop (\textit{see lines 64 through 73}) is required to propagate the carry through
$a$ and copy the result to $c$.  

\subsection{High Level Addition}
Now that both lower level addition and subtraction algorithms have been established an effective high level signed addition algorithm can be
established.  This high level addition algorithm will be what other algorithms and developers will use to perform addition of mp\_int data 
types.  

Recall from section 5.2 that an mp\_int represents an integer with an unsigned mantissa (\textit{the array of digits}) and a \textbf{sign} 
flag.  A high level addition is actually performed as a series of eight seperate cases which can be optimized down to three unique cases.

\newpage\begin{figure}[!here]
\begin{center}
\begin{tabular}{l}
\hline Algorithm \textbf{mp\_add}. \\
\textbf{Input}.   Two mp\_ints $a$ and $b$  \\
\textbf{Output}.  The signed addition $c = a + b$. \\
\hline \\
1.  if $a.sign = b.sign$ then do \\
\hspace{3mm}1.1  $c.sign \leftarrow a.sign$  \\
\hspace{3mm}1.2  $c \leftarrow \vert a \vert + \vert b \vert$ (\textit{hint: use s\_mp\_add})\\
2.  else do \\
\hspace{3mm}2.1  if $\vert a \vert < \vert b \vert$ then do (\textit{hint: use mp\_cmp\_mag})  \\
\hspace{6mm}2.1.1  $c.sign \leftarrow b.sign$ \\
\hspace{6mm}2.1.2  $c \leftarrow \vert b \vert - \vert a \vert$ (\textit{hint: use s\_mp\_sub}) \\
\hspace{3mm}2.2  else do \\
\hspace{6mm}2.2.1  $c.sign \leftarrow a.sign$ \\
\hspace{6mm}2.2.2  $c \leftarrow \vert a \vert - \vert b \vert$ \\
3.  If any of the lower level operations failed return(\textit{MP\_MEM}) \\
4.  Return(\textit{MP\_OKAY}). \\
\hline
\end{tabular}
\end{center}
\caption{Algorithm mp\_add}
\end{figure}

\textbf{Algorithm mp\_add.}
This algorithm performs the signed addition of two mp\_int variables.  There is no reference algorithm to draw upon from either \cite{TAOCPV2} or 
\cite{HAC} since they both only provide unsigned operations.  The algorithm is fairly straightforward but restricted since subtraction can only 
produce positive results.  Consider the following chart of possible inputs.

\begin{figure}[!here]
\begin{small}
\begin{center}
\begin{tabular}{|c|c|c|c|c|}
\hline \textbf{Sign of $a$} & \textbf{Sign of $b$} & \textbf{$\vert a \vert > \vert b \vert $} & \textbf{Unsigned Operation} & \textbf{Result Sign Flag} \\
\hline $+$ & $+$ & Yes & $c = a + b$ & $a.sign$ \\
\hline $+$ & $+$ & No  & $c = a + b$ & $a.sign$ \\
\hline $-$ & $-$ & Yes & $c = a + b$ & $a.sign$ \\
\hline $-$ & $-$ & No  & $c = a + b$ & $a.sign$ \\
\hline &&&&\\

\hline $+$ & $-$ & No  & $c = b - a$ & $b.sign$ \\
\hline $-$ & $+$ & No  & $c = b - a$ & $b.sign$ \\

\hline &&&&\\

\hline $+$ & $-$ & Yes & $c = a - b$ & $a.sign$ \\
\hline $-$ & $+$ & Yes & $c = a - b$ & $a.sign$ \\

\hline
\end{tabular}
\end{center}
\end{small}
\caption{Addition Guide Chart}
\end{figure}

The chart lists all of the eight possible input combinations and is sorted to show that only three specific cases need to be handled.  The 
return code of the unsigned operations at step 1.2, 2.1.2 and 2.2.2 are forwarded to step 3 to check for errors.  This simpliies the description
of the algorithm considerably and best follows how the implementation actually was achieved.

Also note how the \textbf{sign} is set before the unsigned addition or subtraction is performed.  Recall from the descriptions of algorithms
s\_mp\_add and s\_mp\_sub that the mp\_clamp function is used at the end to trim excess digits.  The mp\_clamp algorithm will set the \textbf{sign}
to \textbf{MP\_ZPOS} when the \textbf{used} digit count reaches zero.  

For example, consider performing $-a + a$ with algorithm mp\_add.  By the description of the algorithm the sign is set to \textbf{MP\_NEG} which would
produce a result of $-0$.  However, since the sign is set first then the unsigned addition is performed the subsequent usage of algorithm mp\_clamp 
within algorithm s\_mp\_add will force $-0$ to become $0$.  

\index{bn\_mp\_add.c}
\vspace{+3mm}\begin{small}
\hspace{-5.1mm}{\bf File}: bn\_mp\_add.c
\vspace{-3mm}
\begin{alltt}
016   
017   /* high level addition (handles signs) */
018   int
019   mp_add (mp_int * a, mp_int * b, mp_int * c)
020   \{
021     int     sa, sb, res;
022   
023     /* get sign of both inputs */
024     sa = a->sign;
025     sb = b->sign;
026   
027     /* handle two cases, not four */
028     if (sa == sb) \{
029       /* both positive or both negative */
030       /* add their magnitudes, copy the sign */
031       c->sign = sa;
032       res = s_mp_add (a, b, c);
033     \} else \{
034       /* one positive, the other negative */
035       /* subtract the one with the greater magnitude from */
036       /* the one of the lesser magnitude.  The result gets */
037       /* the sign of the one with the greater magnitude. */
038       if (mp_cmp_mag (a, b) == MP_LT) \{
039         c->sign = sb;
040         res = s_mp_sub (b, a, c);
041       \} else \{
042         c->sign = sa;
043         res = s_mp_sub (a, b, c);
044       \}
045     \}
046     return res;
047   \}
048   
\end{alltt}
\end{small}

The source code follows the algorithm fairly closely.  The most notable new source code addition is the usage of the $res$ integer variable which
is used to pass result of the unsigned operations forward.  Unlike in the algorithm, the variable $res$ is merely returned as is without
explicitly checking it and returning the constant \textbf{MP\_OKAY}.  The observation is this algorithm will succeed or fail only if the lower
level functions do so.  Returning their return code is sufficient.

\subsection{High Level Subtraction}
The high level signed subtraction algorithm is essentially the same as the high level signed addition algorithm.  

\begin{figure}[!here]
\begin{center}
\begin{tabular}{l}
\hline Algorithm \textbf{mp\_sub}. \\
\textbf{Input}.   Two mp\_ints $a$ and $b$  \\
\textbf{Output}.  The signed subtraction $c = a - b$. \\
\hline \\
1.  if $a.sign \ne b.sign$ then do \\
\hspace{3mm}1.1  $c.sign \leftarrow a.sign$ \\
\hspace{3mm}1.2  $c \leftarrow \vert a \vert + \vert b \vert$ (\textit{hint: use s\_mp\_add}) \\
2.  else do \\
\hspace{3mm}2.1  if $\vert a \vert \ge \vert b \vert$ then do (\textit{hint: use mp\_cmp\_mag}) \\
\hspace{6mm}2.1.1  $c.sign \leftarrow a.sign$ \\
\hspace{6mm}2.1.2  $c \leftarrow \vert a \vert  - \vert b \vert$ (\textit{hint: use s\_mp\_sub}) \\
\hspace{3mm}2.2  else do \\
\hspace{6mm}2.2.1  $c.sign \leftarrow  \left \lbrace \begin{array}{ll}
                              MP\_ZPOS &  \mbox{if }a.sign = MP\_NEG \\
                              MP\_NEG  &  \mbox{otherwise} \\
                              \end{array} \right .$ \\
\hspace{6mm}2.2.2  $c \leftarrow \vert b \vert  - \vert a \vert$ \\
3.  If any of the lower level operations failed return(\textit{MP\_MEM}). \\
4.  Return(\textit{MP\_OKAY}). \\
\hline
\end{tabular}
\end{center}
\caption{Algorithm mp\_sub}
\end{figure}

\textbf{Algorithm mp\_sub.}
This algorithm performs the signed subtraction of two inputs.  Similar to algorithm mp\_add there is no reference in either \cite{TAOCPV2} or 
\cite{HAC}.  Also this algorithm is restricted by algorithm s\_mp\_sub.  The following chart lists the eight possible inputs and
the operations required.

\newpage\begin{figure}[!here]
\begin{small}
\begin{center}
\begin{tabular}{|c|c|c|c|c|}
\hline \textbf{Sign of $a$} & \textbf{Sign of $b$} & \textbf{$\vert a \vert \ge \vert b \vert $} & \textbf{Unsigned Operation} & \textbf{Result Sign Flag} \\
\hline $+$ & $-$ & Yes & $c = a + b$ & $a.sign$ \\
\hline $+$ & $-$ & No  & $c = a + b$ & $a.sign$ \\
\hline $-$ & $+$ & Yes & $c = a + b$ & $a.sign$ \\
\hline $-$ & $+$ & No  & $c = a + b$ & $a.sign$ \\
\hline &&&& \\
\hline $+$ & $+$ & Yes & $c = a - b$ & $a.sign$ \\
\hline $-$ & $-$ & Yes & $c = a - b$ & $a.sign$ \\
\hline &&&& \\
\hline $+$ & $+$ & No  & $c = b - a$ & $\mbox{opposite of }a.sign$ \\
\hline $-$ & $-$ & No  & $c = b - a$ & $\mbox{opposite of }a.sign$ \\
\hline
\end{tabular}
\end{center}
\end{small}
\caption{Subtraction Guide Chart}
\end{figure}

Similar to the case of algorithm mp\_add the \textbf{sign} is set first before the unsigned addition or subtraction.  That is to prevent the 
algorithm from producing $-a - -a = -0$ as a result.  

\index{bn\_mp\_sub.c}
\vspace{+3mm}\begin{small}
\hspace{-5.1mm}{\bf File}: bn\_mp\_sub.c
\vspace{-3mm}
\begin{alltt}
016   
017   /* high level subtraction (handles signs) */
018   int
019   mp_sub (mp_int * a, mp_int * b, mp_int * c)
020   \{
021     int     sa, sb, res;
022   
023     sa = a->sign;
024     sb = b->sign;
025   
026     if (sa != sb) \{
027       /* subtract a negative from a positive, OR */
028       /* subtract a positive from a negative. */
029       /* In either case, ADD their magnitudes, */
030       /* and use the sign of the first number. */
031       c->sign = sa;
032       res = s_mp_add (a, b, c);
033     \} else \{
034       /* subtract a positive from a positive, OR */
035       /* subtract a negative from a negative. */
036       /* First, take the difference between their */
037       /* magnitudes, then... */
038       if (mp_cmp_mag (a, b) != MP_LT) \{
039         /* Copy the sign from the first */
040         c->sign = sa;
041         /* The first has a larger or equal magnitude */
042         res = s_mp_sub (a, b, c);
043       \} else \{
044         /* The result has the *opposite* sign from */
045         /* the first number. */
046         c->sign = (sa == MP_ZPOS) ? MP_NEG : MP_ZPOS;
047         /* The second has a larger magnitude */
048         res = s_mp_sub (b, a, c);
049       \}
050     \}
051     return res;
052   \}
053   
\end{alltt}
\end{small}

Much like the implementation of algorithm mp\_add the variable $res$ is used to catch the return code of the unsigned addition or subtraction operations
and forward it to the end of the function.  On line 38 the ``not equal to'' \textbf{MP\_LT} expression is used to emulate a 
``greater than or equal to'' comparison.  

\section{Bit and Digit Shifting}
It is quite common to think of a multiple precision integer as a polynomial in $x$, that is $y = f(\beta)$ where $f(x) = \sum_{i=0}^{n-1} a_i x^i$.  
This notation arises within discussion of Montgomery and Diminished Radix Reduction as well as Karatsuba multiplication and squaring.  

In order to facilitate operations on polynomials in $x$ as above a series of simple ``digit'' algorithms have to be established.  That is to shift
the digits left or right as well to shift individual bits of the digits left and right.  It is important to note that not all ``shift'' operations
are on radix-$\beta$ digits.  

\subsection{Multiplication by Two}

In a binary system where the radix is a power of two multiplication by two not only arises often in other algorithms it is a fairly efficient 
operation to perform.  A single precision logical shift left is sufficient to multiply a single digit by two.  

\newpage\begin{figure}[!here]
\begin{small}
\begin{center}
\begin{tabular}{l}
\hline Algorithm \textbf{mp\_mul\_2}. \\
\textbf{Input}.   One mp\_int $a$ \\
\textbf{Output}.  $b = 2a$. \\
\hline \\
1.  If $b.alloc < a.used + 1$ then grow $b$ to hold $a.used + 1$ digits.  (\textit{hint: use mp\_grow}) \\
2.  If the reallocation failed return(\textit{MP\_MEM}). \\
3.  $oldused \leftarrow b.used$ \\
4.  $b.used \leftarrow a.used$ \\
5.  $r \leftarrow 0$ \\
6.  for $n$ from 0 to $a.used - 1$ do \\
\hspace{3mm}6.1  $rr \leftarrow a_n >> (lg(\beta) - 1)$ \\
\hspace{3mm}6.2  $b_n \leftarrow (a_n << 1) + r \mbox{ (mod }\beta\mbox{)}$ \\
\hspace{3mm}6.3  $r \leftarrow rr$ \\
7.  If $r \ne 0$ then do \\
\hspace{3mm}7.1  $b_{a.used} = 1$ \\
\hspace{3mm}7.2  $b.used \leftarrow b.used + 1$ \\
8.  If $b.used < oldused - 1$ then do \\
\hspace{3mm}8.1  for $n$ from $b.used$ to $oldused - 1$ do \\
\hspace{6mm}8.1.1  $b_n \leftarrow 0$ \\
9.  $b.sign \leftarrow a.sign$ \\
10.  Return(\textit{MP\_OKAY}).\\
\hline
\end{tabular}
\end{center}
\end{small}
\caption{Algorithm mp\_mul\_2}
\end{figure}

\textbf{Algorithm mp\_mul\_2.}
This algorithm will quickly multiply a mp\_int by two provided $\beta$ is a power of two.  Neither \cite{TAOCPV2} nor \cite{HAC} describe such 
an algorithm despite the fact it arises often in other algorithms.  The algorithm is setup much like the lower level algorithm s\_mp\_add since 
it is for all intents and purposes equivalent to the operation $b = \vert a \vert + \vert a \vert$.  

Step 1 and 2 grow the input as required to accomodate the maximum number of \textbf{used} digits in the result.  The initial \textbf{used} count
is set to $a.used$ at step 4.  Only if there is a final carry will the \textbf{used} count require adjustment.

Step 6 is an optimization implementation of the addition loop for this specific case.  That is since the two values being added together 
are the same there is no need to perform two reads from the digits of $a$.  Step 6.1 performs a single precision shift on the current digit $a_n$ to
obtain what will be the carry for the next iteration.  Step 6.2 calculates the $n$'th digit of the result as single precision shift of $a_n$ plus
the previous carry.  Recall from section 5.1 that $a_n << 1$ is equivalent to $a_n \cdot 2$.  An iteration of the addition loop is finished with 
forwarding the carry to the next iteration.

Step 7 takes care of any final carry by setting the $a.used$'th digit of the result to one and augmenting the \textbf{used} count.  Step 8 clears
any original leading digits of $b$.

\index{bn\_mp\_mul\_2.c}
\vspace{+3mm}\begin{small}
\hspace{-5.1mm}{\bf File}: bn\_mp\_mul\_2.c
\vspace{-3mm}
\begin{alltt}
016   
017   /* b = a*2 */
018   int
019   mp_mul_2 (mp_int * a, mp_int * b)
020   \{
021     int     x, res, oldused;
022   
023     /* grow to accomodate result */
024     if (b->alloc < a->used + 1) \{
025       if ((res = mp_grow (b, a->used + 1)) != MP_OKAY) \{
026         return res;
027       \}
028     \}
029   
030     oldused = b->used;
031     b->used = a->used;
032   
033     \{
034       register mp_digit r, rr, *tmpa, *tmpb;
035   
036       /* alias for source */
037       tmpa = a->dp;
038       
039       /* alias for dest */
040       tmpb = b->dp;
041   
042       /* carry */
043       r = 0;
044       for (x = 0; x < a->used; x++) \{
045       
046         /* get what will be the *next* carry bit from the 
047          * MSB of the current digit 
048          */
049         rr = *tmpa >> ((mp_digit)(DIGIT_BIT - 1));
050         
051         /* now shift up this digit, add in the carry [from the previous] */
052         *tmpb++ = ((*tmpa++ << ((mp_digit)1)) | r) & MP_MASK;
053         
054         /* copy the carry that would be from the source 
055          * digit into the next iteration 
056          */
057         r = rr;
058       \}
059   
060       /* new leading digit? */
061       if (r != 0) \{
062         /* add a MSB which is always 1 at this point */
063         *tmpb = 1;
064         ++b->used;
065       \}
066   
067       /* now zero any excess digits on the destination 
068        * that we didn't write to 
069        */
070       tmpb = b->dp + b->used;
071       for (x = b->used; x < oldused; x++) \{
072         *tmpb++ = 0;
073       \}
074     \}
075     b->sign = a->sign;
076     return MP_OKAY;
077   \}
\end{alltt}
\end{small}

This implementation is essentially an optimized implementation of s\_mp\_add for the case of doubling an input.  The only noteworthy difference
is the use of the logical shift operator on line 52 to perform a single precision doubling.  

\subsection{Division by Two}
A division by two can just as easily be accomplished with a logical shift right as multiplication by two can be with a logical shift left.

\newpage\begin{figure}[!here]
\begin{small}
\begin{center}
\begin{tabular}{l}
\hline Algorithm \textbf{mp\_div\_2}. \\
\textbf{Input}.   One mp\_int $a$ \\
\textbf{Output}.  $b = a/2$. \\
\hline \\
1.  If $b.alloc < a.used$ then grow $b$ to hold $a.used$ digits.  (\textit{hint: use mp\_grow}) \\
2.  If the reallocation failed return(\textit{MP\_MEM}). \\
3.  $oldused \leftarrow b.used$ \\
4.  $b.used \leftarrow a.used$ \\
5.  $r \leftarrow 0$ \\
6.  for $n$ from $b.used - 1$ to $0$ do \\
\hspace{3mm}6.1  $rr \leftarrow a_n \mbox{ (mod }2\mbox{)}$\\
\hspace{3mm}6.2  $b_n \leftarrow (a_n >> 1) + (r << (lg(\beta) - 1)) \mbox{ (mod }\beta\mbox{)}$ \\
\hspace{3mm}6.3  $r \leftarrow rr$ \\
7.  If $b.used < oldused - 1$ then do \\
\hspace{3mm}7.1  for $n$ from $b.used$ to $oldused - 1$ do \\
\hspace{6mm}7.1.1  $b_n \leftarrow 0$ \\
8.  $b.sign \leftarrow a.sign$ \\
9.  Return(\textit{MP\_OKAY}).\\
\hline
\end{tabular}
\end{center}
\end{small}
\caption{Algorithm mp\_div\_2}
\end{figure}

\textbf{Algorithm mp\_div\_2.}
This algorithm will divide an mp\_int by two using logical shifts to the right.  Like mp\_mul\_2 it uses a modified low level addition
core as the basis of the algorithm.  Unlike mp\_mul\_2 the shift operations work from the leading digit to the trailing digit.  The algorithm
could be written to work from the trailing digit to the leading digit however, it would have to stop one short of $a.used - 1$ digits to prevent
reading passed the end of the array of digits.

Essentially the loop at step 6 is similar to that of mp\_mul\_2 except the logical shifts go in the opposite direction and the carry is at the 
least significant bit not the most significant bit.  

\index{bn\_mp\_div\_2.c}
\vspace{+3mm}\begin{small}
\hspace{-5.1mm}{\bf File}: bn\_mp\_div\_2.c
\vspace{-3mm}
\begin{alltt}
016   
017   /* b = a/2 */
018   int
019   mp_div_2 (mp_int * a, mp_int * b)
020   \{
021     int     x, res, oldused;
022   
023     /* copy */
024     if (b->alloc < a->used) \{
025       if ((res = mp_grow (b, a->used)) != MP_OKAY) \{
026         return res;
027       \}
028     \}
029   
030     oldused = b->used;
031     b->used = a->used;
032     \{
033       register mp_digit r, rr, *tmpa, *tmpb;
034   
035       /* source alias */
036       tmpa = a->dp + b->used - 1;
037   
038       /* dest alias */
039       tmpb = b->dp + b->used - 1;
040   
041       /* carry */
042       r = 0;
043       for (x = b->used - 1; x >= 0; x--) \{
044         /* get the carry for the next iteration */
045         rr = *tmpa & 1;
046   
047         /* shift the current digit, add in carry and store */
048         *tmpb-- = (*tmpa-- >> 1) | (r << (DIGIT_BIT - 1));
049   
050         /* forward carry to next iteration */
051         r = rr;
052       \}
053   
054       /* zero excess digits */
055       tmpb = b->dp + b->used;
056       for (x = b->used; x < oldused; x++) \{
057         *tmpb++ = 0;
058       \}
059     \}
060     b->sign = a->sign;
061     mp_clamp (b);
062     return MP_OKAY;
063   \}
\end{alltt}
\end{small}

\section{Polynomial Basis Operations}
Recall from section 5.3 that any integer can be represented as a polynomial in $x$ as $y = f(\beta)$.  Such a representation is also known as
the polynomial basis \cite[pp. 48]{ROSE}. Given such a notation a multiplication or division by $x$ amounts to shifting whole digits a single 
place.  The need for such operations arises in several other higher level algorithms such as Barrett and Montgomery reduction, integer
division and Karatsuba multiplication.  

Converting from an array of digits to polynomial basis is very simple.  Consider the integer $y \equiv (a_2, a_1, a_0)_{\beta}$ and recall that
$y = \sum_{i=0}^{2} a_i \beta^i$.  Simply replace $\beta$ with $x$ and the expression is in polynomial basis.  For example, $f(x) = 8x + 9$ is the
polynomial basis representation for $89$ using radix ten.  That is, $f(10) = 8(10) + 9 = 89$.  

\subsection{Multiplication by $x$}

Given a polynomial in $x$ such as $f(x) = a_n x^n + a_{n-1} x^{n-1} + ... + a_0$ multiplying by $x$ amounts to shifting the coefficients up one 
degree.  In this case $f(x) \cdot x = a_n x^{n+1} + a_{n-1} x^n + ... + a_0 x$.  From a scalar basis point of view multiplying by $x$ is equivalent to
multiplying by the integer $\beta$.  

\newpage\begin{figure}[!here]
\begin{small}
\begin{center}
\begin{tabular}{l}
\hline Algorithm \textbf{mp\_lshd}. \\
\textbf{Input}.   One mp\_int $a$ and an integer $b$ \\
\textbf{Output}.  $a \leftarrow a \cdot \beta^b$ (Multiply by $x^b$). \\
\hline \\
1.  If $b \le 0$ then return(\textit{MP\_OKAY}). \\
2.  If $a.alloc < a.used + b$ then grow $a$ to at least $a.used + b$ digits.  (\textit{hint: use mp\_grow}). \\
3.  If the reallocation failed return(\textit{MP\_MEM}). \\
4.  $a.used \leftarrow a.used + b$ \\
5.  $i \leftarrow a.used - 1$ \\
6.  $j \leftarrow a.used - 1 - b$ \\
7.  for $n$ from $a.used - 1$ to $b$ do \\
\hspace{3mm}7.1  $a_{i} \leftarrow a_{j}$ \\
\hspace{3mm}7.2  $i \leftarrow i - 1$ \\
\hspace{3mm}7.3  $j \leftarrow j - 1$ \\
8.  for $n$ from 0 to $b - 1$ do \\
\hspace{3mm}8.1  $a_n \leftarrow 0$ \\
9.  Return(\textit{MP\_OKAY}). \\
\hline
\end{tabular}
\end{center}
\end{small}
\caption{Algorithm mp\_lshd}
\end{figure}

\textbf{Algorithm mp\_lshd.}
This algorithm multiplies an mp\_int by the $b$'th power of $x$.  This is equivalent to multiplying by $\beta^b$.  The algorithm differs 
from the other algorithms presented so far as it performs the operation in place instead storing the result in a seperate location.  The algorithm
will return success immediately if $b \le 0$ since the rest of algorithm is only valid when $b > 0$.  

First the destination $a$ is grown as required to accomodate the result.  The counters $i$ and $j$ are used to form a \textit{sliding window} over
the digits of $a$ of length $b$.  The head of the sliding window is at $i$ (\textit{the leading digit}) and the tail at $j$ (\textit{the trailing digit}).  
The loop on step 7 copies the digit from the tail to the head.  In each iteration the window is moved down one digit.   The last loop on 
step 8 sets the lower $b$ digits to zero.

\newpage
\begin{center}
\begin{figure}[here]
\includegraphics{pics/sliding_window.ps}
\caption{Sliding Window Movement}
\end{figure}
\end{center}

\index{bn\_mp\_lshd.c}
\vspace{+3mm}\begin{small}
\hspace{-5.1mm}{\bf File}: bn\_mp\_lshd.c
\vspace{-3mm}
\begin{alltt}
016   
017   /* shift left a certain amount of digits */
018   int
019   mp_lshd (mp_int * a, int b)
020   \{
021     int     x, res;
022   
023     /* if its less than zero return */
024     if (b <= 0) \{
025       return MP_OKAY;
026     \}
027   
028     /* grow to fit the new digits */
029     if (a->alloc < a->used + b) \{
030        if ((res = mp_grow (a, a->used + b)) != MP_OKAY) \{
031          return res;
032        \}
033     \}
034   
035     \{
036       register mp_digit *tmpa, *tmpaa;
037   
038       /* increment the used by the shift amount than copy upwards */
039       a->used += b;
040   
041       /* top */
042       tmpa = a->dp + a->used - 1;
043   
044       /* base */
045       tmpaa = a->dp + a->used - 1 - b;
046   
047       /* much like mp_rshd this is implemented using a sliding window
048        * except the window goes the otherway around.  Copying from
049        * the bottom to the top.  see bn_mp_rshd.c for more info.
050        */
051       for (x = a->used - 1; x >= b; x--) \{
052         *tmpa-- = *tmpaa--;
053       \}
054   
055       /* zero the lower digits */
056       tmpa = a->dp;
057       for (x = 0; x < b; x++) \{
058         *tmpa++ = 0;
059       \}
060     \}
061     return MP_OKAY;
062   \}
\end{alltt}
\end{small}

The if statement on line 24 ensures that the $b$ variable is greater than zero.  The \textbf{used} count is incremented by $b$ before
the copy loop begins.  This elminates the need for an additional variable in the for loop.  The variable $tmpa$ on line 42 is an alias
for the leading digit while $tmpaa$ on line 45 is an alias for the trailing edge.  The aliases form a window of exactly $b$ digits
over the input.  

\subsection{Division by $x$}

Division by powers of $x$ is easily achieved by shifting the digits right and removing any that will end up to the right of the zero'th digit.  

\newpage\begin{figure}[!here]
\begin{small}
\begin{center}
\begin{tabular}{l}
\hline Algorithm \textbf{mp\_rshd}. \\
\textbf{Input}.   One mp\_int $a$ and an integer $b$ \\
\textbf{Output}.  $a \leftarrow a / \beta^b$ (Divide by $x^b$). \\
\hline \\
1.  If $b \le 0$ then return. \\
2.  If $a.used \le b$ then do \\
\hspace{3mm}2.1  Zero $a$.  (\textit{hint: use mp\_zero}). \\
\hspace{3mm}2.2  Return. \\
3.  $i \leftarrow 0$ \\
4.  $j \leftarrow b$ \\
5.  for $n$ from 0 to $a.used - b - 1$ do \\
\hspace{3mm}5.1  $a_i \leftarrow a_j$ \\
\hspace{3mm}5.2  $i \leftarrow i + 1$ \\
\hspace{3mm}5.3  $j \leftarrow j + 1$ \\
6.  for $n$ from $a.used - b$ to $a.used - 1$ do \\
\hspace{3mm}6.1  $a_n \leftarrow 0$ \\
7.  Clamp excess digits.  (\textit{hint: use mp\_clamp}). \\
8.  Return. \\
\hline
\end{tabular}
\end{center}
\end{small}
\caption{Algorithm mp\_rshd}
\end{figure}

\textbf{Algorithm mp\_rshd.}
This algorithm divides the input in place by the $b$'th power of $x$.  It is analogous to dividing by a $\beta^b$ but much quicker since
it does not require single precision division.  This algorithm does not actually return an error code as it cannot fail.  

If the input $b$ is less than one the algorithm quickly returns without performing any work.  If the \textbf{used} count is less than or equal
to the shift count $b$ then it will simply zero the input and return.

After the trivial cases of inputs have been handled the sliding window is setup.  Much like the case of algorithm mp\_lshd a sliding window that
is $b$ digits wide is used to copy the digits.  Unlike mp\_lshd the window slides in the opposite direction from the trailing to the leading digit.  
Also the digits are copied from the leading to the trailing edge.

Once the window copy is complete the upper digits must be zeroed.  Finally algorithm mp\_clamp is used to trim excess digits.

\index{bn\_mp\_rshd.c}
\vspace{+3mm}\begin{small}
\hspace{-5.1mm}{\bf File}: bn\_mp\_rshd.c
\vspace{-3mm}
\begin{alltt}
016   
017   /* shift right a certain amount of digits */
018   void
019   mp_rshd (mp_int * a, int b)
020   \{
021     int     x;
022   
023     /* if b <= 0 then ignore it */
024     if (b <= 0) \{
025       return;
026     \}
027   
028     /* if b > used then simply zero it and return */
029     if (a->used <= b) \{
030       mp_zero (a);
031       return;
032     \}
033   
034     \{
035       register mp_digit *tmpa, *tmpaa;
036   
037       /* shift the digits down */
038   
039       /* base */
040       tmpa = a->dp;
041   
042       /* offset into digits */
043       tmpaa = a->dp + b;
044   
045       /* this is implemented as a sliding window where 
046        * the window is b-digits long and digits from 
047        * the top of the window are copied to the bottom
048        *
049        * e.g.
050   
051        b-2 | b-1 | b0 | b1 | b2 | ... | bb |   ---->
052                    /\symbol{92}                   |      ---->
053                     \symbol{92}-------------------/      ---->
054        */
055       for (x = 0; x < (a->used - b); x++) \{
056         *tmpa++ = *tmpaa++;
057       \}
058   
059       /* zero the top digits */
060       for (; x < a->used; x++) \{
061         *tmpa++ = 0;
062       \}
063     \}
064     mp_clamp (a);
065   \}
\end{alltt}
\end{small}

The only noteworthy element of this routine is the lack of a return type.  This function cannot fail and as such it is more optimal to not
return anything.

\section{Powers of Two}

Now that algorithms for moving single bits as well as whole digits exist algorithms for moving the ``in between'' distances are required.  For 
example, to quickly multiply by $2^k$ for any $k$ without using a full multiplier algorithm would prove useful.  Instead of performing single
shifts $k$ times to achieve a multiplication by $2^{\pm k}$ a mixture of whole digit shifting and partial digit shifting is employed.  

\subsection{Multiplication by Power of Two}

\newpage\begin{figure}[!here]
\begin{small}
\begin{center}
\begin{tabular}{l}
\hline Algorithm \textbf{mp\_mul\_2d}. \\
\textbf{Input}.   One mp\_int $a$ and an integer $b$ \\
\textbf{Output}.  $c \leftarrow a \cdot 2^b$. \\
\hline \\
1.  $c \leftarrow a$.  (\textit{hint: use mp\_copy}) \\
2.  If $c.alloc < c.used + \lfloor b / lg(\beta) \rfloor + 2$ then grow $c$ accordingly. \\
3.  If the reallocation failed return(\textit{MP\_MEM}). \\
4.  If $b \ge lg(\beta)$ then \\
\hspace{3mm}4.1  $c \leftarrow c \cdot \beta^{\lfloor b / lg(\beta) \rfloor}$ (\textit{hint: use mp\_lshd}). \\
\hspace{3mm}4.2  If step 4.1 failed return(\textit{MP\_MEM}). \\
5.  $d \leftarrow b \mbox{ (mod }lg(\beta)\mbox{)}$ \\
6.  If $d \ne 0$ then do \\
\hspace{3mm}6.1  $mask \leftarrow 2^d$ \\
\hspace{3mm}6.2  $r \leftarrow 0$ \\
\hspace{3mm}6.3  for $n$ from $0$ to $c.used - 1$ do \\
\hspace{6mm}6.3.1  $rr \leftarrow c_n >> (lg(\beta) - d) \mbox{ (mod }mask\mbox{)}$ \\
\hspace{6mm}6.3.2  $c_n \leftarrow (c_n << d) + r \mbox{ (mod }\beta\mbox{)}$ \\
\hspace{6mm}6.3.3  $r \leftarrow rr$ \\
\hspace{3mm}6.4  If $r > 0$ then do \\
\hspace{6mm}6.4.1  $c_{c.used} \leftarrow r$ \\
\hspace{6mm}6.4.2  $c.used \leftarrow c.used + 1$ \\
7.  Return(\textit{MP\_OKAY}). \\
\hline
\end{tabular}
\end{center}
\end{small}
\caption{Algorithm mp\_mul\_2d}
\end{figure}

\textbf{Algorithm mp\_mul\_2d.}
This algorithm multiplies $a$ by $2^b$ and stores the result in $c$.  The algorithm uses algorithm mp\_lshd and a derivative of algorithm mp\_mul\_2 to
quickly compute the product.

First the algorithm will multiply $a$ by $x^{\lfloor b / lg(\beta) \rfloor}$ which will ensure that the remainder multiplicand is less than 
$\beta$.  For example, if $b = 37$ and $\beta = 2^{28}$ then this step will multiply by $x$ leaving a multiplication by $2^{37 - 28} = 2^{9}$ 
left.

The logarithm of the residue is calculated on step 5.  If it is non-zero a modified shift loop is used to calculate the remaining product.  
Essentially the loop is a generic version of algorith mp\_mul2 designed to handle any shift count in the range $1 \le x < lg(\beta)$.  The $mask$
variable is used to extract the upper $d$ bits to form the carry for the next iteration.  

This algorithm is loosely measured as a $O(2n)$ algorithm which means that if the input is $n$-digits that it takes $2n$ ``time'' to 
complete.  It is possible to optimize this algorithm down to a $O(n)$ algorithm at a cost of making the algorithm slightly harder to follow.

\index{bn\_mp\_mul\_2d.c}
\vspace{+3mm}\begin{small}
\hspace{-5.1mm}{\bf File}: bn\_mp\_mul\_2d.c
\vspace{-3mm}
\begin{alltt}
016   
017   /* NOTE:  This routine requires updating.  For instance the c->used = c->all
      oc bit
018      is wrong.  We should just shift c->used digits then set the carry as c->d
      p[c->used] = carry
019    
020      To be fixed for LTM 0.18
021    */
022   
023   /* shift left by a certain bit count */
024   int
025   mp_mul_2d (mp_int * a, int b, mp_int * c)
026   \{
027     mp_digit d;
028     int      res;
029   
030     /* copy */
031     if (a != c) \{
032        if ((res = mp_copy (a, c)) != MP_OKAY) \{
033          return res;
034        \}
035     \}
036   
037     if (c->alloc < (int)(c->used + b/DIGIT_BIT + 2)) \{
038        if ((res = mp_grow (c, c->used + b / DIGIT_BIT + 2)) != MP_OKAY) \{
039          return res;
040        \}
041     \}
042   
043     /* shift by as many digits in the bit count */
044     if (b >= (int)DIGIT_BIT) \{
045       if ((res = mp_lshd (c, b / DIGIT_BIT)) != MP_OKAY) \{
046         return res;
047       \}
048     \}
049     c->used = c->alloc;
050   
051     /* shift any bit count < DIGIT_BIT */
052     d = (mp_digit) (b % DIGIT_BIT);
053     if (d != 0) \{
054       register mp_digit *tmpc, mask, r, rr;
055       register int x;
056   
057       /* bitmask for carries */
058       mask = (((mp_digit)1) << d) - 1;
059   
060       /* alias */
061       tmpc = c->dp;
062   
063       /* carry */
064       r    = 0;
065       for (x = 0; x < c->used; x++) \{
066         /* get the higher bits of the current word */
067         rr = (*tmpc >> (DIGIT_BIT - d)) & mask;
068   
069         /* shift the current word and OR in the carry */
070         *tmpc = ((*tmpc << d) | r) & MP_MASK;
071         ++tmpc;
072   
073         /* set the carry to the carry bits of the current word */
074         r = rr;
075       \}
076     \}
077     mp_clamp (c);
078     return MP_OKAY;
079   \}
\end{alltt}
\end{small}

Notes to be revised when code is updated. -- Tom

\subsection{Division by Power of Two}

\newpage\begin{figure}[!here]
\begin{small}
\begin{center}
\begin{tabular}{l}
\hline Algorithm \textbf{mp\_div\_2d}. \\
\textbf{Input}.   One mp\_int $a$ and an integer $b$ \\
\textbf{Output}.  $c \leftarrow \lfloor a / 2^b \rfloor, d \leftarrow a \mbox{ (mod }2^b\mbox{)}$. \\
\hline \\
1.  If $b \le 0$ then do \\
\hspace{3mm}1.1  $c \leftarrow a$ (\textit{hint: use mp\_copy}) \\
\hspace{3mm}1.2  $d \leftarrow 0$ (\textit{hint: use mp\_zero}) \\
\hspace{3mm}1.3  Return(\textit{MP\_OKAY}). \\
2.  $c \leftarrow a$ \\
3.  $d \leftarrow a \mbox{ (mod }2^b\mbox{)}$ (\textit{hint: use mp\_mod\_2d}) \\
4.  If $b \ge lg(\beta)$ then do \\
\hspace{3mm}4.1  $c \leftarrow \lfloor c/\beta^{\lfloor b/lg(\beta) \rfloor} \rfloor$ (\textit{hint: use mp\_rshd}). \\
5.  $k \leftarrow b \mbox{ (mod }lg(\beta)\mbox{)}$ \\
6.  If $k \ne 0$ then do \\
\hspace{3mm}6.1  $mask \leftarrow 2^k$ \\
\hspace{3mm}6.2  $r \leftarrow 0$ \\
\hspace{3mm}6.3  for $n$ from $c.used - 1$ to $0$ do \\
\hspace{6mm}6.3.1  $rr \leftarrow c_n \mbox{ (mod }mask\mbox{)}$ \\
\hspace{6mm}6.3.2  $c_n \leftarrow (c_n >> k) + (r << (lg(\beta) - k))$ \\
\hspace{6mm}6.3.3  $r \leftarrow rr$ \\
7.  Clamp excess digits of $c$.  (\textit{hint: use mp\_clamp}) \\
8.  Return(\textit{MP\_OKAY}). \\
\hline
\end{tabular}
\end{center}
\end{small}
\caption{Algorithm mp\_div\_2d}
\end{figure}

\textbf{Algorithm mp\_div\_2d.}
This algorithm will divide an input $a$ by $2^b$ and produce the quotient and remainder.  The algorithm is designed much like algorithm 
mp\_mul\_2d by first using whole digit shifts then single precision shifts.  This algorithm will also produce the remainder of the division
by using algorithm mp\_mod\_2d.

\index{bn\_mp\_div\_2d.c}
\vspace{+3mm}\begin{small}
\hspace{-5.1mm}{\bf File}: bn\_mp\_div\_2d.c
\vspace{-3mm}
\begin{alltt}
016   
017   /* shift right by a certain bit count (store quotient in c, remainder in d) 
      */
018   int
019   mp_div_2d (mp_int * a, int b, mp_int * c, mp_int * d)
020   \{
021     mp_digit D, r, rr;
022     int     x, res;
023     mp_int  t;
024   
025   
026     /* if the shift count is <= 0 then we do no work */
027     if (b <= 0) \{
028       res = mp_copy (a, c);
029       if (d != NULL) \{
030         mp_zero (d);
031       \}
032       return res;
033     \}
034   
035     if ((res = mp_init (&t)) != MP_OKAY) \{
036       return res;
037     \}
038   
039     /* get the remainder */
040     if (d != NULL) \{
041       if ((res = mp_mod_2d (a, b, &t)) != MP_OKAY) \{
042         mp_clear (&t);
043         return res;
044       \}
045     \}
046   
047     /* copy */
048     if ((res = mp_copy (a, c)) != MP_OKAY) \{
049       mp_clear (&t);
050       return res;
051     \}
052   
053     /* shift by as many digits in the bit count */
054     if (b >= (int)DIGIT_BIT) \{
055       mp_rshd (c, b / DIGIT_BIT);
056     \}
057   
058     /* shift any bit count < DIGIT_BIT */
059     D = (mp_digit) (b % DIGIT_BIT);
060     if (D != 0) \{
061       register mp_digit *tmpc, mask;
062   
063       /* mask */
064       mask = (((mp_digit)1) << D) - 1;
065   
066       /* alias */
067       tmpc = c->dp + (c->used - 1);
068   
069       /* carry */
070       r = 0;
071       for (x = c->used - 1; x >= 0; x--) \{
072         /* get the lower  bits of this word in a temp */
073         rr = *tmpc & mask;
074   
075         /* shift the current word and mix in the carry bits from the previous 
      word */
076         *tmpc = (*tmpc >> D) | (r << (DIGIT_BIT - D));
077         --tmpc;
078   
079         /* set the carry to the carry bits of the current word found above */
080         r = rr;
081       \}
082     \}
083     mp_clamp (c);
084     res = MP_OKAY;
085     if (d != NULL) \{
086       mp_exch (&t, d);
087     \}
088     mp_clear (&t);
089     return MP_OKAY;
090   \}
\end{alltt}
\end{small}

The implementation of algorithm mp\_div\_2d is slightly different than the algorithm specifies.  The remainder $d$ may be optionally 
ignored by passing \textbf{NULL} as the pointer to the mp\_int variable.    The temporary mp\_int variable $t$ is used to hold the 
result of the remainder operation until the end.  This allows $d = a$ to be true without overwriting the input before they are no longer required.  

The remainder of the source code is essentially the same as the source code for mp\_mul\_2d.  (-- Fix this paragraph up later, Tom).

\subsection{Remainder of Division by Power of Two}

The last algorithm in the series of polynomial basis power of two algorithms is calculating the remainder of division by $2^b$.  This
algorithm benefits from the fact that in twos complement arithmetic $a \mbox{ (mod }2^b\mbox{)}$ is the same as $a$ AND $2^b - 1$.  

\begin{figure}[!here]
\begin{small}
\begin{center}
\begin{tabular}{l}
\hline Algorithm \textbf{mp\_mod\_2d}. \\
\textbf{Input}.   One mp\_int $a$ and an integer $b$ \\
\textbf{Output}.  $c \leftarrow a \mbox{ (mod }2^b\mbox{)}$. \\
\hline \\
1.  If $b \le 0$ then do \\
\hspace{3mm}1.1  $c \leftarrow 0$ (\textit{hint: use mp\_zero}) \\
\hspace{3mm}1.2  Return(\textit{MP\_OKAY}). \\
2.  If $b > a.used \cdot lg(\beta)$ then do \\
\hspace{3mm}2.1  $c \leftarrow a$ (\textit{hint: use mp\_copy}) \\
\hspace{3mm}2.2  Return the result of step 2.1. \\
3.  $c \leftarrow a$ \\
4.  If step 3 failed return(\textit{MP\_MEM}). \\
5.  for $n$ from $\lceil b / lg(\beta) \rceil$ to $c.used$ do \\
\hspace{3mm}5.1  $c_n \leftarrow 0$ \\
6.  $k \leftarrow b \mbox{ (mod }lg(\beta)\mbox{)}$ \\
7.  $c_{\lfloor b / lg(\beta) \rfloor} \leftarrow c_{\lfloor b / lg(\beta) \rfloor} \mbox{ (mod }2^{k}\mbox{)}$. \\
8.  Return(\textit{MP\_OKAY}). \\
\hline
\end{tabular}
\end{center}
\end{small}
\caption{Algorithm mp\_mod\_2d}
\end{figure}

\textbf{Algorithm mp\_mod\_2d.}
This algorithm will quickly calculate the value of $a \mbox{ (mod }2^b\mbox{)}$.  First if $b$ is less than or equal to zero the 
result is set to zero.  If $b$ is greater than the number of bits in $a$ then it simply copies $a$ to $c$ and returns.  Otherwise, $a$ 
is copied to $b$, leading digits are removed and the remaining leading digit is trimed to the exact bit count.

\index{bn\_mp\_mod\_2d.c}
\vspace{+3mm}\begin{small}
\hspace{-5.1mm}{\bf File}: bn\_mp\_mod\_2d.c
\vspace{-3mm}
\begin{alltt}
016   
017   /* calc a value mod 2\b */
018   int
019   mp_mod_2d (mp_int * a, int b, mp_int * c)
020   \{
021     int     x, res;
022   
023   
024     /* if b is <= 0 then zero the int */
025     if (b <= 0) \{
026       mp_zero (c);
027       return MP_OKAY;
028     \}
029   
030     /* if the modulus is larger than the value than return */
031     if (b > (int) (a->used * DIGIT_BIT)) \{
032       res = mp_copy (a, c);
033       return res;
034     \}
035   
036     /* copy */
037     if ((res = mp_copy (a, c)) != MP_OKAY) \{
038       return res;
039     \}
040   
041     /* zero digits above the last digit of the modulus */
042     for (x = (b / DIGIT_BIT) + ((b % DIGIT_BIT) == 0 ? 0 : 1); x < c->used; x+
      +) \{
043       c->dp[x] = 0;
044     \}
045     /* clear the digit that is not completely outside/inside the modulus */
046     c->dp[b / DIGIT_BIT] &=
047       (mp_digit) ((((mp_digit) 1) << (((mp_digit) b) % DIGIT_BIT)) - ((mp_digi
      t) 1));
048     mp_clamp (c);
049     return MP_OKAY;
050   \}
\end{alltt}
\end{small}

-- Add comments later, Tom.

\section*{Exercises}
\begin{tabular}{cl}
$\left [ 3 \right ] $ & Devise an algorithm that performs $a \cdot 2^b$ for generic values of $b$ \\
                      & in $O(n)$ time. \\
                      &\\
$\left [ 3 \right ] $ & Devise an efficient algorithm to multiply by small low hamming  \\
                      & weight values such as $3$, $5$ and $9$.  Extend it to handle all values \\
                      & upto $64$ with a hamming weight less than three. \\
                      &\\
$\left [ 2 \right ] $ & Modify the preceding algorithm to handle values of the form \\
                      & $2^k - 1$ as well. \\
                      &\\
$\left [ 3 \right ] $ & Using only algorithms mp\_mul\_2, mp\_div\_2 and mp\_add create an \\
                      & algorithm to multiply two integers in roughly $O(2n^2)$ time for \\
                      & any $n$-bit input.  Note that the time of addition is ignored in the \\
                      & calculation.  \\
                      & \\
$\left [ 5 \right ] $ & Improve the previous algorithm to have a working time of at most \\
                      & $O \left (2^{(k-1)}n + \left ({2n^2 \over k} \right ) \right )$ for an appropriate choice of $k$.  Again ignore \\
                      & the cost of addition. \\
                      & \\
$\left [ 1 \right ] $ & There exists an improvement on the previous algorithm to \\
                      & slightly reduce the number of additions required.  Modify the \\
                      & previous algorithm to include this improvement. \\
                      & \\
$\left [ 2 \right ] $ & Devise a chart to find optimal values of $k$ for the previous problem \\
                      & for $n = 64 \ldots 1024$ in steps of $64$. \\
                      & \\
$\left [ 2 \right ] $ & Using only algorithms mp\_abs and mp\_sub devise another method for \\
                      & calculating the result of a signed comparison. \\
                      &
\end{tabular}

\chapter{Multiplication and Squaring}
\section{The Multipliers}
For most number theoretic systems including public key cryptographic algorithms the set of algorithms collectively known as the
``multipliers'' form the most important subset of algorithms of any multiple precision integer package.  The set of multipliers 
include multiplication, squaring and modular reduction algorithms.  

The importance of these algorithms is driven by the fact that most popular public key algorithms are based on modular 
exponentiation.  That is performing $d \equiv a^b \mbox{ (mod }c\mbox{)}$ for some arbitrary choice of $a$, $b$, $c$ and $d$.  Roughly
speaking the a modular exponentiation will spend about 40\% of the time in modular reductions, 35\% of the time in squaring and 25\% of
the time in multiplications.  Only a small trivial amount of time is spent on lower level algorithms such as mp\_clamp, mp\_init, etc...

This chapter will discuss only two of the multipliers algorithms, multiplication and squaring.  As will be discussed shortly very efficient
multiplier algorithms are not always straightforward and deserve a lot of attention.

\section{Multiplication}
\subsection{The Baseline Multiplication}
\index{baseline multiplication}
Computing the product of two integers in software can be achieved using a trivial adaptation of the standard $O(n^2)$ long-hand multiplication
algorithm school children are taught.  The ``baseline multiplication'' algorithm is designed to act as the ``catch-all'' algorithm only called
when the faster algorithms cannot be used.  This algorithm does not use any particularly interesting optimizations.

The first algorithm to review is the unsigned multiplication algorithm from which a signed multiplication algorithm can be established.  One important 
facet of this algorithm to note is that it has been modified to only produce a certain amount of output digits as resolution.  Recall that for
a $n$ and $m$ digit input the product will be at most $n + m + 1$ digits.  Therefore, this algorithm can be reduced to a full multiplier by
telling it to produce $n + m + 1$ digits.  

Recall from sub-section 5.2.2 the definition of $\gamma$ as the number of bits in the type \textbf{mp\_digit}.  We shall now extend this variable set to 
include $\alpha$ which shall represent the number of bits in the type \textbf{mp\_word}.  This implies that $2^{\alpha} > 2 \cdot \beta^2$.  The 
constant $\delta = 2^{\alpha - 2lg(\beta)}$ will represent the maximal weight of any column in a product (\textit{see sub-section 6.2.2 for more information}).

\newpage\begin{figure}[!here]
\begin{small}
\begin{center}
\begin{tabular}{l}
\hline Algorithm \textbf{s\_mp\_mul\_digs}. \\
\textbf{Input}.   mp\_int $a$, mp\_int $b$ and an integer $digs$ \\
\textbf{Output}.  $c \leftarrow \vert a \vert \cdot \vert b \vert \mbox{ (mod }\beta^{digs}\mbox{)}$. \\
\hline \\
1.  If min$(a.used, b.used) < \delta$ then do \\
\hspace{3mm}1.1  Calculate $c = \vert a \vert \cdot \vert b \vert$ by the Comba method.  \\
\hspace{3mm}1.2  Return the result of step 1.1 \\
\\
Allocate and initialize a temporary mp\_int. \\
2.  Init $t$ to be of size $digs$ \\
3.  If step 2 failed return(\textit{MP\_MEM}). \\
4.  $t.used \leftarrow digs$ \\
\\
Compute the product. \\
5.  for $ix$ from $0$ to $a.used - 1$ do \\
\hspace{3mm}5.1  $u \leftarrow 0$ \\
\hspace{3mm}5.2  $pb \leftarrow \mbox{min}(b.used, digs - ix)$ \\
\hspace{3mm}5.3  If $pb < 1$ then goto step 6. \\
\hspace{3mm}5.4  for $iy$ from $0$ to $pb - 1$ do \\
\hspace{6mm}5.4.1  $\hat r \leftarrow t_{iy + ix} + a_{ix} \cdot b_{iy} + u$ \\
\hspace{6mm}5.4.2  $t_{iy + ix} \leftarrow \hat r \mbox{ (mod }\beta\mbox{)}$ \\
\hspace{6mm}5.4.3  $u \leftarrow \lfloor \hat r / \beta \rfloor$ \\
\hspace{3mm}5.5  if $ix + iy < digs$ then do \\
\hspace{6mm}5.5.1  $t_{ix + pb} \leftarrow u$ \\
6.  Clamp excess digits of $t$. \\
7.  Swap $c$ with $t$ \\
8.  Clear $t$ \\
9.  Return(\textit{MP\_OKAY}). \\
\hline
\end{tabular}
\end{center}
\end{small}
\caption{Algorithm s\_mp\_mul\_digs}
\end{figure}

\textbf{Algorithm s\_mp\_mul\_digs.}
This algorithm computes the unsigned product of two inputs $a$ and $c$ limited to an output precision of $digs$ digits.  While it may seem
a bit awkward to modify the function from its simple $O(n^2)$ description the usefulness of partial multipliers will arise in a future 
algorithm.  The algorithm is loosely based on algorithm 14.12 from \cite[pp. 595]{HAC} and is similar to Algorithm M \cite[pp. 268]{TAOCPV2}.  The
algorithm differs from those cited references because it can produce a variable output precision regardless of the precision of the inputs.

The first thing this algorithm checks for is whether a Comba multiplier can be used instead.   That is if the minimal digit count of either
input is less than $\delta$ the Comba method is used.    After the Comba method is ruled out the baseline algorithm begins.  A 
temporary mp\_int variable $t$ is used to hold the intermediate result of the product.  This allows the algorithm to be used to 
compute products when either $a = c$ or $b = c$ without overwriting the inputs.  

All of step 5 is the infamous $O(n^2)$ multiplication loop slightly modified to only produce upto $digs$ digits of output.  The $pb$ variable
is given the count of digits to read from $b$ inside the nested loop.  If $pb < 0$ then no more output digits can be produced and the algorithm
will exit the loop.  The best way to think of the loops are as a series of $pb \times 1$ multiplication.    That is, in each pass of the 
innermost loop $a_{ix}$ is multiplied against $b$ and the result is added (\textit{with an appropriate shift}) to $t$.  

For example, consider multiplying $576$ by $241$.  That is equivalent to computing $10^0(1)(576) + 10^1(4)(576) + 10^2(2)(576)$ which is best
visualized as the following table.

\begin{figure}[here]
\begin{center}
\begin{tabular}{|c|c|c|c|c|c|c|}
\hline   &&          & 5 & 7 & 6 & \\
\hline   $\times$&&  & 2 & 4 & 1 & \\
\hline &&&&&&\\
  &&          & 5 & 7 & 6 & $10^0(1)(576)$ \\
  &2 &   3    & 0 & 4 & 0 & $10^1(4)(576)$ \\
  1 & 1 & 5 & 2 & 0 & 0 &  $10^2(2)(576)$ \\
\hline  
\end{tabular}
\end{center}
\caption{Long-Hand Multiplication Diagram}
\end{figure}

Each row of the product is added to the result after being shifted to the left (\textit{multiplied by a power of the radix}) by the appropriate 
count.  That is in pass $ix$ of the inner loop the product is added starting at the $ix$'th digit of the reult.

Step 5.4.1 introduces the hat symbol (\textit{e.g. $\hat x$}) which represents a double precision variable.  The multiplication on that step
is assumed to be a double wide output single precision multiplication.  That is, two single precision variables are multiplied to produce a
double precision result.  The step is somewhat optimized from a long-hand multiplication algorithm because the carry from the addition in step
5.4.1 is forwarded through the nested loop.  If the carry was ignored it would overflow the single precision digit $t_{ix+iy}$ and the result
would be lost.  

At step 5.5 the nested loop is finished and any carry that was left over should be forwarded.  That is provided $ix + iy < digs$ otherwise the
carry is ignored since it will not be part of the result anyways.  

\index{bn\_s\_mp\_mul\_digs.c}
\vspace{+3mm}\begin{small}
\hspace{-5.1mm}{\bf File}: bn\_s\_mp\_mul\_digs.c
\vspace{-3mm}
\begin{alltt}
016   
017   /* multiplies |a| * |b| and only computes upto digs digits of result
018    * HAC pp. 595, Algorithm 14.12  Modified so you can control how 
019    * many digits of output are created.
020    */
021   int
022   s_mp_mul_digs (mp_int * a, mp_int * b, mp_int * c, int digs)
023   \{
024     mp_int  t;
025     int     res, pa, pb, ix, iy;
026     mp_digit u;
027     mp_word r;
028     mp_digit tmpx, *tmpt, *tmpy;
029   
030     /* can we use the fast multiplier? */
031     if (((digs) < MP_WARRAY) &&
032         MIN (a->used, b->used) < 
033             (1 << ((CHAR_BIT * sizeof (mp_word)) - (2 * DIGIT_BIT)))) \{
034       return fast_s_mp_mul_digs (a, b, c, digs);
035     \}
036   
037     if ((res = mp_init_size (&t, digs)) != MP_OKAY) \{
038       return res;
039     \}
040     t.used = digs;
041   
042     /* compute the digits of the product directly */
043     pa = a->used;
044     for (ix = 0; ix < pa; ix++) \{
045       /* set the carry to zero */
046       u = 0;
047   
048       /* limit ourselves to making digs digits of output */
049       pb = MIN (b->used, digs - ix);
050   
051       /* setup some aliases */
052       /* copy of the digit from a used within the nested loop */
053       tmpx = a->dp[ix];
054       
055       /* an alias for the destination shifted ix places */
056       tmpt = t.dp + ix;
057       
058       /* an alias for the digits of b */
059       tmpy = b->dp;
060   
061       /* compute the columns of the output and propagate the carry */
062       for (iy = 0; iy < pb; iy++) \{
063         /* compute the column as a mp_word */
064         r = ((mp_word) *tmpt) + 
065             ((mp_word) tmpx) * ((mp_word) * tmpy++) + 
066             ((mp_word) u);
067   
068         /* the new column is the lower part of the result */
069         *tmpt++ = (mp_digit) (r & ((mp_word) MP_MASK));
070   
071         /* get the carry word from the result */
072         u = (mp_digit) (r >> ((mp_word) DIGIT_BIT));
073       \}
074       /* set carry if it is placed below digs */
075       if (ix + iy < digs) \{
076         *tmpt = u;
077       \}
078     \}
079   
080     mp_clamp (&t);
081     mp_exch (&t, c);
082   
083     mp_clear (&t);
084     return MP_OKAY;
085   \}
\end{alltt}
\end{small}

Lines 31 to 35 determine if the Comba method can be used first.  The conditions for using the Comba routine are that min$(a.used, b.used) < \delta$ and
the number of digits of output is less than \textbf{MP\_WARRAY}.  This new constant is used to control the stack usage in the Comba routines.  By
default it is set to $\delta$ but can be reduced when memory is at a premium.

Of particular importance is the calculation of the $ix+iy$'th column on lines 64, 65 and 66.  Note how all of the
variables are cast to the type \textbf{mp\_word}.  That is to ensure that double precision operations are used instead of single precision.  The
multiplication on line 65 is a bit of a GCC optimization.  On the outset it looks like the compiler will have to use a double precision
multiplication to produce the result required.  Such an operation would be horribly slow on most processors and drag this to a crawl.  However,
GCC is smart enough to realize that double wide output single precision multipliers can be used.  For example, the instruction ``MUL'' on the
x86 processor can multiply two 32-bit values and produce a 64-bit result.  

\subsection{Faster Multiplication by the ``Comba'' Method}

One of the huge drawbacks of the ``baseline'' algorithms is that at the $O(n^2)$ level the carry must be computed and propagated upwards.  This
makes the nested loop very sequential and hard to unroll and implement in parallel.  The ``Comba'' method is named after little known 
(\textit{in cryptographic venues}) Paul G. Comba where in \cite{COMBA} a method of implementing fast multipliers that do not require nested 
carry fixup operations was presented.

At the heart of algorithm is once again the long-hand algorithm for multiplication.  Except in this case a slight twist is placed on how
the columns of the result are produced.  In the standard long-hand algorithm rows of products are produced then added together to form the 
final result.  In the baseline algorithm the columns are added together to get the result instantaneously.  

In the Comba algorithm however, the columns of the result are produced entirely independently of each other.  That is at the $O(n^2)$ level a 
simple multiplication and addition step is performed.  Or more succintly that 

\begin{equation}
x_n = \sum_{i+j = n} a_ib_j
\end{equation}

Where $x_n$ is the $n'th$ column of the output vector.  To see how this works consider once again multiplying $576$ by $241$.  

\begin{figure}[here]
\begin{small}
\begin{center}
\begin{tabular}{|c|c|c|c|c|c|}
  \hline &          & 5 & 7 & 6 & First Input\\
  \hline $\times$ & & 2 & 4 & 1 & Second Input\\
\hline            &                        & $1 \cdot 5 = 5$   & $1 \cdot 7 = 7$   & $1 \cdot 6 = 6$ & First pass \\
                  &  $4 \cdot 5 = 20$      & $4 \cdot 7+5=33$  & $4 \cdot 6+7=31$  & 6               & Second pass \\
   $2 \cdot 5 = 10$ &  $2 \cdot 7 + 20 = 34$ & $2 \cdot 6+33=45$ & 31                & 6             & Third pass \\
\hline 10 & 34 & 45 & 31 & 6 & Final Result \\   
\hline   
\end{tabular}
\end{center}
\end{small}
\caption{Comba Multiplication Diagram}
\end{figure}

At this point the vector $x = \left < 10, 34, 45, 31, 6 \right >$ is the result of the first step of the Comba multipler.  
Now the columns must be fixed by propagating the carry upwards.  The following trivial algorithm will accomplish this.

\begin{enumerate}
    \item for $n$ from 0 to $k - 1$ do
    \item \hspace{3mm} $x_{n+1} \leftarrow x_{n+1} + \lfloor x_{n}/\beta \rfloor$ 
    \item \hspace{3mm} $x_{n} \leftarrow x_{n} \mbox{ (mod }\beta\mbox{)}$
\end{enumerate}

With that algorithm and $k = 5$ and $\beta = 10$ the following vector is produced $y = \left < 1, 3, 8, 8, 1, 6 \right >$.  In this case 
$241 \cdot 576$ is in fact $138816$ and the procedure succeeded.  If the algorithm is correct and as will be demonstrated shortly more
efficient than the baseline algorithm why not simply always use this algorithm?

\subsubsection{Column Weight.}
At the nested $O(n^2)$ level the Comba method adds the product of two single precision variables to a each column of the output 
independently.  A serious obstacle is if the carry is lost due to lack of precision before the algorithm has a chance to fix
the carries.  For example, in the multiplication of two three-digit numbers the third column of output will be the sum of
three single precision multiplications.  If the precision of the accumulator for the output digits is less then $3 \cdot (\beta - 1)^2$ then
an overflow can occur and the carry information will be lost.  For any $m$ and $n$ digit input the maximal weight of any column is 
min$(m, n)$ which is fairly obvious.

The maximal number of terms in any column of a product is known as the ``column weight'' and strictly governs when the algorithm can be used.  Recall
from earlier that a double precision type has $\alpha$ bits of resolution and a single precision digit has $lg(\beta)$ bits of precision.  Given these
two quantities we may not violate the following

\begin{equation}
k \cdot \left (\beta - 1 \right )^2 < 2^{\alpha}
\end{equation}

Which reduces to 

\begin{equation}
k \cdot \left ( \beta^2 - 2\beta + 1 \right ) < 2^{\alpha}
\end{equation}

Let $\rho = lg(\beta)$ represent the number of bits in a single precision digit.  By further re-arrangement of the equation the final solution is
found.

\begin{equation}
k \cdot \left (2^{2\rho} - 2^{\rho + 1} + 1 \right ) < 2^{\alpha}
\end{equation}

The defaults for LibTomMath are $\beta = 2^{28}, \alpha = 2^{64}$ which simplies to $72057593501057025 \cdot k < 2^{64}$ which when divided out
result in $k < 257$.  This implies that the smallest input may not have more than $256$ digits if the Comba method is to be used in
this configuration.  This is quite satisfactory for most applications since $256$ digits would be allow for numbers in the range of $2^{7168}$ 
which is much larger than the typical $2^{100}$ to $2^{4000}$ range most public key cryptographic algorithms use.  

\newpage\begin{figure}[!here]
\begin{small}
\begin{center}
\begin{tabular}{l}
\hline Algorithm \textbf{fast\_s\_mp\_mul\_digs}. \\
\textbf{Input}.   mp\_int $a$, mp\_int $b$ and an integer $digs$ \\
\textbf{Output}.  $c \leftarrow \vert a \vert \cdot \vert b \vert \mbox{ (mod }\beta^{digs}\mbox{)}$. \\
\hline \\
Place an array of \textbf{MP\_WARRAY} double precision digits named $\hat W$ on the stack. \\
1.  If $c.alloc < digs$ then grow $c$ to $digs$ digits. (\textit{hint: use mp\_grow}) \\
2.  If step 1 failed return(\textit{MP\_MEM}).\\
\\
Zero the temporary array $\hat W$. \\
3.  for $n$ from $0$ to $digs - 1$ do \\
\hspace{3mm}3.1  $\hat W_n \leftarrow 0$ \\
\\
Compute the columns. \\
4.  for $ix$ from $0$ to $a.used - 1$ do \\
\hspace{3mm}4.1  $pb \leftarrow \mbox{min}(b.used, digs - ix)$ \\
\hspace{3mm}4.2  If $pb < 1$ then goto step 5. \\
\hspace{3mm}4.3  for $iy$ from $0$ to $pb - 1$ do \\
\hspace{6mm}4.3.1  $\hat W_{ix+iy} \leftarrow \hat W_{ix+iy} + a_{ix}b_{iy}$ \\
\\
Propagate the carries upwards. \\
5.  $oldused \leftarrow c.used$ \\
6.  $c.used \leftarrow digs$ \\
7.  If $digs > 1$ then do \\
\hspace{3mm}7.1.  for $ix$ from $1$ to $digs - 1$ do \\
\hspace{6mm}7.1.1  $\hat W_{ix} \leftarrow \hat W_{ix} + \lfloor \hat W_{ix-1} / \beta \rfloor$ \\
\hspace{6mm}7.1.2  $c_{ix - 1} \leftarrow \hat W_{ix - 1} \mbox{ (mod }\beta\mbox{)}$ \\
8.  else do \\
\hspace{3mm}8.1  $ix \leftarrow 0$ \\
9.  $c_{ix} \leftarrow \hat W_{ix} \mbox{ (mod }\beta\mbox{)}$ \\
\\
Zero excess digits. \\
10.  If $digs < oldused$ then do \\
\hspace{3mm}10.1  for $n$ from $digs$ to $oldused - 1$ do \\
\hspace{6mm}10.1.1  $c_n \leftarrow 0$ \\
11.  Clamp excessive digits of $c$.  (\textit{hint: use mp\_clamp}) \\
12.  Return(\textit{MP\_OKAY}). \\
\hline
\end{tabular}
\end{center}
\end{small}
\caption{Algorithm fast\_s\_mp\_mul\_digs}
\end{figure}

\textbf{Algorithm fast\_s\_mp\_mul\_digs.}
This algorithm performs the unsigned multiplication of $a$ and $b$ using the Comba method limited to $digs$ digits of precision.  The algorithm
essentially peforms the same calculation as algorithm s\_mp\_mul\_digs but much faster.

The array $\hat W$ is meant to be on the stack when the algorithm is used.  The size of the array does not change which is ideal.  Note also that 
unlike algorithm s\_mp\_mul\_digs no temporary mp\_int is required since the result is calculated in place in $\hat W$.  

The $O(n^2)$ loop on step four is where the Comba method starts to show through.  First there is no carry variable in the loop.  Second the
double precision multiply and add step does not have a carry fixup of any sort.  In fact the nested loop is very simple and can be implemented
in parallel.  

What makes the Comba method so attractive is that the carry propagation only takes place outside the $O(n^2)$ nested loop.  For example, if the 
cost in terms of time of a multiply and add is $p$ and the cost of a carry propagation is $q$ then a baseline multiplication would require 
$O \left ((p + q)n^2 \right )$ time to multiply two $n$-digit numbers.  The Comba method only requires $pn^2 + qn$ time, however, in practice 
the speed increase is actually much more.  With $O(n)$ space the algorithm can be reduced to $O(pn + qn)$ time by implementing the $n$ multiply
and add operations in the nested loop in parallel.  

The carry propagation loop on step 7 is fairly straightforward.  It could have been written phased the other direction, that is, to assign
to $c_{ix}$ instead of $c_{ix-1}$ in each iteration.  However, it would still require pre-caution to make sure that $\hat W_{ix+1}$ is not beyond
the \textbf{MP\_WARRAY} words set aside.  

\index{bn\_fast\_s\_mp\_mul\_digs.c}
\vspace{+3mm}\begin{small}
\hspace{-5.1mm}{\bf File}: bn\_fast\_s\_mp\_mul\_digs.c
\vspace{-3mm}
\begin{alltt}
016   
017   /* Fast (comba) multiplier
018    *
019    * This is the fast column-array [comba] multiplier.  It is 
020    * designed to compute the columns of the product first 
021    * then handle the carries afterwards.  This has the effect 
022    * of making the nested loops that compute the columns very
023    * simple and schedulable on super-scalar processors.
024    *
025    * This has been modified to produce a variable number of 
026    * digits of output so if say only a half-product is required 
027    * you don't have to compute the upper half (a feature 
028    * required for fast Barrett reduction).
029    *
030    * Based on Algorithm 14.12 on pp.595 of HAC.
031    *
032    */
033   int
034   fast_s_mp_mul_digs (mp_int * a, mp_int * b, mp_int * c, int digs)
035   \{
036     int     olduse, res, pa, ix;
037     mp_word W[MP_WARRAY];
038   
039     /* grow the destination as required */
040     if (c->alloc < digs) \{
041       if ((res = mp_grow (c, digs)) != MP_OKAY) \{
042         return res;
043       \}
044     \}
045   
046     /* clear temp buf (the columns) */
047     memset (W, 0, sizeof (mp_word) * digs);
048   
049     /* calculate the columns */
050     pa = a->used;
051     for (ix = 0; ix < pa; ix++) \{
052       /* this multiplier has been modified to allow you to 
053        * control how many digits of output are produced.  
054        * So at most we want to make upto "digs" digits of output.
055        *
056        * this adds products to distinct columns (at ix+iy) of W
057        * note that each step through the loop is not dependent on
058        * the previous which means the compiler can easily unroll
059        * the loop without scheduling problems
060        */
061       \{
062         register mp_digit tmpx, *tmpy;
063         register mp_word *_W;
064         register int iy, pb;
065   
066         /* alias for the the word on the left e.g. A[ix] * A[iy] */
067         tmpx = a->dp[ix];
068   
069         /* alias for the right side */
070         tmpy = b->dp;
071   
072         /* alias for the columns, each step through the loop adds a new
073            term to each column
074          */
075         _W = W + ix;
076   
077         /* the number of digits is limited by their placement.  E.g.
078            we avoid multiplying digits that will end up above the # of
079            digits of precision requested
080          */
081         pb = MIN (b->used, digs - ix);
082   
083         for (iy = 0; iy < pb; iy++) \{
084           *_W++ += ((mp_word) tmpx) * ((mp_word) * tmpy++);
085         \}
086       \}
087   
088     \}
089   
090     /* setup dest */
091     olduse = c->used;
092     c->used = digs;
093   
094     \{
095       register mp_digit *tmpc;
096   
097       /* At this point W[] contains the sums of each column.  To get the
098        * correct result we must take the extra bits from each column and
099        * carry them down
100        *
101        * Note that while this adds extra code to the multiplier it 
102        * saves time since the carry propagation is removed from the 
103        * above nested loop.This has the effect of reducing the work 
104        * from N*(N+N*c)==N**2 + c*N**2 to N**2 + N*c where c is the 
105        * cost of the shifting.  On very small numbers this is slower 
106        * but on most cryptographic size numbers it is faster.
107        */
108       tmpc = c->dp;
109       for (ix = 1; ix < digs; ix++) \{
110         W[ix] += (W[ix - 1] >> ((mp_word) DIGIT_BIT));
111         *tmpc++ = (mp_digit) (W[ix - 1] & ((mp_word) MP_MASK));
112       \}
113       *tmpc++ = (mp_digit) (W[digs - 1] & ((mp_word) MP_MASK));
114   
115       /* clear unused */
116       for (; ix < olduse; ix++) \{
117         *tmpc++ = 0;
118       \}
119     \}
120   
121     mp_clamp (c);
122     return MP_OKAY;
123   \}
\end{alltt}
\end{small}

The memset on line 47 clears the initial $\hat W$ array to zero in a single step. Like the slower baseline multiplication
implementation a series of aliases (\textit{lines 67, 70 and 75}) are used to simplify the inner $O(n^2)$ loop.  
In this case a new alias $\_\hat W$ has been added which refers to the double precision columns offset by $ix$ in each pass.  

The inner loop on line 84 is where the algorithm will spend the majority of the time.  Which is why it has been stripped to the 
bones of any extra baggage\footnote{Hence the pointer aliases.}.  On x86 processors the multiply and add amounts to at the very least five
instructions (\textit{two loads, two additions, one multiply}) while on the ARMv4 processors it amounts to only three (\textit{one load, one store,
one multiply-add}).   On both the x86 and ARMv4 processors GCC v3.2 does a very good job at unrolling the loop and scheduling it so there 
are very few dependency stalls.

In theory the difference between the baseline and comba algorithms is a mere $O(qn)$ time difference.  However, in the $O(n^2)$ nested loop of the
baseline method there are dependency stalls as the algorithm must wait for the multiplier to finish before propagating the carry to the next 
digit.  As a result fewer of the often multiple execution units\footnote{The AMD Athlon has three execution units and the Intel P4 has four.} can
be simultaneously used.  

\subsection{Multiplication at New Bounds by Karatsuba Method}
So far two methods of multiplication have been presented.  Both of the algorithms require asymptotically $O(n^2)$ time to multiply two $n$-digit 
numbers together.  While the Comba method is much faster than the baseline algorithm it still requires far too much time to multiply 
large inputs together.  In fact it was not until \cite{KARA} in 1962 that a faster algorithm had been proposed at all.

The idea behind Karatsubas method is that an input can be represented in polynomial basis as two halves then multiplied.  For example, if 
$f(x) = ax + b$ and $g(x) = cx + b$ then the product of the two polynomials $h(x) = f(x)g(x)$ will allow $h(\beta) = (f(\beta))(g(\beta))$.  

So how does this help?  First expand the product $h(x)$.

\begin{center}
\begin{tabular}{rcl}
$h(x)$ & $=$ & $f(x)g(x)$ \\
       & $=$ & $(ax + b)(cx + d)$ \\
       & $=$ & $acx^2 + adx + bcx + bd$ \\
\end{tabular}
\end{center}

The next equation is a bit of genius on the part of Karatsuba.  He proved that the previous equation is equivalent to 

\begin{equation}
h(x) = acx^2 + ((a - c)(b - d) + bd + ac)x + bd
\end{equation}

Essentially the proof lies in some fairly light algebraic number theory (\textit{see \cite{KARAP} for details}) that is not important for
the discussion.  At first glance it appears that the Karatsuba method is actually harder than the straight $O(n^2)$ approach.  
However, further investigation will prove otherwise.  

The first important observation is that both $f(x)$ and $g(x)$ are the polynomial basis representation of two-digit numbers.  This means that 
$\left < a, b, c, d \right >$ are single digit values.  Using either the baseline or straight polynomial multiplication the old method requires
$O \left (4(n/2)^2 \right ) = O(n^2)$ single precision multiplications.  Looking closer at Karatsubas equation there are only three unique multiplications 
required which are $ac$, $bd$ and $(a - c)(b - d)$.  As a result only $O \left (3 \cdot (n/2)^2 \right ) = O \left ( {3 \over 4}n^2 \right )$ 
multiplications are required.  

So far the algorithm has been discussed from the point of view of ``two-digit'' numbers.  However, there is no reason why two digits implies a range of 
$\beta^2$.  It could just as easily represent a range of $\left (\beta^k \right)^2$ as well.  For example, the polynomial 
$f(x) = a_3x^3 + a_2x^2 + a_1x + a_0$ could also be written as $f'(x) = a'_1x + a'_0$ where $f(\beta) = f'(\beta^2)$.  Fortunately representing an
integer which is already in an array of radix-$\beta$ digits in polynomial basis in terms of a power of $\beta$ is very simple.  

\subsubsection{Recursion}
The Karatsuba multiplication algorithm can be applied to practically any size of input.  Therefore, it is possible that the Karatsuba method itself
be used for the three multiplications required.  For example, when multiplying two four-digit numbers there will be three multiplications of two-digit
numbers.  In this case the smaller multiplication requires $p(n) = {3 \over 4}n^2$ time to complete while the larger multiplication requires
$q(n) = 3 \cdot p(n/2)$ multiplications.  

By expanding $q(n)$ the following equation is achieved. 

\begin{center}
\begin{tabular}{rcl}
$q(n)$ & $=$ & $3 \cdot p(n/2)$ \\
       & $=$ & $3 \cdot (3 \cdot ((n/2)/2)^2)$ \\
       & $=$ & $9 \cdot (n/4)^2$ \\
       & $=$ & ${9 \over 16}n^2$ \\
\end{tabular}
\end{center}

The generic expression for the multiplicand is simply $\left ( {3 \over 4} \right )^k$ for $k \ge 1$ recurisions.  The maximal number of recursions
is approximately $lg(n)$.  Putting this all in terms of a base $n$ logarithm the asymptotic running time can be deduced.

\begin{center}
\begin{tabular}{rcl}
$lg_n \left ( \left ( {3 \over 4} \right )^{lg_2 n} \cdot n^2 \right )$ & $=$ & $lg_2 n \cdot lg_n \left ( { 3 \over 4 } \right ) + 2$ \\
                                                                        & $=$ & $\left ( {log N \over log 2} \right ) \cdot \left ( {log \left ( {3 \over 4} \right ) \over log N } \right ) + 2$ \\
                                                                        & $=$ & ${ log 3 - log 2^2 + 2 \cdot log 2} \over log 2$ \\
                                                                        & $=$ & $log 3 \over log 2$ \\
\end{tabular}
\end{center}

Which leads to a running time of $O \left ( n^{lg(3)} \right )$ which is approximately $O(n^{1.584})$.  This can lead to 
impressive savings with fairly moderate sized numbers.  For example, when multiplying two 128-digit numbers the Karatsuba 
method saves $14,197$ (\textit{or $86\%$ of the total}) single precision multiplications.  

The immediate question becomes why not simply use Karatsuba multiplication all the time and forget about the baseline and Comba algorithms? 

\subsubsection{Overhead}
While the Karatsuba method saves on the number of single precision multiplications required this savings is not entirely free.  The product
of three half size products must be stored somewhere as well as four additions and two subtractions performed.  These operations incur sufficient
overhead that often for fairly trivial sized inputs the Karatsuba method is slower.

\index{cutoff point}
The \textit{cutoff point} for Karatsuba multiplication is the point at which the Karatsuba multiplication and baseline (\textit{or Comba}) meet.  
For the purposes of this discussion call this value $x$.  For any input with $n$ digits such that $n < x$ Karatsuba multiplication will be slower 
and for $n > x$ it will be faster.  Often the break between the two algorithms is not so clean cut in reality.  The cleaner the cut the more 
efficient multiplication will be which is why tuning the multiplication is a very important process.  For example, a properly tuned Karatsuba 
multiplication algorithm can multiply two $4,096$ bit numbers up to five times faster on an Athlon processor compared to the standard baseline
algorithm.  

The exact placement of the value of $x$ depends on several key factors.   The cost of allocating storage for the temporary variables, the cost of 
performing the additions and most importantly the cost of performing a single precision multiplication.  With a processor where single precision 
multiplication is fast\footnote{The AMD Athlon for instance has a six cycle multiplier compared to the Intel P4 which has a 15 cycle multiplier.} the 
cutoff point will move upwards.  Similarly with a slower processor the cutoff point will move downwards.  

\newpage\begin{figure}[!here]
\begin{small}
\begin{center}
\begin{tabular}{l}
\hline Algorithm \textbf{mp\_karatsuba\_mul}. \\
\textbf{Input}.   mp\_int $a$ and mp\_int $b$ \\
\textbf{Output}.  $c \leftarrow \vert a \vert \cdot \vert b \vert$ \\
\hline \\
1.  $B \leftarrow \mbox{min}(a.used, b.used)/2$ \\
2.  Init the following mp\_int variables: $x0$, $x1$, $y0$, $y1$, $t1$, $x0y0$, $x1y1$.\\
3.  If step 2 failed then return(\textit{MP\_MEM}). \\
\\
Split the input.  e.g. $a = x1 \cdot \beta^B + x0$ \\
4.  $x0 \leftarrow a \mbox{ (mod }\beta^B\mbox{)}$ (\textit{hint: use mp\_mod\_2d}) \\
5.  $y0 \leftarrow b \mbox{ (mod }\beta^B\mbox{)}$ \\
6.  $x1 \leftarrow \lfloor a / \beta^B \rfloor$ (\textit{hint: use mp\_rshd}) \\
7.  $y1 \leftarrow \lfloor b / \beta^B \rfloor$ \\
\\
Calculate the three products. \\
8.  $x0y0 \leftarrow x0 \cdot y0$ (\textit{hint: use mp\_mul}) \\
9.  $x1y1 \leftarrow x1 \cdot y1$ \\
10.  $t1 \leftarrow x1 - x0$ (\textit{hint: use mp\_sub}) \\
11.  $x0 \leftarrow y1 - y0$ \\
12.  $t1 \leftarrow t1 \cdot x0$ \\
\\
Calculate the middle term. \\
13.  $x0 \leftarrow x0y0 + x1y1$ \\
14.  $t1 \leftarrow x0 - t1$ \\
\\
Calculate the final product. \\
15.  $t1 \leftarrow t1 \cdot \beta^B$ (\textit{hint: use mp\_lshd}) \\
16.  $x1y1 \leftarrow x1y1 \cdot \beta^{2B}$ \\
17.  $t1 \leftarrow x0y0 + t1$ \\
18.  $c \leftarrow t1 + x1y1$ \\
19.  Clear all of the temporary variables. \\
20.  Return(\textit{MP\_OKAY}).\\
\hline 
\end{tabular}
\end{center}
\end{small}
\caption{Algorithm mp\_karatsuba\_mul}
\end{figure}

\textbf{Algorithm mp\_karatsuba\_mul.}


\section{Squaring}
\subsection{The Baseline Squaring Algorithm}
\subsection{Faster Squaring by the ``Comba'' Method}
\subsection{Karatsuba Squaring}
\section{Tuning Algorithms}
\subsection{How to Tune Karatsuba Algorithms}

\chapter{Modular Reductions}
\section{Basics of Modular Reduction}
\section{The Barrett Reduction}
\section{The Montgomery Reduction}
\subsection{Faster ``Comba'' Montgomery Reduction}
\subsection{Example Montgomery Algorithms}
\section{The Diminished Radix Algorithm}
\section{Algorithm Comparison}

\chapter{Exponentiation}
\section{Single Digit Exponentiation}
\section{Modular Exponentiation}
\subsection{General Case}
\subsection{Odd or Diminished Radix Moduli}
\section{Quick Power of Two}

\chapter{Higher Level Algorithms}
\section{Integer Division with Remainder}
\section{Single Digit Helpers}
\subsection{Single Digit Addition}
\subsection{Single Digit Subtraction}
\subsection{Single Digit Multiplication}
\subsection{Single Digit Division}
\subsection{Single Digit Modulo}
\subsection{Single Digit Root Extraction}
\section{Random Number Generation}
\section{Formatted Output}
\subsection{Getting The Output Size}
\subsection{Generating Radix-n Output}
\subsection{Reading Radix-n Input}
\section{Unformatted Output}
\subsection{Getting The Output Size}
\subsection{Generating Output}
\subsection{Reading Input}

\chapter{Number Theoretic Algorithms}
\section{Greatest Common Divisor}
\section{Least Common Multiple}
\section{Jacobi Symbol Computation}
\section{Modular Inverse}
\subsection{General Case}
\subsection{Odd Moduli}
\section{Primality Tests}
\subsection{Trial Division}
\subsection{The Fermat Test}
\subsection{The Miller-Rabin Test}
\subsection{Primality Test in a Bottle}
\subsection{The Next Prime}
\section{Root Extraction}

\backmatter
\appendix
\begin{thebibliography}{ABCDEF}
\bibitem[1]{TAOCPV2}
Donald Knuth, \textit{The Art of Computer Programming}, Third Edition, Volume Two, Seminumerical Algorithms, Addison-Wesley, 1998

\bibitem[2]{HAC}
A. Menezes, P. van Oorschot, S. Vanstone, \textit{Handbook of Applied Cryptography}, CRC Press, 1996

\bibitem[3]{ROSE}
Michael Rosing, \textit{Implementing Elliptic Curve Cryptography}, Manning Publications, 1999

\bibitem[4]{COMBA}
Paul G. Comba, \textit{Exponentiation Cryptosystems on the IBM PC}. IBM Systems Journal 29(4): 526-538 (1990)

\bibitem[5]{KARA}
A. Karatsuba, Doklay Akad. Nauk SSSR 145 (1962), pp.293-294

\bibitem[6]{KARAP}
Andre Weimerskirch and Christof Paar, \textit{Generalizations of the Karatsuba Algorithm for Polynomial Multiplication}, Submitted to Design, Codes and Cryptography, March 2002

\end{thebibliography}

\documentclass[b5paper]{book}
\usepackage{makeidx}
\usepackage{amssymb}
\usepackage{color}
\usepackage{alltt}
\usepackage{graphicx}
\usepackage{layout}
\def\union{\cup}
\def\intersect{\cap}
\def\getsrandom{\stackrel{\rm R}{\gets}}
\def\cross{\times}
\def\cat{\hspace{0.5em} \| \hspace{0.5em}}
\def\catn{$\|$}
\def\divides{\hspace{0.3em} | \hspace{0.3em}}
\def\nequiv{\not\equiv}
\def\approx{\raisebox{0.2ex}{\mbox{\small $\sim$}}}
\def\lcm{{\rm lcm}}
\def\gcd{{\rm gcd}}
\def\log{{\rm log}}
\def\ord{{\rm ord}}
\def\abs{{\mathit abs}}
\def\rep{{\mathit rep}}
\def\mod{{\mathit\ mod\ }}
\renewcommand{\pmod}[1]{\ ({\rm mod\ }{#1})}
\newcommand{\floor}[1]{\left\lfloor{#1}\right\rfloor}
\newcommand{\ceil}[1]{\left\lceil{#1}\right\rceil}
\def\Or{{\rm\ or\ }}
\def\And{{\rm\ and\ }}
\def\iff{\hspace{1em}\Longleftrightarrow\hspace{1em}}
\def\implies{\Rightarrow}
\def\undefined{{\rm ``undefined"}}
\def\Proof{\vspace{1ex}\noindent {\bf Proof:}\hspace{1em}}
\let\oldphi\phi
\def\phi{\varphi}
\def\Pr{{\rm Pr}}
\newcommand{\str}[1]{{\mathbf{#1}}}
\def\F{{\mathbb F}}
\def\N{{\mathbb N}}
\def\Z{{\mathbb Z}}
\def\R{{\mathbb R}}
\def\C{{\mathbb C}}
\def\Q{{\mathbb Q}}
\definecolor{DGray}{gray}{0.5}
\newcommand{\url}[1]{\mbox{$<${#1}$>$}}
\newcommand{\emailaddr}[1]{\mbox{$<${#1}$>$}}
\def\twiddle{\raisebox{0.3ex}{\mbox{\tiny $\sim$}}}
\def\gap{\vspace{0.5ex}}
\makeindex
\begin{document}
\frontmatter
\pagestyle{empty}
\title{Multiple-Precision Integer Arithmetic, \\ A Case Study Involving the LibTomMath Project \\ - DRAFT - }
\author{\mbox{
%\begin{small}
\begin{tabular}{c}
Tom St Denis \\
Algonquin College \\
\\
Mads Rasmussen \\
Open Communications Security \\
\\
Gregory Rose \\
Qualcomm \\
\end{tabular}
%\end{small}
}
}
\maketitle
This text in its entirety is copyrighted \copyright{}2003 by Tom St Denis.  It may not be redistributed 
electronically or otherwise without the sole permission of the author.  The text is freely re distributable as long as
it is packaged along with the LibTomMath project in a non-commercial project.  Contact the
author for other redistribution rights.

This text corresponds to the v0.17 release of the LibTomMath project.

\begin{alltt}
Tom St Denis
111 Banning Rd
Ottawa, Ontario
K2L 1C3
Canada

Phone: 1-613-836-3160
Email: tomstdenis@iahu.ca
\end{alltt}

This text is formatted to the international B5 paper size of 176mm wide by 250mm tall using the \LaTeX{} 
{\em book} macro package and the Perl {\em booker} package.

\tableofcontents
\listoffigures
\chapter*{Preface}
Blah.

\mainmatter
\pagestyle{headings}
\chapter{Introduction}
\section{Multiple Precision Arithmetic}
\subsection{The Need for Multiple Precision Arithmetic}
The most prevalent use for multiple precision arithmetic (\textit{often referred to as bignum math}) is within public
key cryptography.   Algorithms such as RSA, Diffie-Hellman and Elliptic Curve Cryptography require large integers in order to 
resist known cryptanalytic attacks.  Typical modern programming languages such as C and Java only provide small 
single-precision data types which are incapable of precisely representing integers which are often hundreds of bits long.

For example, consider multiplying $1,234,567$ by $9,876,543$ in C with an ``unsigned long'' data type.  With an 
x86 machine the result is $4,136,875,833$ while the true result is $12,193,254,061,881$.  The original inputs 
were approximately $21$ and $24$ bits respectively.  If the C language cannot multiply two relatively small values 
together precisely how does anyone expect it to multiply two values which are considerably larger?

Most advancements in fast multiple precision arithmetic stems from the desire for faster cryptographic primitives.  However, cryptography
is not the only field of study that can benefit fast large integer routines.  Another auxiliary use for multiple precision integers is 
high precision floating point data types.  The basic IEEE standard floating point type is made up of an integer mantissa $q$ and an exponent $e$.  
Numbers are given in the form $n = q \cdot b^e$ where $b = 2$ is convention.  Since IEEE is meant to be implemented in 
hardware the precision of the mantissa is often fairly small (\textit{roughly 23 bits}).  Since the mantissa is merely an 
integer a large multiple precision integer could be used.  In effect very high precision floating point arithmetic 
could be performed.  This would be useful where scientific applications must minimize the total output error over long simulations.  

\subsection{Multiple Precision Arithmetic}
\index{multiple precision}
Multiple precision arithmetic attempts to the solve the shortcomings of single precision data types such as those from
the C and Java programming languages.  In essence multiple precision arithmetic is a set of operations that can be 
performed on members of an algebraic group whose precision is not fixed.  The algorithms when implemented to be multiple
precision can allow a developer to work with any practical precision required.

Typically the arithmetic is performed over the ring of integers denoted by a $\Z$ and referred to casually as ``bignum'' 
routines.  However, it is possible to have rings of polynomials as well typically denoted by $\Z/p\Z \left [ X \right ]$ 
which could have variable precision (\textit{or degree}).  This text will discuss implementation of the former, however,
implementing polynomial basis routines should be relatively easy after reading this text.

\subsection{Benefits of Multiple Precision Arithmetic}
\index{precision} \index{accuracy}
Precision is defined loosely as the proximity to the real value a given representation is.  Accuracy is defined as the 
reproducibility of the result.  For example, the calculation $1/3 = 0.25$ is imprecise but can be accurate provided 
it is reproducible.

The benefit of multiple precision representations over single precision representations is that 
often no precision is lost while representing the result of an operation which requires excess precision.  For example, 
the multiplication of two $n$-bit integers requires at least $2n$ bits to represent the result.  A multiple precision 
system would augment the precision of the destination to accomodate the result while a single precision system would
truncate excess bits to maintain a fixed level of precision.

Multiple precision representations allow for the precision to be very high (\textit{if not exacting}) but at a cost of
modest computer resources.  The only reasonable case where a multiple precision system will lose precision is when
emulating a floating point data type.  However, with multiple precision integer arithmetic no precision is lost.

\subsection{Basis of Operations}
At the heart of all multiple precision integer operations are the ``long-hand'' algorithms we all learnt as children 
in grade school.  For example, to multiply $1,234$ by $981$ the student is not taught to memorize the times table for 
$1,234$ instead they are taught how to long-multiply.  That is to multiply each column using simple single digit 
multiplications and add the resulting products by column.  The representation that most are familiar with is known as 
decimal or formally as radix-10. A radix-$n$ representation simply means there are $n$ possible values per digit.  
For example, binary would be a radix-2 representation.

In essence computer based multiple precision arithmetic is very much the same.  The most notable difference is the usage
of a binary friendly radix.  That is to use a radix of the form $2^k$ where $k$ is typically the size of a machine 
register.  Also occasionally more optimal algorithms are used to perform certain operations such as multiplication and 
squaring instead of traditional long-hand algorithms.

\section{Purpose of This Text}
The purpose of this text is to instruct the reader regarding how to implement multiple precision algorithms.  That is 
to not only explain the core theoretical algorithms but also the various ``house keeping'' tasks that are neglected by
authors of other texts on the subject.  Texts such as Knuths' ``The Art of Computer Programming, vol 2.'' and the 
Handbook of Applied Cryptography (\textit{HAC}) give considerably detailed explanations of the theoretical aspects of 
the algorithms and very little regarding the practical aspects.  

That is how an algorithm is explained and how it is actually implemented are two very different 
realities.  For example, algorithm 14.7 on page 594 of HAC lists a relatively simple algorithm for performing multiple 
precision integer addition.  However, what the description lacks is any discussion concerning the fact that the two 
integer inputs may be of differing magnitudes.  Similarly the division routine (\textit{Algorithm 14.20, pp. 598}) 
does not discuss how to handle sign or handle the dividends decreasing magnitude in the main loop (\textit{Step \#3}).

As well as the numerous practical oversights both of the texts do not discuss several key optimal algorithms required 
such as ``Comba'' and Karatsuba multipliers and fast modular inversion.  These optimal algorithms are considerably
vital to achieve any form of useful performance in non-trivial applications.  

To solve this problem the focus of this text is on the practical aspects of implementing the algorithms that 
constitute a multiple precision integer package with light cursory discussions on the theoretical aspects.  As a case 
study the ``LibTomMath''\footnote{Available freely at http://math.libtomcrypt.org} package is used to demonstrate 
algorithms with implementations that have been field tested and work very well.

\section{Discussion and Notation}
\subsection{Notation}
A multiple precision integer of $n$-digits shall be denoted as $x = (x_n ... x_1 x_0)_{ \beta }$ to be the 
multiple precision notation for the integer $x \equiv \sum_{i=0}^{n} x_i\beta^i$.  The elements of the array $x$ are
said to be the radix $\beta$ digits of the integer.  For example, $x = (15,0,7)_{\beta}$ would represent the 
integer $15\cdot\beta^2 + 0\cdot\beta^1 + 7\cdot\beta^0$.  

A ``mp\_int'' shall refer to a composite structure which contains the digits of the integer as well as auxilary data
required to manipulate the data.  These additional members are discussed in chapter three.  For the purposes of this text
a ``multiple precision integer'' and a ``mp\_int'' are synonymous.

\index{single-precision} \index{double-precision} \index{mp\_digit} \index{mp\_word}
For the purposes of this text a single-precision variable must be able to represent integers in the range $0 \le x < 2 \beta$ while
a double-precision variable must be able to represent integers in the range $0 \le x < 2 \beta^2$.  Within the source code that will be
presented the data type \textbf{mp\_digit} will represent a single-precision type while \textbf{mp\_word} will represent a 
double-precision type.  In several algorithms (\textit{notably the Comba routines}) temporary results 
will be stored in a double-precision arrays.  For the purposes of this text $x_j$ will refer to the 
$j$'th digit of a single-precision array and $\hat x_j$ will refer to the $j$'th digit of a double-precision
array.

\subsection{Work Effort}
\index{big-O}
To measure the efficiency of various algorithms a modified big-O notation is used.  In this system all 
single precision operations are considered to have the same cost\footnote{Except where explicitly noted.}.  
That is a single precision addition, multiplication and division are assumed to take the same time to 
complete.  While this is generally not true in practice it will simplify the discussions considerably.

Some algorithms have slight advantages over others which is why some constants will not be removed in 
the notation.  For example, a normal multiplication requires $O(n^2)$ work while a squaring requires 
$O({{n^2 + n}\over 2})$ work.  In standard big-O notation these would be said to be equivalent.  However, in the 
context of the this text the magnitude of the inputs will not approach an infinite size.  This means the conventional limit 
notation wisdom does not apply to the cancellation of constants.

Throughout the discussions various ``work levels'' will be discussed.  These levels are the $O(1)$,
$O(n)$, $O(n^2)$, ..., $O(n^k)$ work efforts.  For example, operations at the $O(n^k)$ ``level'' are said to be
executed more frequently than operations at the $O(n^m)$ ``level'' when $k > m$.  Obviously most optimizations will pay
off the most at the higher levels since they represent the bulk of the effort required.  

\section{Exercises}
Within the more advanced chapters a section will be set aside to give the reader some challenging exercises.  These exercises are not 
designed to be prize winning problems yet instead to be thought provoking.  Wherever possible the problems are foreward minded stating 
problems that will be answered in subsequent chapters.  The reader is encouraged to finish the exercises as they appear to get a 
better understanding of the subject material.  

Similar to the exercises of \cite{TAOCPV2} as explained on pp.\textit{ix} these exercises are given a scoring system.  However, unlike 
\cite{TAOCPV2} the problems do not get nearly as hard as often.  The scoring of these exercises ranges from one (\textit{the easiest}) to
five (\textit{the hardest}).  The following table sumarizes the scoring.

\vspace{5mm}
\begin{tabular}{cl}
$\left [ 1 \right ]$ & An easy problem that should only take the reader a manner of \\
                     & minutes to solve.  Usually does not involve much computer time. \\
                     & \\
$\left [ 2 \right ]$ & An easy problem that involves a marginal amount of computer \\
                     & time usage.  Usually requires a program to be written to \\
                     & solve the problem. \\
                     & \\
$\left [ 3 \right ]$ & A moderately hard problem that requires a non-trivial amount \\
                     & of work.  Usually involves trivial research and development of \\
                     & new theory from the perspective of a student. \\
                     & \\
$\left [ 4 \right ]$ & A moderately hard problem that involves a non-trivial amount \\
                     & of work and research.  The solution to which will demonstrate \\
                     & a higher mastery of the subject matter. \\
                     & \\
$\left [ 5 \right ]$ & A hard problem that involves concepts that are non-trivial.  \\
                     & Solutions to these problems will demonstrate a complete mastery \\
                     & of the given subject. \\
                     & \\
\end{tabular}

Essentially problems at the first level are meant to be simple questions that the reader can answer quickly without programming a solution or
devising new theory.  These problems are quick tests to see if the material is understood.  Problems at the second level are also
designed to be easy but will require a program or algorithm to be implemented to arrive at the answer.  

Problems at the third level are meant to be a bit more difficult.  Often the answer is fairly obvious but arriving at an exacting solution
requires some thought and skill.  These problems will almost always involve devising a new algorithm or implementing a variation of
another algorithm.

Problems at the fourth level are meant to be even more difficult as well as involve some research.  The reader will most likely not know
the answer right away nor will this text provide the exact details of the answer (\textit{or at least not until a subsequent chapter}).  Problems
at the fifth level are meant to be the hardest problems relative to all the other problems in the chapter.  People who can correctly 
answer fifth level problems have a mastery of the subject matter at hand.

Often problems will be tied together.  The purpose of this is to start a chain of thought that will be discussed in future chapters.  The reader
is encouraged to answer the follow-up problems and try to draw the relevence of problems.

\chapter{Introduction to LibTomMath}

\section{What is the LibTomMath?}
LibTomMath is a free and open source multiple precision number theoretic library written in portable ISO C
source code.  By portable it is meant that the library does not contain any code that is platform dependent or otherwise
problematic to use on any given platform.  The library has been successfully tested under numerous operating systems 
including Solaris, MacOS, Windows, Linux, PalmOS and on standalone hardware such as the Gameboy Advance.  The 
library is designed to contain enough functionality to be able to develop number theoretic applications such as public 
key cryptosystems.

\section{Goals of the LibTomMath}

Even though the library is written entirely in portable ISO C considerable care has been taken to 
optimize the algorithm implementations within the library.  Specifically the code has been written to work well with
the GNU C Compiler (\textit{GCC}) on both x86 and ARMv4 processors.  Wherever possible optimal 
algorithms (\textit{such as Karatsuba multiplication, sliding window exponentiation and Montgomery reduction.}) have 
been provided to make the library as efficient as possible.  Even with the optimal and sometimes specialized 
algorithms that have been included the API has been kept as simple as possible.  Often generic place holder routines 
will make use of specialized algorithms automatically without the developers attention.  One such example
is the generic multiplication algorithm \textbf{mp\_mul()} which will automatically use Karatsuba multiplication if the 
inputs are of a specific size.

Making LibTomMath as efficient as possible is not the only goal of the LibTomMath project.  Ideally the library should 
be source compatible with another popular library which makes it more attractive for developers to use.  In this case the
MPI library was used as a API template for all the basic functions.

The project is also meant to act as a learning tool for students.  The logic being that no easy to follow ``bignum'' 
library exists which can be used to teach computer science students how to perform fast and reliable multiple precision 
arithmetic.  To this end the source code has been given quite a few comments and algorithm discussion points.  Often 
where applicable routines have more comments than lines of code.

\section{Choice of LibTomMath}
LibTomMath was chosen as the case study of this text not only because the author of both projects is one and the same but
for more worthy reasons.  Other libraries such as GMP, MPI, LIP and OpenSSL have multiple precision 
integer arithmetic routines but would not be ideal for this text for numerous reasons as will be explained in the 
following sub-sections.

\subsection{Code Base}
The LibTomMath code base is all portable ISO C source code.  This means that there are no platform dependent conditional
segments of code littered throughout the source.  This clean and uncluttered approach to the library means that a
developer can more readily ascertain the true intent of a given section of source code without trying to keep track of
what conditional code will be used.

The code base of LibTomMath is also exceptionally well organized.  Each function is in its own separate source code file 
which allows the reader to find a given function very fast.  When compiled with GCC for the x86 processor the entire 
library is a mere 87,760 bytes (\textit{$116,182$ bytes for ARMv4 processors}).  This includes every single function 
LibTomMath provides from basic arithmetic to various number theoretic functions such as modular exponentiation, various 
reduction algorithms and Jacobi symbol computation.  

By comparison MPI which has fewer number theoretic functions than LibTomMath compiled with the same conditions is 
45,429 bytes (\textit{$54,536$ for ARMv4}).  GMP which has rather large collection of functions with the default 
configuration on an x86 Athlon is 2,950,688 bytes.  Note that while LibTomMath has fewer functions than GMP it has been
been used as the sole basis for several public key cryptosystems without having to seek additional outside functions
to supplement the library.

\subsection{API Simplicity}
LibTomMath is designed after the MPI library and shares the API design.  Quite often programs that use MPI will build 
with LibTomMath without change. The function names are relatively straight forward as to what they perform.  Almost all of the 
functions except for a few minor exceptions which as will be discussed are for good reasons share the same parameter passing 
convention.  The learning curve is fairly shallow with the API provided which is an extremely valuable benefit for the 
student and developer alike.  

The LIP library is an example of a library with an API that is awkward to work with.  LIP uses function names that are often ``compressed'' to 
illegible short hand.  LibTomMath does not share this fault.

\subsection{Optimizations}
While LibTomMath is certainly not the fastest library (\textit{GMP often beats LibTomMath by a factor of two}) it does
feature a set of optimal algorithms for tasks ranging from modular reduction to squaring.  GMP and LIP also feature
such optimizations while MPI only uses baseline algorithms with no optimizations.

LibTomMath is almost always a magnitude faster than the MPI library at computationally expensive tasks such as modular
exponentiation.  In the grand scheme of ``bignum'' libraries LibTomMath is faster than the average library and usually  
slower than the best libraries such as GMP and OpenSSL by a small factor.

\subsection{Portability and Stability}
LibTomMath will build ``out of the box'' on any platform equipped with a modern version of the GNU C Compiler 
(\textit{GCC}).  This means that without changes the library will build without configuration or setting up any 
variables.  LIP and MPI will build ``out of the box'' as well but have numerous known bugs.  Most notably the author of 
MPI is not working on his library anymore.  

GMP requires a configuration script to run and will not build out of the box.   GMP and LibTomMath are still in active
development and are very stable across a variety of platforms.

\subsection{Choice}
LibTomMath is a relatively compact, well documented, highly optimized and portable library which seems only natural for
the case study of this text.  Various source files from the LibTomMath project will be included within the text.  However, the 
reader is encouraged to download their own copy of the library to actually be able to work with the library.  

\chapter{Getting Started}
\section{Library Basics}
To get the ``ball rolling'' so to speak a primitive data type and a series of primitive algorithms must be established.  First a data
type that will hold the information required to maintain a multiple precision integer must be designed.  With this basic data type of a series
of low level algorithms for initializing, clearing, growing and clamping integers can be developed to form the basis of the entire
package of algorithms.

\section{The mp\_int structure}
First the data type for storing multiple precision integers must be designed.  This data type must be able to hold information to 
maintain an array of digits, how many are actually used in the representation and the sign.  The ISO C standard does not provide for 
any such data type but it does provide for making composite data types known as structures.  The following is the structure definition 
used within LibTomMath.

\index{mp\_int}
\begin{verbatim}
typedef struct  {
    int used, alloc, sign;
    mp_digit *dp;
} mp_int;
\end{verbatim}

The \textbf{used} parameter denotes how many digits of the array \textbf{dp} are actually being used.  The array 
\textbf{dp} holds the digits that represent the integer desired.  The \textbf{alloc} parameter denotes how 
many digits are available in the array to use by functions before it has to increase in size.  When the \textbf{used} count 
of a result would exceed the \textbf{alloc} count all LibTomMath routines will automatically increase the size of the 
array to accommodate the precision of the result.  The \textbf{sign} parameter denotes the sign as either zero/positive 
(\textbf{MP\_ZPOS}) or negative (\textbf{MP\_NEG}).  

\section{Argument Passing}
A convention of arugment passing must be adopted early on in the development of any library.  Making the function prototypes
consistent will help eliminate many headaches in the future as the library grows to significant complexity.  In LibTomMath the multiple precision 
integer functions accept parameters from left to right as pointers to mp\_int structures.  That means that the source operands are 
placed on the left and the destination on the right.   Consider the following examples.

\begin{verbatim}
   mp_mul(&a, &b, &c);   /* c = a * b */
   mp_add(&a, &b, &a);   /* a = a + b */
   mp_sqr(&a, &b);       /* b = a * a */
\end{verbatim}

The left to right order is a fairly natural way to implement the functions since it lets the developer read aloud the
functions and make sense of them.  For example, the first function would read ``multiply a and b and store in c''.

Certain libraries (\textit{LIP by Lenstra for instance}) accept parameters the other way around.  That is the destination
on the left and arguments on the right.  In truth it is entirely a matter of preference.  

Another very useful design consideration is whether to allow argument sources to also be a destination.  For example, the
second example (\textit{mp\_add}) adds $a$ to $b$ and stores in $a$.  This is an important feature to implement since it
allows the higher up functions to cut down on the number of variables.  However, to implement this feature specific
care has to be given to ensure the destination is not written before the source is fully read.

\section{Return Values}
A well implemented library, no matter what its purpose, should trap as many runtime errors as possible and return them to the 
caller.  By catching runtime errors a library can be guaranteed to prevent undefined behaviour within reason.  In a multiple precision 
library the only errors that are bound to occur are related to inappropriate inputs (\textit{division by zero for instance}) or 
memory allocation errors.

In LibTomMath any function that can cause a runtime error will return an error as an \textbf{int} data type with one of the 
following values.

\index{MP\_OKAY} \index{MP\_VAL} \index{MP\_MEM}
\begin{center}
\begin{tabular}{|l|l|}
\hline \textbf{Value} & \textbf{Meaning} \\
\hline \textbf{MP\_OKAY} & The function was successful \\
\hline \textbf{MP\_VAL}  & One of the input value(s) was invalid \\
\hline \textbf{MP\_MEM}  & The function ran out of heap memory \\
\hline
\end{tabular}
\end{center}

When an error is detected within a function it should free any memory they allocated and return as soon as possible.  The goal
is to leave the system in the same state the system was when the function was called.  Error checking with this style of API is fairly simple.

\begin{verbatim}
   int err;
   if ((err = mp_add(&a, &b, &c)) != MP_OKAY) {
      printf("Error: %d\n", err);
      exit(EXIT_FAILURE);
   }
\end{verbatim}

The GMP library uses C style \textit{signals} to flag errors which is of questionable use.  Not all errors are fatal 
and it is not ideal to force developers to have signal handlers for such cases.

\section{Initialization and Clearing}
The logical starting point when actually writing multiple precision integer functions is the initialization and 
clearing of the integers.  These two functions will be used by far the most throughout the algorithms whenever 
temporary integers are required.

Given the basic mp\_int structure an initialization routine must first allocate memory to hold the digits of
the integer.  Often it is optimal to allocate a sufficiently large pre-set number of digits even considering
the initial integer will represent zero.  If only a single digit were allocated quite a few re-allocations
would occur for the majority of inputs.  There exists a tradeoff between how many default digits to allocate
and how many re-allocations are tolerable.  

If the memory for the digits has been successfully allocated then the rest of the members of the structure must
be initialized.  Since the initial state is to represent a zero integer the digits allocated must all be zeroed.  The
\textbf{used} count set to zero and \textbf{sign} set to \textbf{MP\_ZPOS}.

\subsection{Initializing an mp\_int}
To initialize an mp\_int the mp\_init algorithm shall be used.  The purpose of this algorithm is to allocate 
the memory required and initialize the integer to a default representation of zero.

\begin{figure}[here]
\begin{center}
\begin{tabular}{l}
\hline Algorithm \textbf{mp\_init}. \\
\textbf{Input}.   An mp\_int $a$ \\
\textbf{Output}.  Allocate memory for the digits and set to a zero state. \\
\hline \\
1.  Allocate memory for \textbf{MP\_PREC} digits. \\
2.  If the allocation failed then return(\textit{MP\_MEM}) \\
3.  for $n$ from $0$ to $MP\_PREC - 1$ do  \\
\hspace{3mm}3.1  $a_n \leftarrow 0$\\
4.  $a.sign \leftarrow MP\_ZPOS$\\
5.  $a.used \leftarrow 0$\\
6.  $a.alloc \leftarrow MP\_PREC$\\
7.  Return(\textit{MP\_OKAY})\\
\hline
\end{tabular}
\end{center}
\caption{Algorithm mp\_init}
\end{figure}

\textbf{Algorithm mp\_init.}
The \textbf{MP\_PREC} variable is a simple constant used to dictate minimal precision of allocated integers.  It is ideally at least equal to $32$ but 
can be any reasonable power of two.  Step one and two allocate the memory and account for it.  If the allocation fails the algorithm returns
immediately to signal the failure.  Step three will ensure that all the digits are in the default state of zero.  Finally steps 
four through six set the default settings of the \textbf{sign}, \textbf{used} and \textbf{alloc} members of the mp\_int structure.

\index{bn\_mp\_init.c}
\vspace{+3mm}\begin{small}
\hspace{-5.1mm}{\bf File}: bn\_mp\_init.c
\vspace{-3mm}
\begin{alltt}
016   
017   /* init a new bigint */
018   int
019   mp_init (mp_int * a)
020   \{
021     /* allocate ram required and clear it */
022     a->dp = OPT_CAST calloc (sizeof (mp_digit), MP_PREC);
023     if (a->dp == NULL) \{
024       return MP_MEM;
025     \}
026   
027     /* set the used to zero, allocated digit to the default precision
028      * and sign to positive */
029     a->used  = 0;
030     a->alloc = MP_PREC;
031     a->sign  = MP_ZPOS;
032   
033     return MP_OKAY;
034   \}
\end{alltt}
\end{small}

The \textbf{OPT\_CAST} type cast on line 22 is designed to allow C++ compilers to build the code out of
the box.  Microsoft C V5.00 is known to cause problems without the cast.  Also note that if the memory
allocation fails the other members of the mp\_int will be in an undefined state.  The code from 
line 29 to line 31 sets the default state for a mp\_int which is zero, positive and no used digits.

\subsection{Clearing an mp\_int}
When an mp\_int is no longer required the memory allocated for it can be cleared from the heap with 
the mp\_clear algorithm.

\begin{figure}[here]
\begin{center}
\begin{tabular}{l}
\hline Algorithm \textbf{mp\_clear}. \\
\textbf{Input}.   An mp\_int $a$ \\
\textbf{Output}.  The memory for $a$ is cleared. \\
\hline \\
1.  If $a$ has been previously freed then return(\textit{MP\_OKAY}). \\
2.  Free the digits of $a$ and mark $a$ as freed. \\
3.  $a.used \leftarrow 0$ \\
4.  $a.alloc \leftarrow 0$ \\
5.  Return(\textit{MP\_OKAY}). \\
\hline
\end{tabular}
\end{center}
\caption{Algorithm mp\_clear}
\end{figure}

\textbf{Algorithm mp\_clear.}
In steps one and two the memory for the digits are only free'd if they had not been previously released before.  
This is more of concern for the implementation since it is used to prevent ``double-free'' errors.  It also helps catch
code errors where mp\_ints are used after being cleared.  Simiarly steps three and four set the 
\textbf{used} and \textbf{alloc} to known values which would be easy to spot during debugging.  For example, if an mp\_int is expected
to be non-zero and its \textbf{used} member observed to be zero (\textit{due to being cleared}) then an obvious bug in the code has been
spotted.

\index{bn\_mp\_clear.c}
\vspace{+3mm}\begin{small}
\hspace{-5.1mm}{\bf File}: bn\_mp\_clear.c
\vspace{-3mm}
\begin{alltt}
016   
017   /* clear one (frees)  */
018   void
019   mp_clear (mp_int * a)
020   \{
021     if (a->dp != NULL) \{
022   
023       /* first zero the digits */
024       memset (a->dp, 0, sizeof (mp_digit) * a->used);
025   
026       /* free ram */
027       free (a->dp);
028   
029       /* reset members to make debugging easier */
030       a->dp = NULL;
031       a->alloc = a->used = 0;
032     \}
033   \}
\end{alltt}
\end{small}

The \textbf{if} statement on line 21 prevents the heap from being corrupted if a user double-frees an 
mp\_int.  For example, a trivial case of this bug would be as follows.

\begin{verbatim}
mp_int a;
mp_init(&a);
mp_clear(&a);
mp_clear(&a);
\end{verbatim}

Without that check the code would try to free the memory allocated for the digits twice which will cause most standard C
libraries to cause a fault.  Also by setting the pointer to \textbf{NULL} it helps debug code that may inadvertently 
free the mp\_int before it is truly not needed.  The allocated digits are set to zero before being freed on line 24.  
This is ideal for cryptographic situations where the mp\_int is a secret parameter.

The following snippet is an example of using both the init and clear functions.  

\begin{small}
\begin{verbatim}
#include <tommath.h>
#include <stdio.h>
#include <stdlib.h>
int main(void)
{
   mp_int num;
   int err;
   
   /* init the bignum */
   if ((err = mp_init(&num)) != MP_OKAY) {
      printf("Error: %d\n", err);
      return EXIT_FAILURE;
   }
   
   /* do work with it ... */
   
   /* clear up */
   mp_clear(&num);
   
   return EXIT_SUCCESS;
}
\end{verbatim}
\end{small}

\section{Other Initialization Routines}

It is often helpful to have specialized initialization algorithms to simplify the design of other algorithms.  For example, an 
initialization followed by a copy is a common operation when temporary copies of integers are required.  It is quite
beneficial to have a series of simple helper functions available.

\subsection{Initializing Variable Sized mp\_int Structures}
Occasionally the number of digits required will be known in advance of an initialization.  In these
cases the mp\_init\_size algorithm can be of use.  The purpose of this algorithm is similar to mp\_init except that 
it will allocate \textit{at least} a specified number of digits.  This is ideal to prevent re-allocations when the 
input size is known.

\newpage\begin{figure}[here]
\begin{center}
\begin{tabular}{l}
\hline Algorithm \textbf{mp\_init\_size}. \\
\textbf{Input}.   An mp\_int $a$ and the requested number of digits $b$\\
\textbf{Output}.  $a$ is initialized to hold at least $b$ digits. \\
\hline \\
1.  $u \leftarrow b\mbox{ (mod }MP\_PREC\mbox{)}$ \\
2.  $v \leftarrow b + 2 \cdot MP\_PREC - u$ \\
3.  Allocate $v$ digits. \\
4.  If the allocation failed then return(\textit{MP\_MEM}). \\
5.  for $n$ from $0$ to $v - 1$ do \\
\hspace{3mm}5.1  $a_n \leftarrow 0$ \\
6.  $a.sign \leftarrow MP\_ZPOS$\\
7.  $a.used \leftarrow 0$\\
8.  $a.alloc \leftarrow v$\\
9.  Return(\textit{MP\_OKAY})\\
\hline
\end{tabular}
\end{center}
\caption{Algorithm mp\_init\_size}
\end{figure}

\textbf{Algorithm mp\_init\_size.}
The value of $v$ is calculated to be at least the requested amount of digits $b$ plus additional padding.  The padding is calculated
to be at least \textbf{MP\_PREC} digits plus enough digits to make the digit count a multiple of \textbf{MP\_PREC}.  This padding is used to 
prevent trivial allocations from becomming a bottleneck in the rest of the algorithms that depend on this.

\index{bn\_mp\_init\_size.c}
\vspace{+3mm}\begin{small}
\hspace{-5.1mm}{\bf File}: bn\_mp\_init\_size.c
\vspace{-3mm}
\begin{alltt}
016   
017   /* init a mp_init and grow it to a given size */
018   int
019   mp_init_size (mp_int * a, int size)
020   \{
021   
022     /* pad size so there are always extra digits */
023     size += (MP_PREC * 2) - (size & (MP_PREC - 1));    
024     
025     /* alloc mem */
026     a->dp = OPT_CAST calloc (sizeof (mp_digit), size);
027     if (a->dp == NULL) \{
028       return MP_MEM;
029     \}
030     a->used = 0;
031     a->alloc = size;
032     a->sign = MP_ZPOS;
033   
034     return MP_OKAY;
035   \}
\end{alltt}
\end{small}

Line 23 will ensure that the number of digits actually allocated is padded up to the next multiple of 
\textbf{MP\_PREC} plus an additional \textbf{MP\_PREC}.  This ensures that the number of allocated digit is 
always greater than the amount requested.  As a result it prevents many trivial memory allocations.  The value of 
\textbf{MP\_PREC} is defined in ``tommath.h'' and must be a power of two.

\subsection{Creating a Clone}
Another common sequence of operations is to make a local temporary copy of an argument.  To initialize then copy a mp\_int will be known as 
creating a clone.  This is useful within functions that need to modify an integer argument but do not wish to actually modify the original copy.  
The mp\_init\_copy algorithm will perform this very task.

\begin{figure}[here]
\begin{center}
\begin{tabular}{l}
\hline Algorithm \textbf{mp\_init\_copy}. \\
\textbf{Input}.   An mp\_int $a$ and $b$\\
\textbf{Output}.  $a$ is initialized to be a copy of $b$. \\
\hline \\
1.  Init $a$.  (\textit{hint: use mp\_init}) \\
2.  If the init of $a$ was unsuccessful return(\textit{MP\_MEM}) \\
3.  Copy $b$ to $a$.  (\textit{hint: use mp\_copy}) \\
4.  Return the status of the copy operation. \\
\hline
\end{tabular}
\end{center}
\caption{Algorithm mp\_init\_copy}
\end{figure}

\textbf{Algorithm mp\_init\_copy.}
This algorithm will initialize a mp\_int variable and copy another previously initialized mp\_int variable into it.  The algorithm will
detect when the initialization fails and returns the error to the calling algorithm.  As such this algorithm will perform two operations
in one step.  

\index{bn\_mp\_init\_copy.c}
\vspace{+3mm}\begin{small}
\hspace{-5.1mm}{\bf File}: bn\_mp\_init\_copy.c
\vspace{-3mm}
\begin{alltt}
016   
017   /* creates "a" then copies b into it */
018   int
019   mp_init_copy (mp_int * a, mp_int * b)
020   \{
021     int     res;
022   
023     if ((res = mp_init (a)) != MP_OKAY) \{
024       return res;
025     \}
026     return mp_copy (b, a);
027   \}
\end{alltt}
\end{small}

This will initialize \textbf{a} and make it a verbatim copy of the contents of \textbf{b}.  Note that 
\textbf{a} will have its own memory allocated which means that \textbf{b} may be cleared after the call
and \textbf{a} will be left intact.  

\subsection{Multiple Integer Initializations}
Occasionally a function will require a series of mp\_int data types to be made available.  The mp\_init\_multi algorithm
is provided to simplify such cases.  The purpose of this algorithm is to initialize a variable length array of mp\_int 
structures at once.  As a result algorithms that require multiple integers only has to use 
one algorithm to initialize all the mp\_int variables.

\begin{figure}[here]
\begin{center}
\begin{tabular}{l}
\hline Algorithm \textbf{mp\_init\_multi}. \\
\textbf{Input}.   Variable length array of mp\_int variables of length $k$. \\
\textbf{Output}.  The array is initialized such that each each mp\_int is ready to use. \\
\hline \\
1.  for $n$ from 0 to $k - 1$ do \\
\hspace{+3mm}1.1.  Initialize the $n$'th mp\_int (\textit{hint: use mp\_init}) \\
\hspace{+3mm}1.2.  If initialization failed then do \\
\hspace{+6mm}1.2.1.  for $j$ from $0$ to $n$ do \\
\hspace{+9mm}1.2.1.1.  Free the $j$'th mp\_int (\textit{hint: use mp\_clear}) \\
\hspace{+6mm}1.2.2.   Return(\textit{MP\_MEM}) \\
2.  Return(\textit{MP\_OKAY}) \\
\hline
\end{tabular}
\end{center}
\caption{Algorithm mp\_init\_multi}
\end{figure}

\textbf{Algorithm mp\_init\_multi.}
The algorithm will initialize the array of mp\_int variables one at a time.  As soon as an runtime error is detected (\textit{step 1.2}) all of
the previously initialized variables are cleared.  The goal is an ``all or nothing'' initialization which allows for quick recovery from runtime 
errors.

\subsection{Multiple Integer Clearing}
Similarly to clear a variable length list of mp\_int structures the mp\_clear\_multi algorithm will be used.

\index{bn\_mp\_multi.c}
\vspace{+3mm}\begin{small}
\hspace{-5.1mm}{\bf File}: bn\_mp\_multi.c
\vspace{-3mm}
\begin{alltt}
016   #include <stdarg.h>
017   
018   int mp_init_multi(mp_int *mp, ...) 
019   \{
020       mp_err res = MP_OKAY;      /* Assume ok until proven otherwise */
021       int n = 0;                 /* Number of ok inits */
022       mp_int* cur_arg = mp;
023       va_list args;
024   
025       va_start(args, mp);        /* init args to next argument from caller */
026       while (cur_arg != NULL) \{
027           if (mp_init(cur_arg) != MP_OKAY) \{
028               /* Oops - error! Back-track and mp_clear what we already
029                  succeeded in init-ing, then return error.
030               */
031               va_list clean_args;
032               
033               /* end the current list */
034               va_end(args);
035               
036               /* now start cleaning up */            
037               cur_arg = mp;
038               va_start(clean_args, mp);
039               while (n--) \{
040                   mp_clear(cur_arg);
041                   cur_arg = va_arg(clean_args, mp_int*);
042               \}
043               va_end(clean_args);
044               res = MP_MEM;
045               break;
046           \}
047           n++;
048           cur_arg = va_arg(args, mp_int*);
049       \}
050       va_end(args);
051       return res;                /* Assumed ok, if error flagged above. */
052   \}
053   
054   void mp_clear_multi(mp_int *mp, ...) 
055   \{
056       mp_int* next_mp = mp;
057       va_list args;
058       va_start(args, mp);
059       while (next_mp != NULL) \{
060           mp_clear(next_mp);
061           next_mp = va_arg(args, mp_int*);
062       \}
063       va_end(args);
064   \}
\end{alltt}
\end{small}

Consider the following snippet which demonstrates how to use both routines.
\begin{small}
\begin{verbatim}
#include <tommath.h>
#include <stdio.h>
#include <stdlib.h>
int main(void)
{
   mp_int num1, num2, num3;
   int err;
   
   if ((err = mp_init_multi(&num1, &num2, &num3, NULL)) !- MP_OKAY) {
      printf("Error: %d\n", err);
      return EXIT_FAILURE;
   }
   
   /* at this point num1/num2/num3 are ready */
   
   /* free them */
   mp_clear_multi(&num1, &num2, &num3, NULL);
   
   return EXIT_SUCCESS;
}
\end{verbatim}
\end{small}

\section{Maintenance}
A small useful collection of mp\_int maintenance functions will also prove useful.  

\subsection{Augmenting Integer Precision}
When storing a value in an mp\_int sufficient digits must be available to accomodate the entire value without
loss of precision.  Quite often the size of the array given by the \textbf{alloc} member is large enough to simply
increase the \textbf{used} digit count.  However, when the size of the array is too small it must be re-sized 
appropriately to accomodate the result.  The mp\_grow algorithm will provide this functionality.

\begin{figure}[here]
\begin{center}
\begin{tabular}{l}
\hline Algorithm \textbf{mp\_grow}. \\
\textbf{Input}.   An mp\_int $a$ and an integer $b$. \\
\textbf{Output}.  $a$ is expanded to accomodate $b$ digits. \\
\hline \\
1.  if $a.alloc \ge b$ then return(\textit{MP\_OKAY}) \\
2.  $u \leftarrow b\mbox{ (mod }MP\_PREC\mbox{)}$ \\
3.  $v \leftarrow b + 2 \cdot MP\_PREC - u$ \\
4.  Re-Allocate the array of digits $a$ to size $v$ \\
5.  If the allocation failed then return(\textit{MP\_MEM}). \\
6.  for n from a.alloc to $v - 1$ do  \\
\hspace{+3mm}6.1  $a_n \leftarrow 0$ \\
7.  $a.alloc \leftarrow v$ \\
8.  Return(\textit{MP\_OKAY}) \\
\hline
\end{tabular}
\end{center}
\caption{Algorithm mp\_grow}
\end{figure}

\textbf{Algorithm mp\_grow.}
Step one will prevent a re-allocation from being performed if it was not required.  This is useful to prevent mp\_ints
from growing excessively in code that erroneously calls mp\_grow.  Similar to mp\_init\_size the requested digit count
is padded to provide more digits than requested.  

In step four it is assumed that the reallocation leaves the lower $a.alloc$ digits intact.  Much akin to how the 
\textit{realloc} function from the standard C library works.  Since the newly allocated digits are assumed to contain
undefined values they are also initially zeroed.

\index{bn\_mp\_grow.c}
\vspace{+3mm}\begin{small}
\hspace{-5.1mm}{\bf File}: bn\_mp\_grow.c
\vspace{-3mm}
\begin{alltt}
016   
017   /* grow as required */
018   int
019   mp_grow (mp_int * a, int size)
020   \{
021     int     i;
022   
023     /* if the alloc size is smaller alloc more ram */
024     if (a->alloc < size) \{
025       /* ensure there are always at least MP_PREC digits extra on top */
026       size += (MP_PREC * 2) - (size & (MP_PREC - 1));     
027   
028       a->dp = OPT_CAST realloc (a->dp, sizeof (mp_digit) * size);
029       if (a->dp == NULL) \{
030         return MP_MEM;
031       \}
032   
033       /* zero excess digits */
034       i        = a->alloc;
035       a->alloc = size;
036       for (; i < a->alloc; i++) \{
037         a->dp[i] = 0;
038       \}
039     \}
040     return MP_OKAY;
041   \}
\end{alltt}
\end{small}

The first step is to see if we actually need to perform a re-allocation at all.  This is tested for on line 
24.  Similar to mp\_init\_size the same code on line 26 was used to resize the 
digits requested.  A simple for loop from line 34 to line 38 will zero all digits that were above the 
old \textbf{alloc} limit to make sure the integer is in a known state.

\subsection{Clamping Excess Digits}
When a function anticipates a result will be $n$ digits it is simpler to assume this is true within the body of 
the function.  For example, a multiplication of a $i$ digit number by a $j$ digit produces a result of at most 
$i + j + 1$ digits.  It is entirely possible that the result is $i + j$ though, with no final carry into the last 
position.  However, suppose the destination had to be first expanded (\textit{via mp\_grow}) to accomodate $i + j$
digits than further expanded to accomodate the final carry.  That would be a considerable waste of time since heap
operations are relatively slow.

The ideal solution is to always assume the result is $i + j + 1$ and fix up the \textbf{used} count after the function
terminates.  This way a single heap operation (\textit{at most}) is required.  However, if the result was not checked
there would be an excess high order zero digit.  

For example, suppose the product of two integers was $x_n = (0x_{n-1}x_{n-2}...x_0)_{\beta}$.  The leading zero digit 
will not contribute to the precision of the result.  In fact, through subsequent operations more leading zero digits would
accumulate to the point the size of the integer would be prohibitive.  As a result even though the precision is very 
low the representation is excessively large.  

The mp\_clamp algorithm is designed to solve this very problem.  It will trim leading zeros by decrementing the 
\textbf{used} count until a non-zero leading digit is found.  Also in this system, zero is considered to be a positive 
number which means that if the \textbf{used} count is decremented to zero the sign must be set to \textbf{MP\_ZPOS}.

\begin{figure}[here]
\begin{center}
\begin{tabular}{l}
\hline Algorithm \textbf{mp\_clamp}. \\
\textbf{Input}.   An mp\_int $a$ \\
\textbf{Output}.  Any excess leading zero digits of $a$ are removed \\
\hline \\
1.  while $a.used > 0$ and $a_{a.used - 1} = 0$ do \\
\hspace{+3mm}1.1  $a.used \leftarrow a.used - 1$ \\
2.  if $a.used = 0$ then do \\
\hspace{+3mm}2.1  $a.sign \leftarrow MP\_ZPOS$ \\
\hline \\
\end{tabular}
\end{center}
\caption{Algorithm mp\_clamp}
\end{figure}

\textbf{Algorithm mp\_clamp.}
As can be expected this algorithm is very simple.  The loop on step one is indended to be iterate only once or twice at
the most.  For example, for cases where there is not a carry to fill the last position.  Step two fixes the sign for 
when all of the digits are zero to ensure that the mp\_int is valid at all times.

\index{bn\_mp\_clamp.c}
\vspace{+3mm}\begin{small}
\hspace{-5.1mm}{\bf File}: bn\_mp\_clamp.c
\vspace{-3mm}
\begin{alltt}
016   
017   /* trim unused digits 
018    *
019    * This is used to ensure that leading zero digits are
020    * trimed and the leading "used" digit will be non-zero
021    * Typically very fast.  Also fixes the sign if there
022    * are no more leading digits
023    */
024   void
025   mp_clamp (mp_int * a)
026   \{
027     while (a->used > 0 && a->dp[a->used - 1] == 0) \{
028       --(a->used);
029     \}
030     if (a->used == 0) \{
031       a->sign = MP_ZPOS;
032     \}
033   \}
\end{alltt}
\end{small}

Note on line 27 how to test for the \textbf{used} count is made on the left of the \&\& operator.  In the C programming
language the terms to \&\& are evaluated left to right with a boolean short-circuit if any condition fails.  This is 
important since if the \textbf{used} is zero the test on the right would fetch below the array.  That is obviously 
undesirable.  The parenthesis on line 28 is used to make sure the \textbf{used} count is decremented and not
the pointer ``a''.  

\section*{Exercises}
\begin{tabular}{cl}
$\left [ 1 \right ]$ & Discuss the relevance of the \textbf{used} member of the mp\_int structure. \\
                     & \\
$\left [ 1 \right ]$ & Discuss the consequences of not using padding when performing allocations.  \\
                     & \\
$\left [ 2 \right ]$ & Estimate an ideal value for \textbf{MP\_PREC} when performing 1024-bit RSA \\
                     & encryption when $\beta = 2^{28}$.  \\
                     & \\
$\left [ 1 \right ]$ & Discuss the relevance of the algorithm mp\_clamp.  What does it prevent? \\
                     & \\
$\left [ 1 \right ]$ & Give an example of when the algorithm  mp\_init\_copy might be useful. \\
                     & \\
\end{tabular}


\chapter{Basic Operations}
\section{Copying an Integer}
After the various house-keeping routines are in place, simpl algorithms can be designed to take advantage of them.  Being able
to make a verbatim copy of an integer is a very useful function to have.  To copy an integer the mp\_copy algorithm will be used.

\newpage\begin{figure}[here]
\begin{center}
\begin{tabular}{l}
\hline Algorithm \textbf{mp\_copy}. \\
\textbf{Input}.  An mp\_int $a$ and $b$. \\
\textbf{Output}.  Store a copy of $a$ in $b$. \\
\hline \\
1.  Check if $a$ and $b$ point to the same location in memory. \\
2.  If true then return(\textit{MP\_OKAY}). \\
3.  If $b.alloc < a.used$ then grow $b$ to $a.used$ digits.  (\textit{hint: use mp\_grow}) \\
4.  If failed to grow then return(\textit{MP\_MEM}). \\
5.  for $n$ from 0 to $a.used - 1$ do \\
\hspace{3mm}5.1  $b_{n} \leftarrow a_{n}$ \\
6.  if $a.used < b.used - 1$ then \\ 
\hspace{3mm}6.1.  for $n$ from $a.used$ to $b.used - 1$ do \\
\hspace{6mm}6.1.1  $b_{n} \leftarrow 0$ \\
7.  $b.used \leftarrow a.used$ \\
8.  $b.sign \leftarrow a.sign$ \\
9.  return(\textit{MP\_OKAY}) \\
\hline
\end{tabular}
\end{center}
\caption{Algorithm mp\_copy}
\end{figure}

\textbf{Algorithm mp\_copy.}
Step 1 and 2 make sure that the two mp\_ints are unique.  This allows the user to call the copy function with
potentially the same input and not waste time.  Step 3 and 4 ensure that the destination is large enough to
hold a copy of the input $a$.  Note that the \textbf{used} member of $b$ may be smaller than the \textbf{used}
member of $a$ but a memory re-allocation is only required if the \textbf{alloc} member of $b$ is smaller.  This
prevents trivial memory reallocations.

Step 5 copies the digits from $a$ to $b$ while step 6 ensures that if initially $\vert b \vert > \vert a \vert$,
the leading digits of $b$ will be zeroed.  Finally steps 7 and 8 copies the \textbf{used} and \textbf{sign} members over 
which completes the copy operation.

\index{bn\_mp\_copy.c}
\vspace{+3mm}\begin{small}
\hspace{-5.1mm}{\bf File}: bn\_mp\_copy.c
\vspace{-3mm}
\begin{alltt}
016   
017   /* copy, b = a */
018   int
019   mp_copy (mp_int * a, mp_int * b)
020   \{
021     int     res, n;
022   
023     /* if dst == src do nothing */
024     if (a == b || a->dp == b->dp) \{
025       return MP_OKAY;
026     \}
027   
028     /* grow dest */
029     if ((res = mp_grow (b, a->used)) != MP_OKAY) \{
030       return res;
031     \}
032   
033     /* zero b and copy the parameters over */
034     \{
035       register mp_digit *tmpa, *tmpb;
036   
037       /* pointer aliases */
038       tmpa = a->dp;
039       tmpb = b->dp;
040   
041       /* copy all the digits */
042       for (n = 0; n < a->used; n++) \{
043         *tmpb++ = *tmpa++;
044       \}
045   
046       /* clear high digits */
047       for (; n < b->used; n++) \{
048         *tmpb++ = 0;
049       \}
050     \}
051     b->used = a->used;
052     b->sign = a->sign;
053     return MP_OKAY;
054   \}
\end{alltt}
\end{small}

Source lines 23-31 do the initial house keeping.  That is to see if the input is unique and if so to 
make sure there is enough room.  If not enough space is available it returns the error and leaves the destination variable
intact.

The inner loop of the copy operation is contained between lines 34 and 50.  Many LibTomMath routines are designed with this source code style
in mind, making aliases to shorten lengthy pointers (\textit{see line 38 and 39}) for rapid to use.  Also the
use of nested braces creates a simple way to denote various portions of code that reside on various work levels.  Here, the copy loop is at the 
$O(n)$ level.  

\section{Zeroing an Integer}
Reseting an mp\_int to the default state is a common step in many algorithms.  The mp\_zero algorithm will be the algorithm used to
perform this task.

\begin{figure}[here]
\begin{center}
\begin{tabular}{l}
\hline Algorithm \textbf{mp\_zero}. \\
\textbf{Input}.   An mp\_int $a$ \\
\textbf{Output}.  Zero the contents of $a$ \\
\hline \\
1.  $a.used \leftarrow 0$ \\
2.  $a.sign \leftarrow$ MP\_ZPOS \\
3.  for $n$ from 0 to $a.alloc - 1$ do \\
\hspace{3mm}3.1  $a_n \leftarrow 0$ \\
\hline
\end{tabular}
\end{center}
\caption{Algorithm mp\_zero}
\end{figure}

\textbf{Algorithm mp\_zero.}
This algorithm simply resets a mp\_int to the default state.  

\index{bn\_mp\_zero.c}
\vspace{+3mm}\begin{small}
\hspace{-5.1mm}{\bf File}: bn\_mp\_zero.c
\vspace{-3mm}
\begin{alltt}
016   
017   /* set to zero */
018   void
019   mp_zero (mp_int * a)
020   \{
021     a->sign = MP_ZPOS;
022     a->used = 0;
023     memset (a->dp, 0, sizeof (mp_digit) * a->alloc);
024   \}
\end{alltt}
\end{small}

After the function is completed, all of the digits are zeroed, the \textbf{used} count is zeroed and the 
\textbf{sign} variable is set to \textbf{MP\_ZPOS}.

\section{Sign Manipulation}
\subsection{Absolute Value}
With the mp\_int representation of an integer, calculating the absolute value is trivial.  The mp\_abs algorithm will compute
the absolute value of an mp\_int.

\begin{figure}[here]
\begin{center}
\begin{tabular}{l}
\hline Algorithm \textbf{mp\_abs}. \\
\textbf{Input}.   An mp\_int $a$ \\
\textbf{Output}.  Computes $b = \vert a \vert$ \\
\hline \\
1.  Copy $a$ to $b$.  (\textit{hint: use mp\_copy}) \\
2.  If the copy failed return(\textit{MP\_MEM}). \\
3.  $b.sign \leftarrow MP\_ZPOS$ \\
4.  Return(\textit{MP\_OKAY}) \\
\hline
\end{tabular}
\end{center}
\caption{Algorithm mp\_abs}
\end{figure}

\textbf{Algorithm mp\_abs.}
This algorithm computes the absolute of an mp\_int input.  As can be expected the algorithm is very trivial.

\index{bn\_mp\_abs.c}
\vspace{+3mm}\begin{small}
\hspace{-5.1mm}{\bf File}: bn\_mp\_abs.c
\vspace{-3mm}
\begin{alltt}
016   
017   /* b = |a| 
018    *
019    * Simple function copies the input and fixes the sign to positive
020    */
021   int
022   mp_abs (mp_int * a, mp_int * b)
023   \{
024     int     res;
025     if ((res = mp_copy (a, b)) != MP_OKAY) \{
026       return res;
027     \}
028     b->sign = MP_ZPOS;
029     return MP_OKAY;
030   \}
\end{alltt}
\end{small}

\subsection{Integer Negation}
With the mp\_int representation of an integer, calculating the negation is also trivial.  The mp\_neg algorithm will compute
the negative of an mp\_int input.

\newpage\begin{figure}[here]
\begin{center}
\begin{tabular}{l}
\hline Algorithm \textbf{mp\_neg}. \\
\textbf{Input}.   An mp\_int $a$ \\
\textbf{Output}.  Computes $b = -a$ \\
\hline \\
1.  Copy $a$ to $b$.  (\textit{hint: use mp\_copy}) \\
2.  If the copy failed return(\textit{MP\_MEM}). \\
3.  If $a.sign = MP\_ZPOS$ then do \\
\hspace{3mm}3.1  $b.sign = MP\_NEG$. \\
4.  else do \\
\hspace{3mm}4.1  $b.sign = MP\_ZPOS$. \\
5.  Return(\textit{MP\_OKAY}) \\
\hline
\end{tabular}
\end{center}
\caption{Algorithm mp\_neg}
\end{figure}

\textbf{Algorithm mp\_neg.}
This algorithm computes the negation of an input.  

\index{bn\_mp\_neg.c}
\vspace{+3mm}\begin{small}
\hspace{-5.1mm}{\bf File}: bn\_mp\_neg.c
\vspace{-3mm}
\begin{alltt}
016   
017   /* b = -a */
018   int
019   mp_neg (mp_int * a, mp_int * b)
020   \{
021     int     res;
022     if ((res = mp_copy (a, b)) != MP_OKAY) \{
023       return res;
024     \}
025     b->sign = (a->sign == MP_ZPOS) ? MP_NEG : MP_ZPOS;
026     return MP_OKAY;
027   \}
\end{alltt}
\end{small}

\section{Small Constants}
\subsection{Setting Small Constants}
Often a mp\_int must be set to a relatively small value such as $1$ or $2$.  For these cases the mp\_set algorithm is useful.

\newpage\begin{figure}
\begin{center}
\begin{tabular}{l}
\hline Algorithm \textbf{mp\_set}. \\
\textbf{Input}.   An mp\_int $a$ and a digit $b$ \\
\textbf{Output}.  Make $a$ equivalent to $b$ \\
\hline \\
1.  Zero $a$ (\textit{hint: use mp\_zero}). \\
2.  $a_0 \leftarrow b \mbox{ (mod }\beta\mbox{)}$ \\
3.  $a.used \leftarrow  \left \lbrace \begin{array}{ll}
                              1 &  \mbox{if }a_0 > 0 \\
                              0 &  \mbox{if }a_0 = 0 
                              \end{array} \right .$ \\
\hline                              
\end{tabular}
\end{center}
\caption{Algorithm mp\_set}
\end{figure}

\textbf{Algorithm mp\_set.}
This algorithm sets a mp\_int to a small single digit value.  Step number 1 ensures that the integer is reset to the default state.  The
single digit is set (\textit{modulo $\beta$}) and the \textbf{used} count is adjusted accordingly.

\index{bn\_mp\_set.c}
\vspace{+3mm}\begin{small}
\hspace{-5.1mm}{\bf File}: bn\_mp\_set.c
\vspace{-3mm}
\begin{alltt}
016   
017   /* set to a digit */
018   void
019   mp_set (mp_int * a, mp_digit b)
020   \{
021     mp_zero (a);
022     a->dp[0] = b & MP_MASK;
023     a->used = (a->dp[0] != 0) ? 1 : 0;
024   \}
\end{alltt}
\end{small}

Line 21 calls mp\_zero() to clear the mp\_int and reset the sign.  Line 22 actually copies digit 
into the least significant location.  Note the usage of a new constant \textbf{MP\_MASK}.  This constant is used to quickly
reduce an integer modulo $\beta$.  Since $\beta = 2^k$ it suffices to perform a binary AND with $MP\_MASK = 2^k - 1$ to perform
the reduction.  Finally line 23 will set the \textbf{used} member with respect to the digit actually set. This function 
will always make the integer positive.

One important limitation of this function is that it will only set one digit.  The size of a digit is not fixed, meaning source that uses 
this function should take that into account.  The define \textbf{DIGIT\_BIT} in ``tommath.h'' 
defines how many bits per digit are available.  Generally at least seven bits are guaranteed to be available per 
digit.  This means that trivially small constants can be set using this function.

\subsection{Setting Large Constants}
To overcome the limitations of the mp\_set algorithm the mp\_set\_int algorithm is provided.  It accepts a ``long''
data type as input and will always treat it as a 32-bit integer.

\begin{figure}[here]
\begin{center}
\begin{tabular}{l}
\hline Algorithm \textbf{mp\_set\_int}. \\
\textbf{Input}.   An mp\_int $a$ and a ``long'' integer $b$ \\
\textbf{Output}.  Make $a$ equivalent to $b$ \\
\hline \\
1.  Zero $a$ (\textit{hint: use mp\_zero}) \\
2.  for $n$ from 0 to 7 do \\
\hspace{3mm}2.1  $a \leftarrow a \cdot 16$ (\textit{hint: use mp\_mul2d}) \\
\hspace{3mm}2.2  $u \leftarrow \lfloor b / 2^{4(7 - n)} \rfloor \mbox{ (mod }16\mbox{)}$\\
\hspace{3mm}2.3  $a_0 \leftarrow a_0 + u$ \\
\hspace{3mm}2.4  $a.used \leftarrow a.used + \lfloor 32 / lg(\beta) \rfloor + 1$ \\
3.  Clamp excess used digits (\textit{hint: use mp\_clamp}) \\
\hline
\end{tabular}
\end{center}
\caption{Algorithm mp\_set\_int}
\end{figure}

\textbf{Algorithm mp\_set\_int.}
The algorithm performs eight iterations of a simple loop where in each iteration four bits from the source are added to the 
mp\_int.  Step 2.1 will multiply the current result by sixteen making room for four more bits.  In step 2.2 the
next four bits from the source are extracted.  The four bits are added to the mp\_int and the \textbf{used} digit count is 
incremented.  The \textbf{used} digit counter is incremented since if any of the leading digits were zero the mp\_int would have
zero digits used and the newly added four bits would be ignored.

Excess zero digits are trimmed in steps 2.1 and 3 by using higher level algorithms mp\_mul2d and mp\_clamp.

\index{bn\_mp\_set\_int.c}
\vspace{+3mm}\begin{small}
\hspace{-5.1mm}{\bf File}: bn\_mp\_set\_int.c
\vspace{-3mm}
\begin{alltt}
016   
017   /* set a 32-bit const */
018   int
019   mp_set_int (mp_int * a, unsigned int b)
020   \{
021     int     x, res;
022   
023     mp_zero (a);
024     /* set four bits at a time */
025     for (x = 0; x < 8; x++) \{
026       /* shift the number up four bits */
027       if ((res = mp_mul_2d (a, 4, a)) != MP_OKAY) \{
028         return res;
029       \}
030   
031       /* OR in the top four bits of the source */
032       a->dp[0] |= (b >> 28) & 15;
033   
034       /* shift the source up to the next four bits */
035       b <<= 4;
036   
037       /* ensure that digits are not clamped off */
038       a->used += 32 / DIGIT_BIT + 2;
039     \}
040     mp_clamp (a);
041     return MP_OKAY;
042   \}
\end{alltt}
\end{small}

This function sets four bits of the number at a time to handle all practical \textbf{DIGIT\_BIT} sizes.  The weird
addition on line 38 ensures that the newly added in bits are added to the number of digits.  While it may not 
seem obvious as to why the digit counter does not grow exceedingly large it is because of the shift on line 27 
as well as the  call to mp\_clamp() on line 40.  Both functions will clamp excess leading digits which keeps 
the number of used digits low.

\section{Comparisons}
\subsection{Unsigned Comparisions}
Comparing a multiple precision integer is performed with the exact same algorithm used to compare two decimal numbers.  For example,
to compare $1,234$ to $1,264$ the digits are extracted by their positions.  That is we compare $1 \cdot 10^3 + 2 \cdot 10^2 + 3 \cdot 10^1 + 4 \cdot 10^0$
to $1 \cdot 10^3 + 2 \cdot 10^2 + 6 \cdot 10^1 + 4 \cdot 10^0$ by comparing single digits at a time starting with the highest magnitude 
positions.  If any leading digit of one integer is greater than a digit in the same position of another integer then obviously it must be greater.  

The first comparision routine that will be developed is the unsigned magnitude compare which will perform a comparison based on the digits of two
mp\_int variables alone.  It will ignore the sign of the two inputs.  Such a function is useful when an absolute comparison is required or if the 
signs are known to agree in advance.

To facilitate working with the results of the comparison functions three constants are required.  

\begin{figure}[here]
\begin{center}
\begin{tabular}{|r|l|}
\hline \textbf{Constant} & \textbf{Meaning} \\
\hline \textbf{MP\_GT} & Greater Than \\
\hline \textbf{MP\_EQ} & Equal To \\
\hline \textbf{MP\_LT} & Less Than \\
\hline
\end{tabular}
\end{center}
\caption{Comparison Return Codes}
\end{figure}

\begin{figure}[here]
\begin{center}
\begin{tabular}{l}
\hline Algorithm \textbf{mp\_cmp\_mag}. \\
\textbf{Input}.   Two mp\_ints $a$ and $b$.  \\
\textbf{Output}.  Unsigned comparison results ($a$ to the left of $b$). \\
\hline \\
1.  If $a.used > b.used$ then return(\textit{MP\_GT}) \\
2.  If $a.used < b.used$ then return(\textit{MP\_LT}) \\
3.  for n from $a.used - 1$ to 0 do \\
\hspace{+3mm}3.1  if $a_n > b_n$ then return(\textit{MP\_GT}) \\
\hspace{+3mm}3.2  if $a_n < b_n$ then return(\textit{MP\_LT}) \\
4.  Return(\textit{MP\_EQ}) \\
\hline
\end{tabular}
\end{center}
\caption{Algorithm mp\_cmp\_mag}
\end{figure}

\textbf{Algorithm mp\_cmp\_mag.}
By saying ``$a$ to the left of $b$'' it is meant that the comparison is with respect to $a$, that is if $a$ is greater than $b$ it will return
\textbf{MP\_GT} and similar with respect to when $a = b$ and $a < b$.  The first two steps compare the number of digits used in both $a$ and $b$.  
Obviously if the digit counts differ there would be an imaginary zero digit in the smaller number where the leading digit of the larger number is.  
If both have the same number of digits than the actual digits themselves must be compared starting at the leading digit.  

By step three both inputs must have the same number of digits so its safe to start from either $a.used - 1$ or $b.used - 1$ and count down to
the zero'th digit.  If after all of the digits have been compared and no difference found the algorithm simply returns \textbf{MP\_EQ}.

\index{bn\_mp\_cmp\_mag.c}
\vspace{+3mm}\begin{small}
\hspace{-5.1mm}{\bf File}: bn\_mp\_cmp\_mag.c
\vspace{-3mm}
\begin{alltt}
016   
017   /* compare maginitude of two ints (unsigned) */
018   int
019   mp_cmp_mag (mp_int * a, mp_int * b)
020   \{
021     int     n;
022   
023     /* compare based on # of non-zero digits */
024     if (a->used > b->used) \{
025       return MP_GT;
026     \} 
027     
028     if (a->used < b->used) \{
029       return MP_LT;
030     \}
031   
032     /* compare based on digits  */
033     for (n = a->used - 1; n >= 0; n--) \{
034       if (a->dp[n] > b->dp[n]) \{
035         return MP_GT;
036       \} 
037       
038       if (a->dp[n] < b->dp[n]) \{
039         return MP_LT;
040       \}
041     \}
042     return MP_EQ;
043   \}
\end{alltt}
\end{small}

The two if statements on lines 24 and 28 compare the number of digits in the two inputs.  These two are performed before all of the digits
are compared since it is a very cheap test to perform and can potentially save considerable time.  The implementation given is also not valid 
without those two statements.  $b.alloc$ may be smaller than $a.used$, meaning that undefined values will be read from $b$ passed the end of the 
array of digits.

\subsection{Signed Comparisons}
Comparing with sign considerations is also fairly critical in several routines (\textit{division for example}).  Based on an unsigned magnitude 
comparison a trivial signed comparison algorithm can be written.

\newpage\begin{figure}[here]
\begin{center}
\begin{tabular}{l}
\hline Algorithm \textbf{mp\_cmp}. \\
\textbf{Input}.   Two mp\_ints $a$ and $b$ \\
\textbf{Output}.  Signed Comparison Results ($a$ to the left of $b$) \\
\hline \\
1.  if $a.sign = MP\_NEG$ and $b.sign = MP\_ZPOS$ then return(\textit{MP\_LT}) \\
2.  if $a.sign = MP\_ZPOS$ and $b.sign = MP\_NEG$ then return(\textit{MP\_GT}) \\
3.  if $a.sign = MP\_NEG$ then \\
\hspace{+3mm}3.1  Return the unsigned comparison of $b$ and $a$ (\textit{hint: use mp\_cmp\_mag}) \\
4   Otherwise \\
\hspace{+3mm}4.1  Return the unsigned comparison of $a$ and $b$ \\
\hline
\end{tabular}
\end{center}
\caption{Algorithm mp\_cmp}
\end{figure}

\textbf{Algorithm mp\_cmp.}
The first two steps compare the signs of the two inputs.  If the signs do not agree then it can return right away with the appropriate 
comparison code.  When the signs are equal the digits of the inputs must be compared to determine the correct result.  In step 
three the unsigned comparision flips the order of the arguments since they are both negative.  For instance, if $-a > -b$ then 
$\vert a \vert < \vert b \vert$.  Step number four will compare the two when they are both positive.

\index{bn\_mp\_cmp.c}
\vspace{+3mm}\begin{small}
\hspace{-5.1mm}{\bf File}: bn\_mp\_cmp.c
\vspace{-3mm}
\begin{alltt}
016   
017   /* compare two ints (signed)*/
018   int
019   mp_cmp (mp_int * a, mp_int * b)
020   \{
021     /* compare based on sign */
022     if (a->sign == MP_NEG && b->sign == MP_ZPOS) \{
023       return MP_LT;
024     \} 
025     
026     if (a->sign == MP_ZPOS && b->sign == MP_NEG) \{
027       return MP_GT;
028     \}
029     
030     /* compare digits */
031     if (a->sign == MP_NEG) \{
032        /* if negative compare opposite direction */
033        return mp_cmp_mag(b, a);
034     \} else \{
035        return mp_cmp_mag(a, b);
036     \}
037   \}
\end{alltt}
\end{small}

The two if statements on lines 22 and 26 perform the initial sign comparison.  If the signs are not the equal then which ever
has the positive sign is larger.   At line 31, the inputs are compared based on magnitudes.  If the signs were both negative then 
the unsigned comparison is performed in the opposite direction (\textit{line 33}).  Otherwise, the signs are assumed to 
be both positive and a forward direction unsigned comparison is performed.

\section*{Exercises}
\begin{tabular}{cl}
$\left [ 2 \right ]$ & Modify algorithm mp\_set\_int to accept as input a variable length array of bits. \\
                     & \\
$\left [ 3 \right ]$ & Give the probability that algorithm mp\_cmp\_mag will have to compare $k$ digits  \\
                     & of two random digits (of equal magnitude) before a difference is found. \\
                     & \\
$\left [ 1 \right ]$ & Suggest a simple method to speed up the implementation of mp\_cmp\_mag based  \\
                     & on the observations made in the previous problem. \\
                     &
\end{tabular}

\chapter{Basic Arithmetic}
\section{Building Blocks}
At this point algorithms for initialization, de-initialization, zeroing, copying, comparing and setting small constants have been 
established.  The next logical set of algorithms to develop are the addition, subtraction and digit movement algorithms.  These 
algorithms make use of the lower level algorithms and are the cruicial building block for the multipliers.  It is very important that these 
algorithms are highly optimized.  On their own they are simple $O(n)$ algorithms but they can be called from higher level algorithms 
which easily places them at $O(n^2)$ or even $O(n^3)$ work levels.  

All nine algorithms within this chapter make use of the logical bit shift operations denoted by $<<$ and $>>$ for left and right 
logical shifts respectively.  A logical shift is analogous to sliding the decimal point of radix-10 representations.  For example, the real 
number $0.9345$ is equivalent to $93.45\%$ which is found by sliding the the decimal two places to the right (\textit{multiplying by $10^2$}).  
Mathematically a logical shift is equivalent to a division or multiplication by a power of two.  
For example, $a << k = a \cdot 2^k$ while $a >> k = \lfloor a/2^k \rfloor$.

One significant difference between a logical shift and the way decimals are shifted is that digits below the zero'th position are removed
from the number.  For example, consider $1101_2 >> 1$ using decimal notation this would produce $110.1_2$.  However, with a logical shift the 
result is $110_2$.  

\section{Addition and Subtraction}
In normal fixed precision arithmetic negative numbers are easily represented by subtraction from the modulus.  For example, with 32-bit integers
$a - b\mbox{ (mod }2^{32}\mbox{)}$ is the same as $a + (2^{32} - b) \mbox{ (mod }2^{32}\mbox{)}$  since $2^{32} \equiv 0 \mbox{ (mod }2^{32}\mbox{)}$.  
As a result subtraction can be performed with a trivial series of logical operations and an addition.

However, in multiple precision arithmetic negative numbers are not represented in the same way.  Instead a sign flag is used to keep track of the
sign of the integer.  As a result signed addition and subtraction are actually implemented as conditional usage of lower level addition or 
subtraction algorithms with the sign fixed up appropriately.

The lower level algorithms will add or subtract integers without regard to the sign flag.  That is they will add or subtract the magnitude of
the integers respectively.

\subsection{Low Level Addition}
An unsigned addition of multiple precision integers is performed with the same long-hand algorithm used to add decimal numbers.  That is to add the 
trailing digits first and propagate the resulting carry upwards.  Since this is a lower level algorithm the name will have a ``s\_'' prefix.  
Historically that convention stems from the MPI library where ``s\_'' stood for static functions that were hidden from the developer entirely.

\newpage
\begin{figure}[!here]
\begin{center}
\begin{small}
\begin{tabular}{l}
\hline Algorithm \textbf{s\_mp\_add}. \\
\textbf{Input}.   Two mp\_ints $a$ and $b$ \\
\textbf{Output}.  The unsigned addition $c = \vert a \vert + \vert b \vert$. \\
\hline \\
1.  if $a.used > b.used$ then \\
\hspace{+3mm}1.1  $min \leftarrow b.used$ \\
\hspace{+3mm}1.2  $max \leftarrow a.used$ \\
\hspace{+3mm}1.3  $x   \leftarrow a$ \\
2.  else  \\
\hspace{+3mm}2.1  $min \leftarrow a.used$ \\
\hspace{+3mm}2.2  $max \leftarrow b.used$ \\
\hspace{+3mm}2.3  $x   \leftarrow b$ \\
3.  If $c.alloc < max + 1$ then grow $c$ to hold at least $max + 1$ digits (\textit{hint: use mp\_grow}) \\
4.  If failed to grow $c$ return(\textit{MP\_MEM}) \\
5.  $oldused \leftarrow c.used$ \\
6.  $c.used \leftarrow max + 1$ \\
7.  $u \leftarrow 0$ \\
8.  for $n$ from $0$ to $min - 1$ do \\
\hspace{+3mm}8.1  $c_n \leftarrow a_n + b_n + u$ \\
\hspace{+3mm}8.2  $u \leftarrow c_n >> lg(\beta)$ \\
\hspace{+3mm}8.3  $c_n \leftarrow c_n \mbox{ (mod }\beta\mbox{)}$ \\
9.  if $min \ne max$ then do \\
\hspace{+3mm}9.1  for $n$ from $min$ to $max - 1$ do \\
\hspace{+6mm}9.1.1  $c_n \leftarrow x_n + u$ \\
\hspace{+6mm}9.1.2  $u \leftarrow c_n >> lg(\beta)$ \\
\hspace{+6mm}9.1.3  $c_n \leftarrow c_n \mbox{ (mod }\beta\mbox{)}$ \\
10.  $c_{max} \leftarrow u$ \\
11.  if $olduse > max$ then \\
\hspace{+3mm}11.1  for $n$ from $max + 1$ to $olduse - 1$ do \\
\hspace{+6mm}11.1.1  $c_n \leftarrow 0$ \\
12.  Clamp excess digits in $c$.  (\textit{hint: use mp\_clamp}) \\
13.  Return(\textit{MP\_OKAY}) \\
\hline
\end{tabular}
\end{small}
\end{center}
\caption{Algorithm s\_mp\_add}
\end{figure}

\textbf{Algorithm s\_mp\_add.}
This algorithm is loosely based on algorithm 14.7 of \cite[pp. 594]{HAC} but has been extended to allow the inputs to have different magnitudes.  
Coincidentally the description of algorithm A in \cite[pp. 266]{TAOCPV2} shares the same flaw as that from \cite{HAC}.  Even the MIX pseudo 
machine code presented  \cite[pp. 266-267]{TAOCPV2} is incapable of handling inputs which are of different magnitudes.

Steps 1 and 2 will sort the two inputs based on their \textbf{used} digit count.  This allows the inputs to have varying magnitudes which not 
only makes it more efficient than the trivial algorithm presented in the other references but more flexible.  The variable $min$ is given the lowest 
digit count while $max$ is given the highest digit count.  If both inputs have the same \textbf{used} digit count both $min$ and $max$ are 
set to the same.  The variable $x$ is an \textit{alias} for the largest input and not meant to be a copy of it.  After the inputs are sorted steps 
3 and 4 will ensure that the destination $c$ can accommodate the result.  The old \textbf{used} count from $c$ is copied to $oldused$ and the 
new count is set to $max + 1$.  

At step 7 the carry variable $u$ is set to zero and the first leg of the addition loop can begin.  The first step of the loop (\textit{8.1}) adds
digits from the two inputs together along with the carry variable $u$.  The following step extracts the carry bit by shifting the result of the
preceding step right $lg(\beta)$ positions.  The shift to extract the carry is similar to how carry extraction works with decimal addition.

Consider adding $77$ to $65$, the first addition of the first column is $7 + 5$ which produces the result $12$.  The trailing digit of the result
is $2 \equiv 12 \mbox{ (mod }10\mbox{)}$ and the carry is found by dividing (\textit{and ignoring the remainder}) $12$ by the radix or in this case $10$.  The
division and multiplication of $10$ is simply a logical shift right or left respectively of the digits.  In otherwords the carry can be extracted
by shifting one digit to the right.

Note that $lg()$ is simply the base two logarithm such that $lg(2^k) = k$.  This implies that $lg(\beta)$ is the number of bits in a radix-$\beta$ 
digit.  Therefore, a logical shift right of the single digit by $lg(\beta)$ will extract the carry.  The final step of the  loop reduces the digit 
modulo the radix $\beta$ to ensure it is in range.

After step 8 the smallest input (\textit{or both if they are the same magnitude}) has been exhausted.  Step 9 decides whether
the inputs were of equal magnitude.  If not than another loop similar to that in step 8 must be executed.  The loop at step
number 9.1 differs from the previous loop since it only adds the mp\_int $x$ along with the carry.  

Step 10 finishes the addition phase by copying the final carry to the highest location in the result $c_{max}$.  Step 11 ensures that 
leading digits that were originally present in $c$ are cleared.  Finally excess leading digits are clamped and the algorithm returns success.

\index{bn\_s\_mp\_add.c}
\vspace{+3mm}\begin{small}
\hspace{-5.1mm}{\bf File}: bn\_s\_mp\_add.c
\vspace{-3mm}
\begin{alltt}
016   
017   /* low level addition, based on HAC pp.594, Algorithm 14.7 */
018   int
019   s_mp_add (mp_int * a, mp_int * b, mp_int * c)
020   \{
021     mp_int *x;
022     int     olduse, res, min, max;
023   
024     /* find sizes, we let |a| <= |b| which means we have to sort
025      * them.  "x" will point to the input with the most digits
026      */
027     if (a->used > b->used) \{
028       min = b->used;
029       max = a->used;
030       x = a;
031     \} else \{
032       min = a->used;
033       max = b->used;
034       x = b;
035     \}
036   
037     /* init result */
038     if (c->alloc < max + 1) \{
039       if ((res = mp_grow (c, max + 1)) != MP_OKAY) \{
040         return res;
041       \}
042     \}
043   
044     /* get old used digit count and set new one */
045     olduse = c->used;
046     c->used = max + 1;
047   
048     /* set the carry to zero */
049     \{
050       register mp_digit u, *tmpa, *tmpb, *tmpc;
051       register int i;
052   
053       /* alias for digit pointers */
054   
055       /* first input */
056       tmpa = a->dp;
057   
058       /* second input */
059       tmpb = b->dp;
060   
061       /* destination */
062       tmpc = c->dp;
063   
064       /* zero the carry */
065       u = 0;
066       for (i = 0; i < min; i++) \{
067         /* Compute the sum at one digit, T[i] = A[i] + B[i] + U */
068         *tmpc = *tmpa++ + *tmpb++ + u;
069   
070         /* U = carry bit of T[i] */
071         u = *tmpc >> ((mp_digit)DIGIT_BIT);
072   
073         /* take away carry bit from T[i] */
074         *tmpc++ &= MP_MASK;
075       \}
076   
077       /* now copy higher words if any, that is in A+B 
078        * if A or B has more digits add those in 
079        */
080       if (min != max) \{
081         for (; i < max; i++) \{
082           /* T[i] = X[i] + U */
083           *tmpc = x->dp[i] + u;
084   
085           /* U = carry bit of T[i] */
086           u = *tmpc >> ((mp_digit)DIGIT_BIT);
087   
088           /* take away carry bit from T[i] */
089           *tmpc++ &= MP_MASK;
090         \}
091       \}
092   
093       /* add carry */
094       *tmpc++ = u;
095   
096       /* clear digits above oldused */
097       for (i = c->used; i < olduse; i++) \{
098         *tmpc++ = 0;
099       \}
100     \}
101   
102     mp_clamp (c);
103     return MP_OKAY;
104   \}
\end{alltt}
\end{small}

Lines 27 to 35 perform the initial sorting of the inputs and determine the $min$ and $max$ variables.  Note that $x$ is pointer to a 
mp\_int assigned to the largest input, in effect it is a local alias.  Lines 37 to 42 ensure that the destination is grown to 
accomodate the result of the addition. 

Similar to the implementation of mp\_copy this function uses the braced code and local aliases coding style.  The three aliases on 
lines 56, 59 and 62 are the for the two inputs and destination respectively.  These aliases are used to ensure the
compiler does not have to dereference $a$, $b$ or $c$ (respectively) to access the digits of the respective mp\_int.

The initial carry $u$ is cleared on line 65, note that $u$ is of type mp\_digit which ensures type compatibility within the 
implementation.  The initial addition loop begins on line 66 and ends on line 75.  Similarly the conditional addition loop
begins on line 81 and ends on line 90.  The addition is finished with the final carry being stored in $tmpc$ on line 94.  
Note the ``++'' operator on the same line.  After line 94 $tmpc$ will point to the $c.used$'th digit of the mp\_int $c$.  This is useful
for the next loop on lines 97 to 99 which set any old upper digits to zero.

\subsection{Low Level Subtraction}
The low level unsigned subtraction algorithm is very similar to the low level unsigned addition algorithm.  The principle difference is that the
unsigned subtraction algorithm requires the result to be positive.  That is when computing $a - b$ the condition $\vert a \vert \ge \vert b\vert$ must 
be met for this algorithm to function properly.  Keep in mind this low level algorithm is not meant to be used in higher level algorithms directly.  
This algorithm as will be shown can be used to create functional signed addition and subtraction algorithms.


For this algorithm a new variable is required to make the description simpler.  Recall from section 1.3.1 that a mp\_digit must be able to represent
the range $0 \le x < 2\beta$.  It is allowable that a mp\_digit represent a larger range of values.  For this algorithm we will assume that
the variable $\gamma$ represents the number of bits available in a mp\_digit (\textit{this implies $2^{\gamma} > \beta$}).

\newpage\begin{figure}[!here]
\begin{center}
\begin{small}
\begin{tabular}{l}
\hline Algorithm \textbf{s\_mp\_sub}. \\
\textbf{Input}.   Two mp\_ints $a$ and $b$ ($\vert a \vert \ge \vert b \vert$) \\
\textbf{Output}.  The unsigned subtraction $c = \vert a \vert - \vert b \vert$. \\
\hline \\
1.  $min \leftarrow b.used$ \\
2.  $max \leftarrow a.used$ \\
3.  If $c.alloc < max$ then grow $c$ to hold at least $max$ digits.  (\textit{hint: use mp\_grow}) \\
4.  If the reallocation failed return(\textit{MP\_MEM}). \\
5.  $oldused \leftarrow c.used$ \\ 
6.  $c.used \leftarrow max$ \\
7.  $u \leftarrow 0$ \\
8.  for $n$ from $0$ to $min - 1$ do \\
\hspace{3mm}8.1  $c_n \leftarrow a_n - b_n - u$ \\
\hspace{3mm}8.2  $u   \leftarrow c_n >> (\gamma - 1)$ \\
\hspace{3mm}8.3  $c_n \leftarrow c_n \mbox{ (mod }\beta\mbox{)}$ \\
9.  if $min < max$ then do \\
\hspace{3mm}9.1  for $n$ from $min$ to $max - 1$ do \\
\hspace{6mm}9.1.1  $c_n \leftarrow a_n - u$ \\
\hspace{6mm}9.1.2  $u   \leftarrow c_n >> (\gamma - 1)$ \\
\hspace{6mm}9.1.3  $c_n \leftarrow c_n \mbox{ (mod }\beta\mbox{)}$ \\
10. if $oldused > max$ then do \\
\hspace{3mm}10.1  for $n$ from $max$ to $oldused - 1$ do \\
\hspace{6mm}10.1.1  $c_n \leftarrow 0$ \\
11. Clamp excess digits of $c$.  (\textit{hint: use mp\_clamp}). \\
12. Return(\textit{MP\_OKAY}). \\
\hline
\end{tabular}
\end{small}
\end{center}
\caption{Algorithm s\_mp\_sub}
\end{figure}

\textbf{Algorithm s\_mp\_sub.}
This algorithm performs the unsigned subtraction of two mp\_int variables under the restriction that the result must be positive.  That is when
passing variables $a$ and $b$ the condition that $\vert a \vert \ge \vert b \vert$ must be met for the algorithm to function correctly.  This
algorithm is loosely based on algorithm 14.9 \cite[pp. 595]{HAC} and is similar to algorithm S in \cite[pp. 267]{TAOCPV2} as well.  As was the case
of the algorithm s\_mp\_add both other references lack discussion concerning various practical details such as when the inputs differ in magnitude.

The initial sorting of the inputs is trivial in this algorithm since $a$ is guaranteed to have at least the same magnitude of $b$.  Steps 1 and 2 
set the $min$ and $max$ variables.  Unlike the addition routine there is guaranteed to be no carry which means that the final result can be at 
most $max$ digits in length as oppose to $max + 1$.  Similar to the addition algorithm the \textbf{used} count of $c$ is copied locally and 
set to the maximal count for the operation.

The subtraction loop that begins on step 8 is essentially the same as the addition loop of algorithm s\_mp\_add except single precision 
subtraction is used instead.  Note the use of the $\gamma$ variable to extract the carry within the subtraction loops.  Under the assumption
that two's complement single precision arithmetic is used this will successfully extract the carry.  

For example, consider subtracting $0101_2$ from
$0100_2$ where $\gamma = 4$.  The least significant bit will force a carry upwards to the third bit which will be set to zero after the borrow.  After
the very first bit has been subtracted $4 - 1 \equiv 0011_2$ will remain,  When the third bit of $0101_2$ is subtracted from the result it will cause
another carry.  In this case though the carry will be forced to propagate all the way to the most significant bit.  

Recall that $\beta < 2^{\gamma}$.  This means that if a carry does occur it will propagate all the way to the most significant bit.  Therefore a single
logical shift right by $\gamma - 1$ positions is sufficient to extract the carry.  This method of carry extraction may seem awkward but the reason for 
it becomes apparent when the implementation is discussed.  

If $b$ has a smaller magnitude than $a$ then step 9 will force the carry and copy operation to propagate through the larger input $a$ into $c$.  Step
10 will ensure that any leading digits of $c$ above the $max$'th position are zeroed.

\index{bn\_s\_mp\_sub.c}
\vspace{+3mm}\begin{small}
\hspace{-5.1mm}{\bf File}: bn\_s\_mp\_sub.c
\vspace{-3mm}
\begin{alltt}
016   
017   /* low level subtraction (assumes |a| > |b|), HAC pp.595 Algorithm 14.9 */
018   int
019   s_mp_sub (mp_int * a, mp_int * b, mp_int * c)
020   \{
021     int     olduse, res, min, max;
022   
023     /* find sizes */
024     min = b->used;
025     max = a->used;
026   
027     /* init result */
028     if (c->alloc < max) \{
029       if ((res = mp_grow (c, max)) != MP_OKAY) \{
030         return res;
031       \}
032     \}
033     olduse = c->used;
034     c->used = max;
035   
036     /* sub digits from lower part */
037     \{
038       register mp_digit u, *tmpa, *tmpb, *tmpc;
039       register int i;
040   
041       /* alias for digit pointers */
042       tmpa = a->dp;
043       tmpb = b->dp;
044       tmpc = c->dp;
045   
046       /* set carry to zero */
047       u = 0;
048       for (i = 0; i < min; i++) \{
049         /* T[i] = A[i] - B[i] - U */
050         *tmpc = *tmpa++ - *tmpb++ - u;
051   
052         /* U = carry bit of T[i]
053          * Note this saves performing an AND operation since
054          * if a carry does occur it will propagate all the way to the
055          * MSB.  As a result a single shift is required to get the carry
056          */
057         u = *tmpc >> ((mp_digit)(CHAR_BIT * sizeof (mp_digit) - 1));
058   
059         /* Clear carry from T[i] */
060         *tmpc++ &= MP_MASK;
061       \}
062   
063       /* now copy higher words if any, e.g. if A has more digits than B  */
064       for (; i < max; i++) \{
065         /* T[i] = A[i] - U */
066         *tmpc = *tmpa++ - u;
067   
068         /* U = carry bit of T[i] */
069         u = *tmpc >> ((mp_digit)(CHAR_BIT * sizeof (mp_digit) - 1));
070   
071         /* Clear carry from T[i] */
072         *tmpc++ &= MP_MASK;
073       \}
074   
075       /* clear digits above used (since we may not have grown result above) */
      
076       for (i = c->used; i < olduse; i++) \{
077         *tmpc++ = 0;
078       \}
079     \}
080   
081     mp_clamp (c);
082     return MP_OKAY;
083   \}
\end{alltt}
\end{small}

Line 24 and 25 perform the initial hardcoded sorting.  In reality they are only aliases and are only used to make the source easier to 
read.  Again the pointer alias optimization is used within this algorithm.  Lines 42, 43 and 44 initialize the aliases for 
$a$, $b$ and $c$ respectively.

The first subtraction loop occurs on lines 47 through 61.  The theory behind the subtraction loop is exactly the same as that for
the addition loop.  As remarked earlier there is an implementation reason for using the ``awkward'' method of extracting the carry 
(\textit{see line 57}).  The traditional method for extracting the carry would be to shift by $lg(\beta)$ positions and logically AND 
the least significant bit.  The AND operation is required because all of the bits above the $\lg(\beta)$'th bit will be set to one after a carry
occurs from subtraction.  This carry extraction requires two relatively cheap operations to extract the carry.  The other method is to simply 
shift the most significant bit to the least significant bit thus extracting the carry with a single cheap operation.  This optimization only works on
twos compliment machines which is a safe assumption to make.

If $a$ has a higher magnitude than $b$ an additional loop (\textit{see lines 64 through 73}) is required to propagate the carry through
$a$ and copy the result to $c$.  

\subsection{High Level Addition}
Now that both lower level addition and subtraction algorithms have been established an effective high level signed addition algorithm can be
established.  This high level addition algorithm will be what other algorithms and developers will use to perform addition of mp\_int data 
types.  

Recall from section 5.2 that an mp\_int represents an integer with an unsigned mantissa (\textit{the array of digits}) and a \textbf{sign} 
flag.  A high level addition is actually performed as a series of eight seperate cases which can be optimized down to three unique cases.

\newpage\begin{figure}[!here]
\begin{center}
\begin{tabular}{l}
\hline Algorithm \textbf{mp\_add}. \\
\textbf{Input}.   Two mp\_ints $a$ and $b$  \\
\textbf{Output}.  The signed addition $c = a + b$. \\
\hline \\
1.  if $a.sign = b.sign$ then do \\
\hspace{3mm}1.1  $c.sign \leftarrow a.sign$  \\
\hspace{3mm}1.2  $c \leftarrow \vert a \vert + \vert b \vert$ (\textit{hint: use s\_mp\_add})\\
2.  else do \\
\hspace{3mm}2.1  if $\vert a \vert < \vert b \vert$ then do (\textit{hint: use mp\_cmp\_mag})  \\
\hspace{6mm}2.1.1  $c.sign \leftarrow b.sign$ \\
\hspace{6mm}2.1.2  $c \leftarrow \vert b \vert - \vert a \vert$ (\textit{hint: use s\_mp\_sub}) \\
\hspace{3mm}2.2  else do \\
\hspace{6mm}2.2.1  $c.sign \leftarrow a.sign$ \\
\hspace{6mm}2.2.2  $c \leftarrow \vert a \vert - \vert b \vert$ \\
3.  If any of the lower level operations failed return(\textit{MP\_MEM}) \\
4.  Return(\textit{MP\_OKAY}). \\
\hline
\end{tabular}
\end{center}
\caption{Algorithm mp\_add}
\end{figure}

\textbf{Algorithm mp\_add.}
This algorithm performs the signed addition of two mp\_int variables.  There is no reference algorithm to draw upon from either \cite{TAOCPV2} or 
\cite{HAC} since they both only provide unsigned operations.  The algorithm is fairly straightforward but restricted since subtraction can only 
produce positive results.  Consider the following chart of possible inputs.

\begin{figure}[!here]
\begin{small}
\begin{center}
\begin{tabular}{|c|c|c|c|c|}
\hline \textbf{Sign of $a$} & \textbf{Sign of $b$} & \textbf{$\vert a \vert > \vert b \vert $} & \textbf{Unsigned Operation} & \textbf{Result Sign Flag} \\
\hline $+$ & $+$ & Yes & $c = a + b$ & $a.sign$ \\
\hline $+$ & $+$ & No  & $c = a + b$ & $a.sign$ \\
\hline $-$ & $-$ & Yes & $c = a + b$ & $a.sign$ \\
\hline $-$ & $-$ & No  & $c = a + b$ & $a.sign$ \\
\hline &&&&\\

\hline $+$ & $-$ & No  & $c = b - a$ & $b.sign$ \\
\hline $-$ & $+$ & No  & $c = b - a$ & $b.sign$ \\

\hline &&&&\\

\hline $+$ & $-$ & Yes & $c = a - b$ & $a.sign$ \\
\hline $-$ & $+$ & Yes & $c = a - b$ & $a.sign$ \\

\hline
\end{tabular}
\end{center}
\end{small}
\caption{Addition Guide Chart}
\end{figure}

The chart lists all of the eight possible input combinations and is sorted to show that only three specific cases need to be handled.  The 
return code of the unsigned operations at step 1.2, 2.1.2 and 2.2.2 are forwarded to step 3 to check for errors.  This simpliies the description
of the algorithm considerably and best follows how the implementation actually was achieved.

Also note how the \textbf{sign} is set before the unsigned addition or subtraction is performed.  Recall from the descriptions of algorithms
s\_mp\_add and s\_mp\_sub that the mp\_clamp function is used at the end to trim excess digits.  The mp\_clamp algorithm will set the \textbf{sign}
to \textbf{MP\_ZPOS} when the \textbf{used} digit count reaches zero.  

For example, consider performing $-a + a$ with algorithm mp\_add.  By the description of the algorithm the sign is set to \textbf{MP\_NEG} which would
produce a result of $-0$.  However, since the sign is set first then the unsigned addition is performed the subsequent usage of algorithm mp\_clamp 
within algorithm s\_mp\_add will force $-0$ to become $0$.  

\index{bn\_mp\_add.c}
\vspace{+3mm}\begin{small}
\hspace{-5.1mm}{\bf File}: bn\_mp\_add.c
\vspace{-3mm}
\begin{alltt}
016   
017   /* high level addition (handles signs) */
018   int
019   mp_add (mp_int * a, mp_int * b, mp_int * c)
020   \{
021     int     sa, sb, res;
022   
023     /* get sign of both inputs */
024     sa = a->sign;
025     sb = b->sign;
026   
027     /* handle two cases, not four */
028     if (sa == sb) \{
029       /* both positive or both negative */
030       /* add their magnitudes, copy the sign */
031       c->sign = sa;
032       res = s_mp_add (a, b, c);
033     \} else \{
034       /* one positive, the other negative */
035       /* subtract the one with the greater magnitude from */
036       /* the one of the lesser magnitude.  The result gets */
037       /* the sign of the one with the greater magnitude. */
038       if (mp_cmp_mag (a, b) == MP_LT) \{
039         c->sign = sb;
040         res = s_mp_sub (b, a, c);
041       \} else \{
042         c->sign = sa;
043         res = s_mp_sub (a, b, c);
044       \}
045     \}
046     return res;
047   \}
048   
\end{alltt}
\end{small}

The source code follows the algorithm fairly closely.  The most notable new source code addition is the usage of the $res$ integer variable which
is used to pass result of the unsigned operations forward.  Unlike in the algorithm, the variable $res$ is merely returned as is without
explicitly checking it and returning the constant \textbf{MP\_OKAY}.  The observation is this algorithm will succeed or fail only if the lower
level functions do so.  Returning their return code is sufficient.

\subsection{High Level Subtraction}
The high level signed subtraction algorithm is essentially the same as the high level signed addition algorithm.  

\begin{figure}[!here]
\begin{center}
\begin{tabular}{l}
\hline Algorithm \textbf{mp\_sub}. \\
\textbf{Input}.   Two mp\_ints $a$ and $b$  \\
\textbf{Output}.  The signed subtraction $c = a - b$. \\
\hline \\
1.  if $a.sign \ne b.sign$ then do \\
\hspace{3mm}1.1  $c.sign \leftarrow a.sign$ \\
\hspace{3mm}1.2  $c \leftarrow \vert a \vert + \vert b \vert$ (\textit{hint: use s\_mp\_add}) \\
2.  else do \\
\hspace{3mm}2.1  if $\vert a \vert \ge \vert b \vert$ then do (\textit{hint: use mp\_cmp\_mag}) \\
\hspace{6mm}2.1.1  $c.sign \leftarrow a.sign$ \\
\hspace{6mm}2.1.2  $c \leftarrow \vert a \vert  - \vert b \vert$ (\textit{hint: use s\_mp\_sub}) \\
\hspace{3mm}2.2  else do \\
\hspace{6mm}2.2.1  $c.sign \leftarrow  \left \lbrace \begin{array}{ll}
                              MP\_ZPOS &  \mbox{if }a.sign = MP\_NEG \\
                              MP\_NEG  &  \mbox{otherwise} \\
                              \end{array} \right .$ \\
\hspace{6mm}2.2.2  $c \leftarrow \vert b \vert  - \vert a \vert$ \\
3.  If any of the lower level operations failed return(\textit{MP\_MEM}). \\
4.  Return(\textit{MP\_OKAY}). \\
\hline
\end{tabular}
\end{center}
\caption{Algorithm mp\_sub}
\end{figure}

\textbf{Algorithm mp\_sub.}
This algorithm performs the signed subtraction of two inputs.  Similar to algorithm mp\_add there is no reference in either \cite{TAOCPV2} or 
\cite{HAC}.  Also this algorithm is restricted by algorithm s\_mp\_sub.  The following chart lists the eight possible inputs and
the operations required.

\newpage\begin{figure}[!here]
\begin{small}
\begin{center}
\begin{tabular}{|c|c|c|c|c|}
\hline \textbf{Sign of $a$} & \textbf{Sign of $b$} & \textbf{$\vert a \vert \ge \vert b \vert $} & \textbf{Unsigned Operation} & \textbf{Result Sign Flag} \\
\hline $+$ & $-$ & Yes & $c = a + b$ & $a.sign$ \\
\hline $+$ & $-$ & No  & $c = a + b$ & $a.sign$ \\
\hline $-$ & $+$ & Yes & $c = a + b$ & $a.sign$ \\
\hline $-$ & $+$ & No  & $c = a + b$ & $a.sign$ \\
\hline &&&& \\
\hline $+$ & $+$ & Yes & $c = a - b$ & $a.sign$ \\
\hline $-$ & $-$ & Yes & $c = a - b$ & $a.sign$ \\
\hline &&&& \\
\hline $+$ & $+$ & No  & $c = b - a$ & $\mbox{opposite of }a.sign$ \\
\hline $-$ & $-$ & No  & $c = b - a$ & $\mbox{opposite of }a.sign$ \\
\hline
\end{tabular}
\end{center}
\end{small}
\caption{Subtraction Guide Chart}
\end{figure}

Similar to the case of algorithm mp\_add the \textbf{sign} is set first before the unsigned addition or subtraction.  That is to prevent the 
algorithm from producing $-a - -a = -0$ as a result.  

\index{bn\_mp\_sub.c}
\vspace{+3mm}\begin{small}
\hspace{-5.1mm}{\bf File}: bn\_mp\_sub.c
\vspace{-3mm}
\begin{alltt}
016   
017   /* high level subtraction (handles signs) */
018   int
019   mp_sub (mp_int * a, mp_int * b, mp_int * c)
020   \{
021     int     sa, sb, res;
022   
023     sa = a->sign;
024     sb = b->sign;
025   
026     if (sa != sb) \{
027       /* subtract a negative from a positive, OR */
028       /* subtract a positive from a negative. */
029       /* In either case, ADD their magnitudes, */
030       /* and use the sign of the first number. */
031       c->sign = sa;
032       res = s_mp_add (a, b, c);
033     \} else \{
034       /* subtract a positive from a positive, OR */
035       /* subtract a negative from a negative. */
036       /* First, take the difference between their */
037       /* magnitudes, then... */
038       if (mp_cmp_mag (a, b) != MP_LT) \{
039         /* Copy the sign from the first */
040         c->sign = sa;
041         /* The first has a larger or equal magnitude */
042         res = s_mp_sub (a, b, c);
043       \} else \{
044         /* The result has the *opposite* sign from */
045         /* the first number. */
046         c->sign = (sa == MP_ZPOS) ? MP_NEG : MP_ZPOS;
047         /* The second has a larger magnitude */
048         res = s_mp_sub (b, a, c);
049       \}
050     \}
051     return res;
052   \}
053   
\end{alltt}
\end{small}

Much like the implementation of algorithm mp\_add the variable $res$ is used to catch the return code of the unsigned addition or subtraction operations
and forward it to the end of the function.  On line 38 the ``not equal to'' \textbf{MP\_LT} expression is used to emulate a 
``greater than or equal to'' comparison.  

\section{Bit and Digit Shifting}
It is quite common to think of a multiple precision integer as a polynomial in $x$, that is $y = f(\beta)$ where $f(x) = \sum_{i=0}^{n-1} a_i x^i$.  
This notation arises within discussion of Montgomery and Diminished Radix Reduction as well as Karatsuba multiplication and squaring.  

In order to facilitate operations on polynomials in $x$ as above a series of simple ``digit'' algorithms have to be established.  That is to shift
the digits left or right as well to shift individual bits of the digits left and right.  It is important to note that not all ``shift'' operations
are on radix-$\beta$ digits.  

\subsection{Multiplication by Two}

In a binary system where the radix is a power of two multiplication by two not only arises often in other algorithms it is a fairly efficient 
operation to perform.  A single precision logical shift left is sufficient to multiply a single digit by two.  

\newpage\begin{figure}[!here]
\begin{small}
\begin{center}
\begin{tabular}{l}
\hline Algorithm \textbf{mp\_mul\_2}. \\
\textbf{Input}.   One mp\_int $a$ \\
\textbf{Output}.  $b = 2a$. \\
\hline \\
1.  If $b.alloc < a.used + 1$ then grow $b$ to hold $a.used + 1$ digits.  (\textit{hint: use mp\_grow}) \\
2.  If the reallocation failed return(\textit{MP\_MEM}). \\
3.  $oldused \leftarrow b.used$ \\
4.  $b.used \leftarrow a.used$ \\
5.  $r \leftarrow 0$ \\
6.  for $n$ from 0 to $a.used - 1$ do \\
\hspace{3mm}6.1  $rr \leftarrow a_n >> (lg(\beta) - 1)$ \\
\hspace{3mm}6.2  $b_n \leftarrow (a_n << 1) + r \mbox{ (mod }\beta\mbox{)}$ \\
\hspace{3mm}6.3  $r \leftarrow rr$ \\
7.  If $r \ne 0$ then do \\
\hspace{3mm}7.1  $b_{a.used} = 1$ \\
\hspace{3mm}7.2  $b.used \leftarrow b.used + 1$ \\
8.  If $b.used < oldused - 1$ then do \\
\hspace{3mm}8.1  for $n$ from $b.used$ to $oldused - 1$ do \\
\hspace{6mm}8.1.1  $b_n \leftarrow 0$ \\
9.  $b.sign \leftarrow a.sign$ \\
10.  Return(\textit{MP\_OKAY}).\\
\hline
\end{tabular}
\end{center}
\end{small}
\caption{Algorithm mp\_mul\_2}
\end{figure}

\textbf{Algorithm mp\_mul\_2.}
This algorithm will quickly multiply a mp\_int by two provided $\beta$ is a power of two.  Neither \cite{TAOCPV2} nor \cite{HAC} describe such 
an algorithm despite the fact it arises often in other algorithms.  The algorithm is setup much like the lower level algorithm s\_mp\_add since 
it is for all intents and purposes equivalent to the operation $b = \vert a \vert + \vert a \vert$.  

Step 1 and 2 grow the input as required to accomodate the maximum number of \textbf{used} digits in the result.  The initial \textbf{used} count
is set to $a.used$ at step 4.  Only if there is a final carry will the \textbf{used} count require adjustment.

Step 6 is an optimization implementation of the addition loop for this specific case.  That is since the two values being added together 
are the same there is no need to perform two reads from the digits of $a$.  Step 6.1 performs a single precision shift on the current digit $a_n$ to
obtain what will be the carry for the next iteration.  Step 6.2 calculates the $n$'th digit of the result as single precision shift of $a_n$ plus
the previous carry.  Recall from section 5.1 that $a_n << 1$ is equivalent to $a_n \cdot 2$.  An iteration of the addition loop is finished with 
forwarding the carry to the next iteration.

Step 7 takes care of any final carry by setting the $a.used$'th digit of the result to one and augmenting the \textbf{used} count.  Step 8 clears
any original leading digits of $b$.

\index{bn\_mp\_mul\_2.c}
\vspace{+3mm}\begin{small}
\hspace{-5.1mm}{\bf File}: bn\_mp\_mul\_2.c
\vspace{-3mm}
\begin{alltt}
016   
017   /* b = a*2 */
018   int
019   mp_mul_2 (mp_int * a, mp_int * b)
020   \{
021     int     x, res, oldused;
022   
023     /* grow to accomodate result */
024     if (b->alloc < a->used + 1) \{
025       if ((res = mp_grow (b, a->used + 1)) != MP_OKAY) \{
026         return res;
027       \}
028     \}
029   
030     oldused = b->used;
031     b->used = a->used;
032   
033     \{
034       register mp_digit r, rr, *tmpa, *tmpb;
035   
036       /* alias for source */
037       tmpa = a->dp;
038       
039       /* alias for dest */
040       tmpb = b->dp;
041   
042       /* carry */
043       r = 0;
044       for (x = 0; x < a->used; x++) \{
045       
046         /* get what will be the *next* carry bit from the 
047          * MSB of the current digit 
048          */
049         rr = *tmpa >> ((mp_digit)(DIGIT_BIT - 1));
050         
051         /* now shift up this digit, add in the carry [from the previous] */
052         *tmpb++ = ((*tmpa++ << ((mp_digit)1)) | r) & MP_MASK;
053         
054         /* copy the carry that would be from the source 
055          * digit into the next iteration 
056          */
057         r = rr;
058       \}
059   
060       /* new leading digit? */
061       if (r != 0) \{
062         /* add a MSB which is always 1 at this point */
063         *tmpb = 1;
064         ++b->used;
065       \}
066   
067       /* now zero any excess digits on the destination 
068        * that we didn't write to 
069        */
070       tmpb = b->dp + b->used;
071       for (x = b->used; x < oldused; x++) \{
072         *tmpb++ = 0;
073       \}
074     \}
075     b->sign = a->sign;
076     return MP_OKAY;
077   \}
\end{alltt}
\end{small}

This implementation is essentially an optimized implementation of s\_mp\_add for the case of doubling an input.  The only noteworthy difference
is the use of the logical shift operator on line 52 to perform a single precision doubling.  

\subsection{Division by Two}
A division by two can just as easily be accomplished with a logical shift right as multiplication by two can be with a logical shift left.

\newpage\begin{figure}[!here]
\begin{small}
\begin{center}
\begin{tabular}{l}
\hline Algorithm \textbf{mp\_div\_2}. \\
\textbf{Input}.   One mp\_int $a$ \\
\textbf{Output}.  $b = a/2$. \\
\hline \\
1.  If $b.alloc < a.used$ then grow $b$ to hold $a.used$ digits.  (\textit{hint: use mp\_grow}) \\
2.  If the reallocation failed return(\textit{MP\_MEM}). \\
3.  $oldused \leftarrow b.used$ \\
4.  $b.used \leftarrow a.used$ \\
5.  $r \leftarrow 0$ \\
6.  for $n$ from $b.used - 1$ to $0$ do \\
\hspace{3mm}6.1  $rr \leftarrow a_n \mbox{ (mod }2\mbox{)}$\\
\hspace{3mm}6.2  $b_n \leftarrow (a_n >> 1) + (r << (lg(\beta) - 1)) \mbox{ (mod }\beta\mbox{)}$ \\
\hspace{3mm}6.3  $r \leftarrow rr$ \\
7.  If $b.used < oldused - 1$ then do \\
\hspace{3mm}7.1  for $n$ from $b.used$ to $oldused - 1$ do \\
\hspace{6mm}7.1.1  $b_n \leftarrow 0$ \\
8.  $b.sign \leftarrow a.sign$ \\
9.  Return(\textit{MP\_OKAY}).\\
\hline
\end{tabular}
\end{center}
\end{small}
\caption{Algorithm mp\_div\_2}
\end{figure}

\textbf{Algorithm mp\_div\_2.}
This algorithm will divide an mp\_int by two using logical shifts to the right.  Like mp\_mul\_2 it uses a modified low level addition
core as the basis of the algorithm.  Unlike mp\_mul\_2 the shift operations work from the leading digit to the trailing digit.  The algorithm
could be written to work from the trailing digit to the leading digit however, it would have to stop one short of $a.used - 1$ digits to prevent
reading passed the end of the array of digits.

Essentially the loop at step 6 is similar to that of mp\_mul\_2 except the logical shifts go in the opposite direction and the carry is at the 
least significant bit not the most significant bit.  

\index{bn\_mp\_div\_2.c}
\vspace{+3mm}\begin{small}
\hspace{-5.1mm}{\bf File}: bn\_mp\_div\_2.c
\vspace{-3mm}
\begin{alltt}
016   
017   /* b = a/2 */
018   int
019   mp_div_2 (mp_int * a, mp_int * b)
020   \{
021     int     x, res, oldused;
022   
023     /* copy */
024     if (b->alloc < a->used) \{
025       if ((res = mp_grow (b, a->used)) != MP_OKAY) \{
026         return res;
027       \}
028     \}
029   
030     oldused = b->used;
031     b->used = a->used;
032     \{
033       register mp_digit r, rr, *tmpa, *tmpb;
034   
035       /* source alias */
036       tmpa = a->dp + b->used - 1;
037   
038       /* dest alias */
039       tmpb = b->dp + b->used - 1;
040   
041       /* carry */
042       r = 0;
043       for (x = b->used - 1; x >= 0; x--) \{
044         /* get the carry for the next iteration */
045         rr = *tmpa & 1;
046   
047         /* shift the current digit, add in carry and store */
048         *tmpb-- = (*tmpa-- >> 1) | (r << (DIGIT_BIT - 1));
049   
050         /* forward carry to next iteration */
051         r = rr;
052       \}
053   
054       /* zero excess digits */
055       tmpb = b->dp + b->used;
056       for (x = b->used; x < oldused; x++) \{
057         *tmpb++ = 0;
058       \}
059     \}
060     b->sign = a->sign;
061     mp_clamp (b);
062     return MP_OKAY;
063   \}
\end{alltt}
\end{small}

\section{Polynomial Basis Operations}
Recall from section 5.3 that any integer can be represented as a polynomial in $x$ as $y = f(\beta)$.  Such a representation is also known as
the polynomial basis \cite[pp. 48]{ROSE}. Given such a notation a multiplication or division by $x$ amounts to shifting whole digits a single 
place.  The need for such operations arises in several other higher level algorithms such as Barrett and Montgomery reduction, integer
division and Karatsuba multiplication.  

Converting from an array of digits to polynomial basis is very simple.  Consider the integer $y \equiv (a_2, a_1, a_0)_{\beta}$ and recall that
$y = \sum_{i=0}^{2} a_i \beta^i$.  Simply replace $\beta$ with $x$ and the expression is in polynomial basis.  For example, $f(x) = 8x + 9$ is the
polynomial basis representation for $89$ using radix ten.  That is, $f(10) = 8(10) + 9 = 89$.  

\subsection{Multiplication by $x$}

Given a polynomial in $x$ such as $f(x) = a_n x^n + a_{n-1} x^{n-1} + ... + a_0$ multiplying by $x$ amounts to shifting the coefficients up one 
degree.  In this case $f(x) \cdot x = a_n x^{n+1} + a_{n-1} x^n + ... + a_0 x$.  From a scalar basis point of view multiplying by $x$ is equivalent to
multiplying by the integer $\beta$.  

\newpage\begin{figure}[!here]
\begin{small}
\begin{center}
\begin{tabular}{l}
\hline Algorithm \textbf{mp\_lshd}. \\
\textbf{Input}.   One mp\_int $a$ and an integer $b$ \\
\textbf{Output}.  $a \leftarrow a \cdot \beta^b$ (Multiply by $x^b$). \\
\hline \\
1.  If $b \le 0$ then return(\textit{MP\_OKAY}). \\
2.  If $a.alloc < a.used + b$ then grow $a$ to at least $a.used + b$ digits.  (\textit{hint: use mp\_grow}). \\
3.  If the reallocation failed return(\textit{MP\_MEM}). \\
4.  $a.used \leftarrow a.used + b$ \\
5.  $i \leftarrow a.used - 1$ \\
6.  $j \leftarrow a.used - 1 - b$ \\
7.  for $n$ from $a.used - 1$ to $b$ do \\
\hspace{3mm}7.1  $a_{i} \leftarrow a_{j}$ \\
\hspace{3mm}7.2  $i \leftarrow i - 1$ \\
\hspace{3mm}7.3  $j \leftarrow j - 1$ \\
8.  for $n$ from 0 to $b - 1$ do \\
\hspace{3mm}8.1  $a_n \leftarrow 0$ \\
9.  Return(\textit{MP\_OKAY}). \\
\hline
\end{tabular}
\end{center}
\end{small}
\caption{Algorithm mp\_lshd}
\end{figure}

\textbf{Algorithm mp\_lshd.}
This algorithm multiplies an mp\_int by the $b$'th power of $x$.  This is equivalent to multiplying by $\beta^b$.  The algorithm differs 
from the other algorithms presented so far as it performs the operation in place instead storing the result in a seperate location.  The algorithm
will return success immediately if $b \le 0$ since the rest of algorithm is only valid when $b > 0$.  

First the destination $a$ is grown as required to accomodate the result.  The counters $i$ and $j$ are used to form a \textit{sliding window} over
the digits of $a$ of length $b$.  The head of the sliding window is at $i$ (\textit{the leading digit}) and the tail at $j$ (\textit{the trailing digit}).  
The loop on step 7 copies the digit from the tail to the head.  In each iteration the window is moved down one digit.   The last loop on 
step 8 sets the lower $b$ digits to zero.

\newpage
\begin{center}
\begin{figure}[here]
\includegraphics{pics/sliding_window.ps}
\caption{Sliding Window Movement}
\end{figure}
\end{center}

\index{bn\_mp\_lshd.c}
\vspace{+3mm}\begin{small}
\hspace{-5.1mm}{\bf File}: bn\_mp\_lshd.c
\vspace{-3mm}
\begin{alltt}
016   
017   /* shift left a certain amount of digits */
018   int
019   mp_lshd (mp_int * a, int b)
020   \{
021     int     x, res;
022   
023     /* if its less than zero return */
024     if (b <= 0) \{
025       return MP_OKAY;
026     \}
027   
028     /* grow to fit the new digits */
029     if (a->alloc < a->used + b) \{
030        if ((res = mp_grow (a, a->used + b)) != MP_OKAY) \{
031          return res;
032        \}
033     \}
034   
035     \{
036       register mp_digit *tmpa, *tmpaa;
037   
038       /* increment the used by the shift amount than copy upwards */
039       a->used += b;
040   
041       /* top */
042       tmpa = a->dp + a->used - 1;
043   
044       /* base */
045       tmpaa = a->dp + a->used - 1 - b;
046   
047       /* much like mp_rshd this is implemented using a sliding window
048        * except the window goes the otherway around.  Copying from
049        * the bottom to the top.  see bn_mp_rshd.c for more info.
050        */
051       for (x = a->used - 1; x >= b; x--) \{
052         *tmpa-- = *tmpaa--;
053       \}
054   
055       /* zero the lower digits */
056       tmpa = a->dp;
057       for (x = 0; x < b; x++) \{
058         *tmpa++ = 0;
059       \}
060     \}
061     return MP_OKAY;
062   \}
\end{alltt}
\end{small}

The if statement on line 24 ensures that the $b$ variable is greater than zero.  The \textbf{used} count is incremented by $b$ before
the copy loop begins.  This elminates the need for an additional variable in the for loop.  The variable $tmpa$ on line 42 is an alias
for the leading digit while $tmpaa$ on line 45 is an alias for the trailing edge.  The aliases form a window of exactly $b$ digits
over the input.  

\subsection{Division by $x$}

Division by powers of $x$ is easily achieved by shifting the digits right and removing any that will end up to the right of the zero'th digit.  

\newpage\begin{figure}[!here]
\begin{small}
\begin{center}
\begin{tabular}{l}
\hline Algorithm \textbf{mp\_rshd}. \\
\textbf{Input}.   One mp\_int $a$ and an integer $b$ \\
\textbf{Output}.  $a \leftarrow a / \beta^b$ (Divide by $x^b$). \\
\hline \\
1.  If $b \le 0$ then return. \\
2.  If $a.used \le b$ then do \\
\hspace{3mm}2.1  Zero $a$.  (\textit{hint: use mp\_zero}). \\
\hspace{3mm}2.2  Return. \\
3.  $i \leftarrow 0$ \\
4.  $j \leftarrow b$ \\
5.  for $n$ from 0 to $a.used - b - 1$ do \\
\hspace{3mm}5.1  $a_i \leftarrow a_j$ \\
\hspace{3mm}5.2  $i \leftarrow i + 1$ \\
\hspace{3mm}5.3  $j \leftarrow j + 1$ \\
6.  for $n$ from $a.used - b$ to $a.used - 1$ do \\
\hspace{3mm}6.1  $a_n \leftarrow 0$ \\
7.  Clamp excess digits.  (\textit{hint: use mp\_clamp}). \\
8.  Return. \\
\hline
\end{tabular}
\end{center}
\end{small}
\caption{Algorithm mp\_rshd}
\end{figure}

\textbf{Algorithm mp\_rshd.}
This algorithm divides the input in place by the $b$'th power of $x$.  It is analogous to dividing by a $\beta^b$ but much quicker since
it does not require single precision division.  This algorithm does not actually return an error code as it cannot fail.  

If the input $b$ is less than one the algorithm quickly returns without performing any work.  If the \textbf{used} count is less than or equal
to the shift count $b$ then it will simply zero the input and return.

After the trivial cases of inputs have been handled the sliding window is setup.  Much like the case of algorithm mp\_lshd a sliding window that
is $b$ digits wide is used to copy the digits.  Unlike mp\_lshd the window slides in the opposite direction from the trailing to the leading digit.  
Also the digits are copied from the leading to the trailing edge.

Once the window copy is complete the upper digits must be zeroed.  Finally algorithm mp\_clamp is used to trim excess digits.

\index{bn\_mp\_rshd.c}
\vspace{+3mm}\begin{small}
\hspace{-5.1mm}{\bf File}: bn\_mp\_rshd.c
\vspace{-3mm}
\begin{alltt}
016   
017   /* shift right a certain amount of digits */
018   void
019   mp_rshd (mp_int * a, int b)
020   \{
021     int     x;
022   
023     /* if b <= 0 then ignore it */
024     if (b <= 0) \{
025       return;
026     \}
027   
028     /* if b > used then simply zero it and return */
029     if (a->used <= b) \{
030       mp_zero (a);
031       return;
032     \}
033   
034     \{
035       register mp_digit *tmpa, *tmpaa;
036   
037       /* shift the digits down */
038   
039       /* base */
040       tmpa = a->dp;
041   
042       /* offset into digits */
043       tmpaa = a->dp + b;
044   
045       /* this is implemented as a sliding window where 
046        * the window is b-digits long and digits from 
047        * the top of the window are copied to the bottom
048        *
049        * e.g.
050   
051        b-2 | b-1 | b0 | b1 | b2 | ... | bb |   ---->
052                    /\symbol{92}                   |      ---->
053                     \symbol{92}-------------------/      ---->
054        */
055       for (x = 0; x < (a->used - b); x++) \{
056         *tmpa++ = *tmpaa++;
057       \}
058   
059       /* zero the top digits */
060       for (; x < a->used; x++) \{
061         *tmpa++ = 0;
062       \}
063     \}
064     mp_clamp (a);
065   \}
\end{alltt}
\end{small}

The only noteworthy element of this routine is the lack of a return type.  This function cannot fail and as such it is more optimal to not
return anything.

\section{Powers of Two}

Now that algorithms for moving single bits as well as whole digits exist algorithms for moving the ``in between'' distances are required.  For 
example, to quickly multiply by $2^k$ for any $k$ without using a full multiplier algorithm would prove useful.  Instead of performing single
shifts $k$ times to achieve a multiplication by $2^{\pm k}$ a mixture of whole digit shifting and partial digit shifting is employed.  

\subsection{Multiplication by Power of Two}

\newpage\begin{figure}[!here]
\begin{small}
\begin{center}
\begin{tabular}{l}
\hline Algorithm \textbf{mp\_mul\_2d}. \\
\textbf{Input}.   One mp\_int $a$ and an integer $b$ \\
\textbf{Output}.  $c \leftarrow a \cdot 2^b$. \\
\hline \\
1.  $c \leftarrow a$.  (\textit{hint: use mp\_copy}) \\
2.  If $c.alloc < c.used + \lfloor b / lg(\beta) \rfloor + 2$ then grow $c$ accordingly. \\
3.  If the reallocation failed return(\textit{MP\_MEM}). \\
4.  If $b \ge lg(\beta)$ then \\
\hspace{3mm}4.1  $c \leftarrow c \cdot \beta^{\lfloor b / lg(\beta) \rfloor}$ (\textit{hint: use mp\_lshd}). \\
\hspace{3mm}4.2  If step 4.1 failed return(\textit{MP\_MEM}). \\
5.  $d \leftarrow b \mbox{ (mod }lg(\beta)\mbox{)}$ \\
6.  If $d \ne 0$ then do \\
\hspace{3mm}6.1  $mask \leftarrow 2^d$ \\
\hspace{3mm}6.2  $r \leftarrow 0$ \\
\hspace{3mm}6.3  for $n$ from $0$ to $c.used - 1$ do \\
\hspace{6mm}6.3.1  $rr \leftarrow c_n >> (lg(\beta) - d) \mbox{ (mod }mask\mbox{)}$ \\
\hspace{6mm}6.3.2  $c_n \leftarrow (c_n << d) + r \mbox{ (mod }\beta\mbox{)}$ \\
\hspace{6mm}6.3.3  $r \leftarrow rr$ \\
\hspace{3mm}6.4  If $r > 0$ then do \\
\hspace{6mm}6.4.1  $c_{c.used} \leftarrow r$ \\
\hspace{6mm}6.4.2  $c.used \leftarrow c.used + 1$ \\
7.  Return(\textit{MP\_OKAY}). \\
\hline
\end{tabular}
\end{center}
\end{small}
\caption{Algorithm mp\_mul\_2d}
\end{figure}

\textbf{Algorithm mp\_mul\_2d.}
This algorithm multiplies $a$ by $2^b$ and stores the result in $c$.  The algorithm uses algorithm mp\_lshd and a derivative of algorithm mp\_mul\_2 to
quickly compute the product.

First the algorithm will multiply $a$ by $x^{\lfloor b / lg(\beta) \rfloor}$ which will ensure that the remainder multiplicand is less than 
$\beta$.  For example, if $b = 37$ and $\beta = 2^{28}$ then this step will multiply by $x$ leaving a multiplication by $2^{37 - 28} = 2^{9}$ 
left.

The logarithm of the residue is calculated on step 5.  If it is non-zero a modified shift loop is used to calculate the remaining product.  
Essentially the loop is a generic version of algorith mp\_mul2 designed to handle any shift count in the range $1 \le x < lg(\beta)$.  The $mask$
variable is used to extract the upper $d$ bits to form the carry for the next iteration.  

This algorithm is loosely measured as a $O(2n)$ algorithm which means that if the input is $n$-digits that it takes $2n$ ``time'' to 
complete.  It is possible to optimize this algorithm down to a $O(n)$ algorithm at a cost of making the algorithm slightly harder to follow.

\index{bn\_mp\_mul\_2d.c}
\vspace{+3mm}\begin{small}
\hspace{-5.1mm}{\bf File}: bn\_mp\_mul\_2d.c
\vspace{-3mm}
\begin{alltt}
016   
017   /* NOTE:  This routine requires updating.  For instance the c->used = c->all
      oc bit
018      is wrong.  We should just shift c->used digits then set the carry as c->d
      p[c->used] = carry
019    
020      To be fixed for LTM 0.18
021    */
022   
023   /* shift left by a certain bit count */
024   int
025   mp_mul_2d (mp_int * a, int b, mp_int * c)
026   \{
027     mp_digit d;
028     int      res;
029   
030     /* copy */
031     if (a != c) \{
032        if ((res = mp_copy (a, c)) != MP_OKAY) \{
033          return res;
034        \}
035     \}
036   
037     if (c->alloc < (int)(c->used + b/DIGIT_BIT + 2)) \{
038        if ((res = mp_grow (c, c->used + b / DIGIT_BIT + 2)) != MP_OKAY) \{
039          return res;
040        \}
041     \}
042   
043     /* shift by as many digits in the bit count */
044     if (b >= (int)DIGIT_BIT) \{
045       if ((res = mp_lshd (c, b / DIGIT_BIT)) != MP_OKAY) \{
046         return res;
047       \}
048     \}
049     c->used = c->alloc;
050   
051     /* shift any bit count < DIGIT_BIT */
052     d = (mp_digit) (b % DIGIT_BIT);
053     if (d != 0) \{
054       register mp_digit *tmpc, mask, r, rr;
055       register int x;
056   
057       /* bitmask for carries */
058       mask = (((mp_digit)1) << d) - 1;
059   
060       /* alias */
061       tmpc = c->dp;
062   
063       /* carry */
064       r    = 0;
065       for (x = 0; x < c->used; x++) \{
066         /* get the higher bits of the current word */
067         rr = (*tmpc >> (DIGIT_BIT - d)) & mask;
068   
069         /* shift the current word and OR in the carry */
070         *tmpc = ((*tmpc << d) | r) & MP_MASK;
071         ++tmpc;
072   
073         /* set the carry to the carry bits of the current word */
074         r = rr;
075       \}
076     \}
077     mp_clamp (c);
078     return MP_OKAY;
079   \}
\end{alltt}
\end{small}

Notes to be revised when code is updated. -- Tom

\subsection{Division by Power of Two}

\newpage\begin{figure}[!here]
\begin{small}
\begin{center}
\begin{tabular}{l}
\hline Algorithm \textbf{mp\_div\_2d}. \\
\textbf{Input}.   One mp\_int $a$ and an integer $b$ \\
\textbf{Output}.  $c \leftarrow \lfloor a / 2^b \rfloor, d \leftarrow a \mbox{ (mod }2^b\mbox{)}$. \\
\hline \\
1.  If $b \le 0$ then do \\
\hspace{3mm}1.1  $c \leftarrow a$ (\textit{hint: use mp\_copy}) \\
\hspace{3mm}1.2  $d \leftarrow 0$ (\textit{hint: use mp\_zero}) \\
\hspace{3mm}1.3  Return(\textit{MP\_OKAY}). \\
2.  $c \leftarrow a$ \\
3.  $d \leftarrow a \mbox{ (mod }2^b\mbox{)}$ (\textit{hint: use mp\_mod\_2d}) \\
4.  If $b \ge lg(\beta)$ then do \\
\hspace{3mm}4.1  $c \leftarrow \lfloor c/\beta^{\lfloor b/lg(\beta) \rfloor} \rfloor$ (\textit{hint: use mp\_rshd}). \\
5.  $k \leftarrow b \mbox{ (mod }lg(\beta)\mbox{)}$ \\
6.  If $k \ne 0$ then do \\
\hspace{3mm}6.1  $mask \leftarrow 2^k$ \\
\hspace{3mm}6.2  $r \leftarrow 0$ \\
\hspace{3mm}6.3  for $n$ from $c.used - 1$ to $0$ do \\
\hspace{6mm}6.3.1  $rr \leftarrow c_n \mbox{ (mod }mask\mbox{)}$ \\
\hspace{6mm}6.3.2  $c_n \leftarrow (c_n >> k) + (r << (lg(\beta) - k))$ \\
\hspace{6mm}6.3.3  $r \leftarrow rr$ \\
7.  Clamp excess digits of $c$.  (\textit{hint: use mp\_clamp}) \\
8.  Return(\textit{MP\_OKAY}). \\
\hline
\end{tabular}
\end{center}
\end{small}
\caption{Algorithm mp\_div\_2d}
\end{figure}

\textbf{Algorithm mp\_div\_2d.}
This algorithm will divide an input $a$ by $2^b$ and produce the quotient and remainder.  The algorithm is designed much like algorithm 
mp\_mul\_2d by first using whole digit shifts then single precision shifts.  This algorithm will also produce the remainder of the division
by using algorithm mp\_mod\_2d.

\index{bn\_mp\_div\_2d.c}
\vspace{+3mm}\begin{small}
\hspace{-5.1mm}{\bf File}: bn\_mp\_div\_2d.c
\vspace{-3mm}
\begin{alltt}
016   
017   /* shift right by a certain bit count (store quotient in c, remainder in d) 
      */
018   int
019   mp_div_2d (mp_int * a, int b, mp_int * c, mp_int * d)
020   \{
021     mp_digit D, r, rr;
022     int     x, res;
023     mp_int  t;
024   
025   
026     /* if the shift count is <= 0 then we do no work */
027     if (b <= 0) \{
028       res = mp_copy (a, c);
029       if (d != NULL) \{
030         mp_zero (d);
031       \}
032       return res;
033     \}
034   
035     if ((res = mp_init (&t)) != MP_OKAY) \{
036       return res;
037     \}
038   
039     /* get the remainder */
040     if (d != NULL) \{
041       if ((res = mp_mod_2d (a, b, &t)) != MP_OKAY) \{
042         mp_clear (&t);
043         return res;
044       \}
045     \}
046   
047     /* copy */
048     if ((res = mp_copy (a, c)) != MP_OKAY) \{
049       mp_clear (&t);
050       return res;
051     \}
052   
053     /* shift by as many digits in the bit count */
054     if (b >= (int)DIGIT_BIT) \{
055       mp_rshd (c, b / DIGIT_BIT);
056     \}
057   
058     /* shift any bit count < DIGIT_BIT */
059     D = (mp_digit) (b % DIGIT_BIT);
060     if (D != 0) \{
061       register mp_digit *tmpc, mask;
062   
063       /* mask */
064       mask = (((mp_digit)1) << D) - 1;
065   
066       /* alias */
067       tmpc = c->dp + (c->used - 1);
068   
069       /* carry */
070       r = 0;
071       for (x = c->used - 1; x >= 0; x--) \{
072         /* get the lower  bits of this word in a temp */
073         rr = *tmpc & mask;
074   
075         /* shift the current word and mix in the carry bits from the previous 
      word */
076         *tmpc = (*tmpc >> D) | (r << (DIGIT_BIT - D));
077         --tmpc;
078   
079         /* set the carry to the carry bits of the current word found above */
080         r = rr;
081       \}
082     \}
083     mp_clamp (c);
084     res = MP_OKAY;
085     if (d != NULL) \{
086       mp_exch (&t, d);
087     \}
088     mp_clear (&t);
089     return MP_OKAY;
090   \}
\end{alltt}
\end{small}

The implementation of algorithm mp\_div\_2d is slightly different than the algorithm specifies.  The remainder $d$ may be optionally 
ignored by passing \textbf{NULL} as the pointer to the mp\_int variable.    The temporary mp\_int variable $t$ is used to hold the 
result of the remainder operation until the end.  This allows $d = a$ to be true without overwriting the input before they are no longer required.  

The remainder of the source code is essentially the same as the source code for mp\_mul\_2d.  (-- Fix this paragraph up later, Tom).

\subsection{Remainder of Division by Power of Two}

The last algorithm in the series of polynomial basis power of two algorithms is calculating the remainder of division by $2^b$.  This
algorithm benefits from the fact that in twos complement arithmetic $a \mbox{ (mod }2^b\mbox{)}$ is the same as $a$ AND $2^b - 1$.  

\begin{figure}[!here]
\begin{small}
\begin{center}
\begin{tabular}{l}
\hline Algorithm \textbf{mp\_mod\_2d}. \\
\textbf{Input}.   One mp\_int $a$ and an integer $b$ \\
\textbf{Output}.  $c \leftarrow a \mbox{ (mod }2^b\mbox{)}$. \\
\hline \\
1.  If $b \le 0$ then do \\
\hspace{3mm}1.1  $c \leftarrow 0$ (\textit{hint: use mp\_zero}) \\
\hspace{3mm}1.2  Return(\textit{MP\_OKAY}). \\
2.  If $b > a.used \cdot lg(\beta)$ then do \\
\hspace{3mm}2.1  $c \leftarrow a$ (\textit{hint: use mp\_copy}) \\
\hspace{3mm}2.2  Return the result of step 2.1. \\
3.  $c \leftarrow a$ \\
4.  If step 3 failed return(\textit{MP\_MEM}). \\
5.  for $n$ from $\lceil b / lg(\beta) \rceil$ to $c.used$ do \\
\hspace{3mm}5.1  $c_n \leftarrow 0$ \\
6.  $k \leftarrow b \mbox{ (mod }lg(\beta)\mbox{)}$ \\
7.  $c_{\lfloor b / lg(\beta) \rfloor} \leftarrow c_{\lfloor b / lg(\beta) \rfloor} \mbox{ (mod }2^{k}\mbox{)}$. \\
8.  Return(\textit{MP\_OKAY}). \\
\hline
\end{tabular}
\end{center}
\end{small}
\caption{Algorithm mp\_mod\_2d}
\end{figure}

\textbf{Algorithm mp\_mod\_2d.}
This algorithm will quickly calculate the value of $a \mbox{ (mod }2^b\mbox{)}$.  First if $b$ is less than or equal to zero the 
result is set to zero.  If $b$ is greater than the number of bits in $a$ then it simply copies $a$ to $c$ and returns.  Otherwise, $a$ 
is copied to $b$, leading digits are removed and the remaining leading digit is trimed to the exact bit count.

\index{bn\_mp\_mod\_2d.c}
\vspace{+3mm}\begin{small}
\hspace{-5.1mm}{\bf File}: bn\_mp\_mod\_2d.c
\vspace{-3mm}
\begin{alltt}
016   
017   /* calc a value mod 2\b */
018   int
019   mp_mod_2d (mp_int * a, int b, mp_int * c)
020   \{
021     int     x, res;
022   
023   
024     /* if b is <= 0 then zero the int */
025     if (b <= 0) \{
026       mp_zero (c);
027       return MP_OKAY;
028     \}
029   
030     /* if the modulus is larger than the value than return */
031     if (b > (int) (a->used * DIGIT_BIT)) \{
032       res = mp_copy (a, c);
033       return res;
034     \}
035   
036     /* copy */
037     if ((res = mp_copy (a, c)) != MP_OKAY) \{
038       return res;
039     \}
040   
041     /* zero digits above the last digit of the modulus */
042     for (x = (b / DIGIT_BIT) + ((b % DIGIT_BIT) == 0 ? 0 : 1); x < c->used; x+
      +) \{
043       c->dp[x] = 0;
044     \}
045     /* clear the digit that is not completely outside/inside the modulus */
046     c->dp[b / DIGIT_BIT] &=
047       (mp_digit) ((((mp_digit) 1) << (((mp_digit) b) % DIGIT_BIT)) - ((mp_digi
      t) 1));
048     mp_clamp (c);
049     return MP_OKAY;
050   \}
\end{alltt}
\end{small}

-- Add comments later, Tom.

\section*{Exercises}
\begin{tabular}{cl}
$\left [ 3 \right ] $ & Devise an algorithm that performs $a \cdot 2^b$ for generic values of $b$ \\
                      & in $O(n)$ time. \\
                      &\\
$\left [ 3 \right ] $ & Devise an efficient algorithm to multiply by small low hamming  \\
                      & weight values such as $3$, $5$ and $9$.  Extend it to handle all values \\
                      & upto $64$ with a hamming weight less than three. \\
                      &\\
$\left [ 2 \right ] $ & Modify the preceding algorithm to handle values of the form \\
                      & $2^k - 1$ as well. \\
                      &\\
$\left [ 3 \right ] $ & Using only algorithms mp\_mul\_2, mp\_div\_2 and mp\_add create an \\
                      & algorithm to multiply two integers in roughly $O(2n^2)$ time for \\
                      & any $n$-bit input.  Note that the time of addition is ignored in the \\
                      & calculation.  \\
                      & \\
$\left [ 5 \right ] $ & Improve the previous algorithm to have a working time of at most \\
                      & $O \left (2^{(k-1)}n + \left ({2n^2 \over k} \right ) \right )$ for an appropriate choice of $k$.  Again ignore \\
                      & the cost of addition. \\
                      & \\
$\left [ 1 \right ] $ & There exists an improvement on the previous algorithm to \\
                      & slightly reduce the number of additions required.  Modify the \\
                      & previous algorithm to include this improvement. \\
                      & \\
$\left [ 2 \right ] $ & Devise a chart to find optimal values of $k$ for the previous problem \\
                      & for $n = 64 \ldots 1024$ in steps of $64$. \\
                      & \\
$\left [ 2 \right ] $ & Using only algorithms mp\_abs and mp\_sub devise another method for \\
                      & calculating the result of a signed comparison. \\
                      &
\end{tabular}

\chapter{Multiplication and Squaring}
\section{The Multipliers}
For most number theoretic systems including public key cryptographic algorithms the set of algorithms collectively known as the
``multipliers'' form the most important subset of algorithms of any multiple precision integer package.  The set of multipliers 
include multiplication, squaring and modular reduction algorithms.  

The importance of these algorithms is driven by the fact that most popular public key algorithms are based on modular 
exponentiation.  That is performing $d \equiv a^b \mbox{ (mod }c\mbox{)}$ for some arbitrary choice of $a$, $b$, $c$ and $d$.  Roughly
speaking the a modular exponentiation will spend about 40\% of the time in modular reductions, 35\% of the time in squaring and 25\% of
the time in multiplications.  Only a small trivial amount of time is spent on lower level algorithms such as mp\_clamp, mp\_init, etc...

This chapter will discuss only two of the multipliers algorithms, multiplication and squaring.  As will be discussed shortly very efficient
multiplier algorithms are not always straightforward and deserve a lot of attention.

\section{Multiplication}
\subsection{The Baseline Multiplication}
\index{baseline multiplication}
Computing the product of two integers in software can be achieved using a trivial adaptation of the standard $O(n^2)$ long-hand multiplication
algorithm school children are taught.  The ``baseline multiplication'' algorithm is designed to act as the ``catch-all'' algorithm only called
when the faster algorithms cannot be used.  This algorithm does not use any particularly interesting optimizations.

The first algorithm to review is the unsigned multiplication algorithm from which a signed multiplication algorithm can be established.  One important 
facet of this algorithm to note is that it has been modified to only produce a certain amount of output digits as resolution.  Recall that for
a $n$ and $m$ digit input the product will be at most $n + m + 1$ digits.  Therefore, this algorithm can be reduced to a full multiplier by
telling it to produce $n + m + 1$ digits.  

Recall from sub-section 5.2.2 the definition of $\gamma$ as the number of bits in the type \textbf{mp\_digit}.  We shall now extend this variable set to 
include $\alpha$ which shall represent the number of bits in the type \textbf{mp\_word}.  This implies that $2^{\alpha} > 2 \cdot \beta^2$.  The 
constant $\delta = 2^{\alpha - 2lg(\beta)}$ will represent the maximal weight of any column in a product (\textit{see sub-section 6.2.2 for more information}).

\newpage\begin{figure}[!here]
\begin{small}
\begin{center}
\begin{tabular}{l}
\hline Algorithm \textbf{s\_mp\_mul\_digs}. \\
\textbf{Input}.   mp\_int $a$, mp\_int $b$ and an integer $digs$ \\
\textbf{Output}.  $c \leftarrow \vert a \vert \cdot \vert b \vert \mbox{ (mod }\beta^{digs}\mbox{)}$. \\
\hline \\
1.  If min$(a.used, b.used) < \delta$ then do \\
\hspace{3mm}1.1  Calculate $c = \vert a \vert \cdot \vert b \vert$ by the Comba method.  \\
\hspace{3mm}1.2  Return the result of step 1.1 \\
\\
Allocate and initialize a temporary mp\_int. \\
2.  Init $t$ to be of size $digs$ \\
3.  If step 2 failed return(\textit{MP\_MEM}). \\
4.  $t.used \leftarrow digs$ \\
\\
Compute the product. \\
5.  for $ix$ from $0$ to $a.used - 1$ do \\
\hspace{3mm}5.1  $u \leftarrow 0$ \\
\hspace{3mm}5.2  $pb \leftarrow \mbox{min}(b.used, digs - ix)$ \\
\hspace{3mm}5.3  If $pb < 1$ then goto step 6. \\
\hspace{3mm}5.4  for $iy$ from $0$ to $pb - 1$ do \\
\hspace{6mm}5.4.1  $\hat r \leftarrow t_{iy + ix} + a_{ix} \cdot b_{iy} + u$ \\
\hspace{6mm}5.4.2  $t_{iy + ix} \leftarrow \hat r \mbox{ (mod }\beta\mbox{)}$ \\
\hspace{6mm}5.4.3  $u \leftarrow \lfloor \hat r / \beta \rfloor$ \\
\hspace{3mm}5.5  if $ix + iy < digs$ then do \\
\hspace{6mm}5.5.1  $t_{ix + pb} \leftarrow u$ \\
6.  Clamp excess digits of $t$. \\
7.  Swap $c$ with $t$ \\
8.  Clear $t$ \\
9.  Return(\textit{MP\_OKAY}). \\
\hline
\end{tabular}
\end{center}
\end{small}
\caption{Algorithm s\_mp\_mul\_digs}
\end{figure}

\textbf{Algorithm s\_mp\_mul\_digs.}
This algorithm computes the unsigned product of two inputs $a$ and $c$ limited to an output precision of $digs$ digits.  While it may seem
a bit awkward to modify the function from its simple $O(n^2)$ description the usefulness of partial multipliers will arise in a future 
algorithm.  The algorithm is loosely based on algorithm 14.12 from \cite[pp. 595]{HAC} and is similar to Algorithm M \cite[pp. 268]{TAOCPV2}.  The
algorithm differs from those cited references because it can produce a variable output precision regardless of the precision of the inputs.

The first thing this algorithm checks for is whether a Comba multiplier can be used instead.   That is if the minimal digit count of either
input is less than $\delta$ the Comba method is used.    After the Comba method is ruled out the baseline algorithm begins.  A 
temporary mp\_int variable $t$ is used to hold the intermediate result of the product.  This allows the algorithm to be used to 
compute products when either $a = c$ or $b = c$ without overwriting the inputs.  

All of step 5 is the infamous $O(n^2)$ multiplication loop slightly modified to only produce upto $digs$ digits of output.  The $pb$ variable
is given the count of digits to read from $b$ inside the nested loop.  If $pb < 0$ then no more output digits can be produced and the algorithm
will exit the loop.  The best way to think of the loops are as a series of $pb \times 1$ multiplication.    That is, in each pass of the 
innermost loop $a_{ix}$ is multiplied against $b$ and the result is added (\textit{with an appropriate shift}) to $t$.  

For example, consider multiplying $576$ by $241$.  That is equivalent to computing $10^0(1)(576) + 10^1(4)(576) + 10^2(2)(576)$ which is best
visualized as the following table.

\begin{figure}[here]
\begin{center}
\begin{tabular}{|c|c|c|c|c|c|c|}
\hline   &&          & 5 & 7 & 6 & \\
\hline   $\times$&&  & 2 & 4 & 1 & \\
\hline &&&&&&\\
  &&          & 5 & 7 & 6 & $10^0(1)(576)$ \\
  &2 &   3    & 0 & 4 & 0 & $10^1(4)(576)$ \\
  1 & 1 & 5 & 2 & 0 & 0 &  $10^2(2)(576)$ \\
\hline  
\end{tabular}
\end{center}
\caption{Long-Hand Multiplication Diagram}
\end{figure}

Each row of the product is added to the result after being shifted to the left (\textit{multiplied by a power of the radix}) by the appropriate 
count.  That is in pass $ix$ of the inner loop the product is added starting at the $ix$'th digit of the reult.

Step 5.4.1 introduces the hat symbol (\textit{e.g. $\hat x$}) which represents a double precision variable.  The multiplication on that step
is assumed to be a double wide output single precision multiplication.  That is, two single precision variables are multiplied to produce a
double precision result.  The step is somewhat optimized from a long-hand multiplication algorithm because the carry from the addition in step
5.4.1 is forwarded through the nested loop.  If the carry was ignored it would overflow the single precision digit $t_{ix+iy}$ and the result
would be lost.  

At step 5.5 the nested loop is finished and any carry that was left over should be forwarded.  That is provided $ix + iy < digs$ otherwise the
carry is ignored since it will not be part of the result anyways.  

\index{bn\_s\_mp\_mul\_digs.c}
\vspace{+3mm}\begin{small}
\hspace{-5.1mm}{\bf File}: bn\_s\_mp\_mul\_digs.c
\vspace{-3mm}
\begin{alltt}
016   
017   /* multiplies |a| * |b| and only computes upto digs digits of result
018    * HAC pp. 595, Algorithm 14.12  Modified so you can control how 
019    * many digits of output are created.
020    */
021   int
022   s_mp_mul_digs (mp_int * a, mp_int * b, mp_int * c, int digs)
023   \{
024     mp_int  t;
025     int     res, pa, pb, ix, iy;
026     mp_digit u;
027     mp_word r;
028     mp_digit tmpx, *tmpt, *tmpy;
029   
030     /* can we use the fast multiplier? */
031     if (((digs) < MP_WARRAY) &&
032         MIN (a->used, b->used) < 
033             (1 << ((CHAR_BIT * sizeof (mp_word)) - (2 * DIGIT_BIT)))) \{
034       return fast_s_mp_mul_digs (a, b, c, digs);
035     \}
036   
037     if ((res = mp_init_size (&t, digs)) != MP_OKAY) \{
038       return res;
039     \}
040     t.used = digs;
041   
042     /* compute the digits of the product directly */
043     pa = a->used;
044     for (ix = 0; ix < pa; ix++) \{
045       /* set the carry to zero */
046       u = 0;
047   
048       /* limit ourselves to making digs digits of output */
049       pb = MIN (b->used, digs - ix);
050   
051       /* setup some aliases */
052       /* copy of the digit from a used within the nested loop */
053       tmpx = a->dp[ix];
054       
055       /* an alias for the destination shifted ix places */
056       tmpt = t.dp + ix;
057       
058       /* an alias for the digits of b */
059       tmpy = b->dp;
060   
061       /* compute the columns of the output and propagate the carry */
062       for (iy = 0; iy < pb; iy++) \{
063         /* compute the column as a mp_word */
064         r = ((mp_word) *tmpt) + 
065             ((mp_word) tmpx) * ((mp_word) * tmpy++) + 
066             ((mp_word) u);
067   
068         /* the new column is the lower part of the result */
069         *tmpt++ = (mp_digit) (r & ((mp_word) MP_MASK));
070   
071         /* get the carry word from the result */
072         u = (mp_digit) (r >> ((mp_word) DIGIT_BIT));
073       \}
074       /* set carry if it is placed below digs */
075       if (ix + iy < digs) \{
076         *tmpt = u;
077       \}
078     \}
079   
080     mp_clamp (&t);
081     mp_exch (&t, c);
082   
083     mp_clear (&t);
084     return MP_OKAY;
085   \}
\end{alltt}
\end{small}

Lines 31 to 35 determine if the Comba method can be used first.  The conditions for using the Comba routine are that min$(a.used, b.used) < \delta$ and
the number of digits of output is less than \textbf{MP\_WARRAY}.  This new constant is used to control the stack usage in the Comba routines.  By
default it is set to $\delta$ but can be reduced when memory is at a premium.

Of particular importance is the calculation of the $ix+iy$'th column on lines 64, 65 and 66.  Note how all of the
variables are cast to the type \textbf{mp\_word}.  That is to ensure that double precision operations are used instead of single precision.  The
multiplication on line 65 is a bit of a GCC optimization.  On the outset it looks like the compiler will have to use a double precision
multiplication to produce the result required.  Such an operation would be horribly slow on most processors and drag this to a crawl.  However,
GCC is smart enough to realize that double wide output single precision multipliers can be used.  For example, the instruction ``MUL'' on the
x86 processor can multiply two 32-bit values and produce a 64-bit result.  

\subsection{Faster Multiplication by the ``Comba'' Method}

One of the huge drawbacks of the ``baseline'' algorithms is that at the $O(n^2)$ level the carry must be computed and propagated upwards.  This
makes the nested loop very sequential and hard to unroll and implement in parallel.  The ``Comba'' method is named after little known 
(\textit{in cryptographic venues}) Paul G. Comba where in \cite{COMBA} a method of implementing fast multipliers that do not require nested 
carry fixup operations was presented.

At the heart of algorithm is once again the long-hand algorithm for multiplication.  Except in this case a slight twist is placed on how
the columns of the result are produced.  In the standard long-hand algorithm rows of products are produced then added together to form the 
final result.  In the baseline algorithm the columns are added together to get the result instantaneously.  

In the Comba algorithm however, the columns of the result are produced entirely independently of each other.  That is at the $O(n^2)$ level a 
simple multiplication and addition step is performed.  Or more succintly that 

\begin{equation}
x_n = \sum_{i+j = n} a_ib_j
\end{equation}

Where $x_n$ is the $n'th$ column of the output vector.  To see how this works consider once again multiplying $576$ by $241$.  

\begin{figure}[here]
\begin{small}
\begin{center}
\begin{tabular}{|c|c|c|c|c|c|}
  \hline &          & 5 & 7 & 6 & First Input\\
  \hline $\times$ & & 2 & 4 & 1 & Second Input\\
\hline            &                        & $1 \cdot 5 = 5$   & $1 \cdot 7 = 7$   & $1 \cdot 6 = 6$ & First pass \\
                  &  $4 \cdot 5 = 20$      & $4 \cdot 7+5=33$  & $4 \cdot 6+7=31$  & 6               & Second pass \\
   $2 \cdot 5 = 10$ &  $2 \cdot 7 + 20 = 34$ & $2 \cdot 6+33=45$ & 31                & 6             & Third pass \\
\hline 10 & 34 & 45 & 31 & 6 & Final Result \\   
\hline   
\end{tabular}
\end{center}
\end{small}
\caption{Comba Multiplication Diagram}
\end{figure}

At this point the vector $x = \left < 10, 34, 45, 31, 6 \right >$ is the result of the first step of the Comba multipler.  
Now the columns must be fixed by propagating the carry upwards.  The following trivial algorithm will accomplish this.

\begin{enumerate}
    \item for $n$ from 0 to $k - 1$ do
    \item \hspace{3mm} $x_{n+1} \leftarrow x_{n+1} + \lfloor x_{n}/\beta \rfloor$ 
    \item \hspace{3mm} $x_{n} \leftarrow x_{n} \mbox{ (mod }\beta\mbox{)}$
\end{enumerate}

With that algorithm and $k = 5$ and $\beta = 10$ the following vector is produced $y = \left < 1, 3, 8, 8, 1, 6 \right >$.  In this case 
$241 \cdot 576$ is in fact $138816$ and the procedure succeeded.  If the algorithm is correct and as will be demonstrated shortly more
efficient than the baseline algorithm why not simply always use this algorithm?

\subsubsection{Column Weight.}
At the nested $O(n^2)$ level the Comba method adds the product of two single precision variables to a each column of the output 
independently.  A serious obstacle is if the carry is lost due to lack of precision before the algorithm has a chance to fix
the carries.  For example, in the multiplication of two three-digit numbers the third column of output will be the sum of
three single precision multiplications.  If the precision of the accumulator for the output digits is less then $3 \cdot (\beta - 1)^2$ then
an overflow can occur and the carry information will be lost.  For any $m$ and $n$ digit input the maximal weight of any column is 
min$(m, n)$ which is fairly obvious.

The maximal number of terms in any column of a product is known as the ``column weight'' and strictly governs when the algorithm can be used.  Recall
from earlier that a double precision type has $\alpha$ bits of resolution and a single precision digit has $lg(\beta)$ bits of precision.  Given these
two quantities we may not violate the following

\begin{equation}
k \cdot \left (\beta - 1 \right )^2 < 2^{\alpha}
\end{equation}

Which reduces to 

\begin{equation}
k \cdot \left ( \beta^2 - 2\beta + 1 \right ) < 2^{\alpha}
\end{equation}

Let $\rho = lg(\beta)$ represent the number of bits in a single precision digit.  By further re-arrangement of the equation the final solution is
found.

\begin{equation}
k \cdot \left (2^{2\rho} - 2^{\rho + 1} + 1 \right ) < 2^{\alpha}
\end{equation}

The defaults for LibTomMath are $\beta = 2^{28}, \alpha = 2^{64}$ which simplies to $72057593501057025 \cdot k < 2^{64}$ which when divided out
result in $k < 257$.  This implies that the smallest input may not have more than $256$ digits if the Comba method is to be used in
this configuration.  This is quite satisfactory for most applications since $256$ digits would be allow for numbers in the range of $2^{7168}$ 
which is much larger than the typical $2^{100}$ to $2^{4000}$ range most public key cryptographic algorithms use.  

\newpage\begin{figure}[!here]
\begin{small}
\begin{center}
\begin{tabular}{l}
\hline Algorithm \textbf{fast\_s\_mp\_mul\_digs}. \\
\textbf{Input}.   mp\_int $a$, mp\_int $b$ and an integer $digs$ \\
\textbf{Output}.  $c \leftarrow \vert a \vert \cdot \vert b \vert \mbox{ (mod }\beta^{digs}\mbox{)}$. \\
\hline \\
Place an array of \textbf{MP\_WARRAY} double precision digits named $\hat W$ on the stack. \\
1.  If $c.alloc < digs$ then grow $c$ to $digs$ digits. (\textit{hint: use mp\_grow}) \\
2.  If step 1 failed return(\textit{MP\_MEM}).\\
\\
Zero the temporary array $\hat W$. \\
3.  for $n$ from $0$ to $digs - 1$ do \\
\hspace{3mm}3.1  $\hat W_n \leftarrow 0$ \\
\\
Compute the columns. \\
4.  for $ix$ from $0$ to $a.used - 1$ do \\
\hspace{3mm}4.1  $pb \leftarrow \mbox{min}(b.used, digs - ix)$ \\
\hspace{3mm}4.2  If $pb < 1$ then goto step 5. \\
\hspace{3mm}4.3  for $iy$ from $0$ to $pb - 1$ do \\
\hspace{6mm}4.3.1  $\hat W_{ix+iy} \leftarrow \hat W_{ix+iy} + a_{ix}b_{iy}$ \\
\\
Propagate the carries upwards. \\
5.  $oldused \leftarrow c.used$ \\
6.  $c.used \leftarrow digs$ \\
7.  If $digs > 1$ then do \\
\hspace{3mm}7.1.  for $ix$ from $1$ to $digs - 1$ do \\
\hspace{6mm}7.1.1  $\hat W_{ix} \leftarrow \hat W_{ix} + \lfloor \hat W_{ix-1} / \beta \rfloor$ \\
\hspace{6mm}7.1.2  $c_{ix - 1} \leftarrow \hat W_{ix - 1} \mbox{ (mod }\beta\mbox{)}$ \\
8.  else do \\
\hspace{3mm}8.1  $ix \leftarrow 0$ \\
9.  $c_{ix} \leftarrow \hat W_{ix} \mbox{ (mod }\beta\mbox{)}$ \\
\\
Zero excess digits. \\
10.  If $digs < oldused$ then do \\
\hspace{3mm}10.1  for $n$ from $digs$ to $oldused - 1$ do \\
\hspace{6mm}10.1.1  $c_n \leftarrow 0$ \\
11.  Clamp excessive digits of $c$.  (\textit{hint: use mp\_clamp}) \\
12.  Return(\textit{MP\_OKAY}). \\
\hline
\end{tabular}
\end{center}
\end{small}
\caption{Algorithm fast\_s\_mp\_mul\_digs}
\end{figure}

\textbf{Algorithm fast\_s\_mp\_mul\_digs.}
This algorithm performs the unsigned multiplication of $a$ and $b$ using the Comba method limited to $digs$ digits of precision.  The algorithm
essentially peforms the same calculation as algorithm s\_mp\_mul\_digs but much faster.

The array $\hat W$ is meant to be on the stack when the algorithm is used.  The size of the array does not change which is ideal.  Note also that 
unlike algorithm s\_mp\_mul\_digs no temporary mp\_int is required since the result is calculated in place in $\hat W$.  

The $O(n^2)$ loop on step four is where the Comba method starts to show through.  First there is no carry variable in the loop.  Second the
double precision multiply and add step does not have a carry fixup of any sort.  In fact the nested loop is very simple and can be implemented
in parallel.  

What makes the Comba method so attractive is that the carry propagation only takes place outside the $O(n^2)$ nested loop.  For example, if the 
cost in terms of time of a multiply and add is $p$ and the cost of a carry propagation is $q$ then a baseline multiplication would require 
$O \left ((p + q)n^2 \right )$ time to multiply two $n$-digit numbers.  The Comba method only requires $pn^2 + qn$ time, however, in practice 
the speed increase is actually much more.  With $O(n)$ space the algorithm can be reduced to $O(pn + qn)$ time by implementing the $n$ multiply
and add operations in the nested loop in parallel.  

The carry propagation loop on step 7 is fairly straightforward.  It could have been written phased the other direction, that is, to assign
to $c_{ix}$ instead of $c_{ix-1}$ in each iteration.  However, it would still require pre-caution to make sure that $\hat W_{ix+1}$ is not beyond
the \textbf{MP\_WARRAY} words set aside.  

\index{bn\_fast\_s\_mp\_mul\_digs.c}
\vspace{+3mm}\begin{small}
\hspace{-5.1mm}{\bf File}: bn\_fast\_s\_mp\_mul\_digs.c
\vspace{-3mm}
\begin{alltt}
016   
017   /* Fast (comba) multiplier
018    *
019    * This is the fast column-array [comba] multiplier.  It is 
020    * designed to compute the columns of the product first 
021    * then handle the carries afterwards.  This has the effect 
022    * of making the nested loops that compute the columns very
023    * simple and schedulable on super-scalar processors.
024    *
025    * This has been modified to produce a variable number of 
026    * digits of output so if say only a half-product is required 
027    * you don't have to compute the upper half (a feature 
028    * required for fast Barrett reduction).
029    *
030    * Based on Algorithm 14.12 on pp.595 of HAC.
031    *
032    */
033   int
034   fast_s_mp_mul_digs (mp_int * a, mp_int * b, mp_int * c, int digs)
035   \{
036     int     olduse, res, pa, ix;
037     mp_word W[MP_WARRAY];
038   
039     /* grow the destination as required */
040     if (c->alloc < digs) \{
041       if ((res = mp_grow (c, digs)) != MP_OKAY) \{
042         return res;
043       \}
044     \}
045   
046     /* clear temp buf (the columns) */
047     memset (W, 0, sizeof (mp_word) * digs);
048   
049     /* calculate the columns */
050     pa = a->used;
051     for (ix = 0; ix < pa; ix++) \{
052       /* this multiplier has been modified to allow you to 
053        * control how many digits of output are produced.  
054        * So at most we want to make upto "digs" digits of output.
055        *
056        * this adds products to distinct columns (at ix+iy) of W
057        * note that each step through the loop is not dependent on
058        * the previous which means the compiler can easily unroll
059        * the loop without scheduling problems
060        */
061       \{
062         register mp_digit tmpx, *tmpy;
063         register mp_word *_W;
064         register int iy, pb;
065   
066         /* alias for the the word on the left e.g. A[ix] * A[iy] */
067         tmpx = a->dp[ix];
068   
069         /* alias for the right side */
070         tmpy = b->dp;
071   
072         /* alias for the columns, each step through the loop adds a new
073            term to each column
074          */
075         _W = W + ix;
076   
077         /* the number of digits is limited by their placement.  E.g.
078            we avoid multiplying digits that will end up above the # of
079            digits of precision requested
080          */
081         pb = MIN (b->used, digs - ix);
082   
083         for (iy = 0; iy < pb; iy++) \{
084           *_W++ += ((mp_word) tmpx) * ((mp_word) * tmpy++);
085         \}
086       \}
087   
088     \}
089   
090     /* setup dest */
091     olduse = c->used;
092     c->used = digs;
093   
094     \{
095       register mp_digit *tmpc;
096   
097       /* At this point W[] contains the sums of each column.  To get the
098        * correct result we must take the extra bits from each column and
099        * carry them down
100        *
101        * Note that while this adds extra code to the multiplier it 
102        * saves time since the carry propagation is removed from the 
103        * above nested loop.This has the effect of reducing the work 
104        * from N*(N+N*c)==N**2 + c*N**2 to N**2 + N*c where c is the 
105        * cost of the shifting.  On very small numbers this is slower 
106        * but on most cryptographic size numbers it is faster.
107        */
108       tmpc = c->dp;
109       for (ix = 1; ix < digs; ix++) \{
110         W[ix] += (W[ix - 1] >> ((mp_word) DIGIT_BIT));
111         *tmpc++ = (mp_digit) (W[ix - 1] & ((mp_word) MP_MASK));
112       \}
113       *tmpc++ = (mp_digit) (W[digs - 1] & ((mp_word) MP_MASK));
114   
115       /* clear unused */
116       for (; ix < olduse; ix++) \{
117         *tmpc++ = 0;
118       \}
119     \}
120   
121     mp_clamp (c);
122     return MP_OKAY;
123   \}
\end{alltt}
\end{small}

The memset on line 47 clears the initial $\hat W$ array to zero in a single step. Like the slower baseline multiplication
implementation a series of aliases (\textit{lines 67, 70 and 75}) are used to simplify the inner $O(n^2)$ loop.  
In this case a new alias $\_\hat W$ has been added which refers to the double precision columns offset by $ix$ in each pass.  

The inner loop on line 84 is where the algorithm will spend the majority of the time.  Which is why it has been stripped to the 
bones of any extra baggage\footnote{Hence the pointer aliases.}.  On x86 processors the multiply and add amounts to at the very least five
instructions (\textit{two loads, two additions, one multiply}) while on the ARMv4 processors it amounts to only three (\textit{one load, one store,
one multiply-add}).   On both the x86 and ARMv4 processors GCC v3.2 does a very good job at unrolling the loop and scheduling it so there 
are very few dependency stalls.

In theory the difference between the baseline and comba algorithms is a mere $O(qn)$ time difference.  However, in the $O(n^2)$ nested loop of the
baseline method there are dependency stalls as the algorithm must wait for the multiplier to finish before propagating the carry to the next 
digit.  As a result fewer of the often multiple execution units\footnote{The AMD Athlon has three execution units and the Intel P4 has four.} can
be simultaneously used.  

\subsection{Multiplication at New Bounds by Karatsuba Method}
So far two methods of multiplication have been presented.  Both of the algorithms require asymptotically $O(n^2)$ time to multiply two $n$-digit 
numbers together.  While the Comba method is much faster than the baseline algorithm it still requires far too much time to multiply 
large inputs together.  In fact it was not until \cite{KARA} in 1962 that a faster algorithm had been proposed at all.

The idea behind Karatsubas method is that an input can be represented in polynomial basis as two halves then multiplied.  For example, if 
$f(x) = ax + b$ and $g(x) = cx + b$ then the product of the two polynomials $h(x) = f(x)g(x)$ will allow $h(\beta) = (f(\beta))(g(\beta))$.  

So how does this help?  First expand the product $h(x)$.

\begin{center}
\begin{tabular}{rcl}
$h(x)$ & $=$ & $f(x)g(x)$ \\
       & $=$ & $(ax + b)(cx + d)$ \\
       & $=$ & $acx^2 + adx + bcx + bd$ \\
\end{tabular}
\end{center}

The next equation is a bit of genius on the part of Karatsuba.  He proved that the previous equation is equivalent to 

\begin{equation}
h(x) = acx^2 + ((a - c)(b - d) + bd + ac)x + bd
\end{equation}

Essentially the proof lies in some fairly light algebraic number theory (\textit{see \cite{KARAP} for details}) that is not important for
the discussion.  At first glance it appears that the Karatsuba method is actually harder than the straight $O(n^2)$ approach.  
However, further investigation will prove otherwise.  

The first important observation is that both $f(x)$ and $g(x)$ are the polynomial basis representation of two-digit numbers.  This means that 
$\left < a, b, c, d \right >$ are single digit values.  Using either the baseline or straight polynomial multiplication the old method requires
$O \left (4(n/2)^2 \right ) = O(n^2)$ single precision multiplications.  Looking closer at Karatsubas equation there are only three unique multiplications 
required which are $ac$, $bd$ and $(a - c)(b - d)$.  As a result only $O \left (3 \cdot (n/2)^2 \right ) = O \left ( {3 \over 4}n^2 \right )$ 
multiplications are required.  

So far the algorithm has been discussed from the point of view of ``two-digit'' numbers.  However, there is no reason why two digits implies a range of 
$\beta^2$.  It could just as easily represent a range of $\left (\beta^k \right)^2$ as well.  For example, the polynomial 
$f(x) = a_3x^3 + a_2x^2 + a_1x + a_0$ could also be written as $f'(x) = a'_1x + a'_0$ where $f(\beta) = f'(\beta^2)$.  Fortunately representing an
integer which is already in an array of radix-$\beta$ digits in polynomial basis in terms of a power of $\beta$ is very simple.  

\subsubsection{Recursion}
The Karatsuba multiplication algorithm can be applied to practically any size of input.  Therefore, it is possible that the Karatsuba method itself
be used for the three multiplications required.  For example, when multiplying two four-digit numbers there will be three multiplications of two-digit
numbers.  In this case the smaller multiplication requires $p(n) = {3 \over 4}n^2$ time to complete while the larger multiplication requires
$q(n) = 3 \cdot p(n/2)$ multiplications.  

By expanding $q(n)$ the following equation is achieved. 

\begin{center}
\begin{tabular}{rcl}
$q(n)$ & $=$ & $3 \cdot p(n/2)$ \\
       & $=$ & $3 \cdot (3 \cdot ((n/2)/2)^2)$ \\
       & $=$ & $9 \cdot (n/4)^2$ \\
       & $=$ & ${9 \over 16}n^2$ \\
\end{tabular}
\end{center}

The generic expression for the multiplicand is simply $\left ( {3 \over 4} \right )^k$ for $k \ge 1$ recurisions.  The maximal number of recursions
is approximately $lg(n)$.  Putting this all in terms of a base $n$ logarithm the asymptotic running time can be deduced.

\begin{center}
\begin{tabular}{rcl}
$lg_n \left ( \left ( {3 \over 4} \right )^{lg_2 n} \cdot n^2 \right )$ & $=$ & $lg_2 n \cdot lg_n \left ( { 3 \over 4 } \right ) + 2$ \\
                                                                        & $=$ & $\left ( {log N \over log 2} \right ) \cdot \left ( {log \left ( {3 \over 4} \right ) \over log N } \right ) + 2$ \\
                                                                        & $=$ & ${ log 3 - log 2^2 + 2 \cdot log 2} \over log 2$ \\
                                                                        & $=$ & $log 3 \over log 2$ \\
\end{tabular}
\end{center}

Which leads to a running time of $O \left ( n^{lg(3)} \right )$ which is approximately $O(n^{1.584})$.  This can lead to 
impressive savings with fairly moderate sized numbers.  For example, when multiplying two 128-digit numbers the Karatsuba 
method saves $14,197$ (\textit{or $86\%$ of the total}) single precision multiplications.  

The immediate question becomes why not simply use Karatsuba multiplication all the time and forget about the baseline and Comba algorithms? 

\subsubsection{Overhead}
While the Karatsuba method saves on the number of single precision multiplications required this savings is not entirely free.  The product
of three half size products must be stored somewhere as well as four additions and two subtractions performed.  These operations incur sufficient
overhead that often for fairly trivial sized inputs the Karatsuba method is slower.

\index{cutoff point}
The \textit{cutoff point} for Karatsuba multiplication is the point at which the Karatsuba multiplication and baseline (\textit{or Comba}) meet.  
For the purposes of this discussion call this value $x$.  For any input with $n$ digits such that $n < x$ Karatsuba multiplication will be slower 
and for $n > x$ it will be faster.  Often the break between the two algorithms is not so clean cut in reality.  The cleaner the cut the more 
efficient multiplication will be which is why tuning the multiplication is a very important process.  For example, a properly tuned Karatsuba 
multiplication algorithm can multiply two $4,096$ bit numbers up to five times faster on an Athlon processor compared to the standard baseline
algorithm.  

The exact placement of the value of $x$ depends on several key factors.   The cost of allocating storage for the temporary variables, the cost of 
performing the additions and most importantly the cost of performing a single precision multiplication.  With a processor where single precision 
multiplication is fast\footnote{The AMD Athlon for instance has a six cycle multiplier compared to the Intel P4 which has a 15 cycle multiplier.} the 
cutoff point will move upwards.  Similarly with a slower processor the cutoff point will move downwards.  

\newpage\begin{figure}[!here]
\begin{small}
\begin{center}
\begin{tabular}{l}
\hline Algorithm \textbf{mp\_karatsuba\_mul}. \\
\textbf{Input}.   mp\_int $a$ and mp\_int $b$ \\
\textbf{Output}.  $c \leftarrow \vert a \vert \cdot \vert b \vert$ \\
\hline \\
1.  $B \leftarrow \mbox{min}(a.used, b.used)/2$ \\
2.  Init the following mp\_int variables: $x0$, $x1$, $y0$, $y1$, $t1$, $x0y0$, $x1y1$.\\
3.  If step 2 failed then return(\textit{MP\_MEM}). \\
\\
Split the input.  e.g. $a = x1 \cdot \beta^B + x0$ \\
4.  $x0 \leftarrow a \mbox{ (mod }\beta^B\mbox{)}$ (\textit{hint: use mp\_mod\_2d}) \\
5.  $y0 \leftarrow b \mbox{ (mod }\beta^B\mbox{)}$ \\
6.  $x1 \leftarrow \lfloor a / \beta^B \rfloor$ (\textit{hint: use mp\_rshd}) \\
7.  $y1 \leftarrow \lfloor b / \beta^B \rfloor$ \\
\\
Calculate the three products. \\
8.  $x0y0 \leftarrow x0 \cdot y0$ (\textit{hint: use mp\_mul}) \\
9.  $x1y1 \leftarrow x1 \cdot y1$ \\
10.  $t1 \leftarrow x1 - x0$ (\textit{hint: use mp\_sub}) \\
11.  $x0 \leftarrow y1 - y0$ \\
12.  $t1 \leftarrow t1 \cdot x0$ \\
\\
Calculate the middle term. \\
13.  $x0 \leftarrow x0y0 + x1y1$ \\
14.  $t1 \leftarrow x0 - t1$ \\
\\
Calculate the final product. \\
15.  $t1 \leftarrow t1 \cdot \beta^B$ (\textit{hint: use mp\_lshd}) \\
16.  $x1y1 \leftarrow x1y1 \cdot \beta^{2B}$ \\
17.  $t1 \leftarrow x0y0 + t1$ \\
18.  $c \leftarrow t1 + x1y1$ \\
19.  Clear all of the temporary variables. \\
20.  Return(\textit{MP\_OKAY}).\\
\hline 
\end{tabular}
\end{center}
\end{small}
\caption{Algorithm mp\_karatsuba\_mul}
\end{figure}

\textbf{Algorithm mp\_karatsuba\_mul.}


\section{Squaring}
\subsection{The Baseline Squaring Algorithm}
\subsection{Faster Squaring by the ``Comba'' Method}
\subsection{Karatsuba Squaring}
\section{Tuning Algorithms}
\subsection{How to Tune Karatsuba Algorithms}

\chapter{Modular Reductions}
\section{Basics of Modular Reduction}
\section{The Barrett Reduction}
\section{The Montgomery Reduction}
\subsection{Faster ``Comba'' Montgomery Reduction}
\subsection{Example Montgomery Algorithms}
\section{The Diminished Radix Algorithm}
\section{Algorithm Comparison}

\chapter{Exponentiation}
\section{Single Digit Exponentiation}
\section{Modular Exponentiation}
\subsection{General Case}
\subsection{Odd or Diminished Radix Moduli}
\section{Quick Power of Two}

\chapter{Higher Level Algorithms}
\section{Integer Division with Remainder}
\section{Single Digit Helpers}
\subsection{Single Digit Addition}
\subsection{Single Digit Subtraction}
\subsection{Single Digit Multiplication}
\subsection{Single Digit Division}
\subsection{Single Digit Modulo}
\subsection{Single Digit Root Extraction}
\section{Random Number Generation}
\section{Formatted Output}
\subsection{Getting The Output Size}
\subsection{Generating Radix-n Output}
\subsection{Reading Radix-n Input}
\section{Unformatted Output}
\subsection{Getting The Output Size}
\subsection{Generating Output}
\subsection{Reading Input}

\chapter{Number Theoretic Algorithms}
\section{Greatest Common Divisor}
\section{Least Common Multiple}
\section{Jacobi Symbol Computation}
\section{Modular Inverse}
\subsection{General Case}
\subsection{Odd Moduli}
\section{Primality Tests}
\subsection{Trial Division}
\subsection{The Fermat Test}
\subsection{The Miller-Rabin Test}
\subsection{Primality Test in a Bottle}
\subsection{The Next Prime}
\section{Root Extraction}

\backmatter
\appendix
\begin{thebibliography}{ABCDEF}
\bibitem[1]{TAOCPV2}
Donald Knuth, \textit{The Art of Computer Programming}, Third Edition, Volume Two, Seminumerical Algorithms, Addison-Wesley, 1998

\bibitem[2]{HAC}
A. Menezes, P. van Oorschot, S. Vanstone, \textit{Handbook of Applied Cryptography}, CRC Press, 1996

\bibitem[3]{ROSE}
Michael Rosing, \textit{Implementing Elliptic Curve Cryptography}, Manning Publications, 1999

\bibitem[4]{COMBA}
Paul G. Comba, \textit{Exponentiation Cryptosystems on the IBM PC}. IBM Systems Journal 29(4): 526-538 (1990)

\bibitem[5]{KARA}
A. Karatsuba, Doklay Akad. Nauk SSSR 145 (1962), pp.293-294

\bibitem[6]{KARAP}
Andre Weimerskirch and Christof Paar, \textit{Generalizations of the Karatsuba Algorithm for Polynomial Multiplication}, Submitted to Design, Codes and Cryptography, March 2002

\end{thebibliography}

\input{tommath.ind}

\chapter{Appendix}
\subsection*{Appendix A -- Source Listing of tommath.h}

The following is the source listing of the header file ``tommath.h'' for the LibTomMath project.  It contains many of 
the definitions used throughout the code such as \textbf{mp\_int}, \textbf{MP\_PREC} and so on.  The header is 
presented here for completeness.

\index{tommath.h}
\vspace{+3mm}\begin{small}
\hspace{-5.1mm}{\bf File}: tommath.h
\vspace{-3mm}
\begin{alltt}
001   /* LibTomMath, multiple-precision integer library -- Tom St Denis
002    *
003    * LibTomMath is library that provides for multiple-precision
004    * integer arithmetic as well as number theoretic functionality.
005    *
006    * The library is designed directly after the MPI library by
007    * Michael Fromberger but has been written from scratch with
008    * additional optimizations in place.
009    *
010    * The library is free for all purposes without any express
011    * guarantee it works.
012    *
013    * Tom St Denis, tomstdenis@iahu.ca, http://math.libtomcrypt.org
014    */
015   #ifndef BN_H_
016   #define BN_H_
017   
018   #include <stdio.h>
019   #include <string.h>
020   #include <stdlib.h>
021   #include <ctype.h>
022   #include <limits.h>
023   
024   #undef MIN
025   #define MIN(x,y) ((x)<(y)?(x):(y))
026   #undef MAX
027   #define MAX(x,y) ((x)>(y)?(x):(y))
028   
029   #ifdef __cplusplus
030   extern "C" \{
031   
032   /* C++ compilers don't like assigning void * to mp_digit * */
033   #define  OPT_CAST  (mp_digit *)
034   
035   #else
036   
037   /* C on the other hand doesn't care */
038   #define  OPT_CAST
039   
040   #endif
041   
042   /* some default configurations.
043    *
044    * A "mp_digit" must be able to hold DIGIT_BIT + 1 bits
045    * A "mp_word" must be able to hold 2*DIGIT_BIT + 1 bits
046    *
047    * At the very least a mp_digit must be able to hold 7 bits
048    * [any size beyond that is ok provided it doesn't overflow the data type]
049    */
050   #ifdef MP_8BIT
051      typedef unsigned char      mp_digit;
052      typedef unsigned short     mp_word;
053   #elif defined(MP_16BIT)
054      typedef unsigned short     mp_digit;
055      typedef unsigned long      mp_word;
056   #elif defined(MP_64BIT)
057      /* for GCC only on supported platforms */
058   #ifndef CRYPT
059      typedef unsigned long long ulong64;
060      typedef signed long long   long64;
061   #endif
062   
063      typedef ulong64            mp_digit;
064      typedef unsigned long      mp_word __attribute__ ((mode(TI)));
065   
066      #define DIGIT_BIT          60
067   #else
068      /* this is the default case, 28-bit digits */
069      
070      /* this is to make porting into LibTomCrypt easier :-) */
071   #ifndef CRYPT
072      #ifdef _MSC_VER
073         typedef unsigned __int64   ulong64;
074         typedef signed __int64     long64;
075      #else
076         typedef unsigned long long ulong64;
077         typedef signed long long   long64;
078      #endif
079   #endif
080   
081      typedef unsigned long      mp_digit;
082      typedef ulong64            mp_word;
083   
084      #define DIGIT_BIT          28
085   #endif
086   
087   /* otherwise the bits per digit is calculated automatically from the size of
       a mp_digit */
088   #ifndef DIGIT_BIT
089      #define DIGIT_BIT     ((CHAR_BIT * sizeof(mp_digit) - 1))  /* bits per di
      git */
090   #endif
091   
092   
093   #define MP_DIGIT_BIT     DIGIT_BIT
094   #define MP_MASK          ((((mp_digit)1)<<((mp_digit)DIGIT_BIT))-((mp_digit)
      1))
095   #define MP_DIGIT_MAX     MP_MASK
096   
097   /* equalities */
098   #define MP_LT        -1   /* less than */
099   #define MP_EQ         0   /* equal to */
100   #define MP_GT         1   /* greater than */
101   
102   #define MP_ZPOS       0   /* positive integer */
103   #define MP_NEG        1   /* negative */
104   
105   #define MP_OKAY       0   /* ok result */
106   #define MP_MEM        -2  /* out of mem */
107   #define MP_VAL        -3  /* invalid input */
108   #define MP_RANGE      MP_VAL
109   
110   typedef int           mp_err;
111   
112   /* you'll have to tune these... */
113   extern int KARATSUBA_MUL_CUTOFF,
114              KARATSUBA_SQR_CUTOFF,
115              MONTGOMERY_EXPT_CUTOFF;
116   
117   /* various build options */
118   #define MP_PREC                 64      /* default digits of precision (must
       be power of two) */
119   
120   /* define this to use lower memory usage routines (exptmods mostly) */
121   /* #define MP_LOW_MEM */
122   
123   /* size of comba arrays, should be at least 2 * 2**(BITS_PER_WORD - BITS_PER
      _DIGIT*2) */
124   #define MP_WARRAY               (1 << (sizeof(mp_word) * CHAR_BIT - 2 * DIGI
      T_BIT + 1))
125   
126   typedef struct  \{
127       int used, alloc, sign;
128       mp_digit *dp;
129   \} mp_int;
130   
131   #define USED(m)    ((m)->used)
132   #define DIGIT(m,k) ((m)->dp[k])
133   #define SIGN(m)    ((m)->sign)
134   
135   /* ---> init and deinit bignum functions <--- */
136   
137   /* init a bignum */
138   int mp_init(mp_int *a);
139   
140   /* free a bignum */
141   void mp_clear(mp_int *a);
142   
143   /* init a null terminated series of arguments */
144   int mp_init_multi(mp_int *mp, ...);
145   
146   /* clear a null terminated series of arguments */
147   void mp_clear_multi(mp_int *mp, ...);
148   
149   /* exchange two ints */
150   void mp_exch(mp_int *a, mp_int *b);
151   
152   /* shrink ram required for a bignum */
153   int mp_shrink(mp_int *a);
154   
155   /* grow an int to a given size */
156   int mp_grow(mp_int *a, int size);
157   
158   /* init to a given number of digits */
159   int mp_init_size(mp_int *a, int size);
160   
161   /* ---> Basic Manipulations <--- */
162   
163   #define mp_iszero(a) (((a)->used == 0) ? 1 : 0)
164   #define mp_iseven(a) (((a)->used == 0 || (((a)->dp[0] & 1) == 0)) ? 1 : 0)
165   #define mp_isodd(a)  (((a)->used > 0 && (((a)->dp[0] & 1) == 1)) ? 1 : 0)
166   
167   /* set to zero */
168   void mp_zero(mp_int *a);
169   
170   /* set to a digit */
171   void mp_set(mp_int *a, mp_digit b);
172   
173   /* set a 32-bit const */
174   int mp_set_int(mp_int *a, unsigned int b);
175   
176   /* copy, b = a */
177   int mp_copy(mp_int *a, mp_int *b);
178   
179   /* inits and copies, a = b */
180   int mp_init_copy(mp_int *a, mp_int *b);
181   
182   /* trim unused digits */
183   void mp_clamp(mp_int *a);
184   
185   /* ---> digit manipulation <--- */
186   
187   /* right shift by "b" digits */
188   void mp_rshd(mp_int *a, int b);
189   
190   /* left shift by "b" digits */
191   int mp_lshd(mp_int *a, int b);
192   
193   /* c = a / 2**b */
194   int mp_div_2d(mp_int *a, int b, mp_int *c, mp_int *d);
195   
196   /* b = a/2 */
197   int mp_div_2(mp_int *a, mp_int *b);
198   
199   /* c = a * 2**b */
200   int mp_mul_2d(mp_int *a, int b, mp_int *c);
201   
202   /* b = a*2 */
203   int mp_mul_2(mp_int *a, mp_int *b);
204   
205   /* c = a mod 2**d */
206   int mp_mod_2d(mp_int *a, int b, mp_int *c);
207   
208   /* computes a = 2**b */
209   int mp_2expt(mp_int *a, int b);
210   
211   /* makes a pseudo-random int of a given size */
212   int mp_rand(mp_int *a, int digits);
213   
214   /* ---> binary operations <--- */
215   /* c = a XOR b  */
216   int mp_xor(mp_int *a, mp_int *b, mp_int *c);
217   
218   /* c = a OR b */
219   int mp_or(mp_int *a, mp_int *b, mp_int *c);
220   
221   /* c = a AND b */
222   int mp_and(mp_int *a, mp_int *b, mp_int *c);
223   
224   /* ---> Basic arithmetic <--- */
225   
226   /* b = -a */
227   int mp_neg(mp_int *a, mp_int *b);
228   
229   /* b = |a| */
230   int mp_abs(mp_int *a, mp_int *b);
231   
232   /* compare a to b */
233   int mp_cmp(mp_int *a, mp_int *b);
234   
235   /* compare |a| to |b| */
236   int mp_cmp_mag(mp_int *a, mp_int *b);
237   
238   /* c = a + b */
239   int mp_add(mp_int *a, mp_int *b, mp_int *c);
240   
241   /* c = a - b */
242   int mp_sub(mp_int *a, mp_int *b, mp_int *c);
243   
244   /* c = a * b */
245   int mp_mul(mp_int *a, mp_int *b, mp_int *c);
246   
247   /* b = a*a  */
248   int mp_sqr(mp_int *a, mp_int *b);
249   
250   /* a/b => cb + d == a */
251   int mp_div(mp_int *a, mp_int *b, mp_int *c, mp_int *d);
252   
253   /* c = a mod b, 0 <= c < b  */
254   int mp_mod(mp_int *a, mp_int *b, mp_int *c);
255   
256   /* ---> single digit functions <--- */
257   
258   /* compare against a single digit */
259   int mp_cmp_d(mp_int *a, mp_digit b);
260   
261   /* c = a + b */
262   int mp_add_d(mp_int *a, mp_digit b, mp_int *c);
263   
264   /* c = a - b */
265   int mp_sub_d(mp_int *a, mp_digit b, mp_int *c);
266   
267   /* c = a * b */
268   int mp_mul_d(mp_int *a, mp_digit b, mp_int *c);
269   
270   /* a/b => cb + d == a */
271   int mp_div_d(mp_int *a, mp_digit b, mp_int *c, mp_digit *d);
272   
273   /* c = a**b */
274   int mp_expt_d(mp_int *a, mp_digit b, mp_int *c);
275   
276   /* c = a mod b, 0 <= c < b  */
277   int mp_mod_d(mp_int *a, mp_digit b, mp_digit *c);
278   
279   /* ---> number theory <--- */
280   
281   /* d = a + b (mod c) */
282   int mp_addmod(mp_int *a, mp_int *b, mp_int *c, mp_int *d);
283   
284   /* d = a - b (mod c) */
285   int mp_submod(mp_int *a, mp_int *b, mp_int *c, mp_int *d);
286   
287   /* d = a * b (mod c) */
288   int mp_mulmod(mp_int *a, mp_int *b, mp_int *c, mp_int *d);
289   
290   /* c = a * a (mod b) */
291   int mp_sqrmod(mp_int *a, mp_int *b, mp_int *c);
292   
293   /* c = 1/a (mod b) */
294   int mp_invmod(mp_int *a, mp_int *b, mp_int *c);
295   
296   /* c = (a, b) */
297   int mp_gcd(mp_int *a, mp_int *b, mp_int *c);
298   
299   /* c = [a, b] or (a*b)/(a, b) */
300   int mp_lcm(mp_int *a, mp_int *b, mp_int *c);
301   
302   /* finds one of the b'th root of a, such that |c|**b <= |a|
303    *
304    * returns error if a < 0 and b is even
305    */
306   int mp_n_root(mp_int *a, mp_digit b, mp_int *c);
307   
308   /* shortcut for square root */
309   #define mp_sqrt(a, b) mp_n_root(a, 2, b)
310   
311   /* computes the jacobi c = (a | n) (or Legendre if b is prime)  */
312   int mp_jacobi(mp_int *a, mp_int *n, int *c);
313   
314   /* used to setup the Barrett reduction for a given modulus b */
315   int mp_reduce_setup(mp_int *a, mp_int *b);
316   
317   /* Barrett Reduction, computes a (mod b) with a precomputed value c
318    *
319    * Assumes that 0 < a <= b*b, note if 0 > a > -(b*b) then you can merely
320    * compute the reduction as -1 * mp_reduce(mp_abs(a)) [pseudo code].
321    */
322   int mp_reduce(mp_int *a, mp_int *b, mp_int *c);
323   
324   /* setups the montgomery reduction */
325   int mp_montgomery_setup(mp_int *a, mp_digit *mp);
326   
327   /* computes a = B**n mod b without division or multiplication useful for
328    * normalizing numbers in a Montgomery system.
329    */
330   int mp_montgomery_calc_normalization(mp_int *a, mp_int *b);
331   
332   /* computes x/R == x (mod N) via Montgomery Reduction */
333   int mp_montgomery_reduce(mp_int *a, mp_int *m, mp_digit mp);
334   
335   /* returns 1 if a is a valid DR modulus */
336   int mp_dr_is_modulus(mp_int *a);
337   
338   /* sets the value of "d" required for mp_dr_reduce */
339   void mp_dr_setup(mp_int *a, mp_digit *d);
340   
341   /* reduces a modulo b using the Diminished Radix method */
342   int mp_dr_reduce(mp_int *a, mp_int *b, mp_digit mp);
343   
344   /* d = a**b (mod c) */
345   int mp_exptmod(mp_int *a, mp_int *b, mp_int *c, mp_int *d);
346   
347   /* ---> Primes <--- */
348   
349   /* number of primes */
350   #ifdef MP_8BIT
351      #define PRIME_SIZE      31
352   #else
353      #define PRIME_SIZE      256
354   #endif
355   
356   /* table of first PRIME_SIZE primes */
357   extern const mp_digit __prime_tab[];
358   
359   /* result=1 if a is divisible by one of the first PRIME_SIZE primes */
360   int mp_prime_is_divisible(mp_int *a, int *result);
361   
362   /* performs one Fermat test of "a" using base "b".
363    * Sets result to 0 if composite or 1 if probable prime
364    */
365   int mp_prime_fermat(mp_int *a, mp_int *b, int *result);
366   
367   /* performs one Miller-Rabin test of "a" using base "b".
368    * Sets result to 0 if composite or 1 if probable prime
369    */
370   int mp_prime_miller_rabin(mp_int *a, mp_int *b, int *result);
371   
372   /* performs t rounds of Miller-Rabin on "a" using the first
373    * t prime bases.  Also performs an initial sieve of trial
374    * division.  Determines if "a" is prime with probability
375    * of error no more than (1/4)**t.
376    *
377    * Sets result to 1 if probably prime, 0 otherwise
378    */
379   int mp_prime_is_prime(mp_int *a, int t, int *result);
380   
381   /* finds the next prime after the number "a" using "t" trials
382    * of Miller-Rabin.
383    */
384   int mp_prime_next_prime(mp_int *a, int t);
385   
386   
387   /* ---> radix conversion <--- */
388   int mp_count_bits(mp_int *a);
389   
390   int mp_unsigned_bin_size(mp_int *a);
391   int mp_read_unsigned_bin(mp_int *a, unsigned char *b, int c);
392   int mp_to_unsigned_bin(mp_int *a, unsigned char *b);
393   
394   int mp_signed_bin_size(mp_int *a);
395   int mp_read_signed_bin(mp_int *a, unsigned char *b, int c);
396   int mp_to_signed_bin(mp_int *a, unsigned char *b);
397   
398   int mp_read_radix(mp_int *a, char *str, int radix);
399   int mp_toradix(mp_int *a, char *str, int radix);
400   int mp_radix_size(mp_int *a, int radix);
401   
402   int mp_fread(mp_int *a, int radix, FILE *stream);
403   int mp_fwrite(mp_int *a, int radix, FILE *stream);
404   
405   #define mp_read_raw(mp, str, len) mp_read_signed_bin((mp), (str), (len))
406   #define mp_raw_size(mp)           mp_signed_bin_size(mp)
407   #define mp_toraw(mp, str)         mp_to_signed_bin((mp), (str))
408   #define mp_read_mag(mp, str, len) mp_read_unsigned_bin((mp), (str), (len))
409   #define mp_mag_size(mp)           mp_unsigned_bin_size(mp)
410   #define mp_tomag(mp, str)         mp_to_unsigned_bin((mp), (str))
411   
412   #define mp_tobinary(M, S)  mp_toradix((M), (S), 2)
413   #define mp_tooctal(M, S)   mp_toradix((M), (S), 8)
414   #define mp_todecimal(M, S) mp_toradix((M), (S), 10)
415   #define mp_tohex(M, S)     mp_toradix((M), (S), 16)
416   
417   /* lowlevel functions, do not call! */
418   int s_mp_add(mp_int *a, mp_int *b, mp_int *c);
419   int s_mp_sub(mp_int *a, mp_int *b, mp_int *c);
420   #define s_mp_mul(a, b, c) s_mp_mul_digs(a, b, c, (a)->used + (b)->used + 1)
421   int fast_s_mp_mul_digs(mp_int *a, mp_int *b, mp_int *c, int digs);
422   int s_mp_mul_digs(mp_int *a, mp_int *b, mp_int *c, int digs);
423   int fast_s_mp_mul_high_digs(mp_int *a, mp_int *b, mp_int *c, int digs);
424   int s_mp_mul_high_digs(mp_int *a, mp_int *b, mp_int *c, int digs);
425   int fast_s_mp_sqr(mp_int *a, mp_int *b);
426   int s_mp_sqr(mp_int *a, mp_int *b);
427   int mp_karatsuba_mul(mp_int *a, mp_int *b, mp_int *c);
428   int mp_karatsuba_sqr(mp_int *a, mp_int *b);
429   int fast_mp_invmod(mp_int *a, mp_int *b, mp_int *c);
430   int fast_mp_montgomery_reduce(mp_int *a, mp_int *m, mp_digit mp);
431   int mp_exptmod_fast(mp_int *G, mp_int *X, mp_int *P, mp_int *Y, int mode);
432   void bn_reverse(unsigned char *s, int len);
433   
434   #ifdef __cplusplus
435      \}
436   #endif
437   
438   #endif
439   
\end{alltt}
\end{small}

\end{document}

\chapter{Appendix}
\subsection*{Appendix A -- Source Listing of tommath.h}

The following is the source listing of the header file ``tommath.h'' for the LibTomMath project.  It contains many of 
the definitions used throughout the code such as \textbf{mp\_int}, \textbf{MP\_PREC} and so on.  The header is 
presented here for completeness.

\index{tommath.h}
\vspace{+3mm}\begin{small}
\hspace{-5.1mm}{\bf File}: tommath.h
\vspace{-3mm}
\begin{alltt}
001   /* LibTomMath, multiple-precision integer library -- Tom St Denis
002    *
003    * LibTomMath is library that provides for multiple-precision
004    * integer arithmetic as well as number theoretic functionality.
005    *
006    * The library is designed directly after the MPI library by
007    * Michael Fromberger but has been written from scratch with
008    * additional optimizations in place.
009    *
010    * The library is free for all purposes without any express
011    * guarantee it works.
012    *
013    * Tom St Denis, tomstdenis@iahu.ca, http://math.libtomcrypt.org
014    */
015   #ifndef BN_H_
016   #define BN_H_
017   
018   #include <stdio.h>
019   #include <string.h>
020   #include <stdlib.h>
021   #include <ctype.h>
022   #include <limits.h>
023   
024   #undef MIN
025   #define MIN(x,y) ((x)<(y)?(x):(y))
026   #undef MAX
027   #define MAX(x,y) ((x)>(y)?(x):(y))
028   
029   #ifdef __cplusplus
030   extern "C" \{
031   
032   /* C++ compilers don't like assigning void * to mp_digit * */
033   #define  OPT_CAST  (mp_digit *)
034   
035   #else
036   
037   /* C on the other hand doesn't care */
038   #define  OPT_CAST
039   
040   #endif
041   
042   /* some default configurations.
043    *
044    * A "mp_digit" must be able to hold DIGIT_BIT + 1 bits
045    * A "mp_word" must be able to hold 2*DIGIT_BIT + 1 bits
046    *
047    * At the very least a mp_digit must be able to hold 7 bits
048    * [any size beyond that is ok provided it doesn't overflow the data type]
049    */
050   #ifdef MP_8BIT
051      typedef unsigned char      mp_digit;
052      typedef unsigned short     mp_word;
053   #elif defined(MP_16BIT)
054      typedef unsigned short     mp_digit;
055      typedef unsigned long      mp_word;
056   #elif defined(MP_64BIT)
057      /* for GCC only on supported platforms */
058   #ifndef CRYPT
059      typedef unsigned long long ulong64;
060      typedef signed long long   long64;
061   #endif
062   
063      typedef ulong64            mp_digit;
064      typedef unsigned long      mp_word __attribute__ ((mode(TI)));
065   
066      #define DIGIT_BIT          60
067   #else
068      /* this is the default case, 28-bit digits */
069      
070      /* this is to make porting into LibTomCrypt easier :-) */
071   #ifndef CRYPT
072      #ifdef _MSC_VER
073         typedef unsigned __int64   ulong64;
074         typedef signed __int64     long64;
075      #else
076         typedef unsigned long long ulong64;
077         typedef signed long long   long64;
078      #endif
079   #endif
080   
081      typedef unsigned long      mp_digit;
082      typedef ulong64            mp_word;
083   
084      #define DIGIT_BIT          28
085   #endif
086   
087   /* otherwise the bits per digit is calculated automatically from the size of
       a mp_digit */
088   #ifndef DIGIT_BIT
089      #define DIGIT_BIT     ((CHAR_BIT * sizeof(mp_digit) - 1))  /* bits per di
      git */
090   #endif
091   
092   
093   #define MP_DIGIT_BIT     DIGIT_BIT
094   #define MP_MASK          ((((mp_digit)1)<<((mp_digit)DIGIT_BIT))-((mp_digit)
      1))
095   #define MP_DIGIT_MAX     MP_MASK
096   
097   /* equalities */
098   #define MP_LT        -1   /* less than */
099   #define MP_EQ         0   /* equal to */
100   #define MP_GT         1   /* greater than */
101   
102   #define MP_ZPOS       0   /* positive integer */
103   #define MP_NEG        1   /* negative */
104   
105   #define MP_OKAY       0   /* ok result */
106   #define MP_MEM        -2  /* out of mem */
107   #define MP_VAL        -3  /* invalid input */
108   #define MP_RANGE      MP_VAL
109   
110   typedef int           mp_err;
111   
112   /* you'll have to tune these... */
113   extern int KARATSUBA_MUL_CUTOFF,
114              KARATSUBA_SQR_CUTOFF,
115              MONTGOMERY_EXPT_CUTOFF;
116   
117   /* various build options */
118   #define MP_PREC                 64      /* default digits of precision (must
       be power of two) */
119   
120   /* define this to use lower memory usage routines (exptmods mostly) */
121   /* #define MP_LOW_MEM */
122   
123   /* size of comba arrays, should be at least 2 * 2**(BITS_PER_WORD - BITS_PER
      _DIGIT*2) */
124   #define MP_WARRAY               (1 << (sizeof(mp_word) * CHAR_BIT - 2 * DIGI
      T_BIT + 1))
125   
126   typedef struct  \{
127       int used, alloc, sign;
128       mp_digit *dp;
129   \} mp_int;
130   
131   #define USED(m)    ((m)->used)
132   #define DIGIT(m,k) ((m)->dp[k])
133   #define SIGN(m)    ((m)->sign)
134   
135   /* ---> init and deinit bignum functions <--- */
136   
137   /* init a bignum */
138   int mp_init(mp_int *a);
139   
140   /* free a bignum */
141   void mp_clear(mp_int *a);
142   
143   /* init a null terminated series of arguments */
144   int mp_init_multi(mp_int *mp, ...);
145   
146   /* clear a null terminated series of arguments */
147   void mp_clear_multi(mp_int *mp, ...);
148   
149   /* exchange two ints */
150   void mp_exch(mp_int *a, mp_int *b);
151   
152   /* shrink ram required for a bignum */
153   int mp_shrink(mp_int *a);
154   
155   /* grow an int to a given size */
156   int mp_grow(mp_int *a, int size);
157   
158   /* init to a given number of digits */
159   int mp_init_size(mp_int *a, int size);
160   
161   /* ---> Basic Manipulations <--- */
162   
163   #define mp_iszero(a) (((a)->used == 0) ? 1 : 0)
164   #define mp_iseven(a) (((a)->used == 0 || (((a)->dp[0] & 1) == 0)) ? 1 : 0)
165   #define mp_isodd(a)  (((a)->used > 0 && (((a)->dp[0] & 1) == 1)) ? 1 : 0)
166   
167   /* set to zero */
168   void mp_zero(mp_int *a);
169   
170   /* set to a digit */
171   void mp_set(mp_int *a, mp_digit b);
172   
173   /* set a 32-bit const */
174   int mp_set_int(mp_int *a, unsigned int b);
175   
176   /* copy, b = a */
177   int mp_copy(mp_int *a, mp_int *b);
178   
179   /* inits and copies, a = b */
180   int mp_init_copy(mp_int *a, mp_int *b);
181   
182   /* trim unused digits */
183   void mp_clamp(mp_int *a);
184   
185   /* ---> digit manipulation <--- */
186   
187   /* right shift by "b" digits */
188   void mp_rshd(mp_int *a, int b);
189   
190   /* left shift by "b" digits */
191   int mp_lshd(mp_int *a, int b);
192   
193   /* c = a / 2**b */
194   int mp_div_2d(mp_int *a, int b, mp_int *c, mp_int *d);
195   
196   /* b = a/2 */
197   int mp_div_2(mp_int *a, mp_int *b);
198   
199   /* c = a * 2**b */
200   int mp_mul_2d(mp_int *a, int b, mp_int *c);
201   
202   /* b = a*2 */
203   int mp_mul_2(mp_int *a, mp_int *b);
204   
205   /* c = a mod 2**d */
206   int mp_mod_2d(mp_int *a, int b, mp_int *c);
207   
208   /* computes a = 2**b */
209   int mp_2expt(mp_int *a, int b);
210   
211   /* makes a pseudo-random int of a given size */
212   int mp_rand(mp_int *a, int digits);
213   
214   /* ---> binary operations <--- */
215   /* c = a XOR b  */
216   int mp_xor(mp_int *a, mp_int *b, mp_int *c);
217   
218   /* c = a OR b */
219   int mp_or(mp_int *a, mp_int *b, mp_int *c);
220   
221   /* c = a AND b */
222   int mp_and(mp_int *a, mp_int *b, mp_int *c);
223   
224   /* ---> Basic arithmetic <--- */
225   
226   /* b = -a */
227   int mp_neg(mp_int *a, mp_int *b);
228   
229   /* b = |a| */
230   int mp_abs(mp_int *a, mp_int *b);
231   
232   /* compare a to b */
233   int mp_cmp(mp_int *a, mp_int *b);
234   
235   /* compare |a| to |b| */
236   int mp_cmp_mag(mp_int *a, mp_int *b);
237   
238   /* c = a + b */
239   int mp_add(mp_int *a, mp_int *b, mp_int *c);
240   
241   /* c = a - b */
242   int mp_sub(mp_int *a, mp_int *b, mp_int *c);
243   
244   /* c = a * b */
245   int mp_mul(mp_int *a, mp_int *b, mp_int *c);
246   
247   /* b = a*a  */
248   int mp_sqr(mp_int *a, mp_int *b);
249   
250   /* a/b => cb + d == a */
251   int mp_div(mp_int *a, mp_int *b, mp_int *c, mp_int *d);
252   
253   /* c = a mod b, 0 <= c < b  */
254   int mp_mod(mp_int *a, mp_int *b, mp_int *c);
255   
256   /* ---> single digit functions <--- */
257   
258   /* compare against a single digit */
259   int mp_cmp_d(mp_int *a, mp_digit b);
260   
261   /* c = a + b */
262   int mp_add_d(mp_int *a, mp_digit b, mp_int *c);
263   
264   /* c = a - b */
265   int mp_sub_d(mp_int *a, mp_digit b, mp_int *c);
266   
267   /* c = a * b */
268   int mp_mul_d(mp_int *a, mp_digit b, mp_int *c);
269   
270   /* a/b => cb + d == a */
271   int mp_div_d(mp_int *a, mp_digit b, mp_int *c, mp_digit *d);
272   
273   /* c = a**b */
274   int mp_expt_d(mp_int *a, mp_digit b, mp_int *c);
275   
276   /* c = a mod b, 0 <= c < b  */
277   int mp_mod_d(mp_int *a, mp_digit b, mp_digit *c);
278   
279   /* ---> number theory <--- */
280   
281   /* d = a + b (mod c) */
282   int mp_addmod(mp_int *a, mp_int *b, mp_int *c, mp_int *d);
283   
284   /* d = a - b (mod c) */
285   int mp_submod(mp_int *a, mp_int *b, mp_int *c, mp_int *d);
286   
287   /* d = a * b (mod c) */
288   int mp_mulmod(mp_int *a, mp_int *b, mp_int *c, mp_int *d);
289   
290   /* c = a * a (mod b) */
291   int mp_sqrmod(mp_int *a, mp_int *b, mp_int *c);
292   
293   /* c = 1/a (mod b) */
294   int mp_invmod(mp_int *a, mp_int *b, mp_int *c);
295   
296   /* c = (a, b) */
297   int mp_gcd(mp_int *a, mp_int *b, mp_int *c);
298   
299   /* c = [a, b] or (a*b)/(a, b) */
300   int mp_lcm(mp_int *a, mp_int *b, mp_int *c);
301   
302   /* finds one of the b'th root of a, such that |c|**b <= |a|
303    *
304    * returns error if a < 0 and b is even
305    */
306   int mp_n_root(mp_int *a, mp_digit b, mp_int *c);
307   
308   /* shortcut for square root */
309   #define mp_sqrt(a, b) mp_n_root(a, 2, b)
310   
311   /* computes the jacobi c = (a | n) (or Legendre if b is prime)  */
312   int mp_jacobi(mp_int *a, mp_int *n, int *c);
313   
314   /* used to setup the Barrett reduction for a given modulus b */
315   int mp_reduce_setup(mp_int *a, mp_int *b);
316   
317   /* Barrett Reduction, computes a (mod b) with a precomputed value c
318    *
319    * Assumes that 0 < a <= b*b, note if 0 > a > -(b*b) then you can merely
320    * compute the reduction as -1 * mp_reduce(mp_abs(a)) [pseudo code].
321    */
322   int mp_reduce(mp_int *a, mp_int *b, mp_int *c);
323   
324   /* setups the montgomery reduction */
325   int mp_montgomery_setup(mp_int *a, mp_digit *mp);
326   
327   /* computes a = B**n mod b without division or multiplication useful for
328    * normalizing numbers in a Montgomery system.
329    */
330   int mp_montgomery_calc_normalization(mp_int *a, mp_int *b);
331   
332   /* computes x/R == x (mod N) via Montgomery Reduction */
333   int mp_montgomery_reduce(mp_int *a, mp_int *m, mp_digit mp);
334   
335   /* returns 1 if a is a valid DR modulus */
336   int mp_dr_is_modulus(mp_int *a);
337   
338   /* sets the value of "d" required for mp_dr_reduce */
339   void mp_dr_setup(mp_int *a, mp_digit *d);
340   
341   /* reduces a modulo b using the Diminished Radix method */
342   int mp_dr_reduce(mp_int *a, mp_int *b, mp_digit mp);
343   
344   /* d = a**b (mod c) */
345   int mp_exptmod(mp_int *a, mp_int *b, mp_int *c, mp_int *d);
346   
347   /* ---> Primes <--- */
348   
349   /* number of primes */
350   #ifdef MP_8BIT
351      #define PRIME_SIZE      31
352   #else
353      #define PRIME_SIZE      256
354   #endif
355   
356   /* table of first PRIME_SIZE primes */
357   extern const mp_digit __prime_tab[];
358   
359   /* result=1 if a is divisible by one of the first PRIME_SIZE primes */
360   int mp_prime_is_divisible(mp_int *a, int *result);
361   
362   /* performs one Fermat test of "a" using base "b".
363    * Sets result to 0 if composite or 1 if probable prime
364    */
365   int mp_prime_fermat(mp_int *a, mp_int *b, int *result);
366   
367   /* performs one Miller-Rabin test of "a" using base "b".
368    * Sets result to 0 if composite or 1 if probable prime
369    */
370   int mp_prime_miller_rabin(mp_int *a, mp_int *b, int *result);
371   
372   /* performs t rounds of Miller-Rabin on "a" using the first
373    * t prime bases.  Also performs an initial sieve of trial
374    * division.  Determines if "a" is prime with probability
375    * of error no more than (1/4)**t.
376    *
377    * Sets result to 1 if probably prime, 0 otherwise
378    */
379   int mp_prime_is_prime(mp_int *a, int t, int *result);
380   
381   /* finds the next prime after the number "a" using "t" trials
382    * of Miller-Rabin.
383    */
384   int mp_prime_next_prime(mp_int *a, int t);
385   
386   
387   /* ---> radix conversion <--- */
388   int mp_count_bits(mp_int *a);
389   
390   int mp_unsigned_bin_size(mp_int *a);
391   int mp_read_unsigned_bin(mp_int *a, unsigned char *b, int c);
392   int mp_to_unsigned_bin(mp_int *a, unsigned char *b);
393   
394   int mp_signed_bin_size(mp_int *a);
395   int mp_read_signed_bin(mp_int *a, unsigned char *b, int c);
396   int mp_to_signed_bin(mp_int *a, unsigned char *b);
397   
398   int mp_read_radix(mp_int *a, char *str, int radix);
399   int mp_toradix(mp_int *a, char *str, int radix);
400   int mp_radix_size(mp_int *a, int radix);
401   
402   int mp_fread(mp_int *a, int radix, FILE *stream);
403   int mp_fwrite(mp_int *a, int radix, FILE *stream);
404   
405   #define mp_read_raw(mp, str, len) mp_read_signed_bin((mp), (str), (len))
406   #define mp_raw_size(mp)           mp_signed_bin_size(mp)
407   #define mp_toraw(mp, str)         mp_to_signed_bin((mp), (str))
408   #define mp_read_mag(mp, str, len) mp_read_unsigned_bin((mp), (str), (len))
409   #define mp_mag_size(mp)           mp_unsigned_bin_size(mp)
410   #define mp_tomag(mp, str)         mp_to_unsigned_bin((mp), (str))
411   
412   #define mp_tobinary(M, S)  mp_toradix((M), (S), 2)
413   #define mp_tooctal(M, S)   mp_toradix((M), (S), 8)
414   #define mp_todecimal(M, S) mp_toradix((M), (S), 10)
415   #define mp_tohex(M, S)     mp_toradix((M), (S), 16)
416   
417   /* lowlevel functions, do not call! */
418   int s_mp_add(mp_int *a, mp_int *b, mp_int *c);
419   int s_mp_sub(mp_int *a, mp_int *b, mp_int *c);
420   #define s_mp_mul(a, b, c) s_mp_mul_digs(a, b, c, (a)->used + (b)->used + 1)
421   int fast_s_mp_mul_digs(mp_int *a, mp_int *b, mp_int *c, int digs);
422   int s_mp_mul_digs(mp_int *a, mp_int *b, mp_int *c, int digs);
423   int fast_s_mp_mul_high_digs(mp_int *a, mp_int *b, mp_int *c, int digs);
424   int s_mp_mul_high_digs(mp_int *a, mp_int *b, mp_int *c, int digs);
425   int fast_s_mp_sqr(mp_int *a, mp_int *b);
426   int s_mp_sqr(mp_int *a, mp_int *b);
427   int mp_karatsuba_mul(mp_int *a, mp_int *b, mp_int *c);
428   int mp_karatsuba_sqr(mp_int *a, mp_int *b);
429   int fast_mp_invmod(mp_int *a, mp_int *b, mp_int *c);
430   int fast_mp_montgomery_reduce(mp_int *a, mp_int *m, mp_digit mp);
431   int mp_exptmod_fast(mp_int *G, mp_int *X, mp_int *P, mp_int *Y, int mode);
432   void bn_reverse(unsigned char *s, int len);
433   
434   #ifdef __cplusplus
435      \}
436   #endif
437   
438   #endif
439   
\end{alltt}
\end{small}

\end{document}

\chapter{Appendix}
\subsection*{Appendix A -- Source Listing of tommath.h}

The following is the source listing of the header file ``tommath.h'' for the LibTomMath project.  It contains many of 
the definitions used throughout the code such as \textbf{mp\_int}, \textbf{MP\_PREC} and so on.  The header is 
presented here for completeness.

\index{tommath.h}
\vspace{+3mm}\begin{small}
\hspace{-5.1mm}{\bf File}: tommath.h
\vspace{-3mm}
\begin{alltt}
001   /* LibTomMath, multiple-precision integer library -- Tom St Denis
002    *
003    * LibTomMath is library that provides for multiple-precision
004    * integer arithmetic as well as number theoretic functionality.
005    *
006    * The library is designed directly after the MPI library by
007    * Michael Fromberger but has been written from scratch with
008    * additional optimizations in place.
009    *
010    * The library is free for all purposes without any express
011    * guarantee it works.
012    *
013    * Tom St Denis, tomstdenis@iahu.ca, http://math.libtomcrypt.org
014    */
015   #ifndef BN_H_
016   #define BN_H_
017   
018   #include <stdio.h>
019   #include <string.h>
020   #include <stdlib.h>
021   #include <ctype.h>
022   #include <limits.h>
023   
024   #undef MIN
025   #define MIN(x,y) ((x)<(y)?(x):(y))
026   #undef MAX
027   #define MAX(x,y) ((x)>(y)?(x):(y))
028   
029   #ifdef __cplusplus
030   extern "C" \{
031   
032   /* C++ compilers don't like assigning void * to mp_digit * */
033   #define  OPT_CAST  (mp_digit *)
034   
035   #else
036   
037   /* C on the other hand doesn't care */
038   #define  OPT_CAST
039   
040   #endif
041   
042   /* some default configurations.
043    *
044    * A "mp_digit" must be able to hold DIGIT_BIT + 1 bits
045    * A "mp_word" must be able to hold 2*DIGIT_BIT + 1 bits
046    *
047    * At the very least a mp_digit must be able to hold 7 bits
048    * [any size beyond that is ok provided it doesn't overflow the data type]
049    */
050   #ifdef MP_8BIT
051      typedef unsigned char      mp_digit;
052      typedef unsigned short     mp_word;
053   #elif defined(MP_16BIT)
054      typedef unsigned short     mp_digit;
055      typedef unsigned long      mp_word;
056   #elif defined(MP_64BIT)
057      /* for GCC only on supported platforms */
058   #ifndef CRYPT
059      typedef unsigned long long ulong64;
060      typedef signed long long   long64;
061   #endif
062   
063      typedef ulong64            mp_digit;
064      typedef unsigned long      mp_word __attribute__ ((mode(TI)));
065   
066      #define DIGIT_BIT          60
067   #else
068      /* this is the default case, 28-bit digits */
069      
070      /* this is to make porting into LibTomCrypt easier :-) */
071   #ifndef CRYPT
072      #ifdef _MSC_VER
073         typedef unsigned __int64   ulong64;
074         typedef signed __int64     long64;
075      #else
076         typedef unsigned long long ulong64;
077         typedef signed long long   long64;
078      #endif
079   #endif
080   
081      typedef unsigned long      mp_digit;
082      typedef ulong64            mp_word;
083   
084      #define DIGIT_BIT          28
085   #endif
086   
087   /* otherwise the bits per digit is calculated automatically from the size of
       a mp_digit */
088   #ifndef DIGIT_BIT
089      #define DIGIT_BIT     ((CHAR_BIT * sizeof(mp_digit) - 1))  /* bits per di
      git */
090   #endif
091   
092   
093   #define MP_DIGIT_BIT     DIGIT_BIT
094   #define MP_MASK          ((((mp_digit)1)<<((mp_digit)DIGIT_BIT))-((mp_digit)
      1))
095   #define MP_DIGIT_MAX     MP_MASK
096   
097   /* equalities */
098   #define MP_LT        -1   /* less than */
099   #define MP_EQ         0   /* equal to */
100   #define MP_GT         1   /* greater than */
101   
102   #define MP_ZPOS       0   /* positive integer */
103   #define MP_NEG        1   /* negative */
104   
105   #define MP_OKAY       0   /* ok result */
106   #define MP_MEM        -2  /* out of mem */
107   #define MP_VAL        -3  /* invalid input */
108   #define MP_RANGE      MP_VAL
109   
110   typedef int           mp_err;
111   
112   /* you'll have to tune these... */
113   extern int KARATSUBA_MUL_CUTOFF,
114              KARATSUBA_SQR_CUTOFF,
115              MONTGOMERY_EXPT_CUTOFF;
116   
117   /* various build options */
118   #define MP_PREC                 64      /* default digits of precision (must
       be power of two) */
119   
120   /* define this to use lower memory usage routines (exptmods mostly) */
121   /* #define MP_LOW_MEM */
122   
123   /* size of comba arrays, should be at least 2 * 2**(BITS_PER_WORD - BITS_PER
      _DIGIT*2) */
124   #define MP_WARRAY               (1 << (sizeof(mp_word) * CHAR_BIT - 2 * DIGI
      T_BIT + 1))
125   
126   typedef struct  \{
127       int used, alloc, sign;
128       mp_digit *dp;
129   \} mp_int;
130   
131   #define USED(m)    ((m)->used)
132   #define DIGIT(m,k) ((m)->dp[k])
133   #define SIGN(m)    ((m)->sign)
134   
135   /* ---> init and deinit bignum functions <--- */
136   
137   /* init a bignum */
138   int mp_init(mp_int *a);
139   
140   /* free a bignum */
141   void mp_clear(mp_int *a);
142   
143   /* init a null terminated series of arguments */
144   int mp_init_multi(mp_int *mp, ...);
145   
146   /* clear a null terminated series of arguments */
147   void mp_clear_multi(mp_int *mp, ...);
148   
149   /* exchange two ints */
150   void mp_exch(mp_int *a, mp_int *b);
151   
152   /* shrink ram required for a bignum */
153   int mp_shrink(mp_int *a);
154   
155   /* grow an int to a given size */
156   int mp_grow(mp_int *a, int size);
157   
158   /* init to a given number of digits */
159   int mp_init_size(mp_int *a, int size);
160   
161   /* ---> Basic Manipulations <--- */
162   
163   #define mp_iszero(a) (((a)->used == 0) ? 1 : 0)
164   #define mp_iseven(a) (((a)->used == 0 || (((a)->dp[0] & 1) == 0)) ? 1 : 0)
165   #define mp_isodd(a)  (((a)->used > 0 && (((a)->dp[0] & 1) == 1)) ? 1 : 0)
166   
167   /* set to zero */
168   void mp_zero(mp_int *a);
169   
170   /* set to a digit */
171   void mp_set(mp_int *a, mp_digit b);
172   
173   /* set a 32-bit const */
174   int mp_set_int(mp_int *a, unsigned int b);
175   
176   /* copy, b = a */
177   int mp_copy(mp_int *a, mp_int *b);
178   
179   /* inits and copies, a = b */
180   int mp_init_copy(mp_int *a, mp_int *b);
181   
182   /* trim unused digits */
183   void mp_clamp(mp_int *a);
184   
185   /* ---> digit manipulation <--- */
186   
187   /* right shift by "b" digits */
188   void mp_rshd(mp_int *a, int b);
189   
190   /* left shift by "b" digits */
191   int mp_lshd(mp_int *a, int b);
192   
193   /* c = a / 2**b */
194   int mp_div_2d(mp_int *a, int b, mp_int *c, mp_int *d);
195   
196   /* b = a/2 */
197   int mp_div_2(mp_int *a, mp_int *b);
198   
199   /* c = a * 2**b */
200   int mp_mul_2d(mp_int *a, int b, mp_int *c);
201   
202   /* b = a*2 */
203   int mp_mul_2(mp_int *a, mp_int *b);
204   
205   /* c = a mod 2**d */
206   int mp_mod_2d(mp_int *a, int b, mp_int *c);
207   
208   /* computes a = 2**b */
209   int mp_2expt(mp_int *a, int b);
210   
211   /* makes a pseudo-random int of a given size */
212   int mp_rand(mp_int *a, int digits);
213   
214   /* ---> binary operations <--- */
215   /* c = a XOR b  */
216   int mp_xor(mp_int *a, mp_int *b, mp_int *c);
217   
218   /* c = a OR b */
219   int mp_or(mp_int *a, mp_int *b, mp_int *c);
220   
221   /* c = a AND b */
222   int mp_and(mp_int *a, mp_int *b, mp_int *c);
223   
224   /* ---> Basic arithmetic <--- */
225   
226   /* b = -a */
227   int mp_neg(mp_int *a, mp_int *b);
228   
229   /* b = |a| */
230   int mp_abs(mp_int *a, mp_int *b);
231   
232   /* compare a to b */
233   int mp_cmp(mp_int *a, mp_int *b);
234   
235   /* compare |a| to |b| */
236   int mp_cmp_mag(mp_int *a, mp_int *b);
237   
238   /* c = a + b */
239   int mp_add(mp_int *a, mp_int *b, mp_int *c);
240   
241   /* c = a - b */
242   int mp_sub(mp_int *a, mp_int *b, mp_int *c);
243   
244   /* c = a * b */
245   int mp_mul(mp_int *a, mp_int *b, mp_int *c);
246   
247   /* b = a*a  */
248   int mp_sqr(mp_int *a, mp_int *b);
249   
250   /* a/b => cb + d == a */
251   int mp_div(mp_int *a, mp_int *b, mp_int *c, mp_int *d);
252   
253   /* c = a mod b, 0 <= c < b  */
254   int mp_mod(mp_int *a, mp_int *b, mp_int *c);
255   
256   /* ---> single digit functions <--- */
257   
258   /* compare against a single digit */
259   int mp_cmp_d(mp_int *a, mp_digit b);
260   
261   /* c = a + b */
262   int mp_add_d(mp_int *a, mp_digit b, mp_int *c);
263   
264   /* c = a - b */
265   int mp_sub_d(mp_int *a, mp_digit b, mp_int *c);
266   
267   /* c = a * b */
268   int mp_mul_d(mp_int *a, mp_digit b, mp_int *c);
269   
270   /* a/b => cb + d == a */
271   int mp_div_d(mp_int *a, mp_digit b, mp_int *c, mp_digit *d);
272   
273   /* c = a**b */
274   int mp_expt_d(mp_int *a, mp_digit b, mp_int *c);
275   
276   /* c = a mod b, 0 <= c < b  */
277   int mp_mod_d(mp_int *a, mp_digit b, mp_digit *c);
278   
279   /* ---> number theory <--- */
280   
281   /* d = a + b (mod c) */
282   int mp_addmod(mp_int *a, mp_int *b, mp_int *c, mp_int *d);
283   
284   /* d = a - b (mod c) */
285   int mp_submod(mp_int *a, mp_int *b, mp_int *c, mp_int *d);
286   
287   /* d = a * b (mod c) */
288   int mp_mulmod(mp_int *a, mp_int *b, mp_int *c, mp_int *d);
289   
290   /* c = a * a (mod b) */
291   int mp_sqrmod(mp_int *a, mp_int *b, mp_int *c);
292   
293   /* c = 1/a (mod b) */
294   int mp_invmod(mp_int *a, mp_int *b, mp_int *c);
295   
296   /* c = (a, b) */
297   int mp_gcd(mp_int *a, mp_int *b, mp_int *c);
298   
299   /* c = [a, b] or (a*b)/(a, b) */
300   int mp_lcm(mp_int *a, mp_int *b, mp_int *c);
301   
302   /* finds one of the b'th root of a, such that |c|**b <= |a|
303    *
304    * returns error if a < 0 and b is even
305    */
306   int mp_n_root(mp_int *a, mp_digit b, mp_int *c);
307   
308   /* shortcut for square root */
309   #define mp_sqrt(a, b) mp_n_root(a, 2, b)
310   
311   /* computes the jacobi c = (a | n) (or Legendre if b is prime)  */
312   int mp_jacobi(mp_int *a, mp_int *n, int *c);
313   
314   /* used to setup the Barrett reduction for a given modulus b */
315   int mp_reduce_setup(mp_int *a, mp_int *b);
316   
317   /* Barrett Reduction, computes a (mod b) with a precomputed value c
318    *
319    * Assumes that 0 < a <= b*b, note if 0 > a > -(b*b) then you can merely
320    * compute the reduction as -1 * mp_reduce(mp_abs(a)) [pseudo code].
321    */
322   int mp_reduce(mp_int *a, mp_int *b, mp_int *c);
323   
324   /* setups the montgomery reduction */
325   int mp_montgomery_setup(mp_int *a, mp_digit *mp);
326   
327   /* computes a = B**n mod b without division or multiplication useful for
328    * normalizing numbers in a Montgomery system.
329    */
330   int mp_montgomery_calc_normalization(mp_int *a, mp_int *b);
331   
332   /* computes x/R == x (mod N) via Montgomery Reduction */
333   int mp_montgomery_reduce(mp_int *a, mp_int *m, mp_digit mp);
334   
335   /* returns 1 if a is a valid DR modulus */
336   int mp_dr_is_modulus(mp_int *a);
337   
338   /* sets the value of "d" required for mp_dr_reduce */
339   void mp_dr_setup(mp_int *a, mp_digit *d);
340   
341   /* reduces a modulo b using the Diminished Radix method */
342   int mp_dr_reduce(mp_int *a, mp_int *b, mp_digit mp);
343   
344   /* d = a**b (mod c) */
345   int mp_exptmod(mp_int *a, mp_int *b, mp_int *c, mp_int *d);
346   
347   /* ---> Primes <--- */
348   
349   /* number of primes */
350   #ifdef MP_8BIT
351      #define PRIME_SIZE      31
352   #else
353      #define PRIME_SIZE      256
354   #endif
355   
356   /* table of first PRIME_SIZE primes */
357   extern const mp_digit __prime_tab[];
358   
359   /* result=1 if a is divisible by one of the first PRIME_SIZE primes */
360   int mp_prime_is_divisible(mp_int *a, int *result);
361   
362   /* performs one Fermat test of "a" using base "b".
363    * Sets result to 0 if composite or 1 if probable prime
364    */
365   int mp_prime_fermat(mp_int *a, mp_int *b, int *result);
366   
367   /* performs one Miller-Rabin test of "a" using base "b".
368    * Sets result to 0 if composite or 1 if probable prime
369    */
370   int mp_prime_miller_rabin(mp_int *a, mp_int *b, int *result);
371   
372   /* performs t rounds of Miller-Rabin on "a" using the first
373    * t prime bases.  Also performs an initial sieve of trial
374    * division.  Determines if "a" is prime with probability
375    * of error no more than (1/4)**t.
376    *
377    * Sets result to 1 if probably prime, 0 otherwise
378    */
379   int mp_prime_is_prime(mp_int *a, int t, int *result);
380   
381   /* finds the next prime after the number "a" using "t" trials
382    * of Miller-Rabin.
383    */
384   int mp_prime_next_prime(mp_int *a, int t);
385   
386   
387   /* ---> radix conversion <--- */
388   int mp_count_bits(mp_int *a);
389   
390   int mp_unsigned_bin_size(mp_int *a);
391   int mp_read_unsigned_bin(mp_int *a, unsigned char *b, int c);
392   int mp_to_unsigned_bin(mp_int *a, unsigned char *b);
393   
394   int mp_signed_bin_size(mp_int *a);
395   int mp_read_signed_bin(mp_int *a, unsigned char *b, int c);
396   int mp_to_signed_bin(mp_int *a, unsigned char *b);
397   
398   int mp_read_radix(mp_int *a, char *str, int radix);
399   int mp_toradix(mp_int *a, char *str, int radix);
400   int mp_radix_size(mp_int *a, int radix);
401   
402   int mp_fread(mp_int *a, int radix, FILE *stream);
403   int mp_fwrite(mp_int *a, int radix, FILE *stream);
404   
405   #define mp_read_raw(mp, str, len) mp_read_signed_bin((mp), (str), (len))
406   #define mp_raw_size(mp)           mp_signed_bin_size(mp)
407   #define mp_toraw(mp, str)         mp_to_signed_bin((mp), (str))
408   #define mp_read_mag(mp, str, len) mp_read_unsigned_bin((mp), (str), (len))
409   #define mp_mag_size(mp)           mp_unsigned_bin_size(mp)
410   #define mp_tomag(mp, str)         mp_to_unsigned_bin((mp), (str))
411   
412   #define mp_tobinary(M, S)  mp_toradix((M), (S), 2)
413   #define mp_tooctal(M, S)   mp_toradix((M), (S), 8)
414   #define mp_todecimal(M, S) mp_toradix((M), (S), 10)
415   #define mp_tohex(M, S)     mp_toradix((M), (S), 16)
416   
417   /* lowlevel functions, do not call! */
418   int s_mp_add(mp_int *a, mp_int *b, mp_int *c);
419   int s_mp_sub(mp_int *a, mp_int *b, mp_int *c);
420   #define s_mp_mul(a, b, c) s_mp_mul_digs(a, b, c, (a)->used + (b)->used + 1)
421   int fast_s_mp_mul_digs(mp_int *a, mp_int *b, mp_int *c, int digs);
422   int s_mp_mul_digs(mp_int *a, mp_int *b, mp_int *c, int digs);
423   int fast_s_mp_mul_high_digs(mp_int *a, mp_int *b, mp_int *c, int digs);
424   int s_mp_mul_high_digs(mp_int *a, mp_int *b, mp_int *c, int digs);
425   int fast_s_mp_sqr(mp_int *a, mp_int *b);
426   int s_mp_sqr(mp_int *a, mp_int *b);
427   int mp_karatsuba_mul(mp_int *a, mp_int *b, mp_int *c);
428   int mp_karatsuba_sqr(mp_int *a, mp_int *b);
429   int fast_mp_invmod(mp_int *a, mp_int *b, mp_int *c);
430   int fast_mp_montgomery_reduce(mp_int *a, mp_int *m, mp_digit mp);
431   int mp_exptmod_fast(mp_int *G, mp_int *X, mp_int *P, mp_int *Y, int mode);
432   void bn_reverse(unsigned char *s, int len);
433   
434   #ifdef __cplusplus
435      \}
436   #endif
437   
438   #endif
439   
\end{alltt}
\end{small}

\end{document}

\chapter{Appendix}
\subsection*{Appendix A -- Source Listing of tommath.h}

The following is the source listing of the header file ``tommath.h'' for the LibTomMath project.  It contains many of 
the definitions used throughout the code such as \textbf{mp\_int}, \textbf{MP\_PREC} and so on.  The header is 
presented here for completeness.

\index{tommath.h}
\vspace{+3mm}\begin{small}
\hspace{-5.1mm}{\bf File}: tommath.h
\vspace{-3mm}
\begin{alltt}
001   /* LibTomMath, multiple-precision integer library -- Tom St Denis
002    *
003    * LibTomMath is library that provides for multiple-precision
004    * integer arithmetic as well as number theoretic functionality.
005    *
006    * The library is designed directly after the MPI library by
007    * Michael Fromberger but has been written from scratch with
008    * additional optimizations in place.
009    *
010    * The library is free for all purposes without any express
011    * guarantee it works.
012    *
013    * Tom St Denis, tomstdenis@iahu.ca, http://math.libtomcrypt.org
014    */
015   #ifndef BN_H_
016   #define BN_H_
017   
018   #include <stdio.h>
019   #include <string.h>
020   #include <stdlib.h>
021   #include <ctype.h>
022   #include <limits.h>
023   
024   #undef MIN
025   #define MIN(x,y) ((x)<(y)?(x):(y))
026   #undef MAX
027   #define MAX(x,y) ((x)>(y)?(x):(y))
028   
029   #ifdef __cplusplus
030   extern "C" \{
031   
032   /* C++ compilers don't like assigning void * to mp_digit * */
033   #define  OPT_CAST  (mp_digit *)
034   
035   #else
036   
037   /* C on the other hand doesn't care */
038   #define  OPT_CAST
039   
040   #endif
041   
042   /* some default configurations.
043    *
044    * A "mp_digit" must be able to hold DIGIT_BIT + 1 bits
045    * A "mp_word" must be able to hold 2*DIGIT_BIT + 1 bits
046    *
047    * At the very least a mp_digit must be able to hold 7 bits
048    * [any size beyond that is ok provided it doesn't overflow the data type]
049    */
050   #ifdef MP_8BIT
051      typedef unsigned char      mp_digit;
052      typedef unsigned short     mp_word;
053   #elif defined(MP_16BIT)
054      typedef unsigned short     mp_digit;
055      typedef unsigned long      mp_word;
056   #elif defined(MP_64BIT)
057      /* for GCC only on supported platforms */
058   #ifndef CRYPT
059      typedef unsigned long long ulong64;
060      typedef signed long long   long64;
061   #endif
062   
063      typedef ulong64            mp_digit;
064      typedef unsigned long      mp_word __attribute__ ((mode(TI)));
065   
066      #define DIGIT_BIT          60
067   #else
068      /* this is the default case, 28-bit digits */
069      
070      /* this is to make porting into LibTomCrypt easier :-) */
071   #ifndef CRYPT
072      #ifdef _MSC_VER
073         typedef unsigned __int64   ulong64;
074         typedef signed __int64     long64;
075      #else
076         typedef unsigned long long ulong64;
077         typedef signed long long   long64;
078      #endif
079   #endif
080   
081      typedef unsigned long      mp_digit;
082      typedef ulong64            mp_word;
083   
084      #define DIGIT_BIT          28
085   #endif
086   
087   /* otherwise the bits per digit is calculated automatically from the size of
       a mp_digit */
088   #ifndef DIGIT_BIT
089      #define DIGIT_BIT     ((CHAR_BIT * sizeof(mp_digit) - 1))  /* bits per di
      git */
090   #endif
091   
092   
093   #define MP_DIGIT_BIT     DIGIT_BIT
094   #define MP_MASK          ((((mp_digit)1)<<((mp_digit)DIGIT_BIT))-((mp_digit)
      1))
095   #define MP_DIGIT_MAX     MP_MASK
096   
097   /* equalities */
098   #define MP_LT        -1   /* less than */
099   #define MP_EQ         0   /* equal to */
100   #define MP_GT         1   /* greater than */
101   
102   #define MP_ZPOS       0   /* positive integer */
103   #define MP_NEG        1   /* negative */
104   
105   #define MP_OKAY       0   /* ok result */
106   #define MP_MEM        -2  /* out of mem */
107   #define MP_VAL        -3  /* invalid input */
108   #define MP_RANGE      MP_VAL
109   
110   typedef int           mp_err;
111   
112   /* you'll have to tune these... */
113   extern int KARATSUBA_MUL_CUTOFF,
114              KARATSUBA_SQR_CUTOFF,
115              MONTGOMERY_EXPT_CUTOFF;
116   
117   /* various build options */
118   #define MP_PREC                 64      /* default digits of precision (must
       be power of two) */
119   
120   /* define this to use lower memory usage routines (exptmods mostly) */
121   /* #define MP_LOW_MEM */
122   
123   /* size of comba arrays, should be at least 2 * 2**(BITS_PER_WORD - BITS_PER
      _DIGIT*2) */
124   #define MP_WARRAY               (1 << (sizeof(mp_word) * CHAR_BIT - 2 * DIGI
      T_BIT + 1))
125   
126   typedef struct  \{
127       int used, alloc, sign;
128       mp_digit *dp;
129   \} mp_int;
130   
131   #define USED(m)    ((m)->used)
132   #define DIGIT(m,k) ((m)->dp[k])
133   #define SIGN(m)    ((m)->sign)
134   
135   /* ---> init and deinit bignum functions <--- */
136   
137   /* init a bignum */
138   int mp_init(mp_int *a);
139   
140   /* free a bignum */
141   void mp_clear(mp_int *a);
142   
143   /* init a null terminated series of arguments */
144   int mp_init_multi(mp_int *mp, ...);
145   
146   /* clear a null terminated series of arguments */
147   void mp_clear_multi(mp_int *mp, ...);
148   
149   /* exchange two ints */
150   void mp_exch(mp_int *a, mp_int *b);
151   
152   /* shrink ram required for a bignum */
153   int mp_shrink(mp_int *a);
154   
155   /* grow an int to a given size */
156   int mp_grow(mp_int *a, int size);
157   
158   /* init to a given number of digits */
159   int mp_init_size(mp_int *a, int size);
160   
161   /* ---> Basic Manipulations <--- */
162   
163   #define mp_iszero(a) (((a)->used == 0) ? 1 : 0)
164   #define mp_iseven(a) (((a)->used == 0 || (((a)->dp[0] & 1) == 0)) ? 1 : 0)
165   #define mp_isodd(a)  (((a)->used > 0 && (((a)->dp[0] & 1) == 1)) ? 1 : 0)
166   
167   /* set to zero */
168   void mp_zero(mp_int *a);
169   
170   /* set to a digit */
171   void mp_set(mp_int *a, mp_digit b);
172   
173   /* set a 32-bit const */
174   int mp_set_int(mp_int *a, unsigned int b);
175   
176   /* copy, b = a */
177   int mp_copy(mp_int *a, mp_int *b);
178   
179   /* inits and copies, a = b */
180   int mp_init_copy(mp_int *a, mp_int *b);
181   
182   /* trim unused digits */
183   void mp_clamp(mp_int *a);
184   
185   /* ---> digit manipulation <--- */
186   
187   /* right shift by "b" digits */
188   void mp_rshd(mp_int *a, int b);
189   
190   /* left shift by "b" digits */
191   int mp_lshd(mp_int *a, int b);
192   
193   /* c = a / 2**b */
194   int mp_div_2d(mp_int *a, int b, mp_int *c, mp_int *d);
195   
196   /* b = a/2 */
197   int mp_div_2(mp_int *a, mp_int *b);
198   
199   /* c = a * 2**b */
200   int mp_mul_2d(mp_int *a, int b, mp_int *c);
201   
202   /* b = a*2 */
203   int mp_mul_2(mp_int *a, mp_int *b);
204   
205   /* c = a mod 2**d */
206   int mp_mod_2d(mp_int *a, int b, mp_int *c);
207   
208   /* computes a = 2**b */
209   int mp_2expt(mp_int *a, int b);
210   
211   /* makes a pseudo-random int of a given size */
212   int mp_rand(mp_int *a, int digits);
213   
214   /* ---> binary operations <--- */
215   /* c = a XOR b  */
216   int mp_xor(mp_int *a, mp_int *b, mp_int *c);
217   
218   /* c = a OR b */
219   int mp_or(mp_int *a, mp_int *b, mp_int *c);
220   
221   /* c = a AND b */
222   int mp_and(mp_int *a, mp_int *b, mp_int *c);
223   
224   /* ---> Basic arithmetic <--- */
225   
226   /* b = -a */
227   int mp_neg(mp_int *a, mp_int *b);
228   
229   /* b = |a| */
230   int mp_abs(mp_int *a, mp_int *b);
231   
232   /* compare a to b */
233   int mp_cmp(mp_int *a, mp_int *b);
234   
235   /* compare |a| to |b| */
236   int mp_cmp_mag(mp_int *a, mp_int *b);
237   
238   /* c = a + b */
239   int mp_add(mp_int *a, mp_int *b, mp_int *c);
240   
241   /* c = a - b */
242   int mp_sub(mp_int *a, mp_int *b, mp_int *c);
243   
244   /* c = a * b */
245   int mp_mul(mp_int *a, mp_int *b, mp_int *c);
246   
247   /* b = a*a  */
248   int mp_sqr(mp_int *a, mp_int *b);
249   
250   /* a/b => cb + d == a */
251   int mp_div(mp_int *a, mp_int *b, mp_int *c, mp_int *d);
252   
253   /* c = a mod b, 0 <= c < b  */
254   int mp_mod(mp_int *a, mp_int *b, mp_int *c);
255   
256   /* ---> single digit functions <--- */
257   
258   /* compare against a single digit */
259   int mp_cmp_d(mp_int *a, mp_digit b);
260   
261   /* c = a + b */
262   int mp_add_d(mp_int *a, mp_digit b, mp_int *c);
263   
264   /* c = a - b */
265   int mp_sub_d(mp_int *a, mp_digit b, mp_int *c);
266   
267   /* c = a * b */
268   int mp_mul_d(mp_int *a, mp_digit b, mp_int *c);
269   
270   /* a/b => cb + d == a */
271   int mp_div_d(mp_int *a, mp_digit b, mp_int *c, mp_digit *d);
272   
273   /* c = a**b */
274   int mp_expt_d(mp_int *a, mp_digit b, mp_int *c);
275   
276   /* c = a mod b, 0 <= c < b  */
277   int mp_mod_d(mp_int *a, mp_digit b, mp_digit *c);
278   
279   /* ---> number theory <--- */
280   
281   /* d = a + b (mod c) */
282   int mp_addmod(mp_int *a, mp_int *b, mp_int *c, mp_int *d);
283   
284   /* d = a - b (mod c) */
285   int mp_submod(mp_int *a, mp_int *b, mp_int *c, mp_int *d);
286   
287   /* d = a * b (mod c) */
288   int mp_mulmod(mp_int *a, mp_int *b, mp_int *c, mp_int *d);
289   
290   /* c = a * a (mod b) */
291   int mp_sqrmod(mp_int *a, mp_int *b, mp_int *c);
292   
293   /* c = 1/a (mod b) */
294   int mp_invmod(mp_int *a, mp_int *b, mp_int *c);
295   
296   /* c = (a, b) */
297   int mp_gcd(mp_int *a, mp_int *b, mp_int *c);
298   
299   /* c = [a, b] or (a*b)/(a, b) */
300   int mp_lcm(mp_int *a, mp_int *b, mp_int *c);
301   
302   /* finds one of the b'th root of a, such that |c|**b <= |a|
303    *
304    * returns error if a < 0 and b is even
305    */
306   int mp_n_root(mp_int *a, mp_digit b, mp_int *c);
307   
308   /* shortcut for square root */
309   #define mp_sqrt(a, b) mp_n_root(a, 2, b)
310   
311   /* computes the jacobi c = (a | n) (or Legendre if b is prime)  */
312   int mp_jacobi(mp_int *a, mp_int *n, int *c);
313   
314   /* used to setup the Barrett reduction for a given modulus b */
315   int mp_reduce_setup(mp_int *a, mp_int *b);
316   
317   /* Barrett Reduction, computes a (mod b) with a precomputed value c
318    *
319    * Assumes that 0 < a <= b*b, note if 0 > a > -(b*b) then you can merely
320    * compute the reduction as -1 * mp_reduce(mp_abs(a)) [pseudo code].
321    */
322   int mp_reduce(mp_int *a, mp_int *b, mp_int *c);
323   
324   /* setups the montgomery reduction */
325   int mp_montgomery_setup(mp_int *a, mp_digit *mp);
326   
327   /* computes a = B**n mod b without division or multiplication useful for
328    * normalizing numbers in a Montgomery system.
329    */
330   int mp_montgomery_calc_normalization(mp_int *a, mp_int *b);
331   
332   /* computes x/R == x (mod N) via Montgomery Reduction */
333   int mp_montgomery_reduce(mp_int *a, mp_int *m, mp_digit mp);
334   
335   /* returns 1 if a is a valid DR modulus */
336   int mp_dr_is_modulus(mp_int *a);
337   
338   /* sets the value of "d" required for mp_dr_reduce */
339   void mp_dr_setup(mp_int *a, mp_digit *d);
340   
341   /* reduces a modulo b using the Diminished Radix method */
342   int mp_dr_reduce(mp_int *a, mp_int *b, mp_digit mp);
343   
344   /* d = a**b (mod c) */
345   int mp_exptmod(mp_int *a, mp_int *b, mp_int *c, mp_int *d);
346   
347   /* ---> Primes <--- */
348   
349   /* number of primes */
350   #ifdef MP_8BIT
351      #define PRIME_SIZE      31
352   #else
353      #define PRIME_SIZE      256
354   #endif
355   
356   /* table of first PRIME_SIZE primes */
357   extern const mp_digit __prime_tab[];
358   
359   /* result=1 if a is divisible by one of the first PRIME_SIZE primes */
360   int mp_prime_is_divisible(mp_int *a, int *result);
361   
362   /* performs one Fermat test of "a" using base "b".
363    * Sets result to 0 if composite or 1 if probable prime
364    */
365   int mp_prime_fermat(mp_int *a, mp_int *b, int *result);
366   
367   /* performs one Miller-Rabin test of "a" using base "b".
368    * Sets result to 0 if composite or 1 if probable prime
369    */
370   int mp_prime_miller_rabin(mp_int *a, mp_int *b, int *result);
371   
372   /* performs t rounds of Miller-Rabin on "a" using the first
373    * t prime bases.  Also performs an initial sieve of trial
374    * division.  Determines if "a" is prime with probability
375    * of error no more than (1/4)**t.
376    *
377    * Sets result to 1 if probably prime, 0 otherwise
378    */
379   int mp_prime_is_prime(mp_int *a, int t, int *result);
380   
381   /* finds the next prime after the number "a" using "t" trials
382    * of Miller-Rabin.
383    */
384   int mp_prime_next_prime(mp_int *a, int t);
385   
386   
387   /* ---> radix conversion <--- */
388   int mp_count_bits(mp_int *a);
389   
390   int mp_unsigned_bin_size(mp_int *a);
391   int mp_read_unsigned_bin(mp_int *a, unsigned char *b, int c);
392   int mp_to_unsigned_bin(mp_int *a, unsigned char *b);
393   
394   int mp_signed_bin_size(mp_int *a);
395   int mp_read_signed_bin(mp_int *a, unsigned char *b, int c);
396   int mp_to_signed_bin(mp_int *a, unsigned char *b);
397   
398   int mp_read_radix(mp_int *a, char *str, int radix);
399   int mp_toradix(mp_int *a, char *str, int radix);
400   int mp_radix_size(mp_int *a, int radix);
401   
402   int mp_fread(mp_int *a, int radix, FILE *stream);
403   int mp_fwrite(mp_int *a, int radix, FILE *stream);
404   
405   #define mp_read_raw(mp, str, len) mp_read_signed_bin((mp), (str), (len))
406   #define mp_raw_size(mp)           mp_signed_bin_size(mp)
407   #define mp_toraw(mp, str)         mp_to_signed_bin((mp), (str))
408   #define mp_read_mag(mp, str, len) mp_read_unsigned_bin((mp), (str), (len))
409   #define mp_mag_size(mp)           mp_unsigned_bin_size(mp)
410   #define mp_tomag(mp, str)         mp_to_unsigned_bin((mp), (str))
411   
412   #define mp_tobinary(M, S)  mp_toradix((M), (S), 2)
413   #define mp_tooctal(M, S)   mp_toradix((M), (S), 8)
414   #define mp_todecimal(M, S) mp_toradix((M), (S), 10)
415   #define mp_tohex(M, S)     mp_toradix((M), (S), 16)
416   
417   /* lowlevel functions, do not call! */
418   int s_mp_add(mp_int *a, mp_int *b, mp_int *c);
419   int s_mp_sub(mp_int *a, mp_int *b, mp_int *c);
420   #define s_mp_mul(a, b, c) s_mp_mul_digs(a, b, c, (a)->used + (b)->used + 1)
421   int fast_s_mp_mul_digs(mp_int *a, mp_int *b, mp_int *c, int digs);
422   int s_mp_mul_digs(mp_int *a, mp_int *b, mp_int *c, int digs);
423   int fast_s_mp_mul_high_digs(mp_int *a, mp_int *b, mp_int *c, int digs);
424   int s_mp_mul_high_digs(mp_int *a, mp_int *b, mp_int *c, int digs);
425   int fast_s_mp_sqr(mp_int *a, mp_int *b);
426   int s_mp_sqr(mp_int *a, mp_int *b);
427   int mp_karatsuba_mul(mp_int *a, mp_int *b, mp_int *c);
428   int mp_karatsuba_sqr(mp_int *a, mp_int *b);
429   int fast_mp_invmod(mp_int *a, mp_int *b, mp_int *c);
430   int fast_mp_montgomery_reduce(mp_int *a, mp_int *m, mp_digit mp);
431   int mp_exptmod_fast(mp_int *G, mp_int *X, mp_int *P, mp_int *Y, int mode);
432   void bn_reverse(unsigned char *s, int len);
433   
434   #ifdef __cplusplus
435      \}
436   #endif
437   
438   #endif
439   
\end{alltt}
\end{small}

\end{document}